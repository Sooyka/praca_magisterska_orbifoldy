\documentclass[a4paper, 12pt]{report}
\synctex=1
\usepackage{polski}
\usepackage[T1]{fontenc}
\usepackage{amssymb}
\usepackage[polish,english]{babel}
\usepackage[utf8]{inputenc}
\usepackage{mathtools}
\usepackage{microtype}
\usepackage{amsthm}
\usepackage{amsmath}
\usepackage{amsfonts}
\usepackage{empheq}
\usepackage{xcolor}
\usepackage[pdftex,
            pdfauthor={Bartosz Sójka},
            pdftitle={Orbifoldy},
        %    pdfsubject={Orbifoldy}
            ]{hyperref}
\usepackage{siunitx}
\usepackage{geometry}
\usepackage{lipsum}
\usepackage{marvosym}
%\usepackage{enumitem}
%\usepackage{alltt}
\usepackage{listings}
\setlength\parindent{0pt}

%\topmargin = -1in

\geometry{a4paper, 
twoside,
%asymmetric,
top = 25mm,
bottom = 40mm,
inner = 35mm,
outer = 25mm
}

\lstset{numbers=left, numberstyle=\small, stepnumber=1, numbersep=5pt, mathescape=true}

%\linespread{1.24}

%top = 25mm,
%bottom = 30mm,
%inner = 30mm,
%outer = 25mm

\newtheorem{theorem}{Theorem}[subsection]
\newtheorem{observation}[theorem]{Observation}
\newtheorem{definition}[theorem]{Definition}
\newtheorem{lemma}[theorem]{Lemma}
\newtheorem{corollary}[theorem]{Corollary}

\newcommand{\todo}[1]{\hfill \break \textbf{\Huge \textcolor{violet}{TO DO: #1} \hfill \break}
\normalsize}
\newcommand{\content}[1]{\hfill \break \textbf{\large \textcolor{violet}{#1} \hfill \break}
\normalsize}
\newcommand{\smalltodo}[1]{\textbf{\ \textcolor{violet}{To do}}}
\newcommand{\smalltodoII}[1]{\hfill \break \textbf{\ \textcolor{violet}{To do: #1}}\hfill \break}
\newcommand{\missingpicture}[1]{\hfill \break\\[16pt] \Huge \textbf{\textcolor{violet}
{Missing picture 
\normalsize
#1}} \hfill
\break \\[16pt] \normalsize}

%\newcommand{\beginproof}{{\noindent\textbf{Proof.}\\}}
%\newcommand{\endproof}{{$_\square$}}

\newcommand{\sIS}{{\mathcal{\sigma}^I(S^2)}}
\newcommand{\sbD}{{\mathcal{\sigma}^b(D^2)}}

\newcommand{\Eoc}{Euler orbicharacteristic} 
\newcommand{\apots}{accumulation point of the set}

\renewcommand{\labelenumii}{\theenumii}
\renewcommand{\theenumii}{\theenumi.\arabic{enumii}.}

\title{Orbifoldy}
\author{Bartosz Sójka}
%\date{}
\begin{document}
% mainfile: ../main/praca_magisterska_orbifoldy.tex
\newpage
\thispagestyle{empty}
\begin{center}
\textbf{\large Uniwersytet Wrocławski\\
Wydział Matematyki i Informatyki\\
Instytut Matematyczny}\\
\textit{\large specjalność: teoretyczna}\\
\vspace{4cm}
\textbf{\textit{\large Bartosz Sójka}\\
\vspace{0.5cm}
{\Large Areas of two dimensional hyperbolic orbifolds}}\\
\end{center}
\vspace{3cm}
{\large \hspace*{6.5cm}Praca magisterska\\
\hspace*{6.5cm}napisana pod kierunkiem\\
\hspace*{6.5cm}prof. dr hab. Tadeusza Januszkiewicza }\\
\vfill
\begin{center}
{\large Wrocław 2021-2023}\\
\end{center}

\tableofcontents
\begin{abstract}
\setcounter{page}{4}
%\begin{center}
%Orbifolds! Yeah! \\
%Spectrums! Yeah! 
%\end{center}
Orbifoldy
\end{abstract}
% mainfile: ../praca_magisterska_orbifoldy.tex
\chapter{Introduction}

% mainfile: ../praca_magisterska_orbifoldy.tex
\chapter{Definition of an orbifold}

In the definition of the orbifold we are following Thurston from \cite{Thurston1979} (chapter 13). 
We will briefly present here the concept, but we encourae the reader to consult 
\cite{Thurston1979}. \\
%We define an orbifold as a generalisation of manifold, where, istead of having homeomorphisms 
%between open sets of a manifold an $\mathbb{R}^n$ we have that for every point in the orbifold 
%there is a neighbourhood of it that is the image of the quotient map from the finito ... 
%\smalltodoII{rozwinąć powyżej}
%It is descriped very well and approunchable in \cite{Thurston1979} (chapter 13) and we would like 
%to redirect reader there.
We define an orbifold as a generalisation of a manifold. The difference is in maps. \\
On a manifolds a map is a homeomorphism between $\mathbb{R^n}$ and some open set on a manifold. 
On an orbifold a map is a homeomorphism between a quotient of $\mathbb{R^n}$ by some 
finite group and some open set on a orbifold.
In addition to that, the orbifold structure consist the informations about that finite group 
and a quotient map for any such open set. \\ 



Wersje: \\
- iloraz powierzchni przez grupe (wychodzą tylko dobre) \\
- rozmaitość z dodanymi punktami osobliwymi \\

Jak ma wyglądać:
jedna definicja (Thurstona)
notacja może byc od Conwaya

skleić drugi i trzeci rozdział

zwięźle własności


dobre i złe w trzecim rozdziale 

orbikompleksy może nie


This chapter and the next will be technical chapters. Later on we will evoke some terms and 
definitions without 
explicitly saying what they mean instead we will put a reference to this chapter with explicit 
saying to what definition it refers. \\
For example in the later chapters there will be phrases like "adding a defect of order $\dots$" or 
"gluing orbifolds by boundaries" and they are explained in this and the next chapter.
 

We will explore various definitions of an orbifold, partially proving they are equivalent, partially 
linking to the sources. \\
Some of these definitions apply only to the special cases. Some of them contain constructions 
with which not all orbifolds can be made (at least some of them can't be derived as such a priori)
. \\

\section{Where did orbifolds come from}


\section{Quotients of manifolds with the respect to group action}
\subsection{Hyperbolic plane tilling}

\section{Manifolds with defects}
\subsection{Disk and sphere with defects}\label{Disk_and_sphere_with_defects}



\subsection{Conway notation}
\cite{Conway2008}

When it is necessary to avoid a confusion, on parts such as $*abcd$, we will be writing 
$*^*a^*b^*c$ instead. \\
We will propose some extension to a notation from \cite{Conway2008}.
We will regard parts of that notation not only as features on an orbifold but also as an operations 
on orbifolds transforming one to another by adding particular feature. \\
We will denote the difference in Euler characteristic which is made by modifying 
an orbifold by such a feature as $\Delta(modification)$ 
which have less capitalistic vibes than "cost". 
For example $\Delta(*^*2) = \frac{1}{4}$. \\
We will denote by $*$ an operation of cutting out a disk and by $^{\beta*}n$ an operation of 
adding a kaleidoscopic point of period $n$ on the boundary component $\beta$. 
Last operation is defined only on orbifolds with boundaries.
%identified by Id. 
%When we will not concern ourselves on which component of the boundry kaleidoscopic point will 
%be added we


\section{Quotients of planes}

\section{Generalised manifolds}
This approach is very similar to the previous one. It differs slightly where we put the 
definition burden. 

%\section{Complexes?}


../LaTeX/TeX_files/characteristics_classification_and_properties_of_the_orbifolds.tex
% mainfile: ../praca_magisterska_orbifoldy.tex
%\synctex=1
\chapter{Order structure}

In this chapter we will discuss order type of the set of all possible Euler orbicharacteristics 
of two dimentional orbifolds. \\
For now, until Chapter \ref{Counting_occurrences} Counting occurrences, we will not pay attension 
to how many orbifolds have the same Euler orbicharacteristic. \\ 
Because of that and since Euler orbicharacteristic does not depend on the cyclic order 
of points on the conponents of the boundry we introduce an extension of a notation from 
\cite{Conway2008}. 

%We will write $*\{a,b,c,d,\dots\}$ to denote not a particular orbifold, but a type 
%of orbifold that have a cones on a component of the boundry of orders $a,b,c,d,\dots$, but 
%in any order. \\ 

We will write $*\{a,b,c,d,\dots\}$ to denote a type of a boundry (of an orbifold) that have 
kaleidoscopic points of periods $a,b,c,d,\dots$, but in any order. \\


From what we wrote above (that Euler orbicharacteristic does not depend on the cyclic order 
of points on the conponents of the boundry), we can see that Euler 
orbicharacteristic is well defined 
%on such types of orbifolds. \\ 
when we specify only such a type of the components of the boundry of an orbifold and not 
a particular cyclic order.  \\

\section{Reductions of cases}
Now we want to make some reductions to limit number of cases that we will be dealing with. \\
For this chapter we will consider orbifolds according to a definition from 
(\ref{Disk_and_sphere_with_defects}). \\
Let us observe, that:
\begin{align*}
\hspace{3cm}\Delta(\circ) =& \hspace{-1cm} &-2& \hspace{-1cm} &= \Delta(*(^*2)^4) \\
\hspace{3cm}\Delta(*) =& \hspace{-1cm} &-1& \hspace{-1cm} &= \Delta((^*2)^4) \\
\hspace{3cm}\Delta(n) =& \hspace{-1cm} &\frac{n-1}{n}& \hspace{-1cm} &= \Delta((^*n)^2)
\end{align*}

From this we can conclude, that every Euler orbicharacteristic can be obtained 
by an orbifold of signature of a type ($n$ and $m$ are arbitrary):

\begin{align*}
I_1I_2\dots I_n & \textrm{or} \\
*b_1b_2\dots b_m &.
\end{align*}

Let us denote the set of all possible Euler orbicharacteristics of orbifolds of the form 
$I_1I_2\dots I_n$ by $\sIS$ and the set 
of all possible Euler orbicharacteristics of orbifolds of the form $*b_1b_2\dots b_m$ 
as $\sbD$

Let us observe that the topological structure of $\sIS$ and $\sbD$ are the same since 

\begin{equation*}
2\sbD=\sIS
\end{equation*}
So multiplying by $2$ is the homeomorphism. 
\section{Determining the order structure}
In this chapter we will justify, that the order type of all possible Euler orbicharacteristics 
of two dimentional orbifolds is $\omega^\omega$. 
We will also describe precisely where condensation points lie and of which order 
(see below \ref{condensation_points_definitions}) they are.
\subsection{Definitions regarding order of condensation points}
\label{condensation_points_definitions} 
We start with one technical definition of "transitive order" that will be almost what we want
and then, there will be the the definition of "order", which is the definition that we need. \\ 
\begin{definition}
(Inductive). 
%We say, that the point $a$  from the topological space $X$ is a condensation 
%point of the transitive order 0, when
We say that the point is a condensation point of a transitive order $0$, when it is 
an isolated point. 
We say that the point is a condensation point of a transitive order $n + 1$, when it is 
a condensation point (in the usual sense) of the condensation points of the transitive order $n$. 
\end{definition}  
The only issue of the definition is that the point of the transitive order $n$ is also a point 
of the transitive order $k$, for all $0< k \leq n$. We want a definition of order such that 
for any point, there is at most one integer that is its order. So we define:
\begin{definition}
We say that the point is a condensation point of order $n$ iff it is a condensation point 
of the transitive order $n$ and it is not a condensation point of the transitive order $n+1$. 
If the point is a condensation point of the transitive order for an arbitraly large $n$ we say that 
the  point is a condensation point of order $\omega$.
\end{definition}

\subsection{$\sbD$}
\subsubsection{Some preliminary observations.}
Let us observe, that $\lim\limits_{n \to \infty} \Delta(^*n) = -\frac{1}{2}$. From that, we see, 
that for every point $x \in \sbD$, the point $x - \frac{1}{2}$ is a condensation point. 
Let us observe, that also, for every point $x \in \sbD$, we have that $x - \frac{1}{2} \in \sbD$, 
because $\Delta((^*2)^2) = -\frac{1}{2}$. \\

Now we will show that the order type of $\sbD$ is $\omega^\omega$ and where exactly are 
its condensation points of which orders. For this we will use  
a handfull of lemmas. 

\begin{lemma}
If $x$ is a condensation point of the set $\sbD$ of order $n$, then $x-\frac{1}{2}$ is a
 condensation point of the set $\sbD$ of order at least $n+1$. 
\end{lemma}
\textbf{Proof.} \\
Inductive. \\
$\bullet$ $n = 0$: If $x$ is an isolated point of the set $\sbD$, then $x \in \sbD$. From that, we 
have, that points $x - \frac{k-1}{2k}$ are in $\sbD$, from that, that $x-\frac{1}{2}$ is a 
condensation point of $\sbD$. \\
$\bullet$ inductive step: Let $x$ be a condensation point of the set $\sbD$ of an order $n > 0$. 
Let $a_k$ be a sequence od condensation points of order $n-1$ convergent to $x$. From the 
inductive assumption, we have, that $a_k - \frac{1}{2}$ is a sequence of condensation points 
of order at least $n$. From the basic sequence arithmetic it is convergent to $x-\frac{1}{2}$. 
From that, we have that $x-\frac{1}{2}$ is a condensation point of the set $\sbD$ of order 
at least $n+1$. $_\square$
\begin{lemma}
If $x$ is a condensation point of the set $\sbD$ of order $n+1$, then $x+\frac{1}{2}$ is 
a condensation point of the set $\sbD$ of order at least $n$.  
\end{lemma}
\textbf{Proof.} \\
Inductive \\
$\bullet$ n = 0: 







% mainfile: ../praca_magisterska_orbifoldy.tex
\chapter{Algorithm for counting occurences}\label{algorithm}


\section{Counting occurences algorithm}

%We want to determine this $n$. If $n = 0$, then $\frac{p}{q}$ is not in $\sigma$. 
%If $n > 0$, then $$
%\subsection{Deciding number of occurences}
Searching for all occurences 

The difficulty here is to carefully step other an occurence. 

Compared to the previous version, we also use an occurance counter, starting with it set to 0 
and with the list of orbifolds, wich is empty at the start.
\begin{lstlisting}[firstnumber=1,consecutivenumbers=true]
In the case, the flag is set to: 
{
    "Less", then 
    {
        We increase the pivot counter by one ($b_c \coloneqq b_c + 1$).
        If $b_c = 2$ and the values of all the counters 
        on the left are also equal 2 then 
        {
            We end the whole algorithm with the answer "no".
        }
        We set the value of all counters on the left to $b_c$
        If $\chi^{orb}(*b_1b_2b_3\dots)=\frac{p}{q}$ then
        {
            We found an orbifold, we add it to a list 
            and increase the occurence counter by 1. 
            We set the flag to "Less".
            We put pivot to the $c+1$ counter.
            We go to the line 1..
        }
        If $\chi^{orb}(*b_1b_2b_3\dots)>\frac{p}{q}$ then  
        {
            We set the flag to "Greater".
            We put the pivot on the first counter. 
            We go to the line 1..
        } 
        If $\chi^{orb}(*b_1b_2b_3\dots)<\frac{p}{q}$ then
        {
            We set the flag to "Less".
            We put pivot to the $c+1$ counter.
            We go to the line 1..
        } 
    }

    "Greater", then
    {
        If $\chi^{orb}(*b_1\dots b_{c-1}\infty b_{c+1}\dots)=\frac{p}{q}$ then
        {
            We found an orbifold, we add it to a list 
            and increase the occurence counter by 1. 
            We set the flag to "Less".
            We put pivot to the $c+1$ counter.
            We go to the line 1..
        } 
        If $\chi^{orb}(*b_1\dots b_{c-1}\infty b_{c+1}\dots)>\frac{p}{q}$ then
        {
            We set $b_c$ to $\infty$.
            We set the flag to "Greater".
            We move pivot to the $c+1$ counter.
            We go to the line 1..
        }  
        If $\chi^{orb}(*b_1\dots b_{c-1}\infty b_{c+1}\dots)<\frac{p}{q}$ then
        {
            We search for value $b_c'$ of the $c$ counter 
            such that $\chi^{orb}(*b_1\dots b_{c-1}b_c'b_{c+1}\dots)\leq\frac{p}{q}$ 
            and $\chi^{orb}(*b_1\dots b_{c-1}(b_c'-1)b_{c+1}\dots)>\frac{p}{q}$.
            // More on how we search for it will be told later, 
            // for now we can think that we search one by one,
            // starting from $b_c$ and going up till $b_c'$.
            We set $b_c$ to $b_c'$.
            if $\chi^{orb}(*b_1b_2b_3\dots)=\frac{p}{q}$ then 
            {
                We found an orbifold, we add it to a list 
                and increase the occurence counter by 1. 
                We set flag to "Less".
                We go to the line 1..
            }
            We set all the counters to the left to value $b_c$.
            if $\chi^{orb}(*b_1b_2b_3\dots)=\frac{p}{q}$ then 
            {
                We found an orbifold, we add it to a list 
                and increase the occurence counter by 1. 
                We set flag to "Less".
                We move the pivot to the column $c+1$.
                We go to the line 1..
            }
            If $\chi^{orb}(*b_1b_2b_3\dots)<\frac{p}{q}$ then 
            {
                We set flag to "Less".
                We move the pivot to the column $c+1$.
                We go to the line 1..
            }
            If $\chi^{orb}(*b_1b_2b_3\dots)>\frac{p}{q}$ then 
            {
                We set the flag to "Greater".
                We move the pivot to the first counter.
                We go to the line 1..
            }
        }  
    }
}
\end{lstlisting}

\section{Why this works}

\section{Deciding the order}
Let $m \in \mathbb{N}$ be such that $\frac{p}{q} \in (1-\frac{m}{2},1-
\frac{m+1}{2})$
Let us denote by $r \coloneqq \frac{p}{q} - (1-\frac{m}{2})$. \\ 

We will searching in $\sigma$ as such: \\

If $\frac{p}{q} \in \sigma$, then, from the corollary \ref{predescors} we know, that there 
exist some $n \in \mathbb{N}$, such that $\frac{p}{q} + \frac{n}{2} \in \sigma$ but 
$\frac{p}{q} + \frac{n}{2} \not\in \sigma$. \\

We will be consequently checking points from $1+r$, through $1+r-\frac{l}{2}$, for 
$0 \leq l \leq m$, to the $\frac{p}{q}$. We stop at the first found point. 
If one of these point is in the spectrum, then all smaller (so also $\frac{p}{q}$) are in 
the spectrum and $\frac{p}{q}$ is the accumulation point of the spectrum of order $m-l$ 
(from this, 
we can see some heuristic, that the points that have smaller order will be generally 
harder to find in some sense). If none of this points are in in the spectrum, then $\frac{p}{q}$ 
is not. \\




\section{Implementation}
As an appendix in the separate document, there is a source of a program with implementation 
of this algorithm 
with full  
enhancments described in this chapter. It is written in Rust. 
It can be also found on \smalltodoII{dać ref do github} along with the \LaTeX\ source of 
this thesis.
%It is in the separate file, as it would take too much space in this 
%document and wouldn't be readable. 




../counting_occurrences/counting_occurrences.tex
../LaTeX/TeX_files/power_series_and_generating_functions.tex
%../connection_with_modular_forms/connection_with_modular_forms.tex
% mainfile: ../praca_magisterska_orbifoldy.tex
\chapter{Conclusions}

\section{What was done}
In \ref{reduction_to_arithmetical} we proved that $\speD \not\subseteq \speS$ and 
$\speS \not\subseteq \speD$. 

In chapter \ref{order structure}, among other things, we described the spectrum of possible \Eoc s 
of two dimensional orbifolds
and, as the result, the spectrum of all possible areas 
of two dimensional hyperbolic orbifolds in a ordinal and topological manner. 
It has order type and topology (induced from $\mathbb{R}$) of $\omega^\omega$. 
We also proved, that every accumulation point of $\speS$ is in $\speD$.
%and the problem, whether the given point is in the 
%spectrum is desidable. \\

In chapter \ref{Searching the spectrum} 
we provided algorithm for deciding for a given number $x$, whether 
there exists an orbifold $O$, such that $\cho{O} = x$ and proved its correctness.

In chapter \ref{counting occurrences} we provided some finiteness results, such as that 
there are always only finitely many 
orbifolds for a given \Eoc. 
%So the problem how much are for a given number is also decidable.\\
We also proved that for every $n$, in every neighbourhood of every accumulation point 
of $\spe$ of order at least $2$, there is at least one number $x$, such that there are at least 
$n$ orbifolds such that $\cho{O} = x$.

In chapter \ref{Counting orbifolds -- arithmetical part} 
and chapter \ref{Counting orbifolds -- combinatorical part} 
we provided an algorithm 
 for counting for a given number $x$ number of orbifold such that $\cho{O} = x$,
and proved its correctness. 
 
%of this algorithm and discussed, 
We also discussed that its complexity is low enough for actual implementation 
and practical usage on a reasonably small denominators and reasonably close to zero.
%From \smalltodo{referencja} we know, that there are howe ver, blab la dowolnie dużo na \Eoc. \\



%for every denominator, do they coincide 
%from a sufficiently distatn point? (Yes.) 

\section{Further directions}
It remains unclear how Disk spectrum and Sphere spectrum lies relative to each other. 
In particular we still don't know, whether they coincide from a sufficiently distant point.
%but some result was, shown, namely, that every accumulation point of Sphere spectum
%is also in the disk spectrum.

%\subsection{Asked, but unanswered questions}
%Our ultimate goal would to give the answer to the questions such as: \\
%- For a given $x \in \sigma$, how many orbifolds have $x$ as their \Eoc?\\
We don't really know why there is exactly "this" many orbifolds for a given \Eoc? 
%Giving only the algorithm gave little 
We would like to know, whether there is some underlying geometrical reason for that?

We would like to somehow characterise points $x \in \sigma$ that has "the most" 
orbifolds corresponding to them. With reasonable normalisation of what it means for a number
to have "more" orbifolds as we go to lesser values of \Eoc. 

%- Is there any reasonable normalisation to counter the effect that there are 'more' 
%points as we go 
%to lesser values. (What we mean by 'more' was sted in) \\
%The first equstion we can tackle is steaming from the chapter \ref{order structure} 
%and it is -- Do $\speD$ and $\speS$ coincide? It is easy to answer that $\speD \neq \speS$ 
%(and we will do that along some harder questions in the moment), but do they coincide 
%starting from a sufficiently distant point? 
%Or maybe, for every denominator, do they coincide 
%from a sufficiently distatn point? (Yes.) \\
%\todo{przenieść (przekopiować?) część może do futher direction}
%write about cyclic order
%\subsection{Unasked and unanswered questions}
%\subsection{Power series and generating functions}
%\subsection{Seifert manifolds}


%% mainfile: ../praca_magisterska_orbifoldy.tex
\chapter{test}
abcd

% mainfile: ../praca_magisterska_orbifoldy.tex
\nocite{*}
\bibliography{../bibliography/bibliography.bib}{}
\bibliographystyle{plain}

%~\cite{Conway2002}
\end{document}



