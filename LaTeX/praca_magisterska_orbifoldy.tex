\documentclass[a4paper, 12pt]{report}
\synctex=1

% mainfile: ./praca_magisterska_orbifoldy.tex

\usepackage{polski}
\usepackage[T1]{fontenc}
\usepackage{amssymb}
\usepackage[polish,english]{babel}
\usepackage[utf8]{inputenc}
\usepackage{mathtools}
\usepackage{microtype}
\usepackage{amsthm}
\usepackage{amsmath}
\usepackage{amsfonts}
\usepackage{amsrefs}
\usepackage{empheq}
\usepackage{xcolor}
\usepackage[pdftex,
            pdfauthor={Bartosz Sójka},
            pdftitle={Orbifoldy},
        %    pdfsubject={Orbifoldy}
            ]{hyperref}
\usepackage{siunitx}
\usepackage{geometry}
\usepackage{lipsum}
\usepackage{marvosym}
%\usepackage{enumitem}
%\usepackage{alltt}
\usepackage{listings}
%\usepackage{pifont}
\setlength\parindent{0pt}

%\topmargin = -1in

\geometry{a4paper, 
%twoside,
%asymmetric,
top = 25mm,
bottom = 40mm,
inner = 35mm,
outer = 25mm
}

\lstset{numbers=left, numberstyle=\small, stepnumber=1, numbersep=5pt, mathescape=true}

%\linespread{1.24}

%top = 25mm,
%bottom = 30mm,
%inner = 30mm,
%outer = 25mm

\newtheorem{theorem}{Theorem}[subsection]
\newtheorem{observation}[theorem]{Observation}
\newtheorem{definition}[theorem]{Definition}
\newtheorem{lemma}[theorem]{Lemma}
\newtheorem{corollary}[theorem]{Corollary}

\numberwithin{equation}{theorem}

\newcommand{\todo}[1]{\hfill \break \textbf{\Huge \textcolor{violet}{TO DO: #1} \hfill \break}
\normalsize}
\newcommand{\content}[1]{\hfill \break \textbf{\large \textcolor{violet}{#1} \hfill \break}
\normalsize}
\newcommand{\smalltodo}[1]{\textbf{\ \textcolor{violet}{To do}}}
\newcommand{\smalltodoII}[1]{\hfill \break \textbf{\ \textcolor{violet}{To do: #1}}\hfill \break}
\newcommand{\missingpicture}[1]{\hfill \break\\[16pt] \Huge \textbf{\textcolor{violet}
{Missing picture \normalsize #1}} \hfill
\break \\[16pt] \normalsize}

\newcommand{\lrb}{\textnormal{\Big[}}
\newcommand{\rrb}{\textnormal{\Big]}}

\newcommand{\rba}[1]{\lrb #1\rrb}

\newcommand{\lcb}{\textnormal{\Big(}}
\newcommand{\rcb}{\textnormal{\Big)}}

\newcommand{\cba}[1]{\lcb #1\rcb}

\newcommand{\ds}{\textnormal{\big/\!\!\big/}}
\newcommand{\lds}{\textnormal{\big/\!\!\big/\!}}
\newcommand{\rds}{\!\textnormal{\big/\!\!\big/}}

\newcommand{\dsa}[1]{\ds #1\ds}



%\newcommand{\beginproof}{{\noindent\textbf{Proof.}\\}}
%\newcommand{\endproof}{{$_\square$}}

\renewcommand{\labelenumii}{\theenumii}
\renewcommand{\theenumii}{\theenumi.\arabic{enumii}.}


\newcommand{\cho}[1]{{\chi^{orb}(#1)}}

\newcommand{\srS}{{\sigma^r(S^2)}}
\newcommand{\sdD}{{\sigma^d(D^2)}}

\newcommand{\spe}{{\sigma}}
\newcommand{\speS}{{\spe(S^2)}}
\newcommand{\speD}{{\spe(D^2)}}
\newcommand{\spebr}[1]{{\spe(#1)}}


\newcommand{\Eoc}{Euler orbicharacteristic} 
\newcommand{\apots}{accumulation point of the set}


%\title{Orbifoldy}
\title{Areas of two dimensional hyperbolic orbifolds}
\author{Bartosz Sójka}
%\date{}
\begin{document}
% mainfile: ../main/praca_magisterska_orbifoldy.tex
\newpage
\thispagestyle{empty}
\begin{center}
\textbf{\large Uniwersytet Wrocławski\\
Wydział Matematyki i Informatyki\\
Instytut Matematyczny}\\
\textit{\large specjalność: teoretyczna}\\
\vspace{4cm}
\textbf{\textit{\large Bartosz Sójka}\\
\vspace{0.5cm}
{\Large Areas of two dimensional hyperbolic orbifolds}}\\
\end{center}
\vspace{3cm}
{\large \hspace*{6.5cm}Praca magisterska\\
\hspace*{6.5cm}napisana pod kierunkiem\\
\hspace*{6.5cm}prof. dr hab. Tadeusza Januszkiewicza }\\
\vfill
\begin{center}
{\large Wrocław 2021-2023}\\
\end{center}

\newpage
\null
\thispagestyle{empty}

%% mainfile: ./praca_magisterska_orbifoldy.tex
\newpage
\thispagestyle{empty}
\vspace*{19cm}
\hspace*{10cm}
\textit{dla Wujka}
\newpage
\null
\thispagestyle{empty}

\newpage
\thispagestyle{empty}
\vspace*{19cm}
\hspace*{10cm}
\textit{dla Wujka}
\newpage
\null
\thispagestyle{empty}

\newpage
\tableofcontents
\begin{abstract}
\setcounter{page}{7}
%\begin{center}
%Orbifolds! Yeah! \\
%Spectrums! Yeah! 
%\end{center}
%Orbifoldy
In this thesis we aim to describe the spectrum of all possible areas of 
two dimensional orbifold, in particular those of negative \Eoc. 
We will analyse the spectrum treated both as a subset of $\mathbb{R}$ 
and then study its order type and topology; 
and as an image of $\chi^{orb}$ -- \Eoc\ and then count orbifolds corresponding 
to a particular points of a spectrum. 
\end{abstract}
% mainfile: ../praca_magisterska_orbifoldy.tex
\chapter{Introduction}

%% mainfile: ../praca_magisterska_orbifoldy.tex
\chapter{Definition of an orbifold}

In the definition of the orbifold we are following Thurston from \cite{Thurston1979} (chapter 13). 
We will briefly present here the concept, but we encourae the reader to consult 
\cite{Thurston1979}. \\
%We define an orbifold as a generalisation of manifold, where, istead of having homeomorphisms 
%between open sets of a manifold an $\mathbb{R}^n$ we have that for every point in the orbifold 
%there is a neighbourhood of it that is the image of the quotient map from the finito ... 
%\smalltodoII{rozwinąć powyżej}
%It is descriped very well and approunchable in \cite{Thurston1979} (chapter 13) and we would like 
%to redirect reader there.
We define an orbifold as a generalisation of a manifold. The difference is in maps. \\
On a manifolds a map is a homeomorphism between $\mathbb{R^n}$ and some open set on a manifold. 
On an orbifold a map is a homeomorphism between a quotient of $\mathbb{R^n}$ by some 
finite group and some open set on a orbifold.
In addition to that, the orbifold structure consist the informations about that finite group 
and a quotient map for any such open set. \\ 



Wersje: \\
- iloraz powierzchni przez grupe (wychodzą tylko dobre) \\
- rozmaitość z dodanymi punktami osobliwymi \\

Jak ma wyglądać:
jedna definicja (Thurstona)
notacja może byc od Conwaya

skleić drugi i trzeci rozdział

zwięźle własności


dobre i złe w trzecim rozdziale 

orbikompleksy może nie


This chapter and the next will be technical chapters. Later on we will evoke some terms and 
definitions without 
explicitly saying what they mean instead we will put a reference to this chapter with explicit 
saying to what definition it refers. \\
For example in the later chapters there will be phrases like "adding a defect of order $\dots$" or 
"gluing orbifolds by boundaries" and they are explained in this and the next chapter.
 

We will explore various definitions of an orbifold, partially proving they are equivalent, partially 
linking to the sources. \\
Some of these definitions apply only to the special cases. Some of them contain constructions 
with which not all orbifolds can be made (at least some of them can't be derived as such a priori)
. \\

\section{Where did orbifolds come from}


\section{Quotients of manifolds with the respect to group action}
\subsection{Hyperbolic plane tilling}

\section{Manifolds with defects}
\subsection{Disk and sphere with defects}\label{Disk_and_sphere_with_defects}



\subsection{Conway notation}
\cite{Conway2008}

When it is necessary to avoid a confusion, on parts such as $*abcd$, we will be writing 
$*^*a^*b^*c$ instead. \\
We will propose some extension to a notation from \cite{Conway2008}.
We will regard parts of that notation not only as features on an orbifold but also as an operations 
on orbifolds transforming one to another by adding particular feature. \\
We will denote the difference in Euler characteristic which is made by modifying 
an orbifold by such a feature as $\Delta(modification)$ 
which have less capitalistic vibes than "cost". 
For example $\Delta(*^*2) = \frac{1}{4}$. \\
We will denote by $*$ an operation of cutting out a disk and by $^{\beta*}n$ an operation of 
adding a kaleidoscopic point of period $n$ on the boundary component $\beta$. 
Last operation is defined only on orbifolds with boundaries.
%identified by Id. 
%When we will not concern ourselves on which component of the boundry kaleidoscopic point will 
%be added we


\section{Quotients of planes}

\section{Generalised manifolds}
This approach is very similar to the previous one. It differs slightly where we put the 
definition burden. 

%\section{Complexes?}


%../LaTeX/TeX_files/definition_characteristics_classification_and_properties_of_the_orbifolds.tex
% mainfile: ../praca_magisterska_orbifoldy.tex
\chapter{Reduction to arithmetical questions}\label{reduction_to_arithmetical}
Reductions presented in this chapter will be more in the spirit of chapter \ref{order structure}, 
in the sense that 
%We will see (in the observation \ref{boils_down}) that the problem of determining this boils
% down to the 
% analysis of all 
%the possible 
%values of the expressions:
%\begin{equation}
%2 - \sum_{i=1}^n \frac{I_i-1}{I_i}
%\end{equation}
%and 
%\begin{equation}
%1 - \sum_{j=1}^m \frac{b_j-1}{2b_j},
%\end{equation}
%where $I_i, b_j$ varies over $\mathbb{N}_{>0} \cup \{\infty\}$. \\
%As
%\begin{equation}
%2 - \sum_{i=1}^n \frac{I_i-1}{I_i} = 2 - n + \sum_{i=1}^n \frac{1}{I_i}
%\end{equation}
%and 
%\begin{equation}
%1 - \sum_{j=1}^m \frac{b_j-1}{2b_j} = 1 - m + \sum_{j=1}^m \frac{1}{2b_j},
%\end{equation}
%some questions about the spectrum are equivalent to some regarding Egyptian fractions. 
%More on this connection is discussed in \ref{Egyptian_fractions}.
%\\[4pt]
%\textbf{Disclaimer}\\
for now, until chapter \ref{counting occurrences} 
%named "Counting occurrences" 
, we will not pay attention 
to how many orbifolds have the same Euler orbicharacteristic, only whether a particular 
number is an \Eoc\ for at least one orbifold or not. 

In chapter \ref{counting occurrences} we will explain how these reductions will be relevant 
to the discussion held there. 
%Let us note, that 
%Euler orbicharacteristic does not depend on the cyclic order of points on 
%the components of the boundary. 
%Because of that and since Euler orbicharacteristic does not depend on the cyclic order 
%of points on the components of the boundary we introduce an extension of a notation from 
%\cite{Conway2008}. 

%%We will write $\ast \{a,b,c,d,\cdots\}$ to denote not a particular orbifold, but a type 
%%of orbifold that have a dihedral points on a component of the boundry of orders 
%$a,b,c,d,\cdots$, but 
%%in any order. \\ 

%We will write $\ast \{a,b,c,d,\cdots\}$ to denote a type of a boundary (of an orbifold) that have 
%kaleidoscopic points of periods $a,b,c,d,\cdots$, but in any order. \\


%From what we wrote above (that Euler orbicharacteristic does not depend on the cyclic order 
%of points on the components of the boundary), we can see that Euler 
%orbicharacteristic is well defined 
%%on such types of orbifolds. \\ 
%when we specify only such a type of the components of the boundary of an orbifold and not 
%a particular cyclic order.  \\

\section{Reductions of cases}
The aim of following reductions is to make it easier to answer the question of which 
points lie in $\spe$ and which not. 

The first aspect of the structure of $\spe$ that we would like to simplify is that it is 
the sum of $\spebr{M}$, for 
every two dimensional manifold $M$. 
\begin{equation}
\spe = \bigcup_{M\textrm{ : 2d manifold}} \spebr{M}.
\end{equation}

%In this section we want to make some reductions to 
%limit number of cases that we will be dealing with. 

We aim to find a minimal set $\mathcal{M}$ of base manifolds 
%that "covers all the cases" i.e.  
such that:
\begin{equation}\label{minimal calM}
\spe = \bigcup_{M\in\mathcal{M}}\spebr{M}.
\end{equation}

% any for any $x \in \spe$ there 
%is an orbifold $O$ with a base manifold from $B$ such that $\cho{O} = x$.
%We aim to limit the number of base manifolds as much as possible while keeping entire spectrum. 
It will turn out that $\mathcal{M} = \{S^2, D^2\}$ satisfies \ref{minimal calM} and 
that both $S^2$ and $D^2$ are necessarily. 
%there are
% it can be done such that we are left with $B = \{S^2, D^2\}$ and there is 
%no futher reductions possible. 

%For this chapter we will consider orbifolds according to a definition from 
%(\ref{Disk_and_sphere_with_defects}). \\

\section{Sufficiency of $S^2$ and $D^2$}\label{sufficiency of D2 and S2}

Given an orbifold $O_1$, we want to perform some operations from \ref{Operations} on it, 
such that the resulting orbifold 
$O_2$ will have the same Euler orbicharacteristic, but the base manifold of $O_2$ would 
be $S^2$ or $D^2$. We would then say, that $O_1$ got reduced to $O_2$.
In following subsection, we allow only such operations, that do not 
change \Eoc. When writing that we "can" do something we mean that there is 
possible one of the operations from \ref{Operations}.  %have as 
%big Euler characteristic as possible. 

The Euler characteristic of base manifold depends only on the number of handles, cross caps 
and boundary components. And, as stated in \ref{E_orb} it is: 
\begin{equation}
2-2h-c-b,
\end{equation}
%\smalltodoII{opisać i wybrać oznaczenia}
for $h$ - number of handles, $c$ - number of cross-caps, $b$ - number of boundary components. 

For every such a manifold feature we want to find an orbifold features with the same 
Euler orbicharacteristic delta. 

We will take two approaches, depending on whether the orbifold in question has a boundary or not.

%To do this we want to eliminate handles 
%Let us observe, that:
\subsection{Orbifold without boundary}
We can observe that:
\begin{align}
\hspace{3cm}\Delta(\circ) =& \hspace{-1cm} &-2& \hspace{-1cm} &= \Delta(2^4) \\
\hspace{3cm}\Delta(\times) =& \hspace{-1cm} &-1& \hspace{-1cm} &= \Delta(2^2)
\end{align}
From this we can see that we can remove handles and cross-caps from any orbifold without 
the boundary. 
After such reductions we are left with a $S^2$ orbifold with all orbipoints being rotational 
in the interior.

\subsection{Orbifold with boundary}\label{orbifold_with_boundary}
We can observe that:
\begin{align}
\hspace{3cm}\Delta(\circ) =& \hspace{-1cm} &-2& \hspace{-1cm} &= \Delta((^\ast 2)^8) \\
\hspace{3cm}\Delta(\ast) =& \hspace{-1cm} &-1& \hspace{-1cm} &= \Delta((^\ast 2)^4) \\
\hspace{3cm}\Delta(\times) =& \hspace{-1cm} &-1& \hspace{-1cm} &= \Delta((^\ast 2)^4)
\end{align}
From this we can see that we can remove handles and cross-caps 
from any orbifold with a boundary. 
We can also remove all boundary components except one.         
We can further observe that:
\begin{align}
\hspace{3cm}\Delta(n) =& \hspace{-1cm} &\frac{n-1}{n}& \hspace{-1cm} 
&= 2\frac{n-1}{2n} &= \Delta((^\ast n)^2)\label{third_reduction}
\end{align}
From this we see that we can remove all the rotational orbipoints in favor for 
dihedral orbipoints.
After such reductions we are left with a $D^2$ orbifold with all orbipoints being dihedral on 
the boundary or being reflectional on the boundary. 
%\smalltodoII{reflectional?}

As a fact not necessary for our reductions, but interesting on its own, we can furthermore, 
observe that:
\begin{observation}
If $O_1$ has not $S^2$ as its base manifold it can be reduced to a $D^2$-orbifold.
\end{observation}
\subsubsection{Proof.}
If $O_1$ has not $S^2$ as its base manifold $M$, then $M$ has at least one handle or a cross-cup. We can observe that:
\begin{align}
\hspace{3cm}\Delta(\circ) =& \hspace{-1cm} &-2& \hspace{-1cm} &= \Delta(\ast (^*2)^4) \\
\hspace{3cm}\Delta(\times) =& \hspace{-1cm} &-1& \hspace{-1cm} &= \Delta(\ast).
\end{align}
From this we have that the handle or the cross-cap can be replaced by a boundary component  
and some number of boundary orbipoints. After this reduction, we can proceed with all the 
other reductions from the \ref{orbifold_with_boundary} and obtain an $D^2$-orbifold 
with the same \Eoc as the original one. $_\square$

%\begin{align}
%\Delta(n) =& &\frac{n-1}{n}&  
%&= \Delta((^\ast n)^2)\label{third_reduction}
%\end{align}

%From this we can conclude that every Euler orbicharacteristic can be obtained 
%by an orbifold with base manifold $S^2$ or $D^2$. 
%Examples of rational numbers from $\speS \setminus \speD$ and $\speD \setminus \speS$ are:
%We will provide examples 
%
%\ref{counting_occurences}

%From this we can conclude, that every Euler orbicharacteristic can be obtained 
%by an orbifold of signature of a type ($n$ and $m$ are arbitrary):

%\begin{align*}
%I_1I_2\cdots I_n & \textrm{\ or} \\
%\ast b_1b_2\cdots b_m &.
%\end{align*}

%Let us denote the set of all possible Euler orbicharacteristics of orbifolds of the form 
%$I_1I_2\cdots I_n$ by $\speS$ and the set 
%of all possible Euler orbicharacteristics of orbifolds of the form $\ast b_1b_2\cdots b_m$ 
%as $\speD$. 
%So we have that $\spe = \speD \cup \speS$. \\
%% Let us denote the set of all possible \Eoc s of two-dimentional orbifolds as $\spe$. \\
%Let us also observe that the order type and topology of $\speS$ and $\speD$ are 
%the same since 

%\begin{equation}\label{times_two_fact}
%2\speD=\speS
%\end{equation}
%and multiplying by $2$ is the order preserving homeomorphism of $\mathbb{R}$. \\[16pt]

%Now we can make aforementioned observation:
%\todo{dopisać to czytelniej}
%w szeczgólności żeby było jasne jaką redukcję robimy

The results of our reductions, can be summarised as:
%In the terms of set relations:
\begin{observation}\label{sum of spectras}
If two-dimensional manifold $M$ has no boundary, then
%\begin{equation} 
%\spe{M} = \chi(M) - (\speS - 2) 
%\end{equation}
%and
\begin{equation} 
\spebr{M} \subseteq \speS 
\end{equation} 

If, in addition, $M \neq S^2$, then 
\begin{equation} 
\spebr{M} \subseteq \speD. 
\end{equation}

\end{observation}
%\textbf{Proof} \\
%Let $M$ be a two-dimentional manifold with no boundry, from \ref{\Eoc_as_a_sum} we have, that: 
%\begin{equation}
%\spe{M} = \{\chi(M) - \sum_{i=1}^n 
%\frac{I_i-1}{I_i}\ |\ n \in \mathbb{N} \land \forall_i I_i \in \mathbb{N} \cup \infty\}.
%\end{equation}
%And from 
%$\chi(M) - (\speS - 2)$
\begin{observation}
If two-dimensional manifold  $M$ has a boundary, then 
%\begin{equation}
%\spe{M} = \chi(M) + (\speD - 1)
%\end{equation}
%and 
\begin{equation}
\spebr{M} \subseteq \speD
\end{equation}
\end{observation}
%\textbf{Proof} \\
%\begin{equation}
%\{\chi(M) - \sum_{j=1}^m 
%\frac{b_j-1}{2b_j}\ |\ m \in \mathbb{N} \land \forall_j b_j \in \mathbb{N} \cup \infty\} 
%\end{equation}
%\begin{observation}
%If two-dimentional manifold  $M$ has no boundry, then $\spe{M} \subseteq \speS$. 
%If, in addition, $M \neq S^2$, then 
%$\spe{M} \subseteq \speD$.
%\end{observation}
%\begin{observation}
%If two-dimentional manifold  $M$ has a boundry, then $\spe{M} \subseteq \speD$.
%\end{observation}

%In the terms of arithmetical expressions:
%\todo{sums are the form od sped and spes}

\begin{corollary}
We have that $\spe = \speS \cup \speD$.
\end{corollary}

\begin{observation}\label{only dihedral} 
If a two-dimensional manifold $M$ has a boundary, then:
\begin{equation}
\spebr{M} = \spe^{d}(M).
\end{equation}
\end{observation}

We will postpone our discussion of necessity of both $S^2$ and $D^2$ to 
\ref{neccessity of d2 and s2}, after 
the 
section \ref{Reduction to arithmetical questions section} which will provide us 
with more convenient language. 

\section{Reduction to arithmetical questions}\label{Reduction to arithmetical questions section}
As written in \ref{spectra}, we can express an \Eoc of a $M$-orbifold $O$ as:
\begin{equation}
\cho{O} = \chi (M) - \sum_{i=1}^n \frac{r_i-1}{r_i} - \sum_{j=1}^m \frac{d_j-1}{2d_j},
\end{equation}
where $r_i$ and $d_j$ are degrees of the, respectively, rotational and dihedral orbipoints 
of $O$.

From this we can express $\spebr{M}$ as:
\begin{align}
\spe(M) = \chi (M) &-\left\{\sum\limits_{i=1}^n \frac{r_i-1}{r_i}\ \big|\ n\in\mathbb{N}_0,\ 
r_i\in\mathbb{N}_{>0}\cup \{\infty\}\right\}+ \\
&- \left\{\sum\limits_{j=1}^m\frac{d_j-1}{2d_j}\ 
\big|\ m\in\mathbb{N}_0,\ d_j\in\mathbb{N}_{>0}\cup \{\infty\}\right\} .
\end{align}

As from \ref{sufficiency of D2 and S2} we know that $\spe = \speS \cup \speD$, and that 
$\chi (S^2) = 2$ and $\chi (D^2) = 1$, we can express $\spe$ as a sum ($\cup$) of two sets:
\begin{equation}
2 -\left\{\sum\limits_{i=1}^n \frac{r_i-1}{r_i}\ \big|\ n\in\mathbb{N}_0,\ 
r_i\in\mathbb{N}_{>0}\cup \{\infty\}\right\} = \spe(S^2)
\end{equation}
and
\begin{equation}
1 - \left\{\sum\limits_{j=1}^m\frac{d_j-1}{2d_j}\ 
\big|\ m\in\mathbb{N}_0,\ d_j\in\mathbb{N}_{>0}\cup \{\infty\}\right\} = \spe(D^2) .
\end{equation}

From this we see, that the core of understanding $\spe$ through arithmetical viewpoint 
is to understand possible values of expression:
%We now know, that $\spe = \speS \cup \speD$. To determin 

%\begin{observation}\label{boils_down}
%From above reductions we can conclued that our problem boiles down to the analysis of all
% the possible 
%values of the expressions:
\begin{equation}\label{S2_sum}
2 - \sum_{i=1}^n \frac{r_i-1}{r_i}
\end{equation}
and 
\begin{equation}\label{D2_sum}
1 - \sum_{j=1}^m \frac{d_j-1}{2d_j},
\end{equation}
with $r_i$ and $d_j$ ranging over $\mathbb{N}_{>0}\cup \{\infty\}$.
%, with a convention 
%that $\varphi(\infty) \coloneqq \lim\limits_{k\to \infty} \varphi(k)$. 
%\end{observation}

As $\Delta(\infty) = 1 = \Delta(2^2)$ and $\Delta(^*\infty) = \frac{1}{2} = \Delta((^*2)^2)$, 
we could perform further reductions to have an orbifold with 
particular orbicharacteristic without cusps (if needed) and then (after these reductions) 
we can analyse only expressions with $r_i$ and $d_j$ ranging over $\mathbb{N}_{>0}$ and 
they will still give us full spectrum. 
However, as stated later, it will be more convenient to us to include orbifolds with cusps 
so we are stating this observation only as a side remark.
%for readers information. 
%The fact that this agrees with the definition of the \Eoc\ on the geometrical terms was 
%addressed in \ref{extended_Euler_orbicharacteristic}. 

\section{Hurwitz theorem}\label{największy orbifold}
%\subsection{Hurwitz theorem}
One of the well known facts about two-dimensional orbifolds comes from \cite{Hurwitz1893} 
(seite 424).
Although formulated in different language, it states, that the orbifolds 
with hyperbolic structure can have an \Eoc\ at most $-\frac{1}{84}$. 
Here we will present the proof of this result with the language used in this thesis.
%We will now prove this result 
%in a straighforward way. 
%\todo{dopisać}
\begin{theorem}
If a two dimensional orbifold admits hyperbolic structure, then the maximal \Eoc\ it can 
have is $-\frac{1}{84}$ and the only orbifold that realises this \Eoc\ is $*2\ 3\ 7$.
\end{theorem}
\subsubsection{Proof.}
From \ref{sufficiency of D2 and S2} we know, that to check whether $-\frac{1}{84}$ is maximal 
negative \Eoc\ a two dimensional orbifold can have, we only need to check possible 
\Eoc s of $S^2$ orbifolds and $D^2$ orbifolds. 

Firstly, we will show, that $*2\ 3\ 7$ has biggest negative \Eoc\ from all $D^2$ orbifolds 
and its the only one with this \Eoc\ from $D^2$ orbifolds.

We have that 
\begin{equation}
\cho{*2\ 3\ 7} = 1 - \frac{1}{4} - \frac{2}{6} - \frac{6}{14} = -\frac{1}{84}. 
\end{equation}
From \ref{spectra} we know, that $\speD = \spe^d(D^2)$, so we can consider only dihedral 
orbipoints. 
Let us observe, that since 
$\max\{\Delta(^*n)\ |\ n \in \mathbb{N}_{>0} \cup \{\infty\}\} = \frac{1}{2}$, we have, that 
to have \Eoc\ $<0$, 
$D^2$ orbifold has to have at least $3$ orbipoints.
%Let us observe
 
Let us observe, that for any $D^2$ orbifold that have $5$ orbipoints or more, it has an \Eoc 
equal at most $1 - 5\frac{1}{4} = -\frac{1}{4} < -\frac{1}{84}$. So we can restrict our search 
only to orbifolds with at most $4$ orbipoints. 

Let us observe, that $*2\ 2\ 2\ 3$ and $* 3\ 3\ 4$ are $D^2$ orbifolds with the biggest 
negative \Eoc\ among orbifolds, respectively, with four orbipoints, and, without any 
point of degree $2$. The proof of this observation is, that any other orbifold of these 
respective kinds, would need to have all degrees of orbipoints (when ordered in increasing manner) 
pairwise $\geq$ than $*2\ 2\ 2\ 3$ or $* 3\ 3\ 4$.
%Let us observe, that id $D^2$ orbifold have at least one rotational orbipoint, then 
%the smallest possible \Eoc\ is 
%\begin{equation}
%\cho{2 * 2 3} = 1 - \ 
%\end{equation}

Let us observe, that 
%the orbifold wsith the greatest negative \Eoc, that 
%have no orbipoints of degree $2$, that is, $* 3\ 3\ 4$, have \Eoc\ 
\begin{equation}
\cho{*2\ 2\ 2\ 3} = 1 - \frac{1}{4} - \frac{1}{4} - \frac{1}{4} - \frac{2}{6} = -\frac{1}{12} 
< -\frac{1}{84}
\end{equation}
and
\begin{equation}
\cho{*3\ 3\ 4} = 1 - \frac{2}{6} - \frac{2}{6} - \frac{3}{8} = -\frac{1}{24} < -\frac{1}{84},
\end{equation}
From this, we can conclude, that we can restrict our search only to orbifolds with 
exactly $3$ orbipoints, where at least one of them is equal to $2$. 

Let us observe, that for such orbifold to have negative \Eoc, it needs to have at two orbipoints 
of order at least $3$, otherwise it has \Eoc\ at most 
\begin{equation}
\cho{*2\ 2\ \infty} = 1- \frac{1}{4} - \frac{1}{4} - \frac{1}{2} = 0. 
\end{equation}

Further, let us observe, that $*2\ 4\ 5$ has the greatest negative \Eoc\ among orbifolds 
that have no orbipoint of degree $3$. The proof of this observation is similar to the proof 
of the previous one -- any other orbifold of this kind, 
would need to have all degrees of orbipoints (when ordered in increasing manner) 
pairwise $\geq$ than $*2\ 4\ 5$.

Let us observe, that
%We have that 
\begin{equation}
\cho{*2\ 4\ 5} = 1 - \frac{1}{4} - \frac{3}{8} - \frac{4}{10} = -\frac{1}{40} < - \frac{1}{84}. 
\end{equation}
From this, we conclude, that we can restrict our search to the orbifolds of 
the form $*2\ 3\ n$. 
We can also observe, that all orbifolds of the form $*2\ 3\ n$ have unique \Eoc\ 
among this group.
%and it has to be unique. 
The one with the biggest \Eoc\ among them is $*2\ 3\ 7$. 

Let us observe that no $D^2$ orbifold with rotational orbipoints can't have such \Eoc. 
For the sake of contradiction, let us assume that there is some orbifold 
$r_1 \cdots r_n * d_1 \cdots d_m$, with $n\neq 0$, with \Eoc\ equal to $-\frac{1}{84}$. 
However, then also $*r_1r_1 \cdots r_nr_nd_1\cdots d_m$, 
with only dihedral orbipoints, 
would have \Eoc\ equal to $-\frac{1}{84}$. However, $*2\ 3\ 7$ is unique one with this \Eoc\ 
and it have no repeated degree, so it can't be expressed in the form 
$*r_1r_1 \cdots r_nr_nd_1\cdots d_m$ 
with $n\neq 0$. 

Let us also observe, that the same argument shows that $\mathbb{R}P^1$ orbifolds 
can't have \Eoc\ equal to $-\frac{1}{84}$, since $\chi{\mathbb{R}P^1} = \chi{D^2}$ 
and $\mathbb{R}P^1$ has no boundary, so $\mathbb{R}P^1$ orbifold can have only rotational 
orbipoints. 
 
Now, we will prove, that no $S^2$ orbifold has \Eoc\ $-\frac{1}{84}$. 
%From this, 
Since we have \ref{2times homeomorphism} 
%it will 
%it folows, that no $S^2$ orbifold 
%has greater negative \Eoc. 
%The argument goes as folows:
We can perform following argument:  
 
For the sake of contradiction let us assume, that 
$-\frac{1}{84} \in \speS$, then, from \ref{2times homeomorphism} we know, that 
$\frac{1}{2}\left(-\frac{1}{84}\right)\in \speD$. This is a contradiction as 
$0 > \frac{1}{2}\left(-\frac{1}{84}\right) > -\frac{1}{84}$. 
As such, we ruled out all manifolds with Euler characteristic $>0$. 

For 
%What is left to prove, is that $-\frac{1}{84}$ is not an \Eoc\ for an $M$ orbifold 
%for any 
two dimensional manifolds 
%$M$ 
%We can rule out all manifolds 
with Euler characteristic $\leq 0$, we have that orbifolds having them as a base manifolds 
have \Eoc at most $-\frac{1}{4} < -\frac{1}{84}$. 
%From the arguments above, we can also conclude, that we can also rule out 
%$\mathbb{R}\mathrm{P}^1$ -- the only 
%other manifold left under our consideration, as orbifolds corresponding to it can only have 
%rotational orbipoints. 
$_\square$ 

\section{Necessarily of $S^2$ and $D^2$}\label{neccessity of d2 and s2}
As we know from \ref{Operations} adding an orbipoint to a manifold decreases it's 
orbicharacteristic. As $S^2$ has the highest Euler characteristic: $2$ of all 
two dimensional manifolds, there is no other orbifold with \Eoc\ equal to $2$. 
$S^2$ is then necessary to include $2$. 

As known from \ref{największy orbifold}, the number $-\frac{1}{84}\in\speD$, it is 
the greatest negative 
\Eoc any two dimensional orbifold can have and $-\frac{1}{84}\not\in\speS$. 
%We will now show, that 
%$-\frac{1}{84} \not\in \speS$. For the sake of contradiction let us assume, that 
%$-\frac{1}{84} \in \speS$, then, from \ref{2times homeomorphism} we know, that 
%$\frac{1}{2}\left(-\frac{1}{84}\right)\in \speD$. This is a contradiction as 
%$0 > \frac{1}{2}\left(-\frac{1}{84}\right) > -\frac{1}{84}$. 
$_\square$  

Further examination of connections between $\speD$ and $\speS$ is performed in \ref{D_and_S}.







% mainfile: ../praca_magisterska_orbifoldy.tex
%\synctex=1
\chapter{Order type and topology}\label{order structure}
%
%Order type with zanurzenie w R
%
%1537/137
%

In this chapter we will discuss that both the order type and the topology 
of the set $\spe$ of all possible Euler orbicharacteristics 
of two-dimensional orbifolds are that of $\omega^\omega$. 
%We will call this set $\spe$.
%We will see (in the observation \ref{boils_down}) that the problem of determining this boils
% down to the 
% analysis of all 
%the possible 
%values of the expressions:
%\begin{equation}
%2 - \sum_{i=1}^n \frac{I_i-1}{I_i}
%\end{equation}
%and 
%\begin{equation}
%1 - \sum_{j=1}^m \frac{b_j-1}{2b_j},
%\end{equation}
%where $I_i, b_j$ varies over $\mathbb{N}_{>0} \cup \{\infty\}$. \\
%As
%\begin{equation}
%2 - \sum_{i=1}^n \frac{I_i-1}{I_i} = 2 - n + \sum_{i=1}^n \frac{1}{I_i}
%\end{equation}
%and 
%\begin{equation}
%1 - \sum_{j=1}^m \frac{b_j-1}{2b_j} = 1 - m + \sum_{j=1}^m \frac{1}{2b_j},
%\end{equation}
%some questions about the spectrum are equivalent to some regarding Egyptian fractions. 
%More on this connection is discussed in \ref{Egyptian_fractions}.
\\[4pt]
%\textbf{Disclaimer}\\
For now, until chapter \ref{counting occurrences} named "Counting occurrences", 
we will not pay attention 
to how many orbifolds have the same Euler orbicharacteristic, only whether a particular 
number is an \Eoc\ for at least one orbifold or not. 
%Let us note, that 
%Euler orbicharacteristic does not depend on the cyclic order of points on 
%the components of the boundary. 
%Because of that and since Euler orbicharacteristic does not depend on the cyclic order 
%of points on the components of the boundary we introduce an extension of a notation from 
%\cite{Conway2008}. 

%%We will write $\ast \{a,b,c,d,\cdots\}$ to denote not a particular orbifold, but a type 
%%of orbifold that have a dihedral points on a component of the boundry of orders 
%$a,b,c,d,\cdots$, but 
%%in any order. \\ 

%We will write $\ast \{a,b,c,d,\cdots\}$ to denote a type of a boundary (of an orbifold) that have 
%kaleidoscopic points of periods $a,b,c,d,\cdots$, but in any order. \\


%From what we wrote above (that Euler orbicharacteristic does not depend on the cyclic order 
%of points on the components of the boundary), we can see that Euler 
%orbicharacteristic is well defined 
%%on such types of orbifolds. \\ 
%when we specify only such a type of the components of the boundary of an orbifold and not 
%a particular cyclic order.  \\

\section{Reductions of cases}
In this section we want to make some reductions to 
limit number of cases that we will be dealing with. 

We aim to find a minimal set $B$ of base manifolds that "covers all the cases" i.e. $B$ 
such that any for any $x \in \spe$ there 
is an orbifold $O$ with a base manifold from $B$ such that $\cho{O} = x$.
%We aim to limit the number of base manifolds as much as possible while keeping entire spectrum. 
It will turn out that $B = \{S^2, D^2\}$ and there are
% it can be done such that we are left with $B = \{S^2, D^2\}$ and there is 
no futher reductions possible. 

%For this chapter we will consider orbifolds according to a definition from 
%(\ref{Disk_and_sphere_with_defects}). \\
Given an orbifold $O_1$, we want to perform some operations from \ref{Operations} on it, such that the resulting orbifold 
$O_2$ will have the same Euler orbicharactristic, but the base manifold of $O_2$ would have as 
big Euler characteristic as possible. 

The Euler orbicharacteristic of base manifold depends only on the number of handles, cross caps 
and boundry components. And, as stated in \ref{E_orb} it is: 
\begin{equation}
2-2h-c-b
\end{equation}
\smalltodoII{opisać i wybrać oznaczenia}

For ever such a manifold feature we want to find an orbifold features with the same 
Euler orbicharacteristic delta.  

%To do this we want to eliminate handles 
%Let us observe, that:
One of the ways to do that is by observing that:
\begin{align}
\hspace{3cm}\Delta(\circ) =& \hspace{-1cm} &-2& \hspace{-1cm} &= \Delta(\ast (^\ast 2)^4) \\
\hspace{3cm}\Delta(\ast) =& \hspace{-1cm} &-1& \hspace{-1cm} &= \Delta((^\ast 2)^4) \\
\hspace{3cm}\Delta(\times) =& \hspace{-1cm} &-1& \hspace{-1cm} &= \Delta((^\ast 2)^4)
\end{align}
So we see that from any orbifold we can eradicate handles ....       
\begin{align}
\hspace{3cm}\Delta(n) =& \hspace{-1cm} &\frac{n-1}{n}& \hspace{-1cm} 
&= \Delta((^\ast n)^2)\label{third_reduction}
\end{align}

%\begin{align}
%\Delta(n) =& &\frac{n-1}{n}&  
%&= \Delta((^\ast n)^2)\label{third_reduction}
%\end{align}

From this we can conclude that every Euler orbicharacteristic can be obtained 
by an orbifold with base manifold $S^2$ or $D^2$. 
Examples of rational numbers from $\speS \setminus \speD$ and $\speD \setminus \speS$ are:
We will provide examples 
Futher examination of connections between $\speD$ and $\speS$ is performed in \ref{D_and_S}.
%\ref{counting_occurences}

%From this we can conclude, that every Euler orbicharacteristic can be obtained 
%by an orbifold of signature of a type ($n$ and $m$ are arbitrary):

%\begin{align*}
%I_1I_2\cdots I_n & \textrm{\ or} \\
%\ast b_1b_2\cdots b_m &.
%\end{align*}

%Let us denote the set of all possible Euler orbicharacteristics of orbifolds of the form 
%$I_1I_2\cdots I_n$ by $\speS$ and the set 
%of all possible Euler orbicharacteristics of orbifolds of the form $\ast b_1b_2\cdots b_m$ 
%as $\speD$. 
%So we have that $\spe = \speD \cup \speS$. \\
%% Let us denote the set of all possible \Eoc s of two-dimentional orbifolds as $\spe$. \\
%Let us also observe that the order type and topology of $\speS$ and $\speD$ are 
%the same since 

%\begin{equation}\label{times_two_fact}
%2\speD=\speS
%\end{equation}
%and multiplying by $2$ is the order preserving homeomorphism of $\mathbb{R}$. \\[16pt]

%Now we can make aforementioned observation:
\todo{dopisać to czytelniej}
w szeczgólności żeby było jasne jaką redukcję robimy
The result of our reductions, can be expressed as:

In the terms of set relations:
\begin{observation}\label{sum of spectras}
If two-dimentional manifold $M$ has no boundry, then
%\begin{equation} 
%\spe{M} = \chi(M) - (\speS - 2) 
%\end{equation}
%and
\begin{equation} 
\spebr{M} \subseteq \speS 
\end{equation} 

If, in addition, $M \neq S^2$, then 
\begin{equation} 
\spebr{M} \subseteq \speD. 
\end{equation}

\end{observation}
%\textbf{Proof} \\
%Let $M$ be a two-dimentional manifold with no boundry, from \ref{\Eoc_as_a_sum} we have, that: 
%\begin{equation}
%\spe{M} = \{\chi(M) - \sum_{i=1}^n 
%\frac{I_i-1}{I_i}\ |\ n \in \mathbb{N} \land \forall_i I_i \in \mathbb{N} \cup \infty\}.
%\end{equation}
%And from 
%$\chi(M) - (\speS - 2)$
\begin{observation}
If two-dimentional manifold  $M$ has a boundry, then 
%\begin{equation}
%\spe{M} = \chi(M) + (\speD - 1)
%\end{equation}
%and 
\begin{equation}
\spebr{M} \subseteq \speD
\end{equation}
\end{observation}
%\textbf{Proof} \\
%\begin{equation}
%\{\chi(M) - \sum_{j=1}^m 
%\frac{b_j-1}{2b_j}\ |\ m \in \mathbb{N} \land \forall_j b_j \in \mathbb{N} \cup \infty\} 
%\end{equation}
%\begin{observation}
%If two-dimentional manifold  $M$ has no boundry, then $\spe{M} \subseteq \speS$. 
%If, in addition, $M \neq S^2$, then 
%$\spe{M} \subseteq \speD$.
%\end{observation}
%\begin{observation}
%If two-dimentional manifold  $M$ has a boundry, then $\spe{M} \subseteq \speD$.
%\end{observation}

In the terms of arithmetical expressions:
\todo{sums are the form od sped and spes}
\begin{observation}\label{boils_down}
From above reductions we can conclued that our problem boiles down to the analysis of all
 the possible 
values of the expressions:
\begin{equation}\label{S2_sum}
2 - \sum_{i=1}^n \frac{I_i-1}{I_i}
\end{equation}
and 
\begin{equation}
1 - \sum_{j=1}^m \frac{b_j-1}{2b_j},
\end{equation}
with $I_i$ and $b_j$ ranging over $\mathbb{N}_{>0}\cup \{\infty\}$.
%, with a convention 
%that $\varphi(\infty) \coloneqq \lim\limits_{k\to \infty} \varphi(k)$. 
\end{observation}
As stated in \ref{cusp_reduction} we can perform futher reductions to have an orbifold with 
particular orbicharacteristic without cusps (if needed) and then (after these reductions) 
we can analyse only expressions with $I_i$ and $b_j$ ranging over $\mathbb{N}_{>0}$ and 
they will still give us full spectrum. 

However, as stated later, it will be more convenient to us to include orbifolds with cusps 
so we are stating above remark only for readers information. 

The fact that it agrees with the definition of the \Eoc\ on the geometrical terms was 
addressed in \ref{extended_Euler_orbicharacteristic}. 



To determine order type and topology of $\spe$ we will first study how $\speD$ looks like. 
Then, remembering that 
$\spe = \speS \cup \speD$ and $\speS = 2\speD$ we will make an argument for $\spe$. 

%We also have shown that all possible \Eoc s are achieved without using cusps. As such, 
%we will use 
%cusps, remembering, that we can always get rid of them, if needed. So above $I_i$ and $b_j$ 
%are ranging over $\mathbb{N}_{>0}\cup \{\infty\}$, where expressions for infinity are defined as 
%a limits. The fact that it agrees with the definition of the \Eoc\ on the geometrical terms was 
%addressed in \ref{extended_Euler_orbicharacteristic}. \\
%Here is the notation that will be used: \\
%$\spe$ - spectrum of all possible \Eoc\ of two-dimentional orbifolds \\
%$\spe{M}$ - spectrum of all possible \Eoc\ of $M$ orbifolds \\

%% We would like to make some more observations concerning subsets of $\spe$. 
%% For any two-dimentional manifold $M$, observe, that:

%Now we state what we can already conclude. 


%\begin{theorem}\label{all_spectra_are_isomorphic}
%For any two two-dimentional manifolds $M$, $N$ spectras $\spe{M}$ and $\speM(N)$ have the same 
%order type and topology. 
%\end{theorem}
%\textbf{Proof} \\



\section{Order type and topology of $\speD$}


In this section we will also describe precisely where accumulation points of $\speD$ lie and of 
 which order 
(see below \ref{accumulation_points_definitions}) they are. Analysis of locations of those 
accumulation points, as interesting as it is alone will also be necessery for providing 
our argument about order type and topology of $\speD$. 
\subsection{Definition and properties of order of accumulation points}
\label{accumulation_points_definitions} 
\todo{zmienić to wszystko na rangę cantora bendiksona}
%We start with one technical definition of "transitive order" that will be almost what we want
%and then, there will be the the definition of "order", which is the definition that we need.
%\begin{definition}
%(Inductive). 
%%We say, that the point $a$  from the topological space $X$ is an acccumulation 
%%point of the transitive order 0, when
%We say that the point is an acccumulation point of a transitive order $0$, when it is 
%an isolated point. 
%We say that the point is an acccumulation point of a transitive order $n + 1$, when it is 
%an acccumulation point (in the usual sense) of the accumulation points of the transitive 
%order $n$. 
%\end{definition}  
%The only issue of the above definition is that the point of the transitive order $n$ 
%is also a point 
%of the transitive order $k$, for all $0< k \leq n$. We want a definition of order such that 
%for any point, there is at most one integer that is its order. So we define:
%\begin{definition}
%We say that the point is an acccumulation point of order $n$ iff it is an acccumulation point 
%of the transitive order $n$ and it is not an acccumulation point of the transitive order $n+1$. 
%If the point is an acccumulation point of the transitive order for an arbitrary large 
%$n$ we say that 
%the  point is an acccumulation point of order $\omega$.
%\end{definition}
We start with definition of being "at least of order $n$" that will be almost what we want
and then, there will be the definition of being "order", which is the definition that we need. \\
For a given set we define as follows:
\begin{definition}
(Inductive). 
%We say, that the point $a$  from the topological space $X$ is an acccumulation 
%point of the transitive order 0, when
We say that the point $x$ is an acccumulation point of a set $X$ 
of order at least $0$, when it belongs to the set $X$. 
We say that the point $x$ is an acccumulation point of a set 
of order at least $n + 1$, when it is 
an acccumulation point (in the usual sense) of the accumulation points each of order at least 
$n$ i.e. in every neighbourhood of $x$ there is at least one accumulation point of a set $X$ 
of order at least $n$, distincs from $x$. 
\end{definition}  
%The only issue of the above definition is that the point of the transitive order $n$ 
%is also a point 
%of the transitive order $k$, for all $0< k \leq n$. We want a definition of order such that 
%for any point, there is at most one integer that is its order. So we define:
\begin{definition}
We say that the point is an acccumulation point of order $n$ iff it is an acccumulation point 
of order at least $n$ and it is not an acccumulation point of order at least $n+1$. 
If the point is an acccumulation point of order at least $n$ for an arbitrary large 
$n$ we say that 
the point is an acccumulation point of order $\omega$.
\end{definition}
When we will say that a point is an accumulation point of some set without specifying an order 
then we will mean being an accumulation point in the ussual sense; from the point of view 
of above definitions, that is, an accumulation point of order at least one.
%\todo{dopisać notację do punktów skupienia różnego stopnia}
\begin{lemma}

\end{lemma}
\subsection{Analysis of locations of accumulation points of $\speD$ with respect to their order}
%\subsubsection{Some preliminary observations}
We want to determine where exactly are accumulation points of the set $\speD$ with 
respect to their orders. 

For this we will use  
a handful of observations and lemmas. 
\begin{observation}\label{accumulation_points_are_in_the_spectrum}
Let us observe, that $\lim\limits_{n \to \infty} \Delta(^\ast n) = -\frac{1}{2}$. From that, 
we see, 
that for every point $x \in \speD$, the point $x - \frac{1}{2}$ is an acccumulation point. 
Let us observe, that also, for every point $x \in \speD$, we have that $x - \frac{1}{2} 
\in \speD$, 
because $\Delta(^\ast \infty) = -\frac{1}{2}$. 
\end{observation}
%Now we will show that the order type of $\speD$ is $\omega^\omega$ and where exactly are 
%its accumulation points of which orders. For this we will use  
%a handful of lemmas. 

%\subsubsection{Finiteness lemma}
\begin{lemma}\label{finiteness_lemma}
For all $n \in \mathbb{N}_{\geq 2}$ and $x \in (-\infty, 1]$ there are only finitely 
many Euler orbicharacteristics
in the interval $[x,1] \cap \speD$ of orbifolds that have points of order equal 
at most $n$. 
\end{lemma}
\textbf{Proof.} 

Let $x \in (-\infty, 1]$. There can be at most $\lfloor 4(1-x) \rfloor$ orbipoints on the 
$D^2$ orbifold 
with an \Eoc\ $y \in [x,1]$ since each orbipoint decreases an \Eoc\ by at least $\frac{1}{4}$ 
and the Euler characteristic of $D^2$ is $1$. 

There are only $(n-1)^{\lfloor 4(1-x) \rfloor}$ possible sets of $\lfloor 4(1-x) \rfloor$ 
orbipoints' orders that are less or equal than $n$. Hence, there are only at most 
$(n-1)^{\lfloor 4(1-x) \rfloor}$ possible \Eoc s.


\begin{lemma}\label{first_order_lemma}
If $x$ is an acccumulation point of the set $\speD$ of order $n$, then $x-\frac{1}{2}$ is a
 accumulation point of the set $\speD$ of order at least $n+1$. 
\end{lemma}
\noindent\textbf{Proof.} \\
Inductive. 

$\bullet$ $n = 0$: If $x$ is an isolated point of the set $\speD$, then $x \in \speD$. 
From that, we 
have, that points $x - \frac{k-1}{2k}$ are in $\speD$ for all $k \geq 1$, from that, that 
$x-\frac{1}{2}$ is a 
accumulation point of $\speD$. 

$\bullet$ inductive step: Let $x$ be an acccumulation point of the set $\speD$ of an order 
$n > 0$. 
Let $a_k$ be a sequence of accumulation points of order $n-1$ convergent to $x$. From the 
inductive assumption, we have, that $a_k - \frac{1}{2}$ is a sequence of accumulation points 
of order at least $n$. From the basic sequence arithmetic it is convergent to $x-\frac{1}{2}$. 
From that, we have that $x-\frac{1}{2}$ is an acccumulation point of the set $\speD$ of order 
at least $n+1$. $_\square$
\begin{lemma}\label{second_order_lemma}
If $x$ is an acccumulation point of the set $\speD$ of order $n$, then $x+\frac{1}{2}$ is 
an acccumulation point of the set $\speD$ of order at least $n-1$.  
\end{lemma}
\noindent\textbf{Proof.} \\  
Inductive 

$\bullet$ $n = 1$: We assume, that $x$ is an acccumulation point of isolated points of the set 
$\speD$. 
%Let us observe
From \ref{finiteness_lemma} we know, that for all $m$ there are only finitely many 
Euler orbicharacteristics 
in the interval $[x,1]$ of orbifolds that have dihedral points of order equal at most $m$. \\ 
%\todo{może to dać jako osobny lemat}
From that, for arbitrary small neighborhood $U \ni x$ and arbitrary large $m$ there exist 
an orbifold 
that has a dihedral point of period grater than $m$, whose Euler orbicharacteristic lies in $U$. 
Let us take a sequence of such \Eoc s $a_k$ convergent to $x$, such that we can choose 
a sequence divergent to infinity of periods of dihedral points $b_k$ of orbifolds of \Eoc s 
equal $a_k$. 
\smalltodoII{picture} 
Let us observe, that for all $k$, the number $a_k+\frac{b_k-1}{2b_k}$ is in $\speD$. 
It is so, because $a_k$ is an \Eoc\ of an orbifold that have a dihedral point of period $b_k$, so 
identical orbifold, only without this dihedral point, has an \Eoc\ equal to $a_k + 
\frac{b_k-1}{2b_k}$. 
The sequence $a_k + \frac{b_k-1}{2b_k}$ converge to $x+\frac{1}{2}$. From that we have, that 
$x + \frac{1}{2}$ is an acccumulation point of the set $\speD$ of order at least $0$. \\
$\bullet$ inductive step: Let $x$ be an acccumulation point of the set $\speD$ of order $n > 1$. 
Let $a_k$ be a sequence of accumulation points of the set $\speD$ of order $n-1$ 
convergent to $x$. 
From the inductive assumption the sequence $a_k + \frac{1}{2}$ is a sequence of an acccumulation
 points of the set $\speD$ of order $n-2$ convergent to $x + \frac{1}{2}$. From that 
 $x + \frac{1}{2}$ is an acccumulation point of the set $\speD$ of order at least 
 $n-1$. $_\square$ 
\begin{lemma}\label{third_order_lemma}
If $x$ is an acccumulation point of the set $\speD$ of order $n+1$, then \\
$x - \frac{1}{2}$ is an acccumulation point of the set $\speD$ of order $n+2$ and \\
$x + \frac{1}{2}$ is an acccumulation point of the set $\speD$ of order $n$. 
\end{lemma}
\noindent\textbf{Proof.} 

Let $x$ be an acccumulation point of the set $\speD$ of order $n+1$. From the lemma 
 \ref{first_order_lemma} we know, that $x - \frac{1}{2}$ is an acccumulation point of the set 
 $\speD$ of order at least $n+2$. Now let us assume (for a contradiction), that $x - \frac{1}{2}$ 
 is an \apots $\speD$ of order $k>n+2$. But then from the lemma \ref{second_order_lemma} 
 we have that $x$ is an acccumulation point of the set $\speD$ of order at least $n+2$ and that 
 is a contradiction. 
 
Analogously, from the lemma \ref{second_order_lemma} we know, that $x + \frac{1}{2}$ is a 
accumulation point of the set $\speD$ of order at least $n$. Let us assume (for a contradiction), 
that $x+ \frac{1}{2}$ is an acccumulation point of the set $\speD$ of order $k>n$. But then 
from the lemma \ref{first_order_lemma} we have that $x$ is an acccumulation point of 
the set $\speD$ 
of order at least $n+2$ and that is a contradiction. $_\square$ 
\begin{lemma}\label{accumulation_points_of_the_set}
For all $n \in \mathbb{N}$ all accumulation points of the set $\speD$ of order $n$ are in $\speD$.
\end{lemma}
\noindent\textbf{Proof.} \\
Inductive 

$\bullet$ $n=0$: Clear, as they are isolated points of $\speD$. 

$\bullet$ inductive step: Let $x$ be a \apots\  $\speD$ of order $n>0$. From the lemma 
\ref{third_order_lemma} point $x+\frac{1}{2}$ is an acccumulation point of the set $\speD$  
of order $n-1$. From the inductive assumption $x+\frac{1}{2} \in \speD$. Then, 
from \ref{accumulation_points_are_in_the_spectrum}, we have that $x \in \speD$. 
$_\square$ 

\begin{theorem}\label{greatest \apots}
The greatest \apots\ $\speD$ of order $n$ is $1-\frac{n}{2}$.
\end{theorem}
\noindent\textbf{Proof.}\\
Inductive 

$\bullet$ $n=0$: We know, that $1\in \speD$ and $1$ is the greatest element of $\speD$. 

$\bullet$ an inductive step: From the inductive assumption we know that $1-\frac{n}{2}$ is 
the greatest \apots\ $\speD$ of order $n$. From the lemma \ref{third_order_lemma} we have then 
that $1-\frac{n+1}{2}$ is a \apots\ $\speD$ of order $n+1$. Let us assume (for a contradiction), 
that there exist a bigger accumulation point of order $n+1$ equal to $y > 1-\frac{n+1}{2}$. 
But then, from lemma \ref{third_order_lemma}, point $y+\frac{1}{2}$ would be an acccumulation 
point 
of order $n$, what gives a contradiction, because $y+\frac{1}{2}>1-\frac{n}{2}$. $_\square$ 
\\[8pt]
From the above discussion we can also formulate following corollary that will be useful later: 
% corollary (stated in for deifferent ways 
%as one are sometimes more useful that another): 
%\begin{corollary}\label{predescors}
%Let $x \in \spe$. Then there exists $n \in \mathbb{N}$ such that $x + \frac{n}{2} \in \spe$ 
%but $x+\frac{n+1}{2} \not\in \spe$. For such $n$ we have that $x$ is an \apots\ $\spe$ of 
%order $n$.  
%\end{corollary}
%\begin{corollary}
%\end{corollary}
%%Above corollary can be refolmulated in a way that sometimes is more useful:
%\begin{corollary}\label{predescors_variant_II}
%Let $x \in \spe$ be an \apots\ $\spe$ of order $n$. Then there is $y \in \spe$ that is 
%an isolated point of $\spe$ such that $x = y - \frac{n}{2}$.   
%\end{corollary}
\begin{corollary}\label{predescors}
Let $x \in \speD$. Then:
\begin{itemize}
\item there exists $n_1 \in \mathbb{N}$ such that $x + \frac{n_1}{2} \in \speD$ 
but $x+\frac{n_1+1}{2} \not\in \speD$. 

In other words, there exist $y \in \speD$ and 
$n_1 \in \mathbb{N}$ such that 
$y + \frac{1}{2} \not\in \speD$ and such that $x = y - \frac{n_1}{2}$;
\item there exists $n_2 \in \mathbb{N}$ such that $x$ is an \apots\ $\speD$ of 
order $n_2$
%\item $x$ is an \apots\ $\speD$ of order $n_2$, for some $n_2 \in \mathbb{N}$;  
\end{itemize}
and $n_1 = n_2$.
\end{corollary}

\subsection{Proof that $\speD$ is well ordered}

\begin{definition} 
Let $B_0 = \{1\}$.
For an $n \in \mathbb{N}_{>0}$, let $B_n$ be the set of all possible \Eoc\ realised 
by orbifolds of type 
$*^*b_1, \cdots, ^*b_n$. For a given $n$ these are 
$D^2$ orbifolds with precisely $n$ non trivial orbipoits on their boundry.
\end{definition}


\begin{observation}\label{recursive_relation}
There is a recursive relation, that $B_{n+1}=B_n+\{-\frac{n-1}{2n}\ |\ n\geq 2\}$
\end{observation}
\textbf{Proof.} \\
It is so, because every orbifold with $n+1$ orbipoints can be obtained by adding one point 
to an orbifold with $n$ orbipoints and the set 
$\{-\frac{n-1}{2n}\ |\ n\geq 2\} = \{\Delta(^\ast b)\ |\ b \geq 2\}$. $_\square$
%Let take $x \in B_{n+1}$. Then there exist some orbifold $O_1$ with an \Eoc\ equal to $x$ with 
%signature $\ast b_1, \cdots, b_{n+1}$. 
%Taking an orbifold $O_2$ with signature $\ast b_1, \cdots, b_n$, its \Eoc\ $\cho{O_2}$ 
%is in $B_n$ as $O_2$ has $n$ orbipoints. The difference between $\cho{O_1}$ and $\cho{O_2}$ 
%is $-\frac{b_{n+1} - 1}{2b_{n+1}} \in \{-\frac{n-1}{2n}\ |\ n\geq 2\}$. \\
%Let take $x \in B_n+\{-\frac{n-1}{2n}\ |\ n\geq 2\}$. Then $ x = x_n + x'$ for some 
%$x_n \in B_n$ and $x' \in \{-\frac{n-1}{2n}\ |\ n\geq 2\}$. Let 


\begin{observation}\label{form_of_a_spectrum}
Observe that, as any orbifold has only finitely many orbipoints, we have that $\speD \subseteq 
\bigcup\limits^\infty_{n=0}B_n $. We defined $\speD$ as a set of all possible \Eoc\ of disk 
orbifolds, so $\speD \supseteq 
\bigcup\limits^\infty_{n=0}B_n $. From this we have that $\speD = \bigcup\limits^\infty_{n=0}B_n$.
\end{observation}

\begin{lemma}\label{fixed_number_of_orbipoints}
For any given $n \in \mathbb{N}$ the set $B_n$ is a subset of the interval 
$[1-\frac{n}{2}, 1-\frac{n}{4}]$.
\end{lemma}
\textbf{Proof.} \\
Take $x \in B_n$. There exists an orbifold $O$ with signature $\ast ^*b_1, \cdots, ^*b_n$, 
such that $\cho{O} = x$. We have that $\forall_i -\frac{1}{2} \leq \Delta(^*b_i) \leq 
-\frac{1}{4}$. From this $-\frac{n}{2} \leq \Delta(^*b_1, \cdots, ^*b_n) \leq -\frac{n}{4}$, 
so $\cho{O} \in [1-\frac{n}{2}, 1-\frac{n}{4}]$. $_\square$

\begin{observation}\label{ascending_in_B_n}
From \ref{recursive_relation} and \ref{two_sets_lemma}, we have that $B_n$ do not have 
infinite ascending sequence for all $n$. 

Further, from \ref{sum_lemma} we conclude, 
that $\bigcup\limits_{n=0}^N B_n$ do not have infinite ascending sequence for all $N$.
\end{observation}

\begin{theorem}\label{well_order}
In $\speD$ there are no infinite strictly ascending sequences, hence, it is well ordered.
\end{theorem}
\noindent\textbf{Proof.} \\
%Being well ordered is a consequence of not having infinite strictly ascending sequences. 
For the sake of contradiction lets assume that $c_n$ is an infinite strictly ascending sequence in 
$\speD$. As $c_n$ is bounded from below by $c_0$ and whole $\speD$ is bounded from above 
by $1$, all elements of $c_n$ are in the interval $[c_0, 1]$. 
From \ref{form_of_a_spectrum} we have, that $\speD = \bigcup\limits^\infty_{n=0}B_n$. 

Lemma 
\ref{fixed_number_of_orbipoints} says that for all $n$ we
have $B_n \subset [1-\frac{n}{2}, 1 - \frac{n}{4}]$. From this, we know, that for any 
$n$ such that $1 - \frac{n}{4} < c_0$ 
we have, that $B_n \cap [c_0,1] = \varnothing $. Let $n_0$ be the smallest such that 
$1 - \frac{n_0}{4} < c_0$ (so $n_0 > 4(1-c_0)$). 
Then for all $n > n_0$ we have $1 - \frac{n}{4} < c_0$, meaning, that 
for all $n > n_0$ we have
$B_n \cap [c_0,1] = \varnothing $, so all elements of $c_n$ are in 
$\bigcup\limits_{n=0}^{n_0} B_n$.
But this contradicts \ref{ascending_in_B_n}.  $_\square$
%\todo{rozwinąć ostatni argument}









\subsection{Proof that order structure and topology of $\speD$ are those of $\omega^\omega$}
%\smalltodoII{rozbić to na dwa?} 
\begin{theorem}\label{speD_order_type}
Order type and topology of $\speD$ is $\omega^\omega$. 
\end{theorem}
\textbf{Proof.} 
%that order type of $\speD$ is $\omega^\omega$: \\
%- proof that it is at least $\omega^\omega$ \\

We will first prove, that the order type of $\speD$ is $\omega^\omega$.

$\bullet$ Order type of $\speD$ is at least $\omega^\omega$.

%For the sake of contradiction, let us suppouse that the order type $\eta$ of $\speD$ 
%is less than $\omega^\omega$. Then $\eta$ is smaller than $\omega^n$ for some $n$. 
%However, 
From \ref{greatest \apots} we know, that for every $n\in \mathbb{N}$, in $\speD$ there are 
accumulation points of order $n$. From this, and from \ref{accumulation_points_and_order} 
we know that $\speD$ has an order type at least $\omega^n$, for all $n\in \mathbb{N}$. 
The smallest ordinal number qual at least $\omega^n$, for all $n \in \mathbb{N}$ is 
$\omega^\omega$. Thus, the order type of $\speD$ is at least $\omega^\omega$.\\
% but for sufficiently distant have accumulation point. \\
%done \\

$\bullet$ Order type of $\speD$ is at most $\omega^\omega$.

%- proof that it is at most $\omega^\omega$\\
For the sake of contradiction, let us suppouse, that the order type $\eta$ of $\speD$ is 
strictly greater than $\omega^\omega$. Then, $\speD$ has a set $A$ of an order type 
$\omega^\omega$ as it's prefix. The set $A$ is bounded, as the $\omega^\omega+1$st element of 
$\speD$ is greater than any alement of $A$. 
Let $n$, be such that $1-\frac{n}{2}$ is smaller than any lement of $A$. 
As $A$ is of order type $\omega^\omega$ it has a prefix $B$ of order type $\omega^n$.
From \ref{accumulation_points_and_order} we know, that $B$ has an accumulation point $b$ 
of order $n$. This gives us a contradiction, as $b > 1-\frac{n}{2}$, and from \ref{greatest \apots} 
we know, that $1-\frac{n}{2}$ is the greatest accumulation point of order $n$ in $\speD$. \\

%suppouse it is bigger \\
%then there is a point before which there is omega omega order \\
%wooo, but it cant be since first with accumulation is at something spmething \\
%done \\
%ok, so how to give convincing correspondence between accumulation points and $\omega^n$??
%Lemma \\
%there cant be order of type \\
%i wish I can cite this from somewhere \\ 
%
%\newpage
%

Now, we will prove, that the topology of $\speD$ is that of $\omega^\omega$.

From \ref{accumulation_points_of_the_set} we know that every accumulation point 
of $\speD$ is in $\speD$. 
%We just showed, that the order type of $\speD$ is $\omega^\omega$.
Thus, $\speD$ satisfies the assumtions of the lemma \ref{order_preserving_homeomorphism_theorem} 
and 
%\ref{accumulation_points_of_the_set} and the first part of the proof, that just showed, 
%that the order type 
we have that 
the topology of $\speD$ is $\omega^\omega$. 






\section{Order type and topology of $\spebr{M}$}\label{all_spectra_are_isomorphic}\label{spe_M}

\begin{observation}\label{2times homeomerphism}
We have that $\speS = 2\speD$.
\end{observation}
\textbf{Proof.}\\

Indeed, since: \todo{write the sums}

\begin{observation}
For every two dimentional manifold $M$, we have that $\spe(M)$ is homeomorphic to $\speS$. 
For $M$ with $h$ handles, $c$ cross-cups and $b$ boundary components, this homeomorphism is: \\
$\bullet$ for $b \neq 0$:
\begin{equation}
\spe(M) = \speD - 2h - c - (b - 1), 
\end{equation}
$\bullet$ for $b = 0$:
\begin{equation}
\spe(M) = 2\speD - 2h - c.
\end{equation}  
\end{observation}

%Ok, all isomorphic with $\speD$ \\
%here write about it \\
%Tell me about it!
\textbf{Proof.} \\
For a manifold $M$ with $h$ handles, $c$ cross-cups and $b$ boundary components, it's 
Euler characteristic is given by:
\begin{equation}
\chi(M) = 2-2h-c-b.
\end{equation}
The possible $\Delta$ for possible orbifold features are:\\
$\bullet$ for $b\neq 0$:
\begin{equation}
\{-\frac{n-1}{2n}\ \big|\ n\in\mathbb{N}_{>0}\cup \{\infty\}\}
\end{equation}
$\bullet$ for $b = 0$:
\begin{equation}
\{-\frac{n-1}{n}\ \big|\ n\in\mathbb{N}_{>0}\cup \{\infty\}\}.
\end{equation} 
%with the convention, that $\varphi(\infty) \coloneqq \lim\limits_{k\to \infty} \varphi(k)$.

Thus, we have that:\\
$\bullet$ for $b\neq 0$: 
\begin{equation}
\spe(M) = 2-2h-c-b - \left\{\sum\limits_{i=1}^n\frac{d_i-1}{2d_i}\ 
\big|\ n\in\mathbb{N}_0,\ d_i\in\mathbb{N}_{>0}\cup \{\infty\}\right\}
\end{equation}
$\bullet$ for $b = 0$:
\begin{equation}
\spe(M) = 2-2h-c - \left\{\sum\limits_{i=1}^n \frac{r_i-1}{r_i}\ \big|\ n\in\mathbb{N}_0,\ 
r_i\in\mathbb{N}_{>0}\cup \{\infty\}\right\}.
\end{equation} 

On the other hand, we have that:
\begin{equation}
\speD = 1-\left\{\sum\limits_{i=1}^n\frac{d_i-1}{2d_i}\ 
\big|\ n\in\mathbb{N}_0,\ d_i\in\mathbb{N}_{>0}\cup \{\infty\}\right\}
\end{equation}
\begin{equation}
\speS = 2 - \left\{\sum\limits_{i=1}^n \frac{r_i-1}{r_i}\ \big|\ n\in\mathbb{N}_0,\ 
r_i\in\mathbb{N}_{>0}\cup \{\infty\}\right\}
\end{equation}
and
\begin{equation}
\speS = 2\speD.
\end{equation}
From this, the observation follows immedietly. $_\square$

%In short, for every manifold it either have 

\section{Order type and topology of $\spe$}
\begin{theorem}
The order type of the set of possible Euler orbicharacteristics of two-dimensional orbifolds 
$\spe$ is $\omega^\omega$. 
\end{theorem}
\textbf{Proof.} \\
From \ref{sum of spectras} we know, that $\spe = \speD \cup \speS$.

From \ref{speD_order_type} and \ref{2times homeomerphism}, we have that order types and 
topologies of $\speD$ and $\speS$ both are $\omega^\omega$ and that $\speS = 2\speD$.  

We will now prove that the order type of $\spe$ is $\omega^\omega$.

From \ref{greatest \apots} we know, that the largest \apots\ $\speD$\ of order $n$ is 
$1-\frac{n}{2}$. From, this and from the fact that $\speS = 2\speD$ we know that 
that the largest \apots\ $\speS$\ of order $n$ is 
$2-n$. From this, we have, that for every $m \in \mathbb{N}_{>0}$, order type of 
$(-m, \infty) \cap \speD$ is $\omega^{2m+2}$ and that order type 
of $(-m, \infty) \cap \speS$ is $\omega^{m+2}$. 
%So, for every $m \in \mathbb{N}_{>0}$, the order type of $\speD$ is strictly greater than 
%th
Thus, for every $m \in \mathbb{N}_{>0}$, we have that $(-m, \infty) \cap \speD$ and 
$(-m, \infty) \cap \speS$ satisfies assumtions of \ref{key_lemma}, thus, we have that 
$(-m, \infty) \cap (\speD\cup\speS)$ have an order type $\omega^{2m+2}$. 

From this we have that 
\begin{equation}
\spe = \speD \cup \speS = 
\bigcup\limits_{m=1}^\infty \left( (-m, \infty) \cap (\speD\cup\speS) \right)
\end{equation} 
have an order type $\omega^\omega$.

Now we will prove, that the topology of $\spe$ is that of $\omega^\omega$ too.

We have that for $\speD$ \rba{$\speS$} that every accumulation point of 
$\speD$ \rba{$\speS$} is in $\speD$ \rba{$\speS$}. From this and from 
\ref{derivative and sum is commutative} we have, that all accumulations points of $\spe$ 
are in $\spe$. From this, from lemma \ref{order_preserving_homeomorphism_theorem} we have 
that the topology of $\spe$ is $\omega^\omega$. $_\square$ 

%powiedziec, że wtedy zletmatu suma będzie też taka

%powiedzieć że w takim razie calosć będzie taka

%there are no new accumulation points -- all accumulations points are in sigma -- topology is ok

%and 
%dopisać w apendiksie, że jak wysumuję dwa o różnym typie porządkowym, to wychodzi większy równy
%Provide some argument about being homeomorphic 
%\noindent\textbf{Proof.} \\
%From \ref{speD_theorem} we know, that $\speD$ is homeomorphic with $\omega^\omega$. From 
%\ref{all_spectra_are_isomorphic}, we know, that $\speS$ is homeomorphic 
%with $\omega^\omega$. \\
%$\speS = 2\speD$, so for all $n\in \mathbb{N}$ set $\speS \cap [2,-n)$ has a lower order type then 
%$\speD\cap [2,-n)$. From this and from \ref{sum_lemma}, we have that $\speS \cup \speD$ 
%has an order type of $\omega^\omega$. \\
%For homeomorphicity, the proof of theorem \ref{speD_theorem} works as well in this case.
%$_\square$ \\[4pt]

\section{Order type and topology of n-th cantor bendixon derivative of $\spe$} 
$\spe_n$ \\
taking limit points is the order type of $\omega^\omega$ but not homeomorphic anymore. 



%\section{Which points are in the $\spe$?}

%Here we will try to understand better the conditions that let us determine wether the point 
%lie in $\spe$ or not.
% We will also state some observations about reasoning which points 
%belong to $\spe$ based on the knowlegde of other points belonging there. \\


\section{More about how this $\spe$, $\speD$ and $\speS$ lie in $\mathbb{R}$}
%\todo{movve to some other section maybe}
\begin{observation}
The first (greatest) negative \apots\ of $\spe$ is 
$-\frac{1}{12}$. It is the accumulation point of order $1$. 
\end{observation}
\noindent\textbf{Proof.} 

We will show, that $-\frac{1}{12}$ is the greatest negative accumulation point of the set $\speD$. 
From this we will obtain the thesis, as the set of all possible Euler orbicharacteristics 
of two-dimensional orbifolds is equal to $\speS \cup \speD$ and $\speS = 2\speD$, so 
the greatest negative point of the set $\speS$ is smaller than the greatest negative accumulation 
point of the set $\speD$. 

$\bullet$ $-\frac{1}{12}=\chi^{orb}((2,3))-\frac{1}{2}$, from this we have that $-\frac{1}{12}$ 
an acccumulation point of the set $\speD$ of order at least $1$. 

$\bullet$ Let us assume (for a contradiction), that there exist bigger, negative 
accumulation point of the set $\speD$ of order at least $1$. Let us denote it by $x$. 

However, then, from the lemma \ref{third_order_lemma} point $x+\frac{1}{2}$ is the accumulation 
point of the set $\speD$. What is more, since $x\in (0, -\frac{1}{12})$, then $x+\frac{1}{2} 
\in (\frac{1}{2}, \frac{5}{12}$. From the lemma \ref{accumulation_points_of_the_set} we 
have that $x$ is in $\speD$. But orbifolds of the type $\ast b_1$ can have \Eoc only greater or 
equal $\frac{1}{2}$. Orbifolds of the type $\ast b_1b_2$ can only have \Eoc $\frac{1}{2}$, 
$\frac{5}{12}$ and some smaller. Orbifolds of the type $\ast b_1b_2b_3\cdots$ can have \Eoc only 
lower than $\frac{1}{4}$. This analysis of the cases leads us to the conclusion, that 
$(\frac{1}{2},\frac{5}{12})\cap \speD=\emptyset$ and to the contradiction. 

$\bullet$ Above analysis of the cases leads us also to the conclusion, that $\frac{5}{12}$ 
is 
an isolated point of the set $\speD$, from this $-\frac{1}{12}$ is an acccumulation point 
of order $1$ of the set $\speD$. $_\square$ \\ 
 
\subsection{Connections between $\speD$ and $\speS$}\label{D_and_S}
In this section we would like to 
%develop the tools and 
answer some questions about 
%interrelationships
relations between $\speD$ and $\speS$. 

The first, stated in \ref{} is that $2\speD=\speS$. 
This tells us all about simmilarities of their topological structures -- namely, 
they are the same, 
but it does not directly answers questions about how they lie in $\mathbb{R}$, relative 
to each other.
%\subsubsection{Some preliminary observations}
\todo{dopisać trochę inną motywację}
\\[16pt]
We now state some observations that will be usefull in this section.
\begin{observation}
If an \Eoc\ is an accumulation point of order $n$ in $\speD$ \rba{respectively $\speS$}, 
there exist an $D^2$ \rba{resp. $S^2$} 
orbifold with $n$ dihedral \rba{resp. rotational} points of that \Eoc. 
\end{observation}
prrof. from chapter 3. (todo: dopisać)
\begin{observation}\label{adding_multiplied_differences}
If $x \in \speD$ \rba{respectively $\speS$}, then $1-x$ \rba{resp. $2-x$} is 
a difference in 
\Eoc\ resulting 
from some set of dihedral \rba{resp. rotational} points. From that $1-n(1-x) \in \speD$ \rba{resp. 
$2-n(2-x)\in \speS$}
for all $n \in \mathbb{N}$. 
\end{observation}
\subsubsection{$-\frac{1}{84}$ and $-\frac{1}{42}$}
//Why it is how it is//
%\subsection{All the accumulation points of the $\speS$ are in $\speD$}
\subsubsection{Accumulation points of the $\speS$}
\begin{theorem}
All accumulation points of the $\speS$ are in $\speD$.
\end{theorem}
There are two proofs of this theorem showing nice correnpondence -- one arithmetical and 
one geometrical. 
\\
\textbf{Proof I.}
%\subsubsection{Arithmetical reason}
\textbf{Arithmetical reason} \\
We assume that $x \in \speS$ is an \apots\ $\speS$.\\
%If $x \in \speS$, then $\frac{x}{2} \in \speD$. Then $1 - \frac{x}{2}$ 
%is a difference in \Eoc\ resulting from some set of cone points. We can add to the disc twice as 
%many cone points and resulting orbifold $\mathcal{O}$ will have an \Eoc\ equal to 
%$1 - 2(1-\frac{x}{2}) = x - 1$. From \ref{third_order_lemma} for the thesis it is sufficient 
%to $x - 1$ to be an \apots\ $\speD$ of order at least two. \\
%%$2-x$ is a difference in \Eoc\ resulting from some set of gyration points. We can 
%We asumed that $x \in \speS$ is an \apots\ $\speS$, so, 
By \ref{times_two_fact} we have, that 
$\frac{x}{2} \in \speD$ is an \apots\ $\speD$. From \ref{third_order_lemma} we have that 
$\frac{x}{2} + \frac{1}{2} \in \speD$. From that, from \ref{adding_multiplied_differences} 
we have, 
that \begin{equation}
1-\overbrace{2}^{\substack{"n"\rm\ from\ \\ 
\rm \ref{adding_multiplied_differences}}} 
(\ \overbrace{1-(\frac{x}{2}+\frac{1}{2})}^{\substack{"1-x"\rm\ 
from\ \\ \rm \ref{adding_multiplied_differences}}}\ ) \in \speD.
\end{equation} 
But $1 - 2(1-(\frac{x}{2}+
\frac{1}{2})) = x$, so $x \in \speD$. $_\square$
% for some $y \in \speD$, so $\speD$.
\\[2pt]
\textbf{Proof II.}
%\subsubsection{Geometrical reason}
\textbf{Geometrical reason} \\
We assume that $x \in \speS$ is an \apots\ $\speS$.\\
%If $x \in \speS$
From \ref{predescors} we know, that $x$ can be expressed as $y - 1$ for some $y \in \speS$. \\
Let $\mathcal{O}$ be an orbifold with the base manifold $S^2$, such that $\cho{\mathcal{O}} 
= y$. \\
Let $\mathcal{O}_c$ be the orbifold created from $\mathcal{O}$ by adding one cusp. 
Then $\cho{\mathcal{O}_c} = y - 1 = x$. Topologically $\mathcal{O}_c$ with the cusp point 
removed (which do not change an orbicharacteristic) is $\mathbb{R}^2$. 
We can compactify it with $S^1$. This will not change an \Eoc\ since $\cho{S^1} = 0$ and 
\Eoc\ is additive.
\\ What we get is an orbifold $\mathcal{O}_D$ with the base 
manifold $D^2$ and the same 
orbipoints as $\mathcal{O}$. Since orbipoints of $\mathcal{O}$ create a difference 
in \Eoc\ equal to $2-y$, we have that $\cho{\mathcal{O}_D} = 1 - (2-y) = y - 1 = x$. 
We can then move all orbipoints from the interior of $\mathcal{O}_D$ to its boundry 
by doubling them, so $x \in \speD$. $_\square$\\



%The first equstion we can tackle is steaming from the chapter \ref{order structure} 
%and it is -- 
Do $\speD$ and $\speS$ coincide? It is easy to answer that $\speD \neq \speS$ 
(and we will do that along some harder questions in the moment), but do they coincide 
starting from a sufficiently distant point? Or maybe, for every denominator, do they coincide 
from a sufficiently distatn point? (Yes.) \\






% mainfile: ../praca_magisterska_orbifoldy.tex
\chapter{Algorithm for searching for the spectrum}\label{Searching the spectrum}

In the previous chapter we answered the questions about how $\spe$ looks like -- in particular 
what is it's order type and topology. In this chapter we would like to develop a 
methods for answering the 
following question: 
%In the previous chapter the main question was about which rational numbers are in $\spe$. 
%We can ask, how to answear the question 

%In this chapter we will show that the question 
"For a given rational number, is it in $\spe$?" 

We have some sort of answer to this question -- an algorithm.
%a very long equation that 
%commonly is refered to as an algorithm. 

It is not an ideal answer as it gives little insight of what is a general structure 
of the spectrum. Nevertheless it is a constructive and computable answer. 
%What's more, 
%later, in implementation chapter we discuss that for numbers with denominators of a reasonable 
%size and in reasonable distance from zero this algorithm can be run successfully on a 
%personal computer.

%is computable.

%In this chapter, we will provide the best answear we could find. 

%Turns aout that the number is in the specturm iff the following procedute says "yes". 

%It turns aout that the question for any number is answearable  

%Algorithmical approunch to questions from chapter 3 are only possible after gfiniteness 
%part of chapter four.

%What we can show, is that this question is computable -- i.e. there exists an algorithm 
%that answears this question. 

%Using results from previous chapters, we can now prove, that some computational problems related 
%to spectra are solvable.

%We will do it in a constructive way, by writing explicitly the algorithm and proving its
%correctness.

% and properties.

%We will also be able to actually compute sufficiently small examples of the 
%question.
%unansweared question 

The exact question we will provide algorithm to answer here is: 

\textit{For a given rational number $r$ and manifold $M$, is there at least one 
$M$ orbifold with $r$ as its \Eoc?}

%For a given rational number $r$ and manifold $M$:
%\begin{itemize}
%\item How many $M$ orbifolds with $r$ as their \Eoc\ are there?
%\item The accumulation point of what degree is $r$ in $\spe(M)$?
%\end{itemize}

We start with $r=\frac{p}{q}$, where $p \in \mathbb{Z}$, $q \in \mathbb{N}_{>0}$ and a manifold $M$. 
%\section{Reductions and special cases}

%\section{Decidability}
%\todo{oj dokończyć}
%Here we will show the proof that the problem of "deciding whether a given rational number is in an 
%Euler orbicharacteristic's spectrum or not" is decidable by showing algorithm for doing this. 
%Later, our algorithm will have a bonus property of determining of which order of condensation 
%is given point if it is in fact in $\sigma$. \\
%\smalltodoII{Może od razu postawić pełny problem}
%%It fill get also a performance enhancement by this added property. \\
%First stated algorithm is also very inefficient and is presented, because the idea is the most 
%clear in it. Right after it there is stated an algorithm with two enhancements: 
%\begin{itemize}
%\item determining an accumulation point of which order is a given point, if it is in fact in the 
%spectrum (this enhancement gives also a performance boost) 
%\item faster searching, because some cases do not need to be checked. 
%\end{itemize}
%\subsection{The algorithm}
%and $\textrm{gcd}(p,q)$. \\ 
\section{Reduction from arbitrary $M$ to $D^2$}\label{algorithm reduction to D2}
This reduction is based on \ref{spe_M}.
Note, that this is a different reduction than the one in chapter \ref{reduction_to_arithmetical}. 
In chapter \ref{reduction_to_arithmetical} we are saying that for any $M$, we have $\spebr{M} 
\subseteq \speS \cup \speD$. In \ref{spe_M} on the other hand we have, that 
for a manifold $M$ with $h$ handles, $c$ crosscaps and $b$ boundary components: \\
for $b \neq 0$:
\begin{equation}
\spe(M) = \speD + \chi(M) -1 = \speD - 2h - c - (b - 1)
\end{equation}
and for $b = 0$:
\begin{equation}
\spe(M) = 2\speD + \chi(M) - 2 = 2\speD - 2h - c.
\end{equation}  


%Using \ref{spe_M} 
%and \ref{} 
We conclude that the problem of deciding whether $\frac{p}{q}$ is in $\spe(M)$
is equivalent to deciding: \\
for $b \neq 0$ if:
\begin{equation}\label{translation with b not 0}
\frac{p}{q} - \chi(M) + 1 = \frac{p}{q} + 2h + c + (b-1) 
\end{equation} 
is in $\sdD$; \\
for $b = 0$ if:
\begin{equation}\label{translation with b 0}
\frac{1}{2}\frac{p}{q} + \chi(M) + 2 = \frac{1}{2}\frac{p}{q}+h+\frac{c}{2}
\end{equation}
is in $\sdD$.

Considering this fact, from this point, WLOG we will assume that $M = D^2$ and, 
following \ref{only dihedral}, we will 
be concerned only with dihedral orbipoints.

%\subsection{Arithmetic formulation}
%We want to determine whether there exists $d_1,d_2,\cdots,d_k$, such that 
%$\chi^{orb}(*d_1\cdots d_k) = \frac{p}{q}$. 

\section{Special cases}
In the case that $\frac{p}{q}$ is of the form $l\frac{1}{4}$, for some $l \in \mathbb{Z}$ 
% $q = 4$ 
we can give the answer right away. For $l > 4$ we have that $l\frac{1}{4}$ is not in the set 
and for $l \leq 4$ it is (see \ref{greatest \apots}). 

Moreover for an even $l$ we have that $l\frac{1}{4}$ is an accumultion point of order 
$\frac{4-l}{2}$ 
and for an odd $l$ it is an accumulation point of order $\frac{3-l}{2}$ (see \ref{greatest \apots} 
and \ref{predescors}). 

In the case, where $\frac{p}{q} > 1$, we also can give answer right away and this answer is "no". 

Now we will consider only cases when $\frac{p}{q}$ is not of the form $l\frac{1}{4}$ and is 
$\leq 1$.
%\section{General case}
%
%\section{Simpler version of the question}
%
%To present the idea of searching the spectrum for the orbifolds with a given \Eoc, we will 
%first present the algorithm that answears a little easier question, namely: 
%
%\textit{For a given rational number $r$ and manifold $M$, is there at least one 
%$M$ orbifold with $r$ as their \Eoc?}
%
%This algorithm will mirror what we are focused on in \ref{chapter_three}, giving us the 
%computational tool for deciding whether a given number is in the spectrum or no. 
%
%The first approach of the searching algorithm is of this form: \\
%
%%We start with the 
%We use: 
%\begin{itemize}
%$\mathbb{N}$ counters $d_1d_2\cdots$ 
%(with values ranging from $1$, through all natural numbers, to infinity 
%(with infinity included)) set to $1$. Each counter correspond to one cone point 
%on the boundry of the disk of period equal to the value of the counter (with the note, that 
%if counter is set to $1$ it means a trivial cone point - namely a none cone point, a normal 
%point). 
%Every state of the counters during runtime of the algorith will have only finitely many 
%counters with value non-$1$. Moreover every state in the rutime of the algorithm 
%will have values on consequtive counters ordered in weakly decreasing order. From now we will 
%consider only such states. \\
%The state $d_1d_2\cdots$ correspond to the orbifold of 
%\Eoc equal $\chi^{orb}(*d_1d_2\cdots)$ (where the trailing $1$ are trunkated). \\ 
%%There is also a pivot pointing on one counter at any time.  
\section{Regular cases}
First we will describe what we use in the algorithm, giving the brief semantics. 
The detailed semantics are given in \ref{the idea of the algoritm}.
\subsection{What we use}
We use: 
\begin{itemize}
\item $\mathbb{N}_{>0}$ counters $c_1, c_2, \cdots$ 
with values ranging on $\mathbb{N}_{>0}\cup\{\infty\}$.
%$1$, through all natural numbers, to infinity 
%(with infinity included). 
Each counter correspond to one dihedral point 
on the boundary of the disk of period equal to the value of the counter (with the note, that 
if counter is set to $1$ it means a trivial dihedral point - namely a non-orbi point, 
a normal point). 

We will write the state of the counters without commas, using the letter $d$. 
Note that with this convention, $c_i$ will refer to the $i$-th counter and $d_i$ will 
refer to the value of the $i$-th counter. 

So the state of the counters $d_1d_2\cdots$ correspond to the orbifold 
$*d_1d_2\cdots$ (where the trailing $1$'s are truncated).

We will refer to the counters being "to the left" or "to the right" of each other, as 
the numbering would go from left to right.

\item a pivot pointing at some counter 
% at any time
\item a flag that can be set to: "Greater", "Searching" or "Less" corresponding to what was 
the outcome of comparing \Eoc\ of the orbifold corresponding to counters' state and 
$\frac{p}{q}$ or to the fact, that there is a need for a search of the next state of counters 
to compare with $\frac{p}{q}$.  
\end{itemize}
\subsection{What state are we starting our algorithm with}
We start with:
\begin{itemize}
\item all counters set to $1$. 
\item pivot pointing at the $c_1$
\item flag set to "Greater"
\end{itemize}
%If $\frac{p}{q}$ is of form $\frac{k}{4}$, where $k \in (-\infty,8] \cap \mathbb{Z}$ we give 
%the answear "yes" and end the whole algorithm. If $\frac{p}{q} > 2$ we give the answear "no" and 
%end the whole algorithmBecauseof this, below we assume, that \\
\subsection{Invariants claims}
Now we will state the claims of what properties the state of the counters will maintain 
during all the execution of the algorithm. The proof, that this is indeed the case will 
be performed in 
\ref{memory state proof}
\begin{claim}\label{valid state of counters}
We will do our computation such that:
\begin{itemize}
\item every state of the counters during runtime of the algorithm will have only finitely many 
counters with value non-$1$. 
\item every state in the runtime of the algorithm 
will have values on consecutive counters ordered in weakly decreasing order.
\end{itemize}
\end{claim}
From now we will 
consider only such states. 

%There is also a pivot pointing on one counter at any time.  
%The state of the counters $d_1d_2\cdots$ correspond to the orbifold 
%%of \Eoc\ equal $\chi^{orb}(*d_1d_2\cdots)$
%$*d_1d_2\cdots$ (where the trailing $1$'s are trunkated). 
\subsection{The algorithm for searching for a spectrum}
\label{the algorithm for searching the spectrum itself}
When the algorithm is in the state: 
\begin{itemize}
\item counters with values: $d_1d_2\cdots$
\item pivot: at the counter $c_p$
\item flag: set to the value $flag\_value$,
\end{itemize}
we proceed as follows 
%(the term "We continue." means, that we start the following procedure from the beginning)
:
%// More on how we search for it will be told later, 
%        // for now we can think that we search one by one,
%        // starting from $d_p$ and going up till $d_p'$.
\begin{lstlisting}[firstnumber=1,consecutivenumbers=true]
In the case, the $flag\_value$ is equal to: 
{
    "Greater", then
    {
        If $\chi^{orb}(*d_1\cdots d_{p-1}\infty d_{p+1}\cdots)=\frac{p}{q}$ then
        {
            We found an orbifold and we are ending the whole
            algorithm with answer "yes, $*d_1\cdots d_{p-1}\infty d_{p+1}\cdots$".
            
            
            
        } 
        If $\chi^{orb}(*d_1\cdots d_{p-1}\infty d_{p+1}\cdots)>\frac{p}{q}$ then
        {
            We set $d_p$ to $\infty$.
            We set the flag to "Greater".
            We put the pivot at the $c_{p+1}$.
            We go to the 1st line.
        }  
        If $\chi^{orb}(*d_1\cdots d_{p-1}\infty d_{p+1}\cdots)<\frac{p}{q}$ then
        {
            We set the flag to "Searching".
            We go to the 1st line.
        }  
    }
    
    "Searching", then
    {
        We search one by one 
        for the value $d_p'$ of the $c_p$ such that
        $\chi^{orb}(*d_1\cdots d_{p-1}d_p'd_{p+1}\cdots)\leq\frac{p}{q}$ and
        $\chi^{orb}(*d_1\cdots d_{p-1}(d_p'-1)d_{p+1}\cdots)>\frac{p}{q}$.
        We set $c_p$ and all of the counters 
        to the left of $c_p$ to the value $d_p'$.
        if $\chi^{orb}(*d_1d_2d_3\cdots)=\frac{p}{q}$ then 
        {
            We found an orbifold and we are ending the whole
            algorithm with answer "yes, $*d_1d_2\cdots$".
            
            
            
        }
        If $\chi^{orb}(*d_1d_2d_3\cdots)>\frac{p}{q}$ then 
        {
            We set the flag to "Greater".
            We put the pivot at the $c_1$.
            We go to the 1st line.
        }
        If $\chi^{orb}(*d_1d_2d_3\cdots)<\frac{p}{q}$ then 
        {
            We set the flag to "Less".
            We put the pivot at the $c_{p+1}$.
            We go to the 1st line.
        }
    }
    
    "Less", then 
    {
        If $d_p = 1$ and the values of all the counters 
        on the left of $c_p$ are equal to 2 then 
        {
            We end the whole algorithm with the answer "no".
        }
        We increase $c_p$ by one ($d_p \coloneqq d_p + 1$) and
        we set the value of all counters on the left of $c_p$ to $d_p$.
        If $\chi^{orb}(*d_1d_2d_3\cdots)=\frac{p}{q}$ then
        {
            We found an orbifold and we are ending the whole
            algorithm with answer "yes, $*d_1d_2\cdots$".
            
            
            
        }
        If $\chi^{orb}(*d_1d_2d_3\cdots)>\frac{p}{q}$ then  
        {
            We set the flag to "Greater".
            We put the pivot at the $c_1$. 
            We go to the 1st line.
        } 
        If $\chi^{orb}(*d_1d_2d_3\cdots)<\frac{p}{q}$ then
        {
            We set the flag to "Less".
            We put the pivot at the $c_{p+1}$.
            We go to the 1st line.
        } 
    }
}
\end{lstlisting}
\section{The idea of the algorithm}\label{the idea of the algoritm}
We will now present in more detail what the algorithm is indented to do. 
To do this and for the later sections, we will first introduce an order on the states 
of counters satisfying \ref{valid state of counters} (as mentioned in 
\ref{valid state of counters} we will consider only such states) and prove several lemmas about it. 
\subsection{Order on the space of states of the counters}
\begin{definition}
We define a linear order $\preceq$ on the states of counters as follows:

Let $D_1$ be a state of counters equal to $d_1^1d_2^1\cdots$ and $D_2$ be a state of counters 
equal to $d_1^2d_2^2\cdots$. Let $i$ be the greatest index where $D_1$ and $D_2$ differ, then:\\
$bullet$ If $d_i^1 \leq d_i^2$ then $D_1 \preceq D_2$. 
\end{definition}

This is a suborder of the lexicographical order 
of states of counters after truncation of trailing 1's 
with the counters to the right being more significant. 

\begin{observation}
In general it is not true that if $D_1 \preceq D_2$ then 
$\cho{*D_1} \leq \cho{*D_2}$ nor that if $D_1 \preceq D_2$ then 
$\cho{*D_1} \geq \cho{*D_2}$.
\end{observation}

\begin{observation}\label{good lexicographical order}
Since $\preceq$ is a suborder of a lexicographical order it is a good order. 
\end{observation}

Let us use $S(a)$ for a successor of $a$.
We can explicitly write the form of the successor of any state $d_1d_2d_3\cdots$ in $\preceq$:
%(Quotation marks around the counters' states in the following 
%lemma are added here only for readability 
%and they bare no particular meaning.)
\begin{observation}\label{form of the successor}
The successor of 
the state $d_1d_2d_3\cdots$, of the form
\begin{equation}
\underbrace{\infty\infty\cdots\infty}_{k-1 \rm\ times} d_kd_{k+1}d_{k+2}\cdots,
\end{equation}
where $k$ is 
such that $c_k$ is the first counter 
from the left that is not set to $\infty$,
% in the state $d_1d_2d_3\cdots$, 
is
\begin{equation} 
\underbrace{(d_k+1)(d_k+1)\cdots(d_k+1)}_{k-1\rm\ times}(d_k+1)d_{k+1}d_{k+2}\cdots,
\end{equation}

\end{observation}
%\subsubsection{Proof.}
\begin{definition}
We will call the state $d_1d_2d_3\cdots$, such that no $d_k$ is equal to $\infty$ a 
\textbf{finite} state. 

We will call the state $d_1d_2d_3\cdots$, such that at least one of $d_k$ is equal to $\infty$ 
an \textbf{infinite} state.  
\end{definition}
\begin{observation}
Using \ref{valid state of counters} we have that 
for the state $d_1d_2d_3\cdots$ to be finite (resp. infinite), it is equivalent to 
$d_1$ being different from (resp. being equal to) $\infty$. 
\end{observation}
%\begin{observation}

%\end{observation}
\begin{observation}\label{finiteness of the successor}
For any state $D$, we have that $S(D)$ is a finite state.
\end{observation}
\begin{definition}
We will call the ascending sequence $\{D_n\}$ 
in $\preceq$, such 
that for all $n$, we have that $S(D_n) = D_{n+1}$, a \textbf{connected} sequence in 
$\preceq$.   
\end{definition}
\begin{observation}
Every connected sequence of the finite states is of the form $\{(d_1+n)d_2d_3\cdots\}$, 
where all $d_n$ are different from $\infty$.
\end{observation}
\begin{lemma}\label{Successor lemma}
Let $D_1$ and $D_2$ be finite states and let $S(D_1) = D_2$ in 
$\preceq$. Then $\cho{*D_1} > \cho{*D_2}$.
\end{lemma}
\subsubsection{Proof.}
From \ref{form of the successor} we know, that taking the successor 
of the finite state always changes  
only first counter and it is changing it by increasing it by 1. 
Increasing the order of the orbipoint  
decreases \Eoc. $_\square$

\begin{corollary}\label{connected sequences corollary}
The sequence $\{\cho{*D_n}\}$ is descending for every connected sequence 
of finite states $\{D_n\}$ in $\preceq$. 
\end{corollary}

\begin{lemma}
Let $D_1$ be infinite state and let $D_2 \coloneqq S(D_1)$ in $\preceq$. 
Then $\cho{*D_1} \leq \cho{*D_2}$. Furthermore there is only one element in $\preceq$ for 
which the equality holds: $\infty\ 1\ 1\ 1\cdots$, for all the rest the inequality is strict. 
%For avery infinite steate $D$ and its successor $a'$, w ehave that $chi chi$. 
%The equality holds only for one element -- .
\end{lemma}
\subsubsection{Proof.}
%We have that $\cho{*\infty\ 1\ 1\ 1\cdots} =\frac{1}{2}$. 
%We also have that $S(\infty\ 1\ 1\ 1\cdots) = 2\ 2\ 1\ 1\cdots$ and that 
%$\cho{*2\ 2\ 1\ 1\cdots} = \frac{1}{2}$. 

For the state 
\begin{equation}
\infty d_2d_3d_4\cdots\end{equation} 
and its successor 
\begin{equation}
S(\infty d_2d_3d_4\cdots) = (d_2+1)(d_2+1)d_3\cdots,
\end{equation}
 we have that: 
\begin{align}
\cho{*\infty d_2d_3d_4\cdots} &- \cho{(d_2+1)(d_2+1)d_3d_4\cdots} = \notag \\
1 + \Delta(\infty d_2) + \Delta(d_3d_4\cdots) &- 
(1 + \Delta((d_2+1)(d_2+1)) + \Delta(d_3d_4\cdots)) = \notag \\ 
\Delta(\infty d_2) &- \Delta((d_2+1)(d_2+1)) = \notag \\
-\frac{1}{2}- \frac{d_2-1}{2d_2} &+ 2\frac{(d_2+1)-1}{2(d_2+1)} =  \\
\frac{-d_2(d_2+1) - (d_2-1)(d_2+1) + 2d_2^2}{2d_2(d_2+1)}& = 
\frac{-d_2 -d_2 + d_2 +1 }{2d_2(d_2+1)} = 
\frac{1- d_2}{2d_2(d_2 + 1)}. \notag
\end{align}
So the difference is not negative only for $d_2 = 1$ and for $d_2 = 1$ it is equal 
to $0$. $_\square$ 

\begin{lemma}\label{chi supp functoriality}
%Let $d_1$ be different form $\infty$. 
The supremum of the connected sequence 
of finite states 
\begin{equation}
\{(d_1+n)d_2d_3\cdots\}
\end{equation}
is
\begin{equation}
\infty d_2d_3\cdots
\end{equation}, and 
%the following diagram commutes:
the infimum of the corresponding sequence 
\begin{equation}
\{\cho{*(d_1+n)d_2d_3\cdots}\}\end{equation} 
is  
\begin{equation}
\cho{*\infty d_2d_3\cdots}.
\end{equation}
\end{lemma}
\subsubsection{Proof.} \label{chi supp functoriality proof}
%From \ref{good lexicographical order} we know, that every bounded sequence in $\preceq$ have 
%sup

For every $n$ we have that 
\begin{equation}
(d_1+n)d_2d_3\cdots\preceq\infty d_2d_3\cdots.
\end{equation} 
Furthermore for 
every 
\begin{equation}
d_1'd_2'd_3'\cdots\end{equation} such that 
\begin{equation}
d_1'd_2'd_3'\cdots\preceq \infty d_2d_3\cdots,
\end{equation} there 
exists $n$, such that 
\begin{equation}
d_1'd_2'd_3' \cdots\preceq(d_1+n)d_2d_3\cdots.
\end{equation} 
Thus, 
\begin{equation}
\infty d_2d_3\cdots
\end{equation} 
is the supremum of 
\begin{equation}
\{(d_1+n)d_2d_3\cdots\}.
\end{equation}

For every $n$ we have that: 
\begin{align}
\cho{*(d_1+n)d_2d_3\cdots} &= \cho{*d_1d_2d_3\cdots} 
- \frac{(d_1+n)-1}{2(d_1+n)} + \frac{d_1-1}{2d_1} \notag\\ 
&= \cho{*d_1d_2d_3\cdots} - \frac{1}{2d_1} + \frac{1}{2(d_1+n)}.
\end{align}
We also have that:
\begin{align}
\cho{*\infty d_2d_3\cdots} &= \cho{*d_1d_2d_3\cdots} 
- \frac{1}{2} + \frac{d_1-1}{2d_1} \notag\\ 
&= \cho{*d_1d_2d_3\cdots} - \frac{1}{2d_1} + 0.
\end{align}
Thus $\cho{*\infty d_2d_3\cdots}$ is the infimum of $\{\cho{*(d_1+n)d_2d_3\cdots}\}$.
$_\square$
\begin{observation}
We have that for $d_n \neq \infty$:
\begin{equation}
\cho{\infty\infty\cdots\infty d_n d_{n+1} d_{n+2}\cdots} > 
\cho{\infty\infty\cdots\infty (d_n+1) d_{n+1} d_{n+2}\cdots}. 
\end{equation}
As increasing the counter increases corresponding \Eoc.
\end{observation}

\begin{lemma}\label{chi supp functoriality non finite}
%Let $d_1$ be different form $\infty$. 
The supremum of the sequence 
of states 
\begin{equation}
\{\infty\infty\cdots\infty (d_n+m) d_{n+1} d_{n+2}\cdots\}_m
\end{equation} 
is 
\begin{equation}
\infty\infty\cdots\infty \infty d_{n+1} d_{n+2}\cdots,
\end{equation} and 
%the following diagram commutes:
the infimum of the corresponding sequence 
\begin{equation}
\{\cho{*\infty\infty\cdots\infty (d_n+m) d_{n+1} d_{n+2}\cdots}\}_m
\end{equation} is  
\begin{equation}
\cho{*\infty\infty\cdots\infty \infty d_{n+1} d_{n+2}\cdots}.
\end{equation}
\end{lemma}
\subsubsection{Proof.}
The proof will be analogous to \ref{chi supp functoriality proof}

For every $m$ we have that 
\begin{equation}
\infty\infty\cdots\infty(d_n+m)d_{n+1}d_{n+2}\cdots\preceq
\infty\infty\cdots\infty \infty d_{n+1} d_{n+2}\cdots.
\end{equation} 
Furthermore for 
every $d_1'd_2'd_3'\cdots$ such that 
\begin{equation}
d_1'd_2'd_3'\cdots\preceq \infty\infty\cdots\infty \infty d_{n+1} d_{n+2}\cdots,
\end{equation} there 
exists $m$, such that 
\begin{equation}
d_1'd_2'd_3' \cdots\preceq\infty\infty\cdots\infty(d_n+m)d_{n+1}d_{n+2}\cdots.
\end{equation} 

Thus, 
\begin{equation}
\infty\infty\cdots\infty \infty d_{n+1} d_{n+2}\cdots
\end{equation} 
is the supremum of 
\begin{equation}
\{\infty\infty\cdots\infty (d_n+m) d_{n+1} d_{n+2}\cdots\}_m.
\end{equation}

For every $m$ we have that: 
\begin{align}
\cho{*\infty\infty\cdots\infty (d_n+m) d_{n+1} d_{n+2}\cdots} &= \\ 
\cho{*\infty\infty\cdots\infty d_n d_{n+1} d_{n+2}\cdots} 
&- \frac{(d_n+m)-1}{2(d_n+m)} + \frac{d_n-1}{2d_n} = \notag \\ 
\cho{*\infty\infty\cdots\infty d_n d_{n+1} d_{n+2}\cdots} &- 
\frac{1}{2d_n} + \frac{1}{2(d_n+m)}
\end{align}
We also have that:
\begin{align}
\cho{*\infty\infty\cdots\infty \infty d_{n+1} d_{n+2}\cdots} &= \\
\cho{*\infty\infty\cdots\infty d_n d_{n+1} d_{n+2}\cdots} 
&- \frac{1}{2} + \frac{d_n-1}{2d_n} = \notag \\ 
\cho{*\infty\infty\cdots\infty d_n d_{n+1} d_{n+2}\cdots} &- \frac{1}{2d_n} + 0.
\end{align}
Thus 
\begin{equation}
\cho{*\infty\infty\cdots\infty \infty d_{n+1} d_{n+2}\cdots}
\end{equation} 
is the infimum of 
\begin{equation}
\{\infty\infty\cdots\infty (d_n+m) d_{n+1} d_{n+2}\cdots\}_m. _\square
\end{equation}
\begin{lemma}
The state of the counters in the algorithm is weakly increasing with respect to order $\preceq$. 
\end{lemma}
\subsubsection{Proof.}
The state of the counters is changed only in lines 15, 33-34, 64-65. In each of these lines 
the counter with the greatest index of all changed counters increases in value, so 
the resulting state is bigger with respect to order $\preceq$. $_\square$

\subsection{Basic idea}\label{basic idea}
The basic idea of the algorithm is to search through all the states of the counters going 
from the smallest (in the sense of $\preceq$) state of counters, which will be when all counters 
are set to $1$, up to some upper limit beyond which we are sure that no configuration of 
counters will yield the \Eoc\ that we are looking for. 

Now we will go through several obstacles of how to do so and solutions for them, answering for 
example the 
questions how we go through all the states and what can be this upper limit. 

%More on the upper limit will be addre

%However this can't be done directly as there are infinite ascending sequences in $\preceq$. 
\subsection{Checking all the states}
This can't be done directly as there are infinite ascending sequences in $\preceq$. 
However, it can be done with some use of the properties we derived in the previous subsection.
\subsection{Checking infinite connected sequences in finitely many steps}
\label{searching idea connected}
We will now present the method how to check any infinite connected sequence for solutions 
in finite number of steps.

%However, by \ref{Successor lemma} we know, that for every ascending sequence $\{a_n\}$ 
%in $\preceq$, such 
%that for all $n$, we have that $a_{n+1}$ is the succesor of $a_n$, we have that the sequence 
%$\{\cho{*a_n}\}$ is strictly descending. 

First, we will perform a reduction from 
arbitrary infinite connected sequence to the infinite connected sequence of finite states. 

Let us observe, that, by \ref{finiteness of the successor}, 
there can be at most one infinite state in 
any connected sequence, and if it is present it must be the first one. If such state 
$D_0$ is present, 
we can check it whether $\chi(*D_0))$ is equal to $\frac{p}{q}$ or not (one step), 
and then all states that are left to be checked are finite and form infinite 
connected sequence of finite states, thus ending our reduction. 
%separately from the rest of the sequence 

As this from this point we will present a method or checking for solutions any 
infinite connected sequence of finite states.

First, let us observe that thanks to \ref{connected sequences corollary}, 
when we are searching through the infinite connected sequence of finite states in 
$\preceq$, once we get (without finding any solution) 
to the state $D_n$ for which $\cho{*D_n} < \frac{p}{q}$, we know 
that no state $D_m$ with $m>n$ can have $\cho{*D_m} = \frac{p}{q}$ 
and we can disregard whole sequence. 

There is, however, another problem, namely, that when we are searching through 
the infinite connected sequence of the finite state, 
%of consequtive 
%states of counters, 
initially, we don't now, whether there will be any state 
$D_k = (d_1+k)d_2d_3\cdots$ in it, that 
will have $\cho{*(d_1+k)d_2d_3\cdots} \leq \frac{p}{q}$. 
However, thanks to \ref{chi supp functoriality} 
we can check for this, by first comparing $\frac{p}{q}$ with $\chi(*\infty d_2d_3\cdots)$. 
Since from \ref{chi supp functoriality}, we have that 
$\chi(*\infty d_2d_3\cdots)$ is the infimum of $\{*(d_1+n)d_2d_3\cdots\}$, 
we have that if $\chi(*\infty d_2d_3\cdots) < \frac{p}{q}$, then
there must be state $(d_1+n)d_2d_3\cdots$ such that $\chi(*(d_1+n)d_2d_3\cdots) < \frac{p}{q}$, 
for some $n$ and we can proceed to look for it one by one through the sequence.

One case that is left, is when $\chi(*\infty d_2d_3\cdots) > \frac{p}{q}$, but then we can 
disregard the whole sequence right away, since 
$\chi(*\infty d_2d_3\cdots)$ is the infimum of $\{*(d_1+n)d_2d_3\cdots\}$.

\subsection{What after we checked infinite connected sequence?}
Let us suppose that we just checked the infinite connected sequence, together with its supremum. 
%From this we have the method to check any infinite connected sequence of finite states in finitely 
%many steps.

%THere are two options -- either inf > pq or not. 

The supremum is of the form $\infty d_2 d_3d_4\cdots$.
Then, trying to perform \ref{basic idea}, 
we continue with the successor $S(\infty d_2 d_3d_4\cdots) = (d_2+1)(d_2+1)d_3d_4\cdots$ 
(\ref{form of the successor}). Provided the successor is not our solution, 
there are two options:
\begin{enumerate} 
\item $\cho{*(d_2+1)(d_2+1)d_3d_4\cdots} > \frac{p}{q}$,
\item $\cho{*(d_2+1)(d_2+1)d_3d_4\cdots} < \frac{p}{q}$.
\end{enumerate} 
%We will start with 
%the case, where $\cho{*(d_2+1)(d_2+1)d_3d_4\cdots} > \frac{p}{q}$.

\subsection{Case when \texorpdfstring{$\cho{*(d_2+1)(d_2+1)d_3d_4\cdots} > \frac{p}{q}$}
{chi^orb(*(d_2+1)(d_2+1)d_3d_4... > p/q)}}\label{greater idea}

%There are two options -- either it is smaller or larger. 
%we will strat with the case of being larger. 
%We run ourselves into nother ttrouble ...
%another 
%so instead we  first check whether any finite value hhas chnce of being  yeah yeah
We could start checking through the connected sequence starting at 
\begin{equation}
(d_2+1)(d_2+1)d_3d_4\cdots, 
\end{equation}
however, if 
\begin{equation}
\cho{*\infty(d_2+1)d_3d_4\cdots} > \frac{p}{q}, 
\end{equation}
we would end up in the same place that we are now, only with 
\begin{equation}
(d_2+2)(d_2+2)d_3d_4\cdots.
\end{equation}
Without further changes, this will lead to possibly checking one by one of infinitely many states 
of the form 
\begin{equation}\label{sequence second order}
\infty(d_2+n)d_3d_4\cdots. 
\end{equation}
We can solve this problem, by checking the state 
\begin{equation}
\infty\infty d_3d_4\cdots, 
\end{equation}
that have corresponding \Eoc\ lower than all of \ref{sequence second order}. 
If it will happen that 
\begin{equation}
\cho{*\infty\infty d_3d_4\cdots} > \frac{p}{q},
\end{equation}
we ruled out all states of the form
\begin{equation}
\infty(d_2+n)d_3d_4\cdots,
\end{equation}
and we can continue this pattern on further coordinates, checking: 
\begin{equation}
\infty\infty \cdots \infty d_k\cdots, 
\end{equation}
until we find some $n$, such that
\begin{equation}
\cho{*\infty\infty \cdots \infty d_{n+1}\cdots} < \frac{p}{q}.
\end{equation}
We always find $n$ like that, because at all time only finitely many counters are set 
to non-1 value, so from some point moving to next coordinate will result in 
comparing to $\frac{p}{q}$ the number $\frac{1}{2}$ smaller than from previous coordinate. 

Once we find such $n$, we need to perform actions described in \ref{searching mode idea}.  

%If it will happen that 
%\begin{eqaution}
%\cho{*\infty\infty d_3d_4\cdots} < \frac{p}{q},
%\end{eqaution}
%we can go to 
%\begin{equation}
%S(\infty\infty d_3d_4\cdots) = (d_3+1)(d_3+1)(d_3+1)d_4\cdots
%\end{equation}
%we can apply ganeralised behaviour described in 

\subsection{Case when \texorpdfstring{$\cho{*(d_2+1)(d_2+1)d_3d_4\cdots} < \frac{p}{q}$}
{chi^orb(*(d_2+1)(d_2+1)d_3d_4...) < p/q}}\label{Less idea}
%case a being smaller . This case is easy, if a state is smaller all states that are before are 
%also smaller, they have all counters higher, and we can go right to that one. 

%thas it. 
In this case, we know that every state $D$ such that:
\begin{equation}
(d_2+1)(d_2+1)d_3d_4d_5\cdots \preceq D\prec (d_3+1)(d_3+1)(d_3+1)d_4d_5\cdots
\end{equation}
have $\cho{D} < \frac{p}{q}$, since for any such state $D$, we have that 
counter $c_3$ and all to the right of it are the same as in 
\begin{equation}
(d_2+1)(d_2+1)d_3d_4d_5\cdots,
\end{equation}
but counters $c_1$ and $c_2$ are at least equal to $d_2+1$. For this reason, we can 
go to 
\begin{equation}\label{less hop two}
(d_3+1)(d_3+1)(d_3+1)d_4d_5\cdots, 
\end{equation}
as we ruled out all the sates smaller than \ref{less hop two}. Then, we can continue 
from this state.

This behaviour can be generalised -- whenever, in our algorithm we will have counters 
if the state 
\begin{equation}
(d_n+1)(d_n+1)\cdots (d_n + 1)d_{n+1}d_{n+2}d_{n+3}\cdots,
\end{equation} 
and we will know that 
\begin{equation}
\cho{*(d_n+1)(d_n+1)\cdots (d_n + 1)d_{n+1}d_{n+2}d_{n+3}\cdots} < \frac{p}{q},
\end{equation} 
we can rule out all the states up to (but not including) state:
\begin{equation}\label{less generalised}
(d_{n+1}+1)(d_{n+1}+1)\cdots (d_{n+1} + 1)(d_{n+1}+1)d_{n+2}d_{n+3}\cdots
\end{equation}
by the analogous reasoning as for $n = 2$ and continue from state \ref{less generalised}.

\subsection{Searching}\label{searching mode idea}
We are in the state, as described in \ref{greater idea}, that we found $n$, such that 
\begin{equation}
\cho{*\infty\infty \cdots \infty d_{n+1}\cdots} < \frac{p}{q}.
\end{equation}
The idea of algorithm at this point was, to rule out all the states that have 
corresponding \Eoc greater than $\frac{p}{q}$. We ruled out all smaller or equal to
(in the sense of $\preceq$) than 
\begin{equation}
\infty\infty \cdots \infty 1d_{n+1}\cdots.
\end{equation}
At this point, we can use a procedure analogous to the one from 
\ref{searching idea connected}, checking through the sequence (iterated with respect to $m$)
\begin{equation}
\infty\infty \cdots \infty md_{n+1}\cdots,
\end{equation}
that at some $m_0$ is guaranteed to have
\begin{equation}
\cho{\infty\infty \cdots \infty m_0d_{n+1}\cdots)} \leq \frac{p}{q},
\end{equation}
since 
\begin{equation}
\cho{\infty\infty \cdots \infty \infty d_{n+1}\cdots)} < \frac{p}{q}
\end{equation}
and we have that \ref{chi supp functoriality non finite}. 
%Let us suppouse that it is not a solution.

This way, we know that no state smaller or equal than 
\begin{equation}\label{searching state}
\infty\infty \cdots \infty (m_0-1) d_{n+1}\cdots
\end{equation}
is the solution. 
We know that $m_0 \geq 2$, since we know from  the procedure \ref{greater idea} that 
\begin{equation}
\cho{\infty\infty \cdots \infty 1 d_{n+1}\cdots)} > \frac{p}{q}
\end{equation}
We know proceed to check from the successor:
\begin{equation}
S(\infty\infty \cdots \infty (m_0-1) d_{n+1}\cdots) = m_0m_0\cdots m_0 m_0 d_{n+1}\cdots
\end{equation}
and 
up.

This presents the idea of the algorithm.

\subsection{Three "modes" of the algorithm}
The algorithm has three distinct fragments that coincide with 
the description of the idea above: 
\begin{itemize}
\item fragment in the lines 3-35. that will be called the "Greater" part, 
that corresponds to \ref{greater idea}
\item fragment in the lines 27-55, that will be called the "Searching" part, 
that corresponds to \ref{searching idea connected} and \ref{searching mode idea}
\item fragment in the lines 57-86, that will be called the "Less" part, that 
corresponds to \ref{Less idea}.
\end{itemize}
%The graph of the control flow of these parts looks like this:
%Above idea is structured among three parts of the algorithm 

The control flow of the parts can be seen on the diagram (numbers above arrow indicate lines):
\begin{figure}[H]
\centering
\includegraphics[width=\textwidth]{"../searching_for_a_spectrum/control_flow_2.jpg"}
\caption{Diagram of the control flow of the algorithm. Numbers above arrow indicate lines.}
\end{figure}
The execution of the algorithm then goes as follows:

We start at "Greater" and proceed to do the procedure from \ref{greater idea}. 
Once the procedure stops, we do procedure from \ref{searching mode idea}, then 
dependent whether the result have corresponding \Eoc\ greater or smaller than $\frac{p}{q}$, 
we perform, respectively -- again procedure from \ref{greater idea} or the 
procedure from \ref{less idea}. We repeat \ref{less idea} as long as necessarily. 
Once it gives the state that have \Eoc\ greater than $\frac{p}{q}$ we set the flag to 
"Greater" again and repeat the whole process starting from the procedure in \ref{greater idea}. 
In the case that repeating the procedure from \ref{Less idea} won't give any state with 
corresponding \Eoc\ greater than $\frac{p}{q}$ the algorithm will hit its stopping condition 
and answer "no" as written in the algorithm 
\ref{the algorithm for searching the spectrum itself} itself.   
% for searching the spectrum, that is
\section{Proof of the correctness of the algorithm}\label{memory state proof}
\subsection{Lemmas}\label{lemmas for the proof of the correctness}
Firstly, we will prove that our invariants indeed are conserved during the execution 
of the algorithm. We will also proof some other lemmas regarding the state of memory 
during the algorithm.

%-- THere are only finitely many couters non 1
\begin{lemma}
During any time of the execution of the algorithm, there are only finitely many counters 
that have non-$1$ value.
\end{lemma}
\subsubsection{Proof.}
We start with the state that have only finitely many non-$1$ value. Let us observe, that 
all three of the places -- lines: 15, 33-34, 64-65, 
where the counters are changed, change them in the way 
that preserves this state. As during any time of execution, there were only finitely many 
changes, we have the thesis. $_\square$

\begin{lemma}\label{Greater infinities to the left lemma}
When control is at the line 15. and the pivot is at the counter $c_p$, all counters 
to the left of $c_n$ are set to $\infty$. 
\end{lemma}
\subsubsection{Proof.}
We can get to the line 15th only in two ways: from line 18 in "Greater" section or from line 47 
in "Searching" section. 
In this process "Searching" moves pivot to the first counter and  
"Greater" moves pivot one counter to the right. 
As long as the pivot is not on the 1st counter, control flow must have came then to the line 15th 
from "Greater" section and if pivot is at the 1st counter it must have came from the 
"Searching" section. From this we have,  
that for the pivot, to get to the counter $c_p$, it would need to go through all 
the counters to the left of $c_p$ while being on the line 15th and 
setting them to $\infty$. 
\begin{lemma}\label{Searching infinities to the left lemma}
When control is at the line 33rd and the pivot is then at the counter $c_p$, all counters 
to the left of $c_p$ are set to $\infty$ and the counter $c_p$ is not set to $\infty$. 
\end{lemma}
\subsubsection{Proof.}
The only way to get to the line 33rd is from line 23rd in "Greater" section. 
From \ref{Greater infinities to the left lemma} we know, that then all the counters to the left 
of $c_p$ are set to $\infty$. Counter $c_p$ on the other hand can not be set to infinity, 
since, from the fact that control flow was at the block from line 21st, we know that
\begin{equation}
\cho{*d_1d_2\cdots d_{p-1}\infty d_{p+1}\cdots} < \frac{p}{q}
\end{equation}
and from the fact, that control flow was in the "Greater" block we know, that 
\begin{equation}
\cho{*d_1d_2d_3\cdots } > \frac{p}{q}._\square
\end{equation}

\begin{lemma}\label{state is ordered}
For any state of counters during the execution of the algorithm $D = d_1d_2d_3\cdots$ we have that 
$d_1 \geq d_2\geq d_3 \geq \cdots$.
\end{lemma}
%--order of counters is always decreasing
\subsubsection{Proof.}
We start with the state where $d_1 \geq d_2\geq d_3 \geq \cdots$. Let us observe, that, by 
\ref{Greater infinities to the left lemma}, changing at line 15 preserves this state. 
Changing at lines 33-34 or 64-65, preserve this state as they increase
the value of the counter at which pivot is by one (to some $d_p+1$) and change all counters 
to the left of the pivot to $d_p+1$. $_\square$

\begin{lemma}\label{Less same to the left lemma}
When control is at the line 66th and the pivot is then at the counter $c_p$, all counters 
to the left of $c_p$, are set to the same value $d_{p-1}$ and the counter $c_p$ 
is not set to the value $d_{p-1}$. 
\end{lemma}
\subsubsection{Proof.}
The only way for the control flow to get to the line 66th is from line 53th or 84th. 
In both of these cases, the counters to the left of $c_p$ were set to the same value on, 
respectively lines 33-34 or 64-65. Also, on lines 33-34 or 64-65, the value 
of the counter $c_{p-1}$ was increased by $1$. From this and 
from \ref{state is ordered}, we know, that $d_{p-1} > d_p$ 
%when we are at the line 66th
.$_\square$

\begin{lemma}\label{same value on the counters to the left}
All counters strictly to the left to the pivot have the same value 
%and 
%counter at the pivot have different value than these ones 
%at any of above lines. 
 at any stage of 
the execution of the algorithm.
\end{lemma}
\subsubsection{Proof.}
As state of the counters changes only on lines 15, 33-34 or 64-65, and 
the pivot is moving at most by one position to the right 
between the changes to the state of the counters, this is the 
corollary from \ref{Greater infinities to the left lemma}, 
\ref{Searching infinities to the left lemma} and \ref{Less same to the left lemma}. $_\square$

\begin{lemma}\label{Small searching always terminates}
Searching procedure from lines 29-32 always terminates.
\end{lemma}
\subsubsection{Proof.}
Let $c_p$ be the counter at which pivot is, when the searching procedure from lines 
29-32 stars. Control flow can get to the lines 29-32 only from line 23rd. This guarantees, that 
when starting the searching procedure, we have that:
\begin{equation}
\cho{*d_1d_2\cdots d_{p-1}\infty d_{p+1}\cdots} < \frac{p}{q}
\end{equation}
and  
\begin{equation}
\cho{*d_1d_2d_3\cdots } > \frac{p}{q}.
\end{equation}
From this and from \ref{chi supp functoriality non finite}, 
we know, that there exists some $d_p' < \infty$ such that 
\begin{equation}
\cho{*d_1d_2\cdots d_{p-1}d_p' d_{p+1}\cdots} < \frac{p}{q}
\end{equation}
As such, the searching procedure stops. $_\square$
\begin{lemma}\label{finitely many steps between counter changes}
There are always only finitely many steps in execution of the algorithm before it 
changes the state of the counters. 
\end{lemma}
\subsubsection{Proof.}
The only steps not explicitly listed in the algorithm are from the searching procedure from lines 
29-32. From \ref{Small searching always terminates} we know, that this procedure 
always terminates. 
All other control flow can be check explicitly to have always only finitely many steps between 
the change of the counters. 
The change of the counters itself is also a finite procedure, as we are always only changing 
the counters at pivot or at and to the left of the pivot, and there are only finitely many 
such counters. $_\square$
\begin{lemma}\label{always increases}
For any state $D_1$ that is a state of counters at some point of the execution of the algorithm 
and $D_2$ such that $D_1$ is changed to $D_2$ during the execution, we have that $D_1 \prec D_2$.
\end{lemma}
\subsubsection{Proof.}
Let us observe that in each instance of changing the counters -- in lines 15, 33-34 and 64-65. 
the rightmost counter that is changed is always increased. From this, the lemma follows. 
$_\square$
\subsection{Proof.}\label{proof of the correctness of the algorithm}
Now, we will perform the proof, that the idea of the algorithm presented above in 
\ref{the idea of the algoritm}, as 
well as the algorithm itself \ref{the algorithm for searching the spectrum itself}, 
works as intended.

Firstly, let us observe, that algorithm gives the answer only on lines 7-8, 37-38, 62, 68-69 and 
always ends immediately after giving the answer. Thus, it will always give at most one answer.
Furthermore let us observe that these are the only places where the algorithm terminates, 
so if it terminates it will give at least one answer.
 
There are three things to be checked: \\
$\bullet$ That the algorithm never answers "yes" if there is no orbifold of the \Eoc\ 
$\frac{p}{q}$ (No false positives)\\
$\bullet$ That the algorithm never answers "no" if there is an orbifold of \Eoc\ 
$\frac{p}{q}$ (No false negatives)\\ 
$\bullet$ That the algorithm always ends in a finite number of steps (Guaranteed termination). 

%To do this, we will introduce an order on the states of counters satisfying 
%\ref{valid state of counters}:



%\begin{lemma}

%\end{lemma}
%\subsubsection{Proof.}

%\begin{lemma}

%\end{lemma}
%\subsubsection{Proof.}

\subsection{No false positives}
Algorithm gives answer "yes" at lines 7-8, 37-38, 68-69. At each of these places, 
the answer contains the example of an orbifold with \Eoc\ equal to $\frac{p}{q}$ that was 
explicitly checked for correctness just before giving the answer (see lines 5, 35, 66). 
%$_\square$    
\subsection{No false negatives}
Let $D = d_1d_2d_3\cdots$ be such that $\cho{*d_1d_2d_3\cdots} = \frac{p}{q}$. 

%Let us assume for a contradiction, that algorithm started with $\frac{p}{q}$ answered "no". 
%
%We will show that this is impossible, by showing, that the algorithm will never 
%go beyond $d_1d_2\cdots$ in $\preceq$ order. 

First, we will show that the algorithm will never 
go beyond $d_1d_2d_3\cdots$ counter state in $\preceq$ order. 

Let us observe that the only lines where 
the counters are changed are lines 15, 33-34 and 64-65, 

Right before 
the change from 15 and right after each change from 33-34, 64-65 
(lines, respectively 5, 35, 66), 
the new state is checked if it is a solution and 
if it is a solution, the algorithm stops. From this, we 
have that going beyond $d_1d_2d_3\cdots$ can not happen from $d_1d_2d_3\cdots$, 
it must happen from some state $D' \preceq d_1d_2d_3\cdots$.

Furthermore while changing, 
only counters 
at pivot and to the left of the pivot are changed. 

Because of that:
\begin{observation}\label{position of pointer observation}
Going beyond $d_1d_2d_3\cdots$ counter state 
could happen only in lines 15, 33-34 or 64-65, while pivot would be 
%lovely
at the rightmost counter that is different from $d_1d_2d_3\cdots$ or further to the right. 
That is, if $d_1'd_2'd_3'\cdots$ is the current state, and 
$c_k$ is the rightmost counter on which $d_1d_2d_3\cdots$ 
and $d_1'd_2'd_3'\cdots$ differ and $c_n$ is the counter at which the pivot is on then 
at the moment of change that goes beyond $d_1d_2d_3\cdots$, it must hold that $k \leq n$.
\end{observation}
%This is because these are the only lines where the counters are changed and while changing, 
%only counters 
%at pivot and to the left of the pivot are changed. 

We will show that if the counters are below $D$ in $\preceq$ before the change they still 
be below $D$ or at $D$ after the change. 

We will now eliminate all three options arising from lines 15, 33-34 and 64-65 case by case. 
Let $D' = d_1'd_2'd_3'\cdots$ be current state of counters before the change 
during the execution of the algorithm. 
Let $c_k$ be the rightmost counter on which $d_1d_2d_3\cdots$ 
and $d_1'd_2'd_3'\cdots$ differ. Let $c_n$ be the counter at which the pivot is on. 
\subsubsection{Line 15}
%For the sake of contradiction, let us assume, that 
We will show that under taken assumptions, we will not get to this line. 
First, let us prove, that the pivot must be exactly at the counter $c_k$. 
By \ref{Greater infinities to the left lemma} we know, that if the pivot is 
on the counter $c_n$, while execution is at line 15, 
all the counters to the left of $c_n$ are set to $\infty$. 

This however means, that
\begin{equation}
\underbrace{\infty\infty\cdots\infty}_{n-1\ \mathrm{times}}d_nd{n+1}d_{n+2}\cdots \preceq D.
\end{equation}
Together with the fact, that for $k$, which is $\leq n$, we have that 
for any $l > k$, we have that $d_l = d_l'$ this means that if $n \geq k+1$, we have that 
$D = D'$. This is a contradiction with the assumption of $k$. 
From this we have that $n = k$.

We have then that pivot is on the rightmost counter $c_k$, 
that differs between $D$ and $D'$, and 
that all the counters strictly to the left of $c_k$ in $D'$ are set to $\infty$. 

%From the assumption that $D' \preceq D$ we know, that $d_k' < d_k$. 

From this, since $\cho{*d_1d_2d_3\cdots} = \frac{p}{q}$, we have, that: 
\begin{equation}
\cho{*d_1'd_2'd_3'\cdots d_{k-1}'\infty d_{k+1}'\cdots} \leq \frac{p}{q} 
\end{equation}

From this, we have the contradiction with the assumption that we will get to the line 15. 

\subsubsection{Lines 33-34}
Similarly as in the previous case, this time using \ref{Searching infinities to the left lemma} 
we can prove, that $n=k$, as well as that $d_k' < d_k$, that all 
counters strictly to the left of $c_k$ in $D'$ also must be $\infty$.

This leads us to the conclusion that searching procedure from lines 
29-32 will find some $ d_k'' \leq d_k$ as a result. Then, the resulting state 
still will be $\preceq D$. 

\subsubsection{Lines 64-65}
We will show the contradiction by showing that pivot must be on 
the counter strictly to the left of $c_k$. 
%We will show that $n = k$. 
By \ref{Less same to the left lemma} 
we know, that all the counters to the left of $c_n$ are set to $d_{n-1}'$. 
From this, from the fact that $d_k' < d_k$ and 
from the fact that $\cho{*d_1d_2d_3\cdots} = \frac{p}{q}$, 
we have that, unless $n < k$, we have that 
\begin{equation}
\cho{*d_1'd_2'd_3'\cdots} \geq \frac{p}{q}.
\end{equation}
However, this is a contradiction with being at lines 64-65, as 
the only way to get there with the execution, is either from lines 
51-53 or the lines 82-84, reaching either require for the state of counters to have 
corresponding \Eoc\ $<\frac{p}{q}$.

% We can get to line 64 only if the configuration (all of the pivot to 
%the left the same ) ha \Eoc smaller thsn pq. However, this has no smaller . 

\subsection{Guaranteed termination}
First let us observe, that 
Let 
\begin{equation}
_\infty D = \underbrace{\infty\ \infty\ \infty\cdots 
\infty\ }_{\lfloor 2(1-\frac{p}{q})\rfloor\ \mathrm{times}}1\ 1\ 1\cdots
\end{equation}
and let 
\begin{equation}
_2D = \underbrace{2\ 2\ 2\cdots 
2\ }_{\lceil 4(1-\frac{p}{q})\rceil\ \mathrm{times}}1\ 1\ 1\cdots.
\end{equation}
These are the state consisting of only, respectively, $\infty$ and $2$ as a non-$1$ counters 
values, that have the property, that they have the most non-$1$ counters from all 
states of such form, having the corresponding \Eoc\ $\geq\frac{p}{q}$. 

Let us assume that for some input $M$ and $\frac{p}{q}$ the algorithm does not answer "yes". 
We will show, that then it will answer "no" in finite number of steps. By 
\ref{finitely many steps between counter changes} 
we can show this, by showing, that it will answer "no" after 
finitely many of counters state changes. 
\subsubsection{Going beyond every state smaller or equal than $_2D$}\label{going beyond}
We will show it by firstly showing, that for any state $D = d_1d_2d_3\cdots$ smaller 
or equal to  
$D_2$ the algorithm will go to it or beyond it.
%\begin{enumerate}
%    \item if the pivot is on the zeroth column ($c = 0$), then 
%    \begin{enumerate}[label*=\arabic*.]
%        \item 
%        \item
%    \end{enumerate}
%    \item abcd
%\end{enumerate}
%\section{Improvements}
%We will prove that for any state different than 222222 it will go beyond it.

Proof will be inductive, with respect to order $\preceq$. Our inductive assumption will be, that 
for a given state $D$, that is at most $D_2$, 
%all the states $D'\prec D$ 
%have 
there is some state $D'$, such 
that $D\preceq D''$, and that $D'$ was the state of the counters after a finite 
number of steps of execution of the algorithm. 
%at some point. 

$\bullet$ For the fist state we have that this is true since it is the $D'$ for itself.
%for all the previous ones.   

$\bullet$ Let us suppose that for a state $D = d_1d_2d_3\cdots \preceq D_2$, for  
all the states $\prec D$ our inductive assumption holds. We will show that it holds 
for $D$. 
 
We will have two cases. 
\begin{itemize}
\item $D$ is a successor of some $D_1$. 
There are two options -- either $D_1$ was a state of counters 
at some point of the execution, or it wasn't. 

If it wasn't, from the inductive assumption we know 
that some state equal at least $D$ was the state of counters. From this we have the induction 
thesis for $D$.

In the case $D_1$ was a state of counters, from \ref{always increases} we have, that the next 
state of counters will be equal at least $D$.  
%there are three cases
%, depending on 
%which lines will be the change from $D_1$ to the next state in the execution. 
%\begin{itemize}
%\item Line 15 

%in this case the state is chenged 
%\item Line 


%\item Line

% 
%\end{itemize}
%here there will be 222222 as an exeption


\item $D$ is not a successor of any $D' \prec D$. From this, we have that $D$ has at least one 
$\infty$ on its counters.  
Let $c_l$ be the counter that has rightmost infinity in $D$. 
From \ref{} we have, that all the counters to the left of $c_l$ also are set to $\infty$.

Let us consider the state $^0D = ^0d_1^0d_2^0d_3'\cdots$, such that for any $i > l$, we have that 
$^0d_i = d_i$. From the induction assumption, there exists state $D' = d_1'd_2'd_3'\cdots$ 
such that $^0D\preceq D'$ and such that $D'$ was a state of the counters after finitely 
many steps of the algorithm. If $D' \succeq D$, then we have induction thesis for $D$. 
Let us consider the case, where $D' \prec D$. Then, since $^0D \preceq D' \prec D$, we have 
that for every $i > l$, we have $d_i' = d_i$. From this, we have, that $D'$ differs from 
$D$ only on the counters at, or to the left of $c_l$. We know, that all the counters 
to the left of, or at $c_l$
are set to $\infty$ in $D$ and that there is at least one counter set to $\infty$. 
Moreover, we know, that $D'$ differ from $D$ on at least one counter.

There are two cases:
\begin{itemize}
\item $\cho{D} \geq \frac{p}{q}$ 
From this, we conclude 
that $\cho{D'} > \frac{p}{q}$. 

Regardless of at which line state $D'$ was produced, the execution will then follow, 
through line  or to the line    . From this point, since $\cho{D} \geq \frac{p}{q}$, we will 
have consecutive sequence length at most $l$, of states of counters, where 
values of consecutive counters 
from $D'$ will be 
replaced by $\infty$, up to the counter $c_l$ at which state $D$ will be reached. 
This sequence will be generated by repeatedly going through line 15 in the algorithm, since 
at every step, the corresponding \Eoc of the state of counters will be $> \frac{p}{q}$.

\item $\cho{D} < \frac{p}{q}$
From \ref{chi supp functoriality non finite}, we know, that there must exists state 
\begin{equation}^0D = ^0d_1^0d_2^0d_3'\cdots,
\end{equation} 
such that 
all $^0d_i$ are less than $\infty$ and are the same for 
any $i \leq l$, and that $\cho{^0D} < \frac{p}{q}$. 

From this, we conclude, that $\cho{D'} < \frac{p}{q}$. 
Regardless of at which line state $D'$ was produced, the execution will then follow, 
through line 53 or 84 to the lines 64-65. From this point, since $\cho{D'} < \frac{p}{q}$, 
we will 
have consecutive sequence length at most $l$, of states of counters, where 
values of consecutive counters 
from $D'$ will be increased by one and all the counters to the left of them will be set to 
the value of the counter which was just increased. 
This sequence will be generated by repeatedly going through lines 64-65 in the algorithm, since 
at every step, the corresponding \Eoc of the state of counters will be $< \frac{p}{q}$.

This sequence will end by going with the pivot to the counter $c_{l+1}$, which we 
know from the assumption that has value smaller than $\infty$, increasing it by one 
and setting all the counters to the left of if to value $d_{l+1}+1$, this way obtaining 
the state larger in $\preceq$ than $D$. 
\end{itemize}

%From the inductive assumption, we know that for any state $D' \preceq D$.
%Let us take an ascending sequence of ${D_n}$, such that   
%Let take the smallest ones for each. if any of them 
%are greater than D we are done, so let us suppouse that they are all smaller. 

%This needs to be a finite set, since there was only finitely many states as a stets of counters. 
%This means that, since for every non infty one there are some ddadada there is an upper limit 
%dadada. THere is one that has $infty$ at this . Since they have the same to the right as D 
%it must be 
%greater or equal than D,  
\end{itemize}
\subsubsection{Reaching $_2D$ at finite number of steps}
Now we will show that at some point, algorithm will have as a state of counters $_2D$. 

\begin{lemma}\label{incorporating counters}
Every counter with number greater than $\lfloor 2(1-\frac{p}{q})\rfloor\ \mathrm{times}$ 
has pivot on it for the first time while the execution is in line 
\end{lemma}
%Let us observe that 
\subsubsection{Proof.}


From this, we have, that every time there is a state of counters that for the first time 
involves counter with number greater than 
$\lfloor 2(1-\frac{p}{q})\rfloor\ \mathrm{times}$, it is the state 
of the form $2\ 2\ 2\cdots\ 2\ 1\ 1\ 1\cdots$. 
Since from \ref{going beyond} 
we know that for every state $D \preceq _2D$, 
we will have some bigger state $D'$, that at some point was a state of counters, we 
have that at some point counter $c_{\lceil 4(1-\frac{p}{q})\rceil}$ will 
obtain non-zero value (if the predecessor of $_2 D$ will be the state of counters 
at some point, then $c_{\lceil 4(1-\frac{p}{q})\rceil}$ will have non-$1$ value at the 
next change of counters.). 
%We have 
%some state $_\infty D' \succeq _\infty D$. In the next change of the conters, state of 
%the counters will have non-$1$ value at the counter $\lfloor 2(1-\frac{p}{q})\rfloor + 1$.
From \ref{incorporating counters}, we know, that then there must have been state $_2D$ 
at some point as a state of counters. 
%The smallest biggest state involves additiona lcounter 
%and by what we shouwed then there will be 222222. 
$_\square$





\section{Another questions the algorithm can answer}
\subsection{Deciding the order of accumulation}
Using above algorithm, for a point $\frac{p}{q}$ and a manifold $M$, we can check 
what order of accumulation point $\frac{p}{q}$ is in $\spebr{M}$. 

Analogously to \ref{algorithm reduction to D2}, we will wlog answer the question fr $ M = D^2$. 

First, we check with the algorithm, whether $\frac{p}{q} \in \speD$. If not, it is not 
an accumulation point of $\speD$, since \ref{accumulation_points_of_the_set}.

Let us now consider the case, where $\frac{p}{q} \in \speD$.

From \ref{predescors}, we know, that for a point $\frac{p}{q}$ being an accumulation 
point of the set $\speD$, of order at least $n$ is equivalent to the fact that 
$\frac{p}{q} + \frac{n}{2} \in \speD$. From \ref{spectrum lesser than chi} 
on the other hand, we know, that 
for any $x > 1$, we have $x \not\in \speD$.

%Let us assume, that $\frac{p}{q} \in D^2$
  
As such, for a given $\frac{p}{q}$, we can, using our algorithm, check one by one 
every number of the form $\frac{p}{q} + \frac{n}{2}$, such that $n \in \mathbb{N}_0$ and 
$\frac{p}{q} + \frac{n}{2} < 1$. There are at most 
$\lfloor 2 \left(1 -\frac{p}{q} \right)\rfloor$ such numbers. 
Let $n_0$, be the biggest of the checked numbers, for which 
$\frac{p}{q} + \frac{n}{2} \in \speD$. 
Then, from \ref{predescors} 
we know, that $\frac{p}{q}$ is an accumulation point of order $n_0$ in $\speD$. 

 

%\todo{poprawić}
%Let $m \in \mathbb{N}$ be such that $\frac{p}{q} \in (1-\frac{m}{2},1-
%\frac{m+1}{2})$
%Let us denote by $r \coloneqq \frac{p}{q} - (1-\frac{m}{2})$. \\ 

%We will searching in $\sigma$ as such: \\

%If $\frac{p}{q} \in \sigma$, then, from the corollary \ref{predescors} we know, that there 
%exist some $n \in \mathbb{N}$, such that $\frac{p}{q} + \frac{n}{2} \in \sigma$ but 
%$\frac{p}{q} + \frac{n}{2} \not\in \sigma$. \\

%We will be consequently checking points from $1+r$, through $1+r-\frac{l}{2}$, for 
%$0 \leq l \leq m$, to the $\frac{p}{q}$. We stop at the first found point. 
%If one of these point is in the spectrum, then all smaller (so also $\frac{p}{q}$) are in 
%the spectrum and $\frac{p}{q}$ is the accumulation point of the spectrum of order $m-l$ 
%(from this, 
%we can see some heuristic, that the points that have smaller order will be generally 
%harder to find in some sense). If none of this points are in in the spectrum, then $\frac{p}{q}$ 
%is not. \\

\section{Implementation}
This algorithm is a part of the algorithm from chapter \ref{Counting orbifolds -- arithmetical part} 
where the implementation of the whole will be discussed in \ref{implementation}.

../LaTeX/TeX_files/counting_orbifolds.tex
% mainfile: ../praca_magisterska_orbifoldy.tex
\chapter{Counting orbifolds -- arithmetical part}\label{Counting orbifolds -- arithmetical part}


%\section{Arithmetical part}

%We want to determine this $n$. If $n = 0$, then $\frac{p}{q}$ is not in $\sigma$. 
%If $n > 0$, then $$
%\subsection{Deciding number of occurences}
%Searching for all occurences 

%The difficulty here is to carefully step other an occurence. 

%Compared to the previous version, we also use an occurance counter, starting with it set to 0 
%and with the list of orbifolds, wich is empty at the start.
\section{The idea of the algorithm}
This is an extention of the algorithm from \ref{}. It only differs by 
lines after finding the solution. Then, instead terminating it appends the sollution to the 
initialy empty list and proceeds to search through the states as if the solution was 
the starting configuration. 

\begin{lstlisting}[firstnumber=1,consecutivenumbers=true]
In the case, the $flag\_value$ is equal to: 
{
    "Greater", then
    {
        If $\chi^{orb}(*d_1\dots d_{p-1}\infty d_{p+1}\dots)=\frac{p}{q}$ then
        {
            We found an orbifold, we add it to a list 
            and increase the occurence counter by 1. 
            We set the flag to "Less".
            We put pivot to the $c_{p+1}$ counter.
            We go to the 1st line.
        } 
        If $\chi^{orb}(*d_1\dots d_{p-1}\infty d_{p+1}\dots)>\frac{p}{q}$ then
        {
            We set $d_p$ to $\infty$.
            We set the flag to "Greater".
            We put the pivot at the $c_{p+1}$.
            We go to the 1st line.
        }  
        If $\chi^{orb}(*d_1\dots d_{p-1}\infty d_{p+1}\dots)<\frac{p}{q}$ then
        {
            We set the flag to "Searching".
            We go to the 1st line.
        }  
    }
    
    "Searching", then
    {
        We search one by one 
        for the value $d_p'$ of the $c_p$ such that 
        $\chi^{orb}(*d_1\dots d_{p-1}d_p'd_{p+1}\dots)\leq\frac{p}{q}$ and 
        $\chi^{orb}(*d_1\dots d_{p-1}(d_p'-1)d_{p+1}\dots)>\frac{p}{q}$.
        We set $c_p$ and all of the counters 
        to the left of $c_p$ to the value $d_p'$.
        if $\chi^{orb}(*d_1d_2d_3\dots)=\frac{p}{q}$ then 
        {
            We found an orbifold, we add it to a list 
            and increase the occurence counter by 1. 
            We set the flag to "Less".
            We put the pivot at the $c_{p+1}$.
            We go to the 1st line.
        }
        If $\chi^{orb}(*d_1d_2d_3\dots)<\frac{p}{q}$ then 
        {
            We set the flag to "Less".
            We put the pivot at the $c_{p+1}$.
            We go to the 1st line.
        }
        If $\chi^{orb}(*d_1d_2d_3\dots)>\frac{p}{q}$ then 
        {
            We set the flag to "Greater".
            We put the pivot at the $c_1$.
            We go to the 1st line.
        }
    }
    
    "Less", then 
    {
        If $d_p = 1$ and the values of all the counters 
        on the left of $c_p$ are equal to 2 then 
        {
            We end the whole algorithm with the answer "no".
        }
        We increase $c_p$ by one ($d_p \coloneqq d_p + 1$) and
        we set the value of all counters on the left of $c_p$ to $d_p$.
        If $\chi^{orb}(*d_1d_2d_3\dots)=\frac{p}{q}$ then
        {
            We found an orbifold, we add it to a list 
            and increase the occurence counter by 1. 
            We set the flag to "Less".
            We put pivot at the $c_{p+1}$.
            We go to the line 1..
        }
        If $\chi^{orb}(*d_1d_2d_3\dots)>\frac{p}{q}$ then  
        {
            We set the flag to "Greater".
            We put the pivot at the $c_1$. 
            We go to the 1st line.
        } 
        If $\chi^{orb}(*d_1d_2d_3\dots)<\frac{p}{q}$ then
        {
            We set the flag to "Less".
            We put pivot at the $c_{p+1}$.
            We go to the 1st line.
        } 
    }
}
\end{lstlisting}

\section{Proof of the correctness of the algorithm}
Let us observe, that whole proof from the chapter \ref{} was independent from the choise of 
the starting configuration. As such, since algorithm from this chapter is the repeting iteration 
of the algorithm from chapter \ref{} and since \ref{}, 
above algorithm will hold all nesseserly traits. $_\square$

\section{Implementation}\label{implementation}
As an appendix in the separate document, there is a source of a program with implementation 
of this algorithm with optimisation described below.
%with full  
%enhancments described in this chapter. 
It is written in Rust. 
It can be also found on \smalltodoII{dać ref do github} along with the \LaTeX\ source of 
this thesis.
%It is in the separate file, as it would take too much space in this 
%document and wouldn't be readable. 

\subsection{Optimisations}
Binary search

\subsection{Limitations}
i64


% mainfile: ../praca_magisterska_orbifoldy.tex
\chapter{Counting orbifolds -- combinatorical part}\label{Counting orbifolds -- combinatorical part}
We will go through the steps described in \ref{combinatorical part}.

We will form our answer in a form of answering the 
question for a given number $\frac{P}{q}$ and a given manifold $M$. 
As stated in \ref{combinatorical part} at the end all answers for all $M$ needs to be summed, 
and we know that for $M$ such that $\chi(M) < \frac{p}{q}$, the answer is $0$, so there 
are always only finitely many answers to be summed. 

\section{Manifolds without boundary}
In this case we ask our algorithm from \ref{Counting orbifolds -- arithmetical part}: 
"How many sums of the form
\begin{equation}
1-\sum_{i=1}^m \frac{d_j-1}{2d_j} 
\end{equation}
are equal to 
\begin{equation}
\frac{1}{2}\frac{p}{q} - \frac{1}{2}\chi(M) -1\ ?",
\end{equation}
and the result is our final answer, as (\ref{sameness}) the list of degrees of rotational orbipoints ordered 
in descending order uniquely defines a two dimensional $M$-orbifold without boundary. 
\section{Manifolds with boundary}
\subsection{Using chapter \ref{Counting orbifolds -- arithmetical part}}\label{use algorithm step}
We ask our algorithm from chapter \ref{Counting orbifolds -- arithmetical part} for the list 
of all possible sums of the form
\begin{equation}
1-\sum_{i=1}^m \frac{d_j-1}{2d_j} 
\end{equation}
that are equal to 
\begin{equation}
\frac{p}{q} - \chi(M) - 1. 
\end{equation}
As we know from: 
\begin{itemize}
\item \ref{second_finiteness_theorem} 
\item the fact, that to each 
sum corresponds at least one $M$-orbifold
\item the fact, that to different sums corresponds different $M$-orbifolds
\end{itemize}
this list of sums will be finite. 
%\subsection{Fixed}
\subsection{Reduction to only dihedral orbipoints}\label{reduction only to dihedral}
As stated in \ref{additional things in boundary case -- rotational}, we need to first 
take into account that an orbifold can have both dihedral and rotational orbipoints. 

We have complete list of sums corresponding to degrees of dihedral orbipoints that result in 
the $\frac{p}{q}$ orbicharacteristic on $M$-orbifold. 

We are interested in having complete list of sums corresponding to degrees of both dihedral 
and rotational 
orbipoints that result in 
the $\frac{p}{q}$ orbicharacteristic on $M$-orbifold. 
We will now propose a unique  
reduction ($\ast$) of every list corresponding to both dihedral and 
rotational orbipoints to the list of only dihedral orbipoints. 

The reduction ($\ast$) goes as follows: for a list of degrees of orbipoints consisting of 
$r_1r_2\cdots r_n$ for rotational orbipoints and $d_1d_2\cdots d_m$ for dihedral orbipoints, 
we replace each $r_i$ from the list of rotational orbipoints by to entries of the same 
value on the list of dihedral orbipoints. This does not change the 
corresponding \Eoc, since $\Delta(n) = \Delta(^*n^*n)$ (as stated in \ref{}). 

This procedure is unambiguous and gives only one possible list of dihedral orbipoints degrees 
for every list of both rotational and dihedral orbipoints degrees. 

Based on this, for a given sum $d_1d_2\cdots d_n$,
we can perform the transformation of replacing $n$ by $^*n^*n$ in another direction to 
produce all possible sums consisting of both rotational and dihedral degrees, that 
would be reduced to sum $d_1d_2\cdots d_n$. The uniqueness of reduction ($\ast$) guarantees, that 
we won't arrive to the same unreducted sum from different starting lists of only dihedral degrees.

Based on this, after getting the list of sums from \ref{use algorithm step}, we need to 
add to this list all possible sums that could be reducted by ($\ast$) to some 
of the sums that we got from \ref{use algorithm step}. 

Sums on the new extended list have also rotational orbipoints taken into account. 
At this point we have full list of sums resulting in \Eoc\ equal to $\frac{p}{q}$ on a 
$M$-orbifolds. 
In further considerations we will not explicitly rotational orbipoints degrees as they play 
no role in combinatorics.

\subsection{Fixed sum}

Let us now consider a case with a manifold $M$, and a sum $d_1d_2\cdots d_n$ (as we wrote 
at the end of the previous section we are not writing rotational degrees as 
they will play no role from this point, however, they are possibly present in some sums). 
Then the last step of the procedure will be to sum over all sums produced in 
\ref{reduction only to dihedral}. 



Let $b$ be the number of boundary components of $M$. We know from the assumption, that $b > 0$. 

%Let the sum be of the form $a_1^{k_1}a_2^{k_2}\cdots a_m^{k_m}$, meaning that the 
%list $d_1d_2\cdots d_n$ looks like (let $l_i \coloneqq \sum_{j = 1}^i k_j$):
%\begin{equation}
%\underbrace{d_1d_2\cdots d_{l_1}}_{k_1\ \mathrm{times}\ a_1} 
%\underbrace{d_{l_1+1}d_{l_1+2}\cdots d_{l_2}}_{k_2\ \mathrm{times}\ a_2} \cdots 
%\underbrace{d_{l_{m-1}}d_{l_{m-1}+1}\cdots d_n}_{k_m\ \mathrm{times}\ a_m}. 
%\end{equation}

%We need to partition $a_1^{k_1}a_2^{k_2}\cdots a_m^{k_m}$ among boundary 
%components of $M$. 

We need to partition $d_1d_2\cdots d_n$ among boundary 
components of $M$. 

%In considerations of partitions, we will treat 
%At first we will give each of the boundary component a label and treat them as distinguishable. 
%At the end, we will divide the result by $b!$ reflecting the fact, that topologically, they are 
%not 
%distinguishable. 

At this moment we will treat boundary components as distinguishable. 

%Effective way to iterate over all the partitions 
Let us consider some partition of $d_1d_2\cdots d_n$ among the 
distinguishable  boundary 
components. After this, results from all partitions need to be summed together. 
The fact that boundary components are not distinguishable 
will be taken into account in the next subsection by assigning the proper weights in the summation.
%, but in the 
%case that $M$ is orientable,
%results from partitions where at least . 



\subsection{Fixed partition}
Let us now consider some fixed partition.
% and one boundary component. 
Suppose that in this partition for every $1\leq j \leq b$
boundary component $B_j$ have orbipoints of degrees: 
%$^ja_1^{^js_1}\ ^ja_2^{^js_2}\cdots\ ^ja_{n_j}^{^js_{n_j}}$
$^jd_1\ ^jd_2\cdots ^jd_{n_j}$.  
We want to know how many possible sets of cyclic orders there are on a boundary components with 
$^jd_1\ ^jd_2\cdots ^jd_{n_j}$ on $B_i$. 
%Having this answer we will multiply these answers from each boundary component in 
%a given partition
%%, this way resulting in an answer for a partition. 
%However  

%First, we will start with answearing the question when all $s_i$ are equal to $1$. 
%Then:
%\begin{itemize}
%\item if $M$ is orientable, possible number of orders is the number 
%of linear orders -- $n!$, divided by 
%the number of cyclic permutations -- $n$, giving the answear $(n-1)!$. 
%\item  if $M$ is non orientable and $n \geq 3$, possible number of orders is the number 
%of linear orders -- $n!$, divided by the number of elements in a dihedral group $D_n$, 
%that is -- $2n$, resultng in an answer $\frac{(n-1)!}{2}$.
%\item if $M$ is non orientable and $n < 3$, then the answear is the same as in orientable case 
%and there exists only one cyclic order that is inveriant to reflections.  
%%as cyclic orders of one ore two elements are invariant to reflections. 
%\end{itemize} 

%Then 

%\subsubsection{Fixed cyclic order for the type of the components}
%We can enumerate through all the cyclic orders by iterating though linear orders
We can count sets of cyclic orders un untistinguishable boundary components 
by iterating through tuples of 
linear orders on distinguishable boundary components and summing them with proper weights. 

Given the tuple of linear orders $\mathcal{L} = (L_1, \cdots, L_b)$ on
distinguishable components, to calculate 
the weight $W(\mathcal{L})$, we will first set some weights $W(L_j)$, for every 
$1\leq j \leq b$. 
%Then we will put $W(L) \coloneqq \prod_{i=1}^bL_i$.  

%We 
%, but counting each 
For a linear 
order $L_j$ we set the 
weight $W(L_j)$ to be 
%(let $t \coloneqq \sum_{i = 0}^n s_i$)
:  
%where $\Gamma$ is (let $t \coloneqq \sum_{i = 0}^n s_i$): 
\begin{itemize}
\item in case $M$ is orientable -- $\frac{|\mathbb{Z}_k|}{|\mathbb{Z}_{n_j}|} = \frac{k}{n_j}$, 
where $\mathbb{Z}_k$ 
is the biggest cyclic subgroup of $\mathbb{Z}_{n_j}$, under which $L_j$ 
is invariant as a linear order,
\item in case $M$ is not orientable -- $\frac{|\mathbb{D}_k|}{|\mathbb{D}_{n_j}|} = 
\frac{2k}{2n_j} 
=\frac{k}{n_j}$ , 
where $\mathbb{D}_k$ is 
the biggest dihedral 
subgroup of $\mathbb{D}_{n_j}$, under 
which $L_j$ is invariant as a linear order. 
\end{itemize}
%After this, we need to divide 
%First, let us answear this question for $b = 1$. 

Then, we put $W(L) \coloneqq SR\prod_{j=1}^bL_j$, where:
\begin{itemize}
\item $S \coloneqq \frac{|G|}{|S_b|}$, where $G$ is the biggest subgroup of permutations 
$S_b$ under which 
tuple of orders $\mathcal{L}$ is invariant as a tuple of cyclic orders, 
\item $R = \frac{1}{2}$ if
$M$ is orientable and  
at least one of linear orders $L_j$ is different as a cyclic order than the 
reverse of $L_j$ as a cyclic order; 
otherwise $R = 1$
\end{itemize}


%Let us start with only cosidering dihedral orbipoints.

%Let us first calculate number of possible divisions of $a_1^{k_1}a_2^{k_2}\cdots a_m^{k_m}$ 
%among boundarie components. If we have $k_i$ copies of the orbipoint of degree $a_i$, 
%the problem of counting the possible distributions of them among $b$ boundary is 
%equivalent to the problem of dividing $k_i$ elemnent set into $b$ distinguishable compartemets. 
%This is in turn equivalent to a problem of placing $b-1$ 
%undistinguishable objects into $k_i+1$ distinguishable compartements 
%(each of this compartemett is a places of divisions of 
%$k_i$ element 
%set into the next of consequtive $b$ groups). This in turn is equal to     

%\section{Counting orbifolds -- combinatorical part summary}
\subsection{Comment about possibility of a single equation}
Although, given enough effort, results from this section could be summarise in one 
equation consisting only 
of $b$ and $k_1, k_2, \cdots, k_n$ for a given sum, 
we feel that it would be long enough not to give any new 
insight into the structure of the problem. We are stopping thus at giving the above procedure.
%what is an algorithm if not very elaborate equation after all?



%../LaTeX/TeX_files/power_series_and_generating_functions.tex
%../connection_with_modular_forms/connection_with_modular_forms.tex
% mainfile: ../praca_magisterska_orbifoldy.tex
\chapter{Conclusions}

\section{What was done}
In \ref{reduction_to_arithmetical} we proved that $\speD \not\subseteq \speS$ and 
$\speS \not\subseteq \speD$. 

In chapter \ref{order structure}, among other things, we described the spectrum of possible \Eoc s 
of two dimensional orbifolds
and, as the result, the spectrum of all possible areas 
of two dimensional hyperbolic orbifolds in a ordinal and topological manner. 
It has order type and topology (induced from $\mathbb{R}$) of $\omega^\omega$. 
We also proved, that every accumulation point of $\speS$ is in $\speD$.
%and the problem, whether the given point is in the 
%spectrum is desidable. \\

In chapter \ref{Searching the spectrum} 
we provided algorithm for deciding for a given number $x$, whether 
there exists an orbifold $O$, such that $\cho{O} = x$ and proved its correctness.

In chapter \ref{counting occurrences} we provided some finiteness results, such as that 
there are always only finitely many 
orbifolds for a given \Eoc. 
%So the problem how much are for a given number is also decidable.\\
We also proved that for every $n$, in every neighbourhood of every accumulation point 
of $\spe$ of order at least $2$, there is at least one number $x$, such that there are at least 
$n$ orbifolds such that $\cho{O} = x$.

In chapter \ref{Counting orbifolds -- arithmetical part} 
and chapter \ref{Counting orbifolds -- combinatorical part} 
we provided an algorithm 
 for counting for a given number $x$ number of orbifold such that $\cho{O} = x$,
and proved its correctness. 
 
%of this algorithm and discussed, 
We also discussed that its complexity is low enough for actual implementation 
and practical usage on a reasonably small denominators and reasonably close to zero.
%From \smalltodo{referencja} we know, that there are howe ver, blab la dowolnie dużo na \Eoc. \\



%for every denominator, do they coincide 
%from a sufficiently distatn point? (Yes.) 

\section{Further directions}
It remains unclear how Disk spectrum and Sphere spectrum lies relative to each other. 
In particular we still don't know, whether they coincide from a sufficiently distant point.
%but some result was, shown, namely, that every accumulation point of Sphere spectum
%is also in the disk spectrum.

%\subsection{Asked, but unanswered questions}
%Our ultimate goal would to give the answer to the questions such as: \\
%- For a given $x \in \sigma$, how many orbifolds have $x$ as their \Eoc?\\
We don't really know why there is exactly "this" many orbifolds for a given \Eoc? 
%Giving only the algorithm gave little 
We would like to know, whether there is some underlying geometrical reason for that?

We would like to somehow characterise points $x \in \sigma$ that has "the most" 
orbifolds corresponding to them. With reasonable normalisation of what it means for a number
to have "more" orbifolds as we go to lesser values of \Eoc. 

%- Is there any reasonable normalisation to counter the effect that there are 'more' 
%points as we go 
%to lesser values. (What we mean by 'more' was sted in) \\
%The first equstion we can tackle is steaming from the chapter \ref{order structure} 
%and it is -- Do $\speD$ and $\speS$ coincide? It is easy to answer that $\speD \neq \speS$ 
%(and we will do that along some harder questions in the moment), but do they coincide 
%starting from a sufficiently distant point? 
%Or maybe, for every denominator, do they coincide 
%from a sufficiently distatn point? (Yes.) \\
%\todo{przenieść (przekopiować?) część może do futher direction}
%write about cyclic order
%\subsection{Unasked and unanswered questions}
%\subsection{Power series and generating functions}
%\subsection{Seifert manifolds}


%% mainfile: ../praca_magisterska_orbifoldy.tex
\chapter{Further directions}
\section{Power series and generating functions}
\section{Seifert manifolds}

%% mainfile: ../praca_magisterska_orbifoldy.tex
\chapter{test}
abcd

% mainfile: ../praca_magisterska_orbifoldy.tex
\nocite{*}
\bibliography{../bibliography/bibliography.bib}{}
\bibliographystyle{plain}

%~\cite{Conway2002}
\appendix
%../LaTeX/TeX_files/appendix_the_implementation_of_the_searching_algorithm.tex
../LaTeX/TeX_files/appendix_order_homeomorphism_theorem.tex
%../LaTeX/TeX_files/graphs_of_the_control_flows_of_the_algorithms.tex
\end{document}



