\documentclass[a4paper, 12pt]{report}
\synctex=1

% mainfile: ./praca_magisterska_orbifoldy.tex
%\usepackage{pdfpages}
%\includepdf[<key=val>]{}

\usepackage{polski}
\usepackage[T1]{fontenc}
\usepackage{amssymb}
\usepackage[polish,english]{babel}
\usepackage[utf8]{inputenc}
\usepackage{mathtools}
\usepackage{microtype}
\usepackage{amsthm}
\usepackage{amsmath}
\usepackage{amsfonts}
%\usepackage{amsrefs}
\usepackage{empheq}
\usepackage[pdftex,
            pdfauthor={Bartosz Sójka},
            pdftitle={Orbifoldy},
        %    pdfsubject={Orbifoldy}
            ]{hyperref}
\usepackage{siunitx}
\usepackage{geometry}
\usepackage{lipsum}
\usepackage{marvosym}
%\usepackage{enumitem}
%\usepackage{alltt}
\usepackage{listings}
%\usepackage[eulermath]{classicthesis}
\usepackage{xcolor}
%\usepackage[nottoc,numbib]{tocbibind}
\usepackage[
backend=biber,
style=alphabetic,
sorting=ynt
]{biblatex}
\addbibresource{../bibliography/bibliography.bib}

%\setmathfont{eulermath}
%\usepackage{pifont}
%\setlength\parindent{0pt}

%\topmargin = -1in

\geometry{a4paper, 
twoside,
%asymmetric,
top = 25mm,
bottom = 40mm,
inner = 35mm,
outer = 25mm
}

\lstset{numbers=left, numberstyle=\small, stepnumber=1, numbersep=5pt, mathescape=true}

%\linespread{1.24}

%top = 25mm,
%bottom = 30mm,
%inner = 30mm,
%outer = 25mm

\newtheorem{theorem}{Theorem}[subsection]
\newtheorem{observation}[theorem]{Observation}
\newtheorem{definition}[theorem]{Definition}
\newtheorem{lemma}[theorem]{Lemma}
\newtheorem{corollary}[theorem]{Corollary}
\newtheorem{claim}[theorem]{Claim}

\numberwithin{equation}{theorem}

\newcommand{\todo}[1]{\hfill \break \textbf{\Huge \textcolor{violet}{TO DO: #1} \hfill \break}
\normalsize}
\newcommand{\content}[1]{\hfill \break \textbf{\large \textcolor{violet}{#1} \hfill \break}
\normalsize}
\newcommand{\smalltodo}[1]{\textbf{\ \textcolor{violet}{To do}}}
\newcommand{\smalltodoII}[1]{\hfill \break \textbf{\ \textcolor{violet}{To do: #1}}\hfill \break}
\newcommand{\missingpicture}[1]{\hfill \break\\[16pt] \Huge \textbf{\textcolor{violet}
{Missing picture \normalsize #1}} \hfill
\break \\[16pt] \normalsize}

\newcommand{\lrb}{\textnormal{\big[}}
\newcommand{\rrb}{\textnormal{\big]}}

\newcommand{\rba}[1]{\lrb #1\rrb}

\newcommand{\lcb}{\textnormal{\Big(}}
\newcommand{\rcb}{\textnormal{\Big)}}

\newcommand{\cba}[1]{\lcb #1\rcb}

\newcommand{\ds}{\textnormal{\big/\!\!\big/}}
\newcommand{\lds}{\textnormal{\big/\!\!\big/\!}}
\newcommand{\rds}{\!\textnormal{\big/\!\!\big/}}

\newcommand{\dsa}[1]{\ds #1\ds}





%\newcommand{\beginproof}{{\noindent\textbf{Proof.}\\}}
%\newcommand{\endproof}{{$_\square$}}

\renewcommand{\labelenumii}{\theenumii}
\renewcommand{\theenumii}{\theenumi.\arabic{enumii}.}


\newcommand{\cho}[1]{{\chi^{orb}(#1)}}

\newcommand{\srS}{{\sigma^r(S^2)}}
\newcommand{\sdD}{{\sigma^d(D^2)}}

\newcommand{\spe}{{\sigma}}
\newcommand{\speS}{{\spe(S^2)}}
\newcommand{\speD}{{\spe(D^2)}}
\newcommand{\spebr}[1]{{\spe(#1)}}


\newcommand{\Eoc}{Euler orbicharacteristic} 
\newcommand{\apots}{accumulation point of the set}


%\title{Orbifoldy}
\title{Areas of two dimensional hyperbolic orbifolds}
\author{Bartosz Sójka}
%\date{}
\begin{document}
% mainfile: ../main/praca_magisterska_orbifoldy.tex
\newpage
\thispagestyle{empty}
\begin{center}
\textbf{\large Uniwersytet Wrocławski\\
Wydział Matematyki i Informatyki\\
Instytut Matematyczny}\\
\textit{\large specjalność: teoretyczna}\\
\vspace{4cm}
\textbf{\textit{\large Bartosz Sójka}\\
\vspace{0.5cm}
{\Large Two dimentional orbifolds' volumes' spectrum}}\\
\end{center}
\vspace{3cm}
{\large \hspace*{6.5cm}Praca magisterska\\
\hspace*{6.5cm}napisana pod kierunkiem\\
\hspace*{6.5cm}prof. dr hab. Tadeusza Januszkiewicza }\\
\vfill
\begin{center}
{\large Wrocław 2021}\\
\end{center}

\newpage
\null
\thispagestyle{empty}

%% mainfile: ./praca_magisterska_orbifoldy.tex
\newpage
\thispagestyle{empty}
\vspace*{19cm}
\hspace*{10cm}
\textit{dla Wujka}
\newpage
\null
\thispagestyle{empty}

\newpage
\thispagestyle{empty}
\vspace*{19cm}
\hspace*{10cm}
\textit{dla Wujka}
\newpage
\null
\thispagestyle{empty}

\newpage
\tableofcontents
\begin{abstract}
\setcounter{page}{7}
%\begin{center}
%Orbifolds! Yeah! \\
%Spectrums! Yeah! 
%\end{center}
%Orbifoldy
This thesis aims to describe the spectrum of all possible areas of 
two dimensional orbifold, in particular those of negative \Eoc. 
We will analyse the spectrum both as a set and study its order type and topology 
as well as as a preimage of $\chi^{orb}$ -- \Eoc\ and count orbifolds corresponding 
to a particular points to a spectrum. 
\end{abstract}
% mainfile: ../praca_magisterska_orbifoldy.tex
\chapter{Introduction}
\setcounter{page}{9}
\section{Motivations}
Orbifols are geometrical spaces that encodes some of groups action properties. 

They played a crucial role in Thurstons geometrisation prgram leading to very important 
results in geomerty
%Quotiens by a groups
They are also correlated and find use in many different subjects, 
from geometry to physics and signal analysis, and art -- se e esher art analysis -- 
differnt patterns of fresks and os on,freski.

They are also beatiful symmetrical structers on they own, provideing 
nice and neat  and uniform language to describe platonic solids, parkietarze of the euklidean 
and hyperbolid plane, asher pictures 
as we ll as general notion of symmety

%Praca o ujemnym ale z najwyzszą orbicharakterystyką

The partcular focus of this thesis is on ppossible volumes of orbifodls.
Such analyssis was performed for example in dimension three, where 
cite cite cite.

\todo{dopisać}

\section{Scope}
In this thesis we would like to give some description of the spectrum of the volumes 
of two dimensional orbifolds, in particlar thouse of negative \Eoc. 
We will examine order type and topology of the spectrum \ref{order structure} 
as well as the structure 
of the spectrum based on spectra corresponding to different manifolds \ref{spectra}, 
\ref{reduction_to_arithmetical}.
We will also provide tools for determining whether a given number is in spectrum of volumes 
\ref{Searching the spectrum} as 
well as tools to compute how many orbifolds correspond to a given volume in the spectrum 
\ref{counting occurrences}. 

\section{Technical introductions}
Now, we will proceed to give technical introductions about orbifolds, \Eoc\ and the 
technics we will use in this thesis, alongside with some definitions and notation and 
naming conventions. 
 
\section{Orbifolds}
%\todo{jak sie juz wszystko zbierze co ma tu być, to to dopisać}
\subsection{Definition}
The definition of the orbifold is taken from Thurston \cite{Thurston1979} (chapter 13), 
with slight modification described in \ref{compactness}. 
We briefly recall the concept, but for full discussion we refer to \cite{Thurston1979}. 

An orbifold is a generalisation of a manifold. As manifold, it consists of a Hausdorf space 
(which we will call a base space)
with some additional structure. 
Compared to manifolds, one allows more variety of local behaviour. 
On a manifold a map is a homeomorphism between $\mathbb{R}^n$ and some open set on a manifold. 
On an orbifold a map is a homeomorphism between a quotient of $\mathbb{R}^n$ by some 
finite group and some open set on an orbifold. 
In addition to that, the orbifold structure consist the informations about that finite group 
and a quotient map for any such open set. 

We can make an observation, that since in dimension 2, quatient of $\mathbb{R}^2$ by a finite 
group is topologically always either $\mathbb{R}^2$ or $\mathbb{R}\times\mathbb{R}_{\geq 0}$, 
we have that in dimension 2, the underlying Hausdorf space of any orbifold is a topological 
manifold (possibly with a boundary). For an orbifold $O$ we will call this underlying manifold $M$
a base manifold of $O$ and we will call $O$ an $M$-orbifold.  

\subsection{Good and bad orbifolds}
Above definition says that an orbifold is locally homeomorphic do the quotient of $\mathbb{R}^n$ 
by some finite group. 

When an orbifold as a whole is quotient of some finite group acting on a manifold we say, that 
it is 'good'. Otherwise we say, that it is 'bad'. 

%For this subsection we will also adopting notation from \cite{Thurston1979}. 

In two dimentions there are only four types of bad orbifolds, namely
(adopting notation from \cite{Thurston1979}): 
\begin{itemize}
\item $S^2(n)$ 
\item $D^2(;n)$ 
\item $S^2(n_1,n_2)$ for $n_1 < n_2$ 
\item $D^2(;n_1,n_2)$ for $n_1 < n_2$. 
\end{itemize}
All other orbifolds are good -- \cite{Thurston1979} (theorem 13.3.6).
%As manifolds are special case of orbifolds with all ...
%Write about isomorphism of all spectra

%$M$ - orbifold

Manifolds without boundary can be treated as orbifolds with trivial group for every map and will. 
Manifolds with boundary can be treated as orbifolds with trivial group on all maps from the 
interior and with group $\mathbb{Z}_2$ on the boundary, as described in \cite{Thurston1979} 
(example 13.2.2.).

\subsection{Therminology}
We differ from Thurston in the terms of naming points with maps with non-trivial groups. 
We will call them orbipoints. If the group acts as the group of rotations (so a 
cyclic group) we will call them rotational points. If the group is a dihedreal group we will 
call them 
dihedreal points. And if the point is on the boundry that stabilises reflection we will call it a 
reflection point. 

If a group associated to the orbipoint has degree $n$, we will say that tha orbipoint is 
of degree $n$. 
\subsection{Finite number of orbipoints}\label{finite number of orbipoints}
In this thesis we will consider only orbifolds with finitely many orbipoints and all orbifolds 
mentioned from this point are meant as such without further notice. Reason for this 
choice will be described in \ref{Eoc}.  

\subsection{Compactness}\label{compactness}
Orbifold as a topological space is the same as its base manifold.
We, would like to restrict our interest only to compact orbifolds. 
However, noncompact orbifolds, such as ones from \cite{Conway2016} (chapter 18) 
which are quotients 
of a group action of an infinite group acting on $\mathbb{R}^2$ (that will 
be also frequently interpreted in this setting as a hyperbolic plane $\mathbb{H}^2$), 
also interests us. We would like to accomodate some of noncompact orbifolds that 
will satisfy the condition simmilar to the \ref{finite number of orbipoints}. 
To do this we will slightly expand our definition of an orbifold. 

Let us start with a following construction.

For an noncompact orbifold $O$, let us take it's one point compactification. 
Let the compactification point be named $x_O$.  
For some set $X$, let $\#(X)$ be the number of connected components of $X$.
Now let us consider some connected open set $U \ni x_O$. The set $U\setminus\{x_O\}$ is 
not neccesserly connected.
 
Let 
\begin{equation}
C(O) \coloneqq \sup\limits_{\substack{U \ni x_O \\ \rm connected, \\ \rm open }} 
\#(U\setminus\{x_O\}).
\end{equation}
We will be interested only in the case, when $C(O)$ is finite.
If $C(O)$ is finite, we take some $U$ that realise the supremum and compactify each of 
connected components of $U\setminus\{x_O\}$ with a separate point. We will call these points 
"cusps". If for some cusp $x$, in every $U \ni x$ there are points from the boundary of $O$, 
we will call it a cusp on a boundary. In the other case, we will call it a cusp in the interior.

The result is a compact topological space. We will treat it as an orbifold, with 
cusps as orbipoits in extended definition. 
We extend the definition as such, that the map from the compactification of 
some open subset consisting point $x$ can go to the quationt of compactification of 
$\mathbb{H}^2$ by $S^1$. The group, we will take to act on $\mathbb{H}^2$ will 
be:
\begin{itemize} 
\item in the case of $x$ being in the interior -- infinite cyclic group $\mathbb{Z}$, 
where the generator acts as translation by 1 
in the half-plane model of hyperbolic plane   
\item in the case of $x$ being on the boundary -- infinite dihedral group $D_\infty$, where 
generators will be reflections with respect to vertical lines spaced by 1 in a half-plane 
model on a hyperbolic plane.
\end{itemize}

Note that in both cases, the is exactly one point in the compactification of $\mathbb{H}$, that is 
fixed for every elemnt of the acting group -- point from the compactification at 
the infinity on the top of the plane. This point is will be always mapped to the orbipoint $x$. 
We will call $x$ an, respectively, rotational, or dihedral orbipoint of infinite degree.

%Further implications of this extention of the definition will be 
%the group associated with a map of 
%a compactification  of a cusp is $\mathbb{Z}$ acting 

%However, there 
%is a class of noncompact orbifolds that interests us and we would like 
%to accomodate them. Noncompact orbifolds interests us that are quotiens 
%of a group action of an infinite group acting on $\mathbb{R}^2$ (that will 
%be also frequently interpreted in this setting as a hyperbolic plane $\mathbb{H}^2$). 
%Examples can be found in \cite{Conway}.
%welcome it in our discussion.
%quotiens of groups 
%\subsection{Maps between orbifolds}
%Map between orbifolds can be defined as follows:
%%\cite{kleiner2014geometrization}
\subsection{Equivalence}\label{sameness}
As stated in \cite{Conway16} (chapter 18) two two-dimensional orbifolds are the same iff they have:
\begin{itemize}
\item the same base manifold, 
\item the same number of orbipoints for each type and degree 
\item orbipoints on the boundaries 
in the pairwise same cyclic orders (up to orientation if the base manifold is non-orientable).
\end{itemize}
%From this we can treat that and that orbifolds as the same.

%Without boundary -- uniquely determined by orbipoints list. 

%With boundary -- in general not. 

\section{Euler (orbi)characteristic}\label{E_orb}
\label{\Eoc_as_a_sum}
%\todo{dać cytowanie do charakterystyki}
%quotiens
\subsection{Euler characteristic}
On CW-complexes we can define Euler characteristic as additive topological invariant 
normalised on symplexes.

On a CW-complex it is then an alternating sum of numbers of cells in 
a consecutive dimentions: 
\begin{equation}
\chi(C) = \sum_{d = 0}^n (-1)^d k_d
\end{equation}

Among other properties we have that if a compact manifold $N$ is a quotient of a compact 
manifold $M$, 
by the action of a finite group $G$, that acts properly discontinuously and freely on $M$, then
\begin{equation}
\chi(N) = \frac{\chi(M)}{|G|}
\end{equation}   
%we will treat 
%as we will treat manifolds as orbifolds we will always refer 
%we will 

%from of \Eoc on two dim orbifolds\label{\Eoc on 2d}

\subsection{\Eoc}\label{Eoc}\label{extended_Euler_orbicharacteristic}
We would like to extend the definition of Euler characteristic to orbifolds in a way 
that will reflect on their structre. 
We will call the resulting additive topological invariant the Euler orbicharacteristic 
and denote it by $\chi^{\rm orb}$.

%For spaces where Euler characteristic is already defined \Eoc\ will be the same as 
%Euler characteristic. 

Following \cite{Thurston1979} (definition 13.3.3.), but extending 
definition to orbifolds with cusps, we will define it as follows:
\begin{definition}
When an orbifold $O$ has a cell-division of the base space $X$, such that for each
open cell the group associated to
the interior points of a cell is constant, then the Euler number $\cho{O}$ is defined by
the formula:
\begin{equation}
\cho{O} \coloneqq \sum_{c_i} (-1)^{ \mathrm{dim}(c_i)}\frac{1}{|\Gamma(c_i)|},
\end{equation}
where $c_i$ ranges over all cells and $|\Gamma(c_i)|$ is the order of the group $\Gamma(c_i)$ 
associated to each cell, if $c_i$ is a cusp and $|\Gamma(c_i)| = \infty$ we put 
$\frac{1}{|\Gamma(c_i)|} = 0$. We will call $\frac{1}{|\Gamma(c_i)|}$ a weight of a cell $c_i$.
\end{definition} 

This definition results in the propertie, that if an orbifold $O_2$ is a quotient 
of a orbifold $O_1$, by the action of the finite group $G$, that acts properly 
discontinuosly, but not neceserly freely on $O_1$, then:
\begin{equation}
\cho{O_2} = \frac{\chi(O_1)}{|G|}.
\end{equation}
For a two dimentional orbifolds, the possible cells with weights different than $1$ are 
only in dimensions $0$ and $1$. In dimention 0 they are rotational and dihedral 
orbipoints. In dimention 1 they are fragments of the boundary that stabilises reflections. 
The weights of these cells are (with the convention that $\frac{1}{\infty} = 0$):
\begin{itemize}
\item for a rotational point of degree $n$, the weight is $\frac{1}{n}$,
\item for a dihedral point of degree $n$, the weight is $\frac{1}{2n}$,
\item for a reflection line, the weight is $\frac{1}{2}$.
\end{itemize}
We can see that weights of rotational and dihedral orbipoints are monotonously decrising 
and converges to $0$, as degree diverges to infinity. Moreover, the 
cusps -- rbipoints of an inifinte degree, 
that are stabilised by a groups of infinite degree, has weight $0$.

From this we will obtain the formula for an Euler orbicharacteristic of a two dimensional 
orbifold with rotational points of degrees $r_1, r_2, \cdots, r_n$ and dihedral points 
of degrees $d_1, d_2, \cdots, d_m$:
\begin{align}\label{chi orb expression}
\cho{O} &= \chi (M) - n + \sum_{i=1}^n \frac{1}{r_i} - \frac{m}{2} + \sum_{j=1}^m \frac{1}{2d_j} \\
&= \chi (M) - \sum_{i=1}^n \frac{r_i-1}{r_i} - \sum_{j=1}^m \frac{d_j-1}{2d_j}
\end{align}
For $O$ with only rotational orbipoints:
\begin{equation}\label{chi orb expression rotational}
\cho{O} = \chi (M) - \sum_{i=1}^n \frac{r_i-1}{r_i}.
\end{equation}
For $O$ with only dihedral orbipoints:
\begin{equation}\label{chi orb expression dihedral}
\cho{O} = \chi (M) - \sum_{j=1}^m \frac{d_j-1}{2d_j}.
\end{equation}

From these formulas we can see, that as number of orbipoints diverges to infinity, the \Eoc\ 
diverges to minus infinity. For this reason, we restric ourselves only to orbifolds 
with finitely many orbipoints.
\begin{observation}\label{orbifolds have smaller Eoc than their base manifolds}
A $M$-orbifold that is different than $M$ always have strictly smaller \Eoc\ than $M$. 
\end{observation}

%\subsection{Extended Euler orbicharacteristic}\label{extended_Euler_orbicharacteristic} 
%%(with cusps)
%%Write about cusp as a limit.
%We can extent our definition of \Eoc\ to orbifolds with cusps. 

\section{Metric structures on the orbifolds and volumes of the orbifolds}
%\subsection{Metric structure on the orbifolds}
As described and proved in \cite{Thurston1979} (13.3.6.) all good orbifolds have either:
\begin{itemize}
\item Elliptic structure, if $\cho{O} > 0$,
\item Parabollic structure, if $\cho{O} = 0$,
\item Hyperbolic structure, if $\cho{O} < 0$.
\end{itemize}
With an elliptic or a hyperbollic metric structre, we can meassure the volume of the orbifold as 
\begin{equation}
V(O) = |\int_O K dA|
\end{equation} 
Also, as stated in \cite{Thurston1979} (13.3.5.), the Gauss-Bonnet theorem works also for orbifolds 
and we have that:
\begin{equation}
\int_O K dA = 2\pi|\cho{O}|,
\end{equation}
thus having that 
\begin{equation}\label{volume to chi}
V(O) = 2\pi|\cho{O}|.
\end{equation}
Our main goal in this thesis is to describe possible volumes of two dimensional orbifolds, 
especially thouse with negative \Eoc. As from \ref{volume to chi} we have direct correspondence 
between volumes and \Eoc s, and with \Eoc, we can restrict ourselves to rational numbers, 
we will try to describe possible \Eoc\ of two dimensional orbifolds.  

\section{Classification of two dimensional manifolds}\label{2 dim manifolds}
%\todo{twierdzenie o klasyfikacji powierzchni}
In this thesis we aim to better recognise possible two dimensional orbifolds. 
The fundament that we are relaying on is the classification of two dimensional manifolds. 

It is phrased as a well known theorem:
\cite{}


%\section{Classification of orbifolds}
%The list of all orbifolds with non-negative Euler orbicharacteristic
%%Powiedzieć coś o tym, że orbicharatkeryttyka odpowiada polom (Gauss Bonett itd.)
%\cite{}
%\subsection{Non-negative \Eoc}



\section{Therminology and notation}\label{therminology and notation}
\subsection{Orbifold notation}
For the rest of this thesis we will use slightly modified notation from \cite{Conway2016}.
%"feature"
%Write about manifold Euler characterestic
%We treat manifolds and orbifolds as a sphere with some features added by the operations.
As we know from \ref{sameness}, two dimentional orbifold can be defined 
by specifying:
\begin{itemize} 
\item its base manifold, 
\item the list of degrees of rotational points (we will 
always write them in the decreasing order) 
\item the list of the lists of dihedral points for each boundary component
\end{itemize}
Furthermore from \ref{2 dim manifolds} we know, two dimensional manifold 
can be 
defined, by specifying:
\begin{itemize}
\item how many boundary components it has,
\item how many handles it has,
\item how many cross-caps it has.
\end{itemize} 
\todo{dopisać}

When we will write $\Delta(feature)$ we would mean the difference in \Eoc

%dać na sferę $\varepsilon$ słowo puste.
%We will regard parts of that notation not only as features on an orbifold but also as 
%an operations 
%on orbifolds transforming one to another by adding particular feature. \\
We will denote the difference in Euler characteristic which is made by modifying 
an orbifold by such a feature as $\Delta(modification)$.
%\todo{rozwinąć} 
%dopisać, że w Conwayowej >= 2

\subsection{Expressions involving $\infty$}
If not stated othewise, in the expressions containing $\infty$ symbol, their value is understood 
as $\varphi(\infty) \coloneqq \lim\limits_{n\to \infty}\varphi(n)$. Only expressions 
where such limits exists will occour without futther notice.
%\todo{}

\subsection{Addition of sets and numbers}
For $A, B \subseteqq \mathbb{R}$, we define 
\begin{equation}
A+B \coloneqq \{a+b\ |\ a \in A, b\in B\}.
\end{equation}

\subsection{Two dimensionality}
From this point, throught the whole thesis we will consider only two dimensional manifolds and 
orbifolds, for this reason words "two-dimensional" will be sometimes ommited, nevertheless 
from this poinnt we will always mean only two dimensional manifolds and orbifolds.  

%delta
%h c b

%$\srS, \sdD$ ???

\section{Spectra}\label{spectra}
We will call the set of all possible \Eoc\ of a $M$-orbifolds, the spectrum of $M$ and 
we will denote it by $\spebr{M}$. We will denote the set of all possible \Eoc\ of a $M$-orbifolds 
that have only rotational orbipoints by $\spe^r(M)$. 
We would denote the set of all possible \Eoc\ of a $M$-orbifolds 
that have only dihedral orbipoints by $\spe^d(M)$, but from this section, it follows that 
we always have $\spe^d(M) = \spebr{M}$. 

We will also denote the sum of spectra of all two dimentional manifolds by:
\begin{equation}
\spe \coloneqq \bigcup_{M\textrm{ : 2d manifold}} \spebr{M}.
\end{equation}
This will be the main interest of this thesis. 
%$\spe$. dadada

Now we want to derive the form of the $\spebr{M}$.
For a manifold $M$ with $h$ handles, $c$ cross-cups and $b$ boundary components, it's 
Euler characteristic is given by:
\begin{equation}
\chi(M) = 2-2h-c-b.
\end{equation}
The set of $\Delta$ for possible orbifold features are:\\
$\bullet$ for $b\neq 0$:
\begin{equation}
\{-\frac{n-1}{2n}\ \big|\ n\in\mathbb{N}_{>0}\cup \{\infty\}\}
\end{equation}
$\bullet$ for $b = 0$:
\begin{equation}
\{-\frac{n-1}{n}\ \big|\ n\in\mathbb{N}_{>0}\cup \{\infty\}\}.
\end{equation} 
%with the convention, that $\varphi(\infty) \coloneqq \lim\limits_{k\to \infty} \varphi(k)$.

Thus, we have that the form of the spectrum of two dimensional manifold $M$ is:\\
$\bullet$ for $b\neq 0$: 
\begin{equation}
\spe(M) = \chi(M) - \left\{\sum\limits_{i=1}^n\frac{d_i-1}{2d_i}\ 
\big|\ n\in\mathbb{N}_0,\ d_i\in\mathbb{N}_{>0}\cup \{\infty\}\right\}
\end{equation}
$\bullet$ for $b = 0$:
\begin{equation}
\spe(M) = \chi(M) - \left\{\sum\limits_{i=1}^n \frac{r_i-1}{r_i}\ \big|\ n\in\mathbb{N}_0,\ 
r_i\in\mathbb{N}_{>0}\cup \{\infty\}\right\}.
\end{equation} 
\label{two dim manifold spectrum}


\begin{observation}\label{2times homeomorphism}
We have that $\speS = 2\speD$.
\end{observation}
\subsubsection{Proof.}

Indeed, since: 
\begin{align}
\spe(S^2) = 2 -\left\{\sum\limits_{i=1}^n \frac{r_i-1}{r_i}\ \big|\ n\in\mathbb{N}_0,\ 
r_i\in\mathbb{N}_{>0}\cup \{\infty\}\right\}
\end{align}
and 
\begin{align}
\spe(D^2) = 1 - \left\{\sum\limits_{j=1}^m\frac{d_j-1}{2d_j}\ 
\big|\ m\in\mathbb{N}_0,\ d_j\in\mathbb{N}_{>0}\cup \{\infty\}\right\}._\square
\end{align}

\begin{observation}\label{all_spectra_are_isomorphic}\label{spe_M}
For every two dimentional manifold $M$, we have that $\spe(M)$ is homeomorphic to $\speD$. 
%For $M$ with $h$ handles, $c$ cross-cups and $b$ boundary components, 
This homeomorphism is: \\
$\bullet$ for $b \neq 0$:
\begin{equation}
\spe(M) = \speD + \chi(M) - 1, 
\end{equation}
$\bullet$ for $b = 0$:
\begin{equation}
\spe(M) = 2\speD + \chi(M) - 2.
\end{equation}  
\end{observation}

%Ok, all isomorphic with $\speD$ \\
%here write about it \\
%Tell me about it!
\subsubsection{Proof.}
For a manifold $M$ with $h$ handles, $c$ cross-cups and $b$ boundary components, it's 
$\spebr{M}$ is given by:\\
$\bullet$ for $b\neq 0$: 
\begin{equation}
\spe(M) = \chi(M) - \left\{\sum\limits_{i=1}^n\frac{d_i-1}{2d_i}\ 
\big|\ n\in\mathbb{N}_0,\ d_i\in\mathbb{N}_{>0}\cup \{\infty\}\right\}
\end{equation}
$\bullet$ for $b = 0$:
\begin{equation}
\spe(M) = \chi(M) - \left\{\sum\limits_{i=1}^n \frac{r_i-1}{r_i}\ \big|\ n\in\mathbb{N}_0,\ 
r_i\in\mathbb{N}_{>0}\cup \{\infty\}\right\}.
\end{equation} 

On the other hand, we have that:
\begin{equation}
\speD = 1-\left\{\sum\limits_{i=1}^n\frac{d_i-1}{2d_i}\ 
\big|\ n\in\mathbb{N}_0,\ d_i\in\mathbb{N}_{>0}\cup \{\infty\}\right\}
\end{equation}
\begin{equation}
\speS = 2 - \left\{\sum\limits_{i=1}^n \frac{r_i-1}{r_i}\ \big|\ n\in\mathbb{N}_0,\ 
r_i\in\mathbb{N}_{>0}\cup \{\infty\}\right\}
\end{equation}
and
\begin{equation}
\speS = 2\speD.
\end{equation}
From this, the observation follows immedietly. $_\square$

\begin{observation}
For every manifold $M$, for every $x \in \spebr{M}$, we have that $x \leq \chi(M)$.
\end{observation}

%\todo{dopisać, że spectrum jest poniżejchi rozmaitości}

\section{Egyptian franctions}
Egyptian frantion is a finite sum of fractions, all with numerators one. 
Most of the time it is also required, that the fractions in the sum have pairwise distinct 
denominators. We will however take less usual convention and will drop that requirement, 
calling an egyptian fraction any sum of unitary fractions. 
%\section{Translating questions to ones about Egyptian fractions}\label{Egyptian_fractions}
\subsection{Connection between spectra and Egyptian fractions}\label{Egyptian_fractions}
The terms $-\frac{r_i-1}{r_i}$ in the sum \ref{chi orb expression rotational} 
can be expressed as $-1+ \frac{1}{r_i}$ 
and the term $-\frac{d_j-1}{2d_j}$ in the sum \ref{chi orb expression dihedral} can be expressed as 
$-\frac{1}{2} + \frac{1}{2d_j}$. 
Then the sums become:
\begin{equation}\label{Egyptian S2 sum}
\chi(M) - n + \underbrace{\sum_{i=1}^n \frac{1}{r_i}}_{
\substack{\textrm{Egyptian} \\ \textrm{fraction}}}
\end{equation}
and
\begin{equation}\label{Egyptian D2 sum}
\chi(M) - \frac{m}{2} + \frac{1}{2}
\underbrace{\sum_{j=1}^m \frac{1}{d_j}}_{
\substack{\textrm{Egyptian} \\\textrm{fraction}}}.
\end{equation}

%In expressions there are 
In this form, the egyptian fractions are explicitly present in expresions of 
points in $\spebr{M}$.
%This is very simmilar to the notion of the egyptian franction, 
%which is a rational number expressed as a sum of a fractions with numerator $1$ and 
%different denominators. Sometimes the "distinct denominators" condition is dropped and we are 
%following that convention in this thesis. 

The $-n$ and $-\frac{m}{2}$ terms provide constraints on the number of fractions that 
can apprear in the sum.
%The particular interest is in translating questions related to spectra to the questions 
%of egyptian franctions. 
%What the data of the spectra imposes is the number of fractions to be summed. 

%We are interested in translating 
We will now translate the questions of being in the spectrum 
to the questions of being expressable as egyptian fraction with the particular number 
of summands. It will be used in \ref{saturation theorem} and \ref{unboundness}.

%We can make following corollaries:
%\begin{corollary}\label{from Egyptian fractions}
%If $x$ can be expressed as an egyptian fraction with $n$ summands, then 
%\begin{equation}
%2 - n + x \in \speS
%\end{equation}
%and
%\begin{equation}
%1 - \frac{n}{2} + \frac{x}{2} \in \speD.
%\end{equation}
%\end{corollary}
%\begin{corollary}
%If $y \in \speS$ as an \Eoc of an orbifold which all orbipoints are $n$ rotational orbipoints, 
%then 
%\begin{equation}
%y + n - 2
%\end{equation} 
%can be expressed as an egyptian fraction with $n$ 
%(not nessecerely distincs) summands. 
%
%If $y \in \speD$ as an \Eoc of an orbifold which all orbipoints are $m$ rotational orbipoints, 
%then \begin{equation}
%2y + \frac{m}{2} - 1
%\end{equation}
%can be expressed as an egyptian fraction with $m$ 
%(not nessecerely distincs) summands. 
%\end{corollary}
%We may now expand our corollaries about egyptian fractions to arbitrary $M$:

We will now state two corrolaries that follows immediately from the 
form of expressions \ref{Egyptian S2 sum} and \ref{Egyptian D2 sum}, and from 
\ref{two dim manifold spectrum}.
\begin{corollary}\label{from Egyptian fractions}
If $x$ can be expressed as an egyptian fraction with $n$ summands, then for any two dimensional 
manifold $M$ we have: 
\begin{equation}
\chi(M) - n + x \in \spebr{M}
\end{equation}
and, if $M$ has at least one boundary component also:
\begin{equation}
\chi(M) - \frac{n}{2} + \frac{1}{2}x \in \spebr{M}.
\end{equation}
\end{corollary}
\begin{corollary}\label{to egyptian fractions}
If for some two dimensional manifold $M$ we have that $y \in \spebr{M}$ as an \Eoc\ of 
an orbifold which has $n$ rotational orbipoints and not any other, 
then 
\begin{equation}
y + n - \chi(M)
\end{equation} 
can be expressed as an egyptian fraction with $n$ 
(not nessecerely distincs) summands. 

If for some two dimensional manifold $M$ with at least one boundary component 
we have that $y \in \spebr{M}$ as an \Eoc of an orbifold which has $m$ dihedral orbipoints and 
not any other, 
then 
\begin{equation}
2y + \frac{m}{2} - 2\chi(M)
\end{equation}
can be expressed as an egyptian fraction with $m$ 
(not nessecerely distinct) summands. 
\end{corollary}

\section{Operations on orbifolds}\label{Operations}
As stated in \ref{therminology and notation} we will often see two dimentional orbifold 
as a sphere $S^2$ with a collection of features.
Throughout the thesis we will frequently refer to performing the "operation" on an orbifold 
consisting of removing and adding thouse features. 
What we mean by this is giving as a result of an operation on one orbifold, defined 
by some list of features on a sphere, another one 
with a modified list of features according to the described operation. 

When we will be talking about "adding" or "removing" a feature from an orbifold, we 
will mean adding or removing this feature from a list defining this orbifold and 
taking the orbifold defined by resultin list as a result of the operation.
 
As our main interest is to figure out, for a given rational number 
$\frac{p}{q}$ which orbifolds have $\frac{p}{q}$ as their \Eoc\ and for a given 
orbifold $O$, which orbifolds 
have the same \Eoc\ as $O$, we will be particulary interested in such operations 
that do not change \Eoc, which will be used in \ref{sufficiency of D2 and S2}.



%One of a usefull approuches will be to treat 
%Write about the general 
%operations we are interested in i.e. taking any number of features (handles 
%cross caps, parts of boundry components with orbipoints on it, orbipoints in the interior) and
%replacing it by any other feauters
%(Some preserve the area)
%Write about operations nesseserie for reduction of cases
%write that every operation reduces \Eoc.

%This does not have any particular deep geometrical meaning

\label{moving from interior to boundary}


\section{Questions asked}
There will be two main parts of question: 

$\bullet$ Ones regarding $\spe$ as a set, where we will be asking 
of its order type and topology and relation to other sets such as $\speD$ and $\speS$. 
We will focus on these questions in \ref{order structure}. 

$\bullet$ Ones regarding $\spe$ as an image of a $\chi^{orb}$, sending orbifolds to their \Eoc s. 
There, we will ask for example how namy orbifolds have particular \Eoc\ and 
related questions. We will focus on these questions in the chapter \ref{counting occurrences}.  

%Reduction presented in \ref{reduction_to_arithmetical} are with . 






%../LaTeX/TeX_files/definition_of_an_orbifold.tex
%% mainfile: ../praca_magisterska_orbifoldy.tex
\chapter{Definition, characteristics, classification and properties of the orbifolds}
\section{Definition}
\todo{jak sie juz wszystko zbierze co ma tu być, to to dopisać}
The definition of the orbifold is taken from Thurston \cite{Thurston1979} (chapter 13). 
We briefly recall the concept, but for full discussion we refer to \cite{Thurston1979}. \\
An orbifold is a generalisation of a manifold. One allows more variety of local behaviour. 
On a manifold a map is a homeomorphism between $\mathbb{R}^n$ and some open set on a manifold. 
On an orbifold a map is a homeomorphism between a quotient of $\mathbb{R}^n$ by some 
finite group and some open set on an orbifold. 
In addition to that, the orbifold structure consist the informations about that finite group 
and a quotient map for any such open set. \\

Above definition says that an orbifold is locally homeomorphic do the quotient of $\mathbb{R}^n$ 
by some finite group. \\
When an orbifold as a whole is quotient of some finite group acting on a manifold we say, that 
it is 'good'. Otherwise we say, that it is 'bad'. \\

We are also adopting notation from \cite{Thurston1979}. \\

In two dimentions there are only four types of bad orbifolds, namely: \\
- $S^2(n)$ \\
- $D^2(;n)$ \\
- $S^2(n_1,n_2)$ for $n_1 < n_2$ \\
- $D^2(;n_1,n_2)$ for $n_1 < n_2$. \\
All other orbifolds are good.
As manifolds are special case of orbifolds with all ...
We differ from Thurston in the terms of naming points with maps with non-trivial groups. 
We call them orbipoints. If the group acts as the group of rotations (so a 
cyclic group) we call them rotational points. If the group is a dihedreal group we call them 
dihedreal points. And if it is point on the boundry that stabilises relfecition it is a 
reflection point.
\cite{kleiner2014geometrization}
\section{Euler orbicharacteristic}\label{E_orb}
\label{\Eoc_as_a_sum}
we will treat 
as we will treat manifolds as orbifolds we will always refer 
we will 
\subsection{Classification of orbifolds with non-negative Euler orbicharacteristic}
The list of all orbifolds with non-negative Euler orbicharacteristic
Powiedzieć coś o tym, że orbicharatkeryttyka odpowiada polom (Gauss Bonett itd.)
\subsection{Extended Euler orbicharacteristic}\label{extended_Euler_orbicharacteristic} (with cusps)
Write about cusp as a limit.

Write about isomorphism of all spectra

$M$ - orbifold

\section{Uniformisation theorem (formulation)}
\todo{twierdzenie o klasyfikacji powierzchni}
\section{Operations and constructions on orbifolds}\label{Operations}
Write about the general operations we are interested in i.e. taking any number of features (handles 
cross caps, parts of boundry components with orbipoints on it, orbipoints in the interior) and
replacing it by any other feauters
(Some preserve the area)
Write about surgeries nesseserie for reduction of cases
\section{Notation}
We will regard parts of that notation not only as features on an orbifold but also as an operations 
on orbifolds transforming one to another by adding particular feature. \\
We will denote the difference in Euler characteristic which is made by modifying 
an orbifold by such a feature as $\Delta(modification)$.
\todo{rozwinąć} 
dopisać, że w Conwayowej >= 2

If not stated othewise, in the expressions containing $\infty$ symbol, their value is understood 
as $\varphi(\infty) \coloneqq \lim\limits_{n\to \infty}\varphi(n)$.



% mainfile: ../praca_magisterska_orbifoldy.tex
\chapter{Reduction to arithmetical questions}\label{reduction_to_arithmetical}
Reductions presented in this chapter will be more in the spirit of chapter \ref{order structure}, 
in the sense that 
%We will see (in the observation \ref{boils_down}) that the problem of determining this boils
% down to the 
% analysis of all 
%the possible 
%values of the expressions:
%\begin{equation}
%2 - \sum_{i=1}^n \frac{I_i-1}{I_i}
%\end{equation}
%and 
%\begin{equation}
%1 - \sum_{j=1}^m \frac{b_j-1}{2b_j},
%\end{equation}
%where $I_i, b_j$ varies over $\mathbb{N}_{>0} \cup \{\infty\}$. \\
%As
%\begin{equation}
%2 - \sum_{i=1}^n \frac{I_i-1}{I_i} = 2 - n + \sum_{i=1}^n \frac{1}{I_i}
%\end{equation}
%and 
%\begin{equation}
%1 - \sum_{j=1}^m \frac{b_j-1}{2b_j} = 1 - m + \sum_{j=1}^m \frac{1}{2b_j},
%\end{equation}
%some questions about the spectrum are equivalent to some regarding Egyptian fractions. 
%More on this connection is discussed in \ref{Egyptian_fractions}.
%\\[4pt]
%\textbf{Disclaimer}\\
for now, until chapter \ref{counting occurrences} 
%named "Counting occurrences" 
, we will not pay attention 
to how many orbifolds have the same Euler orbicharacteristic, only whether a particular 
number is an \Eoc\ for at least one orbifold or not. 

In chapter \ref{counting occurrences} we will explain how these reductions will be relevant 
to the discussion holded there. 
%Let us note, that 
%Euler orbicharacteristic does not depend on the cyclic order of points on 
%the components of the boundary. 
%Because of that and since Euler orbicharacteristic does not depend on the cyclic order 
%of points on the components of the boundary we introduce an extension of a notation from 
%\cite{Conway2008}. 

%%We will write $\ast \{a,b,c,d,\cdots\}$ to denote not a particular orbifold, but a type 
%%of orbifold that have a dihedral points on a component of the boundry of orders 
%$a,b,c,d,\cdots$, but 
%%in any order. \\ 

%We will write $\ast \{a,b,c,d,\cdots\}$ to denote a type of a boundary (of an orbifold) that have 
%kaleidoscopic points of periods $a,b,c,d,\cdots$, but in any order. \\


%From what we wrote above (that Euler orbicharacteristic does not depend on the cyclic order 
%of points on the components of the boundary), we can see that Euler 
%orbicharacteristic is well defined 
%%on such types of orbifolds. \\ 
%when we specify only such a type of the components of the boundary of an orbifold and not 
%a particular cyclic order.  \\

\section{Reductions of cases}
The aim of following reductions is to make it easier to answear the question of which 
points lie in $\spe$ and which not. 

The first aspect of the structure of $\spe$ that we would like to simplify is that it is 
the sum of $\spebr{M}$, for 
every two dimentional manifold $M$. 
\begin{equation}
\spe = \bigcup_{M\textrm{ : 2d manifold}} \spebr{M}.
\end{equation}

%In this section we want to make some reductions to 
%limit number of cases that we will be dealing with. 

We aim to find a minimal set $\mathcal{M}$ of base manifolds 
%that "covers all the cases" i.e.  
such that:
\begin{equation}\label{minimal calM}
\spe = \bigcup_{M\in\mathcal{M}}\spebr{M}.
\end{equation}

% any for any $x \in \spe$ there 
%is an orbifold $O$ with a base manifold from $B$ such that $\cho{O} = x$.
%We aim to limit the number of base manifolds as much as possible while keeping entire spectrum. 
It will turn out that $\mathcal{M} = \{S^2, D^2\}$ satisfies \ref{minimal calM} and 
that both $S^2$ and $D^2$ are neccesery. 
%there are
% it can be done such that we are left with $B = \{S^2, D^2\}$ and there is 
%no futher reductions possible. 

%For this chapter we will consider orbifolds according to a definition from 
%(\ref{Disk_and_sphere_with_defects}). \\

\section{Sufficiency of $S^2$ and $D^2$}\label{sufficiency of D2 and S2}

Given an orbifold $O_1$, we want to perform some operations from \ref{Operations} on it, 
such that the resulting orbifold 
$O_2$ will have the same Euler orbicharactristic, but the base manifold of $O_2$ would 
be $S^2$ or $D^2$. We would then say, that $O_1$ got reduced to $O_2$.
In following subsection, we allow only such operations, that do not 
change \Eoc. When writing that we "can" do something we mean that there is 
possible one of the operations from \ref{Operations}.  %have as 
%big Euler characteristic as possible. 

The Euler characteristic of base manifold depends only on the number of handles, cross caps 
and boundry components. And, as stated in \ref{E_orb} it is: 
\begin{equation}
2-2h-c-b,
\end{equation}
%\smalltodoII{opisać i wybrać oznaczenia}
for $h$ - number of handles, $c$ - number of cross-caps, $b$ - number of boundary components. 

For every such a manifold feature we want to find an orbifold features with the same 
Euler orbicharacteristic delta. 

We will take two approuches, depending on whether the orbifold in question has a boundary or not.

%To do this we want to eliminate handles 
%Let us observe, that:
\subsection{Orbifold without boundary}
We can observe that:
\begin{align}
\hspace{3cm}\Delta(\circ) =& \hspace{-1cm} &-2& \hspace{-1cm} &= \Delta(2^4) \\
\hspace{3cm}\Delta(\times) =& \hspace{-1cm} &-1& \hspace{-1cm} &= \Delta(2^2)
\end{align}
From this we can see that we can remove handles and cross-caps from any orbifold without 
the boundary. 
After such reductions we are left with a $S^2$ orbifold with all orbipoints being rotational 
in the interior.

\subsection{Orbifold with boundary}\label{orbifold_with_boundary}
We can observe that:
\begin{align}
\hspace{3cm}\Delta(\circ) =& \hspace{-1cm} &-2& \hspace{-1cm} &= \Delta((^\ast 2)^8) \\
\hspace{3cm}\Delta(\ast) =& \hspace{-1cm} &-1& \hspace{-1cm} &= \Delta((^\ast 2)^4) \\
\hspace{3cm}\Delta(\times) =& \hspace{-1cm} &-1& \hspace{-1cm} &= \Delta((^\ast 2)^4)
\end{align}
From this we can see that we can remove handles and cross-caps 
from any orbifold with a boundary. 
We can also remove all boundary components exept one.         
We can further observe that:
\begin{align}
\hspace{3cm}\Delta(n) =& \hspace{-1cm} &\frac{n-1}{n}& \hspace{-1cm} 
&= 2\frac{n-1}{2n} &= \Delta((^\ast n)^2)\label{third_reduction}
\end{align}
From this we see that we can remove all the rotational orbipoints in favor for 
dihedral orbipoints.
After such reductions we are left with a $D^2$ orbifold with all orbipoints being dihedral on 
the boundary or being reflectional on the boundary. 
%\smalltodoII{reflectional?}

As a fact not necessary for our reductions, but interestung on its own, we can furthermore, 
observe that:
\begin{observation}
If $O_1$ has not $S^2$ as its base manifold it can be reduced to a $D^2$-orbifold.
\end{observation}
\subsubsection{Proof.}
If $O_1$ has not $S^2$ as its base manifold $M$, then $M$ has at least one handle or a cross-cup. We can observe that:
\begin{align}
\hspace{3cm}\Delta(\circ) =& \hspace{-1cm} &-2& \hspace{-1cm} &= \Delta(\ast (^*2)^4) \\
\hspace{3cm}\Delta(\times) =& \hspace{-1cm} &-1& \hspace{-1cm} &= \Delta(\ast).
\end{align}
From this we have that the handle or the cross-cap can be replaced by a boundary component  
and some number of boundary orbipoints. After this reduction, we can proceed with all the 
other reductions from the \ref{orbifold_with_boundary} and obtain an $D^2$-orbifold 
with the same \Eoc as the original one. $_\square$

%\begin{align}
%\Delta(n) =& &\frac{n-1}{n}&  
%&= \Delta((^\ast n)^2)\label{third_reduction}
%\end{align}

%From this we can conclude that every Euler orbicharacteristic can be obtained 
%by an orbifold with base manifold $S^2$ or $D^2$. 
%Examples of rational numbers from $\speS \setminus \speD$ and $\speD \setminus \speS$ are:
%We will provide examples 
%
%\ref{counting_occurences}

%From this we can conclude, that every Euler orbicharacteristic can be obtained 
%by an orbifold of signature of a type ($n$ and $m$ are arbitrary):

%\begin{align*}
%I_1I_2\cdots I_n & \textrm{\ or} \\
%\ast b_1b_2\cdots b_m &.
%\end{align*}

%Let us denote the set of all possible Euler orbicharacteristics of orbifolds of the form 
%$I_1I_2\cdots I_n$ by $\speS$ and the set 
%of all possible Euler orbicharacteristics of orbifolds of the form $\ast b_1b_2\cdots b_m$ 
%as $\speD$. 
%So we have that $\spe = \speD \cup \speS$. \\
%% Let us denote the set of all possible \Eoc s of two-dimentional orbifolds as $\spe$. \\
%Let us also observe that the order type and topology of $\speS$ and $\speD$ are 
%the same since 

%\begin{equation}\label{times_two_fact}
%2\speD=\speS
%\end{equation}
%and multiplying by $2$ is the order preserving homeomorphism of $\mathbb{R}$. \\[16pt]

%Now we can make aforementioned observation:
%\todo{dopisać to czytelniej}
%w szeczgólności żeby było jasne jaką redukcję robimy

The results of our reductions, can be summarised as:
%In the terms of set relations:
\begin{observation}\label{sum of spectras}
If two-dimentional manifold $M$ has no boundry, then
%\begin{equation} 
%\spe{M} = \chi(M) - (\speS - 2) 
%\end{equation}
%and
\begin{equation} 
\spebr{M} \subseteq \speS 
\end{equation} 

If, in addition, $M \neq S^2$, then 
\begin{equation} 
\spebr{M} \subseteq \speD. 
\end{equation}

\end{observation}
%\textbf{Proof} \\
%Let $M$ be a two-dimentional manifold with no boundry, from \ref{\Eoc_as_a_sum} we have, that: 
%\begin{equation}
%\spe{M} = \{\chi(M) - \sum_{i=1}^n 
%\frac{I_i-1}{I_i}\ |\ n \in \mathbb{N} \land \forall_i I_i \in \mathbb{N} \cup \infty\}.
%\end{equation}
%And from 
%$\chi(M) - (\speS - 2)$
\begin{observation}
If two-dimentional manifold  $M$ has a boundry, then 
%\begin{equation}
%\spe{M} = \chi(M) + (\speD - 1)
%\end{equation}
%and 
\begin{equation}
\spebr{M} \subseteq \speD
\end{equation}
\end{observation}
%\textbf{Proof} \\
%\begin{equation}
%\{\chi(M) - \sum_{j=1}^m 
%\frac{b_j-1}{2b_j}\ |\ m \in \mathbb{N} \land \forall_j b_j \in \mathbb{N} \cup \infty\} 
%\end{equation}
%\begin{observation}
%If two-dimentional manifold  $M$ has no boundry, then $\spe{M} \subseteq \speS$. 
%If, in addition, $M \neq S^2$, then 
%$\spe{M} \subseteq \speD$.
%\end{observation}
%\begin{observation}
%If two-dimentional manifold  $M$ has a boundry, then $\spe{M} \subseteq \speD$.
%\end{observation}

%In the terms of arithmetical expressions:
%\todo{sums are the form od sped and spes}

\begin{corollary}
We have that $\spe = \speS \cup \speD$.
\end{corollary}

\begin{observation}\label{only dihedral} 
If a two-dimensional manifold $M$ has a boundary, then:
\begin{equation}
\spebr{M} = \spe^{d}(M).
\end{equation}
\end{observation}

We will postpone our discussion of neccessity of both $S^2$ and $D^2$ to 
\ref{neccessity of d2 and s2}, after 
the 
section \ref{Reduction to arithmetical questions section} which will provide us 
with more convenient language. 

\section{Reduction to arithmetical questions}\label{Reduction to arithmetical questions section}
As written in \ref{spectra}, we can express an \Eoc of a $M$-orbifold $O$ as:
\begin{equation}
\cho{O} = \chi (M) - \sum_{i=1}^n \frac{r_i-1}{r_i} - \sum_{j=1}^m \frac{d_j-1}{2d_j},
\end{equation}
where $r_i$ and $d_j$ are degrees of the, respectively, rotational and diheadral orbipoints 
of $O$.

From this we can express $\spebr{M}$ as:
\begin{align}
\spe(M) = \chi (M) &-\left\{\sum\limits_{i=1}^n \frac{r_i-1}{r_i}\ \big|\ n\in\mathbb{N}_0,\ 
r_i\in\mathbb{N}_{>0}\cup \{\infty\}\right\}+ \\
&- \left\{\sum\limits_{j=1}^m\frac{d_j-1}{2d_j}\ 
\big|\ m\in\mathbb{N}_0,\ d_j\in\mathbb{N}_{>0}\cup \{\infty\}\right\} .
\end{align}

As from \ref{sufficiency of D2 and S2} we know that $\spe = \speS \cup \speD$, and that 
$\chi (S^2) = 2$ and $\chi (D^2) = 1$, we can expresss $\spe$ as a sum ($\cup$) of two sets:
\begin{equation}
2 -\left\{\sum\limits_{i=1}^n \frac{r_i-1}{r_i}\ \big|\ n\in\mathbb{N}_0,\ 
r_i\in\mathbb{N}_{>0}\cup \{\infty\}\right\} = \spe(S^2)
\end{equation}
and
\begin{equation}
1 - \left\{\sum\limits_{j=1}^m\frac{d_j-1}{2d_j}\ 
\big|\ m\in\mathbb{N}_0,\ d_j\in\mathbb{N}_{>0}\cup \{\infty\}\right\} = \spe(D^2) .
\end{equation}

From this we see, that the core of understanding $\spe$ through arithmetical viewpoint 
is to understand possible values of expression:
%We now know, that $\spe = \speS \cup \speD$. To determin 

%\begin{observation}\label{boils_down}
%From above reductions we can conclued that our problem boiles down to the analysis of all
% the possible 
%values of the expressions:
\begin{equation}\label{S2_sum}
2 - \sum_{i=1}^n \frac{r_i-1}{r_i}
\end{equation}
and 
\begin{equation}\label{D2_sum}
1 - \sum_{j=1}^m \frac{d_j-1}{2d_j},
\end{equation}
with $r_i$ and $d_j$ ranging over $\mathbb{N}_{>0}\cup \{\infty\}$.
%, with a convention 
%that $\varphi(\infty) \coloneqq \lim\limits_{k\to \infty} \varphi(k)$. 
%\end{observation}

As $\Delta(\infty) = 1 = \Delta(2^2)$ and $\Delta(^*\infty) = \frac{1}{2} = \Delta((^*2)^2)$, 
we could perform futher reductions to have an orbifold with 
particular orbicharacteristic without cusps (if needed) and then (after these reductions) 
we can analyse only expressions with $r_i$ and $d_j$ ranging over $\mathbb{N}_{>0}$ and 
they will still give us full spectrum. 
However, as stated later, it will be more convenient to us to include orbifolds with cusps 
so we are stating this observation only as a side remark.
%for readers information. 
%The fact that this agrees with the definition of the \Eoc\ on the geometrical terms was 
%addressed in \ref{extended_Euler_orbicharacteristic}. 

\section{Hurwitz theorem}\label{największy orbifold}
%\subsection{Hurwitz theorem}
One of the well known facts about two-dimensional orbifolds comes from \cite{Hurwitz1893} 
(seite 424).
Although formulated in different language, it states, that the orbifolds 
with hyperbolic structure can have an \Eoc\ at most $-\frac{1}{84}$. 
Here we will present the proof of this result with the language used in this thesis.
%We will now prove this result 
%in a straighforward way. 
%\todo{dopisać}
\begin{theorem}
If a two dimensional orbifold admits hyperbolic structure, then the maximal \Eoc\ it can 
have is $-\frac{1}{84}$ and the only orbifold that realises this \Eoc\ is $*2\ 3\ 7$.
\end{theorem}
\subsubsection{Proof.}
From \ref{sufficiency of D2 and S2} we know, that to check whether $-\frac{1}{84}$ is maximal 
negative \Eoc\ a two dimensional orbifold can have, we only need to check possible 
\Eoc s of $S^2$ orbifolds and $D^2$ orbifolds. 

Firstly, we will show, that $*2\ 3\ 7$ has biggest negative \Eoc\ from all $D^2$ orbifolds 
and its the only one with this \Eoc\ from $D^2$ orbifolds.

We have that 
\begin{equation}
\cho{*2\ 3\ 7} = 1 - \frac{1}{4} - \frac{2}{6} - \frac{6}{14} = -\frac{1}{84}. 
\end{equation}
From \ref{spectra} we know, that $\speD = \spe^d(D^2)$, so we can consider only dihedral 
orbipoints. 
Let us observe, that since 
$\max\{\Delta(^*n)\ |\ n \in \mathbb{N}_{>0} \cup \{\infty\}\} = \frac{1}{2}$, we have, that 
to heva \Eoc\ $<0$, 
$D^2$ orbifold has to have at least $3$ orbipoints.
%Let us observe
 
Let us observe, that for any $D^2$ orbifold that have $5$ orbipoints or more, it has an \Eoc 
equal at most $1 - 5\frac{1}{4} = -\frac{1}{4} < -\frac{1}{84}$. So we can restrict our search 
only to orbifolds with at most $4$ orbipoints. 

Let us observe, that $*2\ 2\ 2\ 3$ and $* 3\ 3\ 4$ are $D^2$ orbifolds with the biggest 
negative \Eoc\ among orbifolds, respectively, with four orbipoints, and, without any 
point of degree $2$. The proof of this observation is, that any other orbifold of these 
respective kinds, would need to have all degrees of orbipoints (when ordered in increasing manner) 
pairwise $\geq$ than $*2\ 2\ 2\ 3$ or $* 3\ 3\ 4$.
%Let us observe, that id $D^2$ orbifold have at least one rotational orbipoint, then 
%the smallest possible \Eoc\ is 
%\begin{equation}
%\cho{2 * 2 3} = 1 - \ 
%\end{equation}

Let us observe, that 
%the orbifold wsith the greatest negative \Eoc, that 
%have no orbipoints of degree $2$, that is, $* 3\ 3\ 4$, have \Eoc\ 
\begin{equation}
\cho{*2\ 2\ 2\ 3} = 1 - \frac{1}{4} - \frac{1}{4} - \frac{1}{4} - \frac{2}{6} = -\frac{1}{12} 
< -\frac{1}{84}
\end{equation}
and
\begin{equation}
\cho{*3\ 3\ 4} = 1 - \frac{2}{6} - \frac{2}{6} - \frac{3}{8} = -\frac{1}{24} < -\frac{1}{84},
\end{equation}
From this, we can conclude, that we can restrict our search only to orbifolds with 
exactly $3$ orbipoints, where at least one of them is equal to $2$. 

Let us observe, that for such orbifold to have negative \Eoc, it needs to have at two orbipoints 
of order at least $3$, otherwise it has \Eoc\ at most 
\begin{equation}
\cho{*2\ 2\ \infty} = 1- \frac{1}{4} - \frac{1}{4} - \frac{1}{2} = 0. 
\end{equation}

Further, let us observe, that $*2\ 4\ 5$ has the greatest negative \Eoc\ among orbifolds 
that have no orbipoint of degree $3$. The proof of this observation is simmilar to the proof 
of the previous one -- any other orbifold of this kind, 
would need to have all degrees of orbipoints (when ordered in increasing manner) 
pairwise $\geq$ than $*2\ 4\ 5$.

Let us observe, that
%We have that 
\begin{equation}
\cho{*2\ 4\ 5} = 1 - \frac{1}{4} - \frac{3}{8} - \frac{4}{10} = -\frac{1}{40} < - \frac{1}{84}. 
\end{equation}
From this, we conclude, that we can restric our search to the orbifolds of 
the form $*2\ 3\ n$. 
We can also observe, that all orbifolds of the form $*2\ 3\ n$ have unique \Eoc\ 
among this group.
%and it has to be unique. 
The one with the biggest \Eoc\ among them is $*2\ 3\ 7$. 

Let us observe that no $D^2$ orbifold with rotational orbipoints can't have such \Eoc. 
For the sake of contradiction, let us assume that there is some orbifold 
$r_1 \cdots r_n * d_1 \cdots d_m$, with $n\neq 0$, with \Eoc\ equal to $-\frac{1}{84}$. 
However, then also $*r_1r_1 \cdots r_nr_nd_1\cdots d_m$, 
with only dihedral orbiponts, 
would have \Eoc\ equal to $-\frac{1}{84}$. However, $*2\ 3\ 7$ is unique one with this \Eoc\ 
and it have no repeted degree, so it can't be expressed in the form 
$*r_1r_1 \cdots r_nr_nd_1\cdots d_m$ 
with $n\neq 0$. 

Let us also observe, that the same argument shows that $\mathbb{R}P^1$ orbifolds 
can't have \Eoc\ equal to $-\frac{1}{84}$, since $\chi{\mathbb{R}P^1} = \chi{D^2}$ 
and $\mathbb{R}P^1$ has no boundary, so $\mathbb{R}P^1$ obifold can have only rotational 
orbipoints. 
 
Now, we will prove, that no $S^2$ orbifold has \Eoc\ $-\frac{1}{84}$. 
%From this, 
Since we have \ref{2times homeomorphism} 
%it will 
%it folows, that no $S^2$ orbifold 
%has greater negative \Eoc. 
%The argument goes as folows:
We can perform following argument:  
 
For the sake of contradiction let us assume, that 
$-\frac{1}{84} \in \speS$, then, from \ref{2times homeomorphism} we know, that 
$\frac{1}{2}\left(-\frac{1}{84}\right)\in \speD$. This is a contradiction as 
$0 > \frac{1}{2}\left(-\frac{1}{84}\right) > -\frac{1}{84}$. 
As such, we ruled out all manifolds with Euler characteristic $>0$. 

For 
%What is left to prove, is that $-\frac{1}{84}$ is not an \Eoc\ for an $M$ orbifold 
%for any 
two dimensional manifolds 
%$M$ 
%We can rule out all manifolds 
with Euler characteristic $\leq 0$, we have that orbifolds having them as a base manifolds 
have \Eoc at most $-\frac{1}{4} < -\frac{1}{84}$. 
%From the arguments above, we can also conclude, that we can also rule out 
%$\mathbb{R}\mathrm{P}^1$ -- the only 
%other manifold left under our consideration, as orbifolds corresponding to it can only have 
%rotational orbipoints. 
$_\square$ 

\section{Neccessity of $S^2$ and $D^2$}\label{neccessity of d2 and s2}
As we know from \ref{Operations} adding an orbipoint to a manifold decreases it's 
orbicharacteristic. As $S^2$ has the highest Euler characteristic: $2$ of all 
two dimentional manifolds, there is no other orbifold with \Eoc\ equal to $2$. 
$S^2$ is then necesery to include $2$. 

As known from \ref{największy orbifold}, the number $-\frac{1}{84}\in\speD$, it is 
the greatest negative 
\Eoc any two dimensional orbifold can have and $-\frac{1}{84}\not\in\speS$. 
%We will now show, that 
%$-\frac{1}{84} \not\in \speS$. For the sake of contradiction let us assume, that 
%$-\frac{1}{84} \in \speS$, then, from \ref{2times homeomorphism} we know, that 
%$\frac{1}{2}\left(-\frac{1}{84}\right)\in \speD$. This is a contradiction as 
%$0 > \frac{1}{2}\left(-\frac{1}{84}\right) > -\frac{1}{84}$. 
$_\square$  

Futher examination of connections between $\speD$ and $\speS$ is performed in \ref{D_and_S}.







% mainfile: ../praca_magisterska_orbifoldy.tex
%\synctex=1
\chapter{Order type and topology}\label{order structure}
%
%Order type with zanurzenie w R
%
%1537/137
%
In this chapter we will discuss that both the order type and the topology 
of the set of all possible Euler orbicharacteristics 
of two-dimensional orbifolds are that of $\omega^\omega$. We will call this set $\spe$.
%We will see (in the observation \ref{boils_down}) that the problem of determining this boils
% down to the 
% analysis of all 
%the possible 
%values of the expressions:
%\begin{equation}
%2 - \sum_{i=1}^n \frac{I_i-1}{I_i}
%\end{equation}
%and 
%\begin{equation}
%1 - \sum_{j=1}^m \frac{b_j-1}{2b_j},
%\end{equation}
%where $I_i, b_j$ varies over $\mathbb{N}_{>0} \cup \{\infty\}$. \\
%As
%\begin{equation}
%2 - \sum_{i=1}^n \frac{I_i-1}{I_i} = 2 - n + \sum_{i=1}^n \frac{1}{I_i}
%\end{equation}
%and 
%\begin{equation}
%1 - \sum_{j=1}^m \frac{b_j-1}{2b_j} = 1 - m + \sum_{j=1}^m \frac{1}{2b_j},
%\end{equation}
%some questions about the spectrum are equivalent to some regarding Egyptian fractions. 
%More on this connection is discussed in \ref{Egyptian_fractions}.
\\[4pt]
\textbf{Disclaimer}\\
For now, until Chapter \ref{counting occurrences} named "Counting occurrences", 
we will not pay attention 
to how many orbifolds have the same Euler orbicharacteristic. 
%Let us note, that 
%Euler orbicharacteristic does not depend on the cyclic order of points on 
%the components of the boundary. 
%Because of that and since Euler orbicharacteristic does not depend on the cyclic order 
%of points on the components of the boundary we introduce an extension of a notation from 
%\cite{Conway2008}. 

%%We will write $\ast \{a,b,c,d,\cdots\}$ to denote not a particular orbifold, but a type 
%%of orbifold that have a dihedral points on a component of the boundry of orders 
%$a,b,c,d,\cdots$, but 
%%in any order. \\ 

%We will write $\ast \{a,b,c,d,\cdots\}$ to denote a type of a boundary (of an orbifold) that have 
%kaleidoscopic points of periods $a,b,c,d,\cdots$, but in any order. \\


%From what we wrote above (that Euler orbicharacteristic does not depend on the cyclic order 
%of points on the components of the boundary), we can see that Euler 
%orbicharacteristic is well defined 
%%on such types of orbifolds. \\ 
%when we specify only such a type of the components of the boundary of an orbifold and not 
%a particular cyclic order.  \\

\section{Reductions of cases}
Now we want to make some reductions to limit number of cases that we will be dealing with. \\ 
We aim to find a minimal set $B$ of base manifolds that such that any for any $x \in \spe$ there 
is an orbifold $O$ with a base manifold from $B$ such that $\cho{O} = x$.
%We aim to limit the number of base manifolds as much as possible while keeping entire spectrum. 
It will turn out it can be done such that we are left with $B = \{S^2, D^2\}$ and there is 
no futher reduction possible. \\ 
%For this chapter we will consider orbifolds according to a definition from 
%(\ref{Disk_and_sphere_with_defects}). \\
Given an orbifold $O_1$, we want to perform some surgeries on it such that the resulting orbifold 
$O_2$ will have the same Euler orbicharactristic, but the base manifold of $O_2$ would have as 
big Euler characteristic as possible. \\ 
The Euler orbicharacteristic of base manifold depends only on the number of handles, cross caps 
and boundry components. And, as stated in \ref{} it is: \\ 
For ever such a manifold feature we want to find an orbifold features with the same 
Euler orbicharacteristic delta.  \\
%To do this we want to eliminate handles 
%Let us observe, that:
One of the ways to do that is by observing that:
\begin{align}
\hspace{3cm}\Delta(\circ) =& \hspace{-1cm} &-2& \hspace{-1cm} &= \Delta(\ast (^\ast 2)^4) \\
\hspace{3cm}\Delta(\ast) =& \hspace{-1cm} &-1& \hspace{-1cm} &= \Delta((^\ast 2)^4) \\
\hspace{3cm}\Delta(\times) =& \hspace{-1cm} &-1& \hspace{-1cm} &= \Delta((^\ast 2)^4)
\end{align}
So we see that from any orbifold we can eradicate handles ....       

\begin{align}
\hspace{3cm}\Delta(n) =& \hspace{-1cm} &\frac{n-1}{n}& \hspace{-1cm} 
&= \Delta((^\ast n)^2)\label{third_reduction}
\end{align}

%\begin{align}
%\Delta(n) =& &\frac{n-1}{n}&  
%&= \Delta((^\ast n)^2)\label{third_reduction}
%\end{align}

From this we can conclude that every Euler orbicharacteristic can be obtained 
by an orbifold with base manifold $S^2$ or $D^2$. 
Examples of rational numbers from $\speS \setminus \speD$ and $\speD \setminus \speS$ are:
We will provide examples 
Futher examination of connections between $\speD$ and $\speS$ is performed in 
\ref{counting_occurences}

%From this we can conclude, that every Euler orbicharacteristic can be obtained 
%by an orbifold of signature of a type ($n$ and $m$ are arbitrary):

%\begin{align*}
%I_1I_2\cdots I_n & \textrm{\ or} \\
%\ast b_1b_2\cdots b_m &.
%\end{align*}

%Let us denote the set of all possible Euler orbicharacteristics of orbifolds of the form 
%$I_1I_2\cdots I_n$ by $\speS$ and the set 
%of all possible Euler orbicharacteristics of orbifolds of the form $\ast b_1b_2\cdots b_m$ 
%as $\speD$. 
%So we have that $\spe = \speD \cup \speS$. \\
%% Let us denote the set of all possible \Eoc s of two-dimentional orbifolds as $\spe$. \\
%Let us also observe that the order type and topology of $\speS$ and $\speD$ are 
%the same since 

%\begin{equation}\label{times_two_fact}
%2\speD=\speS
%\end{equation}
%and multiplying by $2$ is the order preserving homeomorphism of $\mathbb{R}$. \\[16pt]

%Now we can make aforementioned observation:
In the terms of set relations:
\begin{observation}
If two-dimentional manifold $M$ has no boundry, then
%\begin{equation} 
%\spe{M} = \chi(M) - (\speS - 2) 
%\end{equation}
%and
\begin{equation} 
\spebr{M} \subseteq \speS 
\end{equation} 

If, in addition, $M \neq S^2$, then 
\begin{equation} 
\spebr{M} \subseteq \speD. 
\end{equation}

\end{observation}
%\textbf{Proof} \\
%Let $M$ be a two-dimentional manifold with no boundry, from \ref{\Eoc_as_a_sum} we have, that: 
%\begin{equation}
%\spe{M} = \{\chi(M) - \sum_{i=1}^n 
%\frac{I_i-1}{I_i}\ |\ n \in \mathbb{N} \land \forall_i I_i \in \mathbb{N} \cup \infty\}.
%\end{equation}
%And from 
%$\chi(M) - (\speS - 2)$
\begin{observation}
If two-dimentional manifold  $M$ has a boundry, then 
%\begin{equation}
%\spe{M} = \chi(M) + (\speD - 1)
%\end{equation}
%and 
\begin{equation}
\spebr{M} \subseteq \speD
\end{equation}
\end{observation}
%\textbf{Proof} \\
%\begin{equation}
%\{\chi(M) - \sum_{j=1}^m 
%\frac{b_j-1}{2b_j}\ |\ m \in \mathbb{N} \land \forall_j b_j \in \mathbb{N} \cup \infty\} 
%\end{equation}
%\begin{observation}
%If two-dimentional manifold  $M$ has no boundry, then $\spe{M} \subseteq \speS$. 
%If, in addition, $M \neq S^2$, then 
%$\spe{M} \subseteq \speD$.
%\end{observation}
%\begin{observation}
%If two-dimentional manifold  $M$ has a boundry, then $\spe{M} \subseteq \speD$.
%\end{observation}

In the terms of arithmetical expressions:
\begin{observation}\label{boils_down}
From above reductions we can conclued that our problem boiles down to the analysis of all
 the possible 
values of the expressions:
\begin{equation}\label{S2_sum}
2 - \sum_{i=1}^n \frac{I_i-1}{I_i}
\end{equation}
and 
\begin{equation}
1 - \sum_{j=1}^m \frac{b_j-1}{2b_j},
\end{equation}
\end{observation}
with $I_i$ and $b_j$ ranging over $\mathbb{N}_{>0}\cup \{\infty\}$. \\
As stated in \ref{cusp_reduction} we can perform futher reductions to have an orbifold with 
particular orbicharacteristic without cusps (if needed) and then (after these reductions) 
we can analyse only expressions with $I_i$ and $b_j$ ranging over $\mathbb{N}_{>0}$ and 
they will still give us full spectrum. \\ 
However, as stated later, it will be more convenient to us to include orbifolds with cusps 
so we are stating above remark only for readers information. \\ 
 The fact that it agrees with the definition of the \Eoc\ on the geometrical terms was 
addressed in \ref{extended_Euler_orbicharacteristic}. 

%We also have shown that all possible \Eoc s are achieved without using cusps. As such, 
%we will use 
%cusps, remembering, that we can always get rid of them, if needed. So above $I_i$ and $b_j$ 
%are ranging over $\mathbb{N}_{>0}\cup \{\infty\}$, where expressions for infinity are defined as 
%a limits. The fact that it agrees with the definition of the \Eoc\ on the geometrical terms was 
%addressed in \ref{extended_Euler_orbicharacteristic}. \\
%Here is the notation that will be used: \\
%$\spe$ - spectrum of all possible \Eoc\ of two-dimentional orbifolds \\
%$\spe{M}$ - spectrum of all possible \Eoc\ of $M$ orbifolds \\

%% We would like to make some more observations concerning subsets of $\spe$. 
%% For any two-dimentional manifold $M$, observe, that:

%Now we state what we can already conclude. 


%\begin{theorem}\label{all_spectra_are_isomorphic}
%For any two two-dimentional manifolds $M$, $N$ spectras $\spe{M}$ and $\speM(N)$ have the same 
%order type and topology. 
%\end{theorem}
%\textbf{Proof} \\



\section{Order type and topology of $\speD$}
To determine order type and topology of $\spe$ we will first study how $\speD$ looks like. 
Then, remembering that 
$\spe = \speS \cup \speD$ and $\speS = 2\speD$ we will make an argument for $\spe$. \\
In this section we will also describe precisely where accumulation points of $\speD$ lie and of 
 which order 
(see below \ref{accumulation_points_definitions}) they are. Analysis of locations of those 
accumulation points, as interesting as it is alone will also be necessery for providing 
our argument about order type and topology of $\speD$. \\
\subsection{Definition and properties of order of accumulation points}
\label{accumulation_points_definitions} 
%We start with one technical definition of "transitive order" that will be almost what we want
%and then, there will be the the definition of "order", which is the definition that we need.
%\begin{definition}
%(Inductive). 
%%We say, that the point $a$  from the topological space $X$ is an acccumulation 
%%point of the transitive order 0, when
%We say that the point is an acccumulation point of a transitive order $0$, when it is 
%an isolated point. 
%We say that the point is an acccumulation point of a transitive order $n + 1$, when it is 
%an acccumulation point (in the usual sense) of the accumulation points of the transitive 
%order $n$. 
%\end{definition}  
%The only issue of the above definition is that the point of the transitive order $n$ 
%is also a point 
%of the transitive order $k$, for all $0< k \leq n$. We want a definition of order such that 
%for any point, there is at most one integer that is its order. So we define:
%\begin{definition}
%We say that the point is an acccumulation point of order $n$ iff it is an acccumulation point 
%of the transitive order $n$ and it is not an acccumulation point of the transitive order $n+1$. 
%If the point is an acccumulation point of the transitive order for an arbitrary large 
%$n$ we say that 
%the  point is an acccumulation point of order $\omega$.
%\end{definition}
We start with definition of being "at least of order $n$" that will be almost what we want
and then, there will be the definition of being "order", which is the definition that we need. \\
For a given set we define as follows:
\begin{definition}
(Inductive). 
%We say, that the point $a$  from the topological space $X$ is an acccumulation 
%point of the transitive order 0, when
We say that the point $x$ is an acccumulation point of a set $X$ 
of order at least $0$, when it belongs to the set $X$. 
We say that the point $x$ is an acccumulation point of a set 
of order at least $n + 1$, when it is 
an acccumulation point (in the usual sense) of the accumulation points each of order at least 
$n$ i.e. in every neighbourhood of $x$ there is at least one accumulation point of a set $X$ 
of order at least $n$, distincs from $x$. 
\end{definition}  
%The only issue of the above definition is that the point of the transitive order $n$ 
%is also a point 
%of the transitive order $k$, for all $0< k \leq n$. We want a definition of order such that 
%for any point, there is at most one integer that is its order. So we define:
\begin{definition}
We say that the point is an acccumulation point of order $n$ iff it is an acccumulation point 
of order at least $n$ and it is not an acccumulation point of order at least $n+1$. 
If the point is an acccumulation point of order at least $n$ for an arbitrary large 
$n$ we say that 
the point is an acccumulation point of order $\omega$.
\end{definition}
When we will say that a point is an accumulation point of some set without specifying an order 
then we will mean being an accumulation point in the ussual sense; from the point of view 
of above definitions, that is, an accumulation point of order at least one.
%\todo{dopisać notację do punktów skupienia różnego stopnia}
\begin{lemma}

\end{lemma}
\subsection{Analysis of locations of accumulation points of $\speD$ with respect to their order}
%\subsubsection{Some preliminary observations}
We want to determine where exactly are accumulation points of the set $\speD$ with 
respect to their orders. \\
For this we will use  
a handful of observations and lemmas. 
\begin{observation}\label{accumulation_points_are_in_the_spectrum}
Let us observe, that $\lim\limits_{n \to \infty} \Delta(^\ast n) = -\frac{1}{2}$. From that, 
we see, 
that for every point $x \in \speD$, the point $x - \frac{1}{2}$ is an acccumulation point. 
Let us observe, that also, for every point $x \in \speD$, we have that $x - \frac{1}{2} 
\in \speD$, 
because $\Delta(^\ast \infty) = -\frac{1}{2}$. 
\end{observation}
%Now we will show that the order type of $\speD$ is $\omega^\omega$ and where exactly are 
%its accumulation points of which orders. For this we will use  
%a handful of lemmas. 

%\subsubsection{Finiteness lemma}
\begin{lemma}\label{finiteness_lemma}
For all $n \in \mathbb{N}_{\geq 2}$ and $x \in (-\infty, 1]$ there are only finitely 
many Euler orbicharacteristics
in the interval $[x,1] \cap \speD$ of orbifolds that have points of order equal 
at most $n$. 
\end{lemma}
\textbf{Proof.} \\
Let $x \in (-\infty, 1]$. There can be at most $\lfloor 4(1-x) \rfloor$ orbipoints on the 
$D^2$ orbifold 
with an \Eoc\ $y \in [x,1]$ since each orbipoint decreases an \Eoc\ by at least $\frac{1}{4}$ 
and the Euler characteristic of $D^2$ is $1$. \\
There are only $(n-1)^{\lfloor 4(1-x) \rfloor}$ possible sets of $\lfloor 4(1-x) \rfloor$ 
orbipoints' orders that are less or equal than $n$. Hence, there are only at most 
$(n-1)^{\lfloor 4(1-x) \rfloor}$ possible \Eoc s.


\begin{lemma}\label{first_order_lemma}
If $x$ is an acccumulation point of the set $\speD$ of order $n$, then $x-\frac{1}{2}$ is a
 accumulation point of the set $\speD$ of order at least $n+1$. 
\end{lemma}
\textbf{Proof.} \\
Inductive. \\
$\bullet$ $n = 0$: If $x$ is an isolated point of the set $\speD$, then $x \in \speD$. 
From that, we 
have, that points $x - \frac{k-1}{2k}$ are in $\speD$ for all $k \geq 1$, from that, that 
$x-\frac{1}{2}$ is a 
accumulation point of $\speD$. \\
$\bullet$ inductive step: Let $x$ be an acccumulation point of the set $\speD$ of an order 
$n > 0$. 
Let $a_k$ be a sequence of accumulation points of order $n-1$ convergent to $x$. From the 
inductive assumption, we have, that $a_k - \frac{1}{2}$ is a sequence of accumulation points 
of order at least $n$. From the basic sequence arithmetic it is convergent to $x-\frac{1}{2}$. 
From that, we have that $x-\frac{1}{2}$ is an acccumulation point of the set $\speD$ of order 
at least $n+1$. $_\square$
\begin{lemma}\label{second_order_lemma}
If $x$ is an acccumulation point of the set $\speD$ of order $n$, then $x+\frac{1}{2}$ is 
an acccumulation point of the set $\speD$ of order at least $n-1$.  
\end{lemma}
\noindent\textbf{Proof.} \\  
Inductive \\
$\bullet$ $n = 1$: We assume, that $x$ is an acccumulation point of isolated points of the set 
$\speD$. 
%Let us observe
From \ref{finiteness_lemma} we know, that for all $m$ there are only finitely many 
Euler orbicharacteristics 
in the interval $[x,1]$ of orbifolds that have dihedral points of order equal at most $m$. \\ 
%\todo{może to dać jako osobny lemat}
From that, for arbitrary small neighborhood $U \ni x$ and arbitrary large $m$ there exist 
an orbifold 
that has a dihedral point of period grater than $m$, whose Euler orbicharacteristic lies in $U$. 
Let us take a sequence of such \Eoc s $a_k$ convergent to $x$, such that we can choose 
a sequence divergent to infinity of periods of dihedral points $b_k$ of orbifolds of \Eoc s 
equal $a_k$. 
\smalltodoII{picture} 
Let us observe, that for all $k$, the number $a_k+\frac{b_k-1}{2b_k}$ is in $\speD$. 
It is so, because $a_k$ is an \Eoc\ of an orbifold that have a dihedral point of period $b_k$, so 
identical orbifold, only without this dihedral point, has an \Eoc\ equal to $a_k + 
\frac{b_k-1}{2b_k}$. 
The sequence $a_k + \frac{b_k-1}{2b_k}$ converge to $x+\frac{1}{2}$. From that we have, that 
$x + \frac{1}{2}$ is an acccumulation point of the set $\speD$ of order at least $0$. \\
$\bullet$ inductive step: Let $x$ be an acccumulation point of the set $\speD$ of order $n > 1$. 
Let $a_k$ be a sequence of accumulation points of the set $\speD$ of order $n-1$ 
convergent to $x$. 
From the inductive assumption the sequence $a_k + \frac{1}{2}$ is a sequence of an acccumulation
 points of the set $\speD$ of order $n-2$ convergent to $x + \frac{1}{2}$. From that 
 $x + \frac{1}{2}$ is an acccumulation point of the set $\speD$ of order at least 
 $n-1$. $_\square$ 
\begin{lemma}\label{third_order_lemma}
If $x$ is an acccumulation point of the set $\speD$ of order $n+1$, then \\
$x - \frac{1}{2}$ is an acccumulation point of the set $\speD$ of order $n+2$ and \\
$x + \frac{1}{2}$ is an acccumulation point of the set $\speD$ of order $n$. 
\end{lemma}
\noindent\textbf{Proof.} \\
Let $x$ be an acccumulation point of the set $\speD$ of order $n+1$. From the lemma 
 \ref{first_order_lemma} we know, that $x - \frac{1}{2}$ is an acccumulation point of the set 
 $\speD$ of order at least $n+2$. Now let us assume (for a contradiction), that $x - \frac{1}{2}$ 
 is an \apots $\speD$ of order $k>n+2$. But then from the lemma \ref{second_order_lemma} 
 we have that $x$ is an acccumulation point of the set $\speD$ of order at least $n+2$ and that 
 is a contradiction. \\
Analogously, from the lemma \ref{second_order_lemma} we know, that $x + \frac{1}{2}$ is a 
accumulation point of the set $\speD$ of order at least $n$. Let us assume (for a contradiction), 
that $x+ \frac{1}{2}$ is an acccumulation point of the set $\speD$ of order $k>n$. But then 
from the lemma \ref{first_order_lemma} we have that $x$ is an acccumulation point of 
the set $\speD$ 
of order at least $n+2$ and that is a contradiction. $_\square$ 
\begin{lemma}\label{accumulation_points_of_the_set}
For all $n \in \mathbb{N}$ all accumulation points of the set $\speD$ of order $n$ are in $\speD$.
\end{lemma}
\noindent\textbf{Proof.} \\
Inductive \\
$\bullet$ $n=0$: Clear, as they are isolated points of $\speD$. \\
$\bullet$ inductive step: Let $x$ be a \apots  $\speD$ of order $n>0$. From the lemma 
\ref{third_order_lemma} point $x+\frac{1}{2}$ is an acccumulation point of the set $\speD$  
of order $n-1$. From the inductive assumption $x+\frac{1}{2} \in \speD$. Then, 
from \ref{accumulation_points_are_in_the_spectrum}, we have that $x \in \speD$. 
$_\square$ 

\begin{theorem}\label{greatest \apots}
The greatest \apots\ $\speD$ of order $n$ is $1-\frac{n}{2}$.
\end{theorem}
\noindent\textbf{Proof.}\\
Inductive \\
$\bullet$ $n=0$: We know, that $1\in \speD$ and $1$ is the greatest element of $\speD$. \\
$\bullet$ an inductive step: From the inductive assumption we know that $1-\frac{n}{2}$ is 
the greatest \apots\ $\speD$ of order $n$. From the lemma \ref{third_order_lemma} we have then 
that $1-\frac{n+1}{2}$ is a \apots\ $\speD$ of order $n+1$. Let us assume (for a contradiction), 
that there exist a bigger accumulation point of order $n+1$ equal to $y > 1-\frac{n+1}{2}$. 
But then, from lemma \ref{third_order_lemma}, point $y+\frac{1}{2}$ would be an acccumulation 
point 
of order $n$, what gives a contradiction, because $y+\frac{1}{2}>1-\frac{n}{2}$. $_\square$ 
\\[8pt]
From the above discussion we can also formulate following corollary that will be useful later: 
% corollary (stated in for deifferent ways 
%as one are sometimes more useful that another): 
%\begin{corollary}\label{predescors}
%Let $x \in \spe$. Then there exists $n \in \mathbb{N}$ such that $x + \frac{n}{2} \in \spe$ 
%but $x+\frac{n+1}{2} \not\in \spe$. For such $n$ we have that $x$ is an \apots\ $\spe$ of 
%order $n$.  
%\end{corollary}
%\begin{corollary}
%\end{corollary}
%%Above corollary can be refolmulated in a way that sometimes is more useful:
%\begin{corollary}\label{predescors_variant_II}
%Let $x \in \spe$ be an \apots\ $\spe$ of order $n$. Then there is $y \in \spe$ that is 
%an isolated point of $\spe$ such that $x = y - \frac{n}{2}$.   
%\end{corollary}
\begin{corollary}\label{predescors}
Let $x \in \speD$. Then:
\begin{itemize}
\item there exists $n_1 \in \mathbb{N}$ such that $x + \frac{n_1}{2} \in \speD$ 
but $x+\frac{n_1+1}{2} \not\in \speD$. \\ In other words, there exist $y \in \speD$ and 
$n_1 \in \mathbb{N}$ such that 
$y + \frac{1}{2} \not\in \speD$ and such that $x = y - \frac{n_1}{2}$;
\item there exists $n_2 \in \mathbb{N}$ such that $x$ is an \apots\ $\speD$ of 
order $n_2$
%\item $x$ is an \apots\ $\speD$ of order $n_2$, for some $n_2 \in \mathbb{N}$;  
\end{itemize}
and $n_1 = n_2$.
\end{corollary}

\subsection{Proof that $\speD$ is well ordered}

\begin{definition} 
Let $B_0 = \{1\}$.
For an $n \in \mathbb{N}_{>0}$, let $B_n$ be the set of all possible \Eoc\ realised 
by orbifolds of type 
$*b_1, \cdots, b_n$. For a given $n$ these are 
$D^2$ orbifolds with precisely $n$ non trivial orbipoits on their boundry.
\end{definition}


\begin{observation}\label{recursive_relation}
There is a recursive relation, that $B_{n+1}=B_n+\{-\frac{n-1}{2n}\ |\ n\geq 2\}$
\end{observation}
\textbf{Proof.} \\
It is so, because every orbifold with $n+1$ orbipoints can be obtained by adding one point 
to an orbifold with $n$ orbipoints and the set 
$\{-\frac{n-1}{2n}\ |\ n\geq 2\} = \{\Delta(^\ast b)\ |\ b \geq 2\}$. $_\square$
%Let take $x \in B_{n+1}$. Then there exist some orbifold $O_1$ with an \Eoc\ equal to $x$ with 
%signature $\ast b_1, \cdots, b_{n+1}$. 
%Taking an orbifold $O_2$ with signature $\ast b_1, \cdots, b_n$, its \Eoc\ $\cho{O_2}$ 
%is in $B_n$ as $O_2$ has $n$ orbipoints. The difference between $\cho{O_1}$ and $\cho{O_2}$ 
%is $-\frac{b_{n+1} - 1}{2b_{n+1}} \in \{-\frac{n-1}{2n}\ |\ n\geq 2\}$. \\
%Let take $x \in B_n+\{-\frac{n-1}{2n}\ |\ n\geq 2\}$. Then $ x = x_n + x'$ for some 
%$x_n \in B_n$ and $x' \in \{-\frac{n-1}{2n}\ |\ n\geq 2\}$. Let 


\begin{observation}\label{form_of_a_spectrum}
Observe that, as any orbifold has only finitely many orbipoints, we have that $\speD \subseteq 
\bigcup\limits^\infty_{n=0}B_n $. We defined $\speD$ as a set of all possible \Eoc\ of disk 
orbifolds, so $\speD \supseteq 
\bigcup\limits^\infty_{n=0}B_n $. From this we have that $\speD = \bigcup\limits^\infty_{n=0}B_n$.
\end{observation}

\begin{lemma}\label{fixed_number_of_orbipoints}
For any given $n \in \mathbb{N}$ the set $B_n$ is a subset of the interval 
$[1-\frac{n}{2}, 1-\frac{n}{4}]$.
\end{lemma}
\textbf{Proof.} \\


\begin{lemma}\label{two_sets_lemma}
If $A, B \subseteqq \mathbb{R}$ have no infinite strictly ascending sequences, then set 
$A + B \coloneqq \{a+b\ |\ a \in A, b \in B\}$ also have no infinite strictly ascending sequences. 
\end{lemma}
\noindent\textbf{Proof.} \\
Let $A$, $B$ have no infinite strictly ascending sequences. 
Let $c_n \in A + B$ are elements of some sequence. With a sequence $c_n$ there are 
two associated sequences $a_n$, $b_n$, such that, for all $n$, we have $a_n \in A$, 
$b_n \in B$ and 
$a_n + b_n = c_n$. Assume (for contradiction), that $c_n$ is an infinite strictly 
ascending sequence. 
Then $\forall_n\ a_{n+1}>a_n\ \lor\ b_{n+1} > b_n$. From the assumption $a_n$ has no infinite 
ascending sequence, so $a_n$ has a weakly decreasing subsequence $a_{n_k}$. But then 
subsequence $b_{n_k}$ must be strictly increasing, as $c_{n_k}$ is strictly increasing, what gives 
us a contradiction. 
%\Lightning 
$_\square$ 

\begin{lemma}\label{sum_lemma}
If $A, B \subseteqq \mathbb{R}$ have no infinite strictly ascending sequences, then set 
$A \cup B$ also have no infinite strictly ascending sequences.
\end{lemma}
\textbf{Proof.} \\
Let $A$, $B$ have no infinite strictly ascending sequences. 
For the sake of contradiction, lets assume, that $A \cup B$ has an infinite strictly 
ascending sequence $c_n$. Let $c_{n_k}$, $c_{n_l}$ be subsequences of $c_n$ consisting 
of elements from, respectively $A$ and $B$. At least one of them must be infinite and 
strictly increasing, which gives us a contradiction. $_\square$ 

\begin{observation}\label{ascending_in_B_n}
From \ref{recursive_relation} and \ref{two_sets_lemma}, we have that $B_n$ do not have 
infinite ascending sequence for all $n$. \\
Further, from \ref{sum_lemma} we conclude, 
that $\bigcup\limits_{n=0}^N B_n$ do not have infinite ascending sequence for all $N$.
\end{observation}

\begin{theorem}\label{well_order}
In $\speD$ there are no infinite strictly ascending sequences, hence, it is well ordered.
\end{theorem}
\noindent\textbf{Proof.} \\
%Being well ordered is a consequence of not having infinite strictly ascending sequences. 
For the sake of contradiction lets assume that $c_n$ is an infinite strictly ascending sequence in 
$\speD$. As $c_n$ is bounded from below by $c_0$ and whole $\speD$ is bounded from above 
by $1$, all elements of $c_n$ are in the interval $[c_0, 1]$. 
From \ref{form_of_a_spectrum} we have, that $\speD = \bigcup\limits^\infty_{n=0}B_n$. \\
Lemma 
\ref{fixed_number_of_orbipoints} says that for all $n$ we
have $B_n \subset [1-\frac{n}{2}, 1 - \frac{n}{4}]$. From this, we know, that for any 
$n$ such that $1 - \frac{n}{4} < c_0$ 
we have, that $B_n \cap [c_0,1] = \varnothing $. Let $n_0$ be such that 
$1 - \frac{n_0}{4} = c_0$ (so $n_0 = 4(1-c_0)$). 
Then for all $n > n_0$ we have $1 - \frac{n}{4} > c_0$, meaning, that 
for all $n > n_0$ we have
$B_n \cap [c_0,1] = \varnothing $, so all elements of $c_n$ are in 
$\bigcup\limits_{n=0}^{n_0} B_n$.
But this contradicts \ref{ascending_in_B_n}.  $_\square$
%\todo{rozwinąć ostatni argument}









\subsection{Proof that order structure and topology of $\speD$ are those of $\omega^\omega$}
\smalltodoII{rozbić to na dwa?} 
\begin{theorem}\label{speD_order_type}
Order type of $\speD$ is $\omega^\omega$. 
\end{theorem}
\textbf{Proof.} \\
%that order type of $\speD$ is $\omega^\omega$: \\
- proof that it is at least $\omega^\omega$ \\
suppouse that it is less than $\omega^\omega$\\
then it is smaller than $\omega^n$ for some $n$. \\
but for sufficiently distant have accumulation point. \\
done \\
- proof that it is at most $\omega^\omega$\\
suppouse it is bigger \\
then there is a point before which there is omega omega order \\
wooo, but it cant be since first with accumulation is at something spmething \\
done \\
ok, so how to give convincing correspondence between accumulation points and $\omega^n$??
Lemma \\
there cant be order of type \\
i wish I can cite this from somewhere \\ 
%
%\newpage
%

From the theorem \ref{order_preserving_homeomorphism_theorem} and ... we have that 
the topology of $\speD$ is $\omega^\omega$. 


\begin{theorem}\label{speD_theorem}
Topology of $\speD$ is that of $\omega^\omega$. 
\end{theorem}

\textbf{Proof} \\
We will prove it by inductively constructing an order preserving homeomorphism $f$ between 
$\omega^\omega$ and $\speD$. \\
For simplycity, we will take reverse (decreasing) order on $\speD$ i.e. $1$ will be the smallest 
 element 
(so for example $0 > 1$ in this order)
% and in general $x < y $ in the ussual order iff $x > y$ in 
the reverse order).  \\
%\todo{dopisać dowód}
%Since $\speD$ is well ordered (as we know from \ref{well_order}) it has an order preserving 
%bijection with some ordinal number. 
We will inductively construct the family of order preserving homeomorphisms $f_\mu$, 
indexed by ordinal numbers less 
%or equal
\smalltodoII{fix this}
than 
$\omega^\omega$ each 
prefix of $\omega^\omega$ homeomorphic to $\nu + 1$ (so on all ordinals less 
or equal to $\nu$) 
and some prefix of $\speD$. We will construct them in such a way, that for any $\mu_1 < \mu_2 
< \omega^\omega$ function $f_{\mu_2}$ restricted to the ordinals less or equal to 
$\mu_1$ coincides with $f_{\mu_1}$.
Then we will take $f \coloneqq \bigcup\limits_{0 \leq \mu < 
\omega^\omega} f_\mu$ (so $f(\mu) \coloneqq f_\mu(\mu)$). \\
Our inductive assumption for a given $\mu$ will be that 
for all $\nu < \mu$ function $f_\nu$ will be an order preserving homeomorphism 
between prefix of $\omega^\omega$ homeomorphic to $\nu + 1$ (so on all ordinals less 
or equal to $\nu$) 
and some prefix of $\speD$ and that for every $\nu_1 < \nu_2 < \mu$ function $f_{\nu_2}$ 
restricted to the ordinals less or equal than $\nu_1$ coincides with $f_{\nu_1}$. \\ 
%Our inductive assumtion will be, that for all $\mu \leq \omega^\omega$ 
%function $f_\mu$ is a homeomorphism 
%between prefix of $\omega^\omega$ homeomorphic to $\mu$ and some \\
%$\bullet$ $\mu = 0$: Function $f_0$ is an empty function and as such it is an order preserving 
%homeomorphism. \\
$\bullet$ $\mu = 0$: We take $f_0$ as a function on $\{0\}$ taking value $1$. 
Both $0$ and $1$ are the smallest elements of, respectively, $\omega^\omega$ and $\speD$ 
so $f_0$ is defined between prefix of $\omega^\omega$ of all ordinals less or equal to $0$, and 
some prefix of $\speD$. 
Function $f_0$ also preserves order on one element set.
Function $f_0$ is an homeomorphism between one element sets, both with discreate topology. \\
$\bullet$ $\mu$ is a successor ordinal less than $\omega^\omega$: From an inductive assumption 
we have an order preserving homeomorphism $f_{\mu - 1}$ between all ordinals 
less or equal to $\mu - 1$ and some prefix of $\speD$. 
We define $f_\mu$ on all numbers less or equal to $\mu - 1$ to be equal $f_{\mu-1}$. \\
%Now we have two cases: \\ 
It remains to define $f_\mu(\mu)$. 
As $\speD$ is well ordered it is well defined to take successor of an element of $\speD$. 
We define $f_\mu(\mu)$ to be a succesor of $f_{\mu - 1}(\mu-1)$ in $\speD$. As such (and from 
inductive assumption) it is indeed defined as a function between prefix of $\omega^\omega$ of all 
ordinals less or equal to $\mu$, and 
some prefix of $\speD$.
\\
Now we want to prove, that $f_\mu$ preservse the order. From the inductive assumption 
it preserves the order up to $\mu - 1$. As $\mu$ is the successor of $\mu - 1$ and 
$f_\mu(\mu)$ is a successor of $f_\mu(\mu-1)$, we have that $f_\mu$ is indeed an order preserving 
function. \\
Now we want to prove that $f_\mu$ is a homeomorphism. As from inductive assumtion we know, 
that $f_{\mu - 1}$ was a homeomorphism it is sufficient to show that preimages of open 
sets containing $f_\mu(\mu)$ and images of open sets containing $\mu$ are open. \\
Since $f_\mu(\mu)$ is a successor and since $\speD$ is well ordered, we have, that $f_\mu(\mu)$ 
is an isolated point in $\speD$. \\
Simmilarly $\mu$ is an isolated point in $\mu + 1$ as an successor ordinal. \\
From this we have, that open sets containing $f_\mu(\mu)$ (resp. $\mu$) are of the form 
$U \cup \{f_\mu(\mu)\}$ (resp. $V \cup \{\mu\}$) for some $U$ -- open set in $\speD$. 
(resp. $V$ -- open set in $\mu + 1$).
%Holding the notation of $U$ and $V$ we have that $f_\mu[V] = $
\smalltodoII{może rozwinąć}
From this this is clear. \\
%Let us observe, that 
%Lets take \todo{finish}
$\bullet$ $\mu$ is a limit ordinal less than $\omega^\omega$:  
%\todo{net and so on}
From the inductive assumption, for each $\nu < \mu$ we have an order preserving homeomorphism 
$f_\nu$ on the ordinals less or equal to $\nu$ and those functions pairwise coincide 
on the intersections 
of their domains. For every ordinal $\nu < \mu$ we define 
$f_\mu(\nu) \coloneqq f_\nu(\nu)$. It remains to define $f_\mu(\mu)$. \\
We consider a net $\phi_\nu \coloneqq \{f_\nu(\nu)\}_{\nu<\mu} \subset \mathbb{R}$
%, indexed by all $\nu < \mu$
. From the inductive assumption we know that the domain of the net $\phi_\nu$, as 
well as it's image is well ordered and that the net $\phi_\nu$.  
is an order preserving homeomorphism.
% is  this is a well ordered net. \todo{napisać jakoś lepiej tę własność}
Now we will show that the net $\phi_\nu$ has a limit in $\speD$. \\
First we will show, that $\phi_\nu$ has a limit in $\mathbb{R}$. For this, we will show that 
$\phi_\nu$ is bounded. \\
Order type of the image of $\phi_\nu$ is equal to $\mu$ and it is a prefix of $\speD$. 

As we have \ref{accumulation_points_of_the_set} 
As $\mathbb{R}$ is Hausdorff, from \cite{Kelley1975} (chapter 2, 
theorem 3, page 67) we know, that .  
\\ 
%\newpage
- show that if $\mu < \omega^n$, then it is earlier than $\frac{n}{2}$ something something\\
\smalltodo 
\\
- done\\
XD \\
$le\ XD$ \\ 
For the sake of contradiction, let us assume, that $\phi_\nu$ is unbounded. \\
There exist $n$ such that $\phi_\nu < \omega^n$. \\
The only unbounded in $\mathbb{R}$ prefix of $\speD$ is a whole $\speD$. \\ 
From this we conclude that $\speD$ schould have order type of $\omega^\omega$.
\\ 
%Firstly we will determine the order type of $\speD$. 
%From the lemma \ref{well_order} we know, that $\speD$ is well ordered, so it has order type 
%of some ordinal number. From this and 
%from the theorem \ref{greatest \apots} we know, that for the point $1-\frac{n}{2}$ there exist 
%a neighborhood $U=(1-\frac{n}{2}-\varepsilon,1-\frac{n}{2}+\varepsilon)$ such that $U \cap 
%\speD$ is homeomorphic to $\omega^n$. From this, and again from theorem \ref{greatest \apots} 
%we have that $\speD \cap [1,1-\frac{n}{2})$ is homeomorphic with $\omega^n$. 
%From this $\speD$ is homeomorphic with $\omega^\omega$.



\section{Order type and topology of $\spebr{M}$}\label{all_spectra_are_isomorphic}
Ok, all isomorphic with $\speD$ \\
here write about it \\
Tell me about it!


\section{Order type and topology of $\spe$}
\begin{theorem}
The order type of the set of possible Euler orbicharacteristics of two-dimensional orbifolds 
$\spe$ is $\omega^\omega$. 
\end{theorem}
\todo{tutaj też dopisać dowód}
Provide some argument about being homeomorphic 
\noindent\textbf{Proof.} \\
From \ref{speD_theorem} we know, that $\speD$ is homeomorphic with $\omega^\omega$. From 
\ref{all_spectra_are_isomorphic}, we know, that $\speS$ is homeomorphic 
with $\omega^\omega$. \\
$\speS = 2\speD$, so for all $n\in \mathbb{N}$ set $\speS \cap [2,-n)$ has a lower order type then 
$\speD\cap [2,-n)$. From this and from \ref{sum_lemma}, we have that $\speS \cup \speD$ 
has an order type of $\omega^\omega$. \\
For homeomorphicity, the proof of theorem \ref{speD_theorem} works as well in this case.
$_\square$ \\[4pt]

\section{Order type and topology of some subsets of $\spe$ and $\spe(M)$} 
$\spe_n$ \\
taking limit points is the order type of $\omega^\omega$ but not homeomorphic anymore. 



%\section{Which points are in the $\spe$?}

%Here we will try to understand better the conditions that let us determine wether the point 
%lie in $\spe$ or not.
% We will also state some observations about reasoning which points 
%belong to $\spe$ based on the knowlegde of other points belonging there. \\


\section{More about how this $\omega^\omega$ lies in $\mathbb{R}$}
%\todo{movve to some other section maybe}
\begin{observation}
The first (greatest) negative \apots\ of $\spe$ is 
$-\frac{1}{12}$. It is the accumulation point of order $1$. 
\end{observation}
\noindent\textbf{Proof.} \\
We will show, that $-\frac{1}{12}$ is the greatest negative accumulation point of the set $\speD$. 
From this we will obtain the thesis, as the set of all possible Euler orbicharacteristics 
of two-dimensional orbifolds is equal to $\speS \cup \speD$ and $\speS = 2\speD$, so 
the greatest negative point of the set $\speS$ is smaller than the greatest negative accumulation 
point of the set $\speD$. \\
$\bullet$ $-\frac{1}{12}=\chi^{orb}((2,3))-\frac{1}{2}$, from this we have that $-\frac{1}{12}$ 
an acccumulation point of the set $\speD$ of order at least $1$. \\
$\bullet$ Let us assume (for a contradiction), that there exist bigger, negative 
accumulation point of the set $\speD$ of order at least $1$. Let us denote it by $x$. \\
However, then, from the lemma \ref{third_order_lemma} point $x+\frac{1}{2}$ is the accumulation 
point of the set $\speD$. What is more, since $x\in (0, -\frac{1}{12})$, then $x+\frac{1}{2} 
\in (\frac{1}{2}, \frac{5}{12}$. From the lemma \ref{accumulation_points_of_the_set} we 
have that $x$ is in $\speD$. But orbifolds of the type $\ast b_1$ can have \Eoc only greater or 
equal $\frac{1}{2}$. Orbifolds of the type $\ast b_1b_2$ can only have \Eoc $\frac{1}{2}$, 
$\frac{5}{12}$ and some smaller. Orbifolds of the type $\ast b_1b_2b_3\cdots$ can have \Eoc only 
lower than $\frac{1}{4}$. This analysis of the cases leads us to the conclusion, that 
$(\frac{1}{2},\frac{5}{12})\cap \speD=\emptyset$ and to the contradiction. \\
$\bullet$ Above analysis of the cases leads us also to the conclusion, that $\frac{5}{12}$ 
is 
an isolated point of the set $\speD$, from this $-\frac{1}{12}$ is an acccumulation point 
of order $1$ of the set $\speD$. $_\square$ \\ 
 









% mainfile: ../praca_magisterska_orbifoldy.tex
\chapter{Algorithm for searching for the spectrum}

In the previous chapter we answeared the questions about how $\spe$ looks like -- in particular 
what is it's order type and topology. In this chapter we would like to develop a 
methods for answearing the 
following question: 
%In the previous chapter the main question was about which rational numbers are in $\spe$. 
%We can ask, how to answear the question 

%In this chapter we will show that the question 
"For a given rational number, is it in $\spe$?" 

We have some sort of answear to this question -- an algorithm.
%a very long equation that 
%commonly is refered to as an algorithm. 

It is not an ideal answear as it gives little insight of what is a general structure 
of the spectrum. Nethertheless it is a constructive and computable answear. 
%What's more, 
%later, in implementation chapter we discuss that for numbers with denominators of a reasonable 
%size and in reasonable distance from zero this algorithm can be run successfully on a 
%personal computer.

%is computable.

%In this chapter, we will provide the best answear we could find. 

%Turns aout that the number is in the specturm iff the following procedute says "yes". 

%It turns aout that the question for any number is answearable  

%Algorithmical approunch to questions from chapter 3 are only possible after gfiniteness 
%part of chapter four.

%What we can show, is that this question is computable -- i.e. there exists an algorithm 
%that answears this question. 

%Using results from previous chapters, we can now prove, that some computational problems related 
%to spectra are solvable.

%We will do it in a constructive way, by writing explicitly the algorithm and proving its
%correctness.

% and properties.

%We will also be able to actually compute sufficiently small examples of the 
%question.
%unansweared question 

The exact question we will provide algorithm to answear here is: 

\textit{For a given rational number $r$ and manifold $M$, is there at least one 
$M$ orbifold with $r$ as its \Eoc?}

%For a given rational number $r$ and manifold $M$:
%\begin{itemize}
%\item How many $M$ orbifolds with $r$ as their \Eoc\ are there?
%\item The accumulation point of what degree is $r$ in $\spe(M)$?
%\end{itemize}

We start with $r=\frac{p}{q}$, where $p \in \mathbb{Z}$, $q \in \mathbb{N}_{>0}$ and a manifold $M$. 
%\section{Reductions and special cases}

%\section{Decidability}
%\todo{oj dokończyć}
%Here we will show the proof that the problem of "deciding whether a given rational number is in an 
%Euler orbicharacteristic's spectrum or not" is decidable by showing algorithm for doing this. 
%Later, our algorithm will have a bonus property of determining of which order of condensation 
%is given point if it is in fact in $\sigma$. \\
%\smalltodoII{Może od razu postawić pełny problem}
%%It fill get also a performance enhancement by this added property. \\
%First stated algorithm is also very inefficient and is presented, because the idea is the most 
%clear in it. Right after it there is stated an algorithm with two enhancements: 
%\begin{itemize}
%\item determining an accumulation point of which order is a given point, if it is in fact in the 
%spectrum (this enhancement gives also a performance boost) 
%\item faster searching, because some cases do not need to be checked. 
%\end{itemize}
%\subsection{The algorithm}
%and $\textrm{gcd}(p,q)$. \\ 
\section{Reduction from arbitrary $M$ do $D^2$}
This reduction is based on \ref{spe_M}.
Note, that this is a different reduction than the one in \ref{reduction_to_arithmetical}. 
In \ref{reduction_to_arithmetical} we are saying that for any $M$, we have $\spebr{M} 
\subseteq \speS \cup \speD$. In \ref{spe_M} on the other hand we have, that 
for a manifold $M$ with $h$ handles, $c$ crosscaps and $b$ boundary components: \\
for $b \neq 0$:
\begin{equation}
\spe(M) = \speD - 2h - c - (b - 1)
\end{equation}
and for $b = 0$:
\begin{equation}
\spe(M) = 2\speD - 2h - c.
\end{equation}  


Using \ref{spe_M} 
%and \ref{} 
we conclude that the problem of deciding whether $\frac{p}{q}$ is in $\spe(M)$
is equivalent to deciding: \\
for $b \neq 0$ if:
\begin{equation}
\frac{p}{q} + 2h + c + (b-1) 
\end{equation} 
is in $\sdD$; \\
for $b = 0$ if:
\begin{equation}
\frac{1}{2}\frac{p}{q}+h+\frac{c}{2}
\end{equation}
is in $\sdD$.

Considering this fact, from this point, WLOG we will assume that $M = D^2$ and, 
following \ref{only dihedral}, we will 
be conserned only with dihedral orbipoints.

%\subsection{Arithmetic formulation}
%We want to determine whether there exists $d_1,d_2,\cdots,d_k$, such that 
%$\chi^{orb}(*d_1\cdots d_k) = \frac{p}{q}$. 

\section{Special cases}
In the case that $\frac{p}{q}$ is of the form $l\frac{1}{4}$, for some whole $l$ 
% $q = 4$ 
we can give the answer right away. For $l > 4$ we have that $l\frac{1}{4}$ is not in the set 
and for $l \leq 4$ it is (see \ref{greatest \apots}). 

Moreover for an even $l$ we have that $l\frac{1}{4}$ is a condensation point of order 
$\frac{4-l}{2}$ 
and for an odd $l$ it is a condensation point of order $\frac{3-l}{2}$ (see \ref{greatest \apots} 
and \ref{predescors}). 

In the case, where $\frac{p}{q} > 1$, we also can give answer rigth away and this answr is "no". 

Now we will consider only cases when $\frac{p}{q}$ is not of the form $l\frac{1}{4}$ and is 
$\leq 1$.
%\section{General case}
%
%\section{Simpler version of the question}
%
%To present the idea of searching the spectrum for the orbifolds with a given \Eoc, we will 
%first present the algorithm that answears a little easier question, namely: 
%
%\textit{For a given rational number $r$ and manifold $M$, is there at least one 
%$M$ orbifold with $r$ as their \Eoc?}
%
%This algorithm will mirror what we are focused on in \ref{chapter_three}, giving us the 
%computational tool for deciding whether a given number is in the spectrum or no. 
%
%The first approach of the searching algorithm is of this form: \\
%
%%We start with the 
%We use: 
%\begin{itemize}
%$\mathbb{N}$ counters $d_1d_2\cdots$ 
%(with values ranging from $1$, through all natural numbers, to infinity 
%(with infinity included)) set to $1$. Each counter correspond to one cone point 
%on the boundry of the disk of period equal to the value of the counter (with the note, that 
%if counter is set to $1$ it means a trivial cone point - namely a none cone point, a normal 
%point). 
%Every state of the counters during runtime of the algorith will have only finitely many 
%counters with value non-$1$. Moreover every state in the rutime of the algorithm 
%will have values on consequtive counters ordered in weakly decreasing order. From now we will 
%consider only such states. \\
%The state $d_1d_2\cdots$ correspond to the orbifold of 
%\Eoc equal $\chi^{orb}(*d_1d_2\cdots)$ (where the trailing $1$ are trunkated). \\ 
%%There is also a pivot pointing on one counter at any time.  
\section{Regular cases}
First we will describe what we use in the algorithm, giving the brief semantics. 
The detailed semantics are given in \ref{idea of the algorithm}.
\subsection{What we use}
We use: 
\begin{itemize}
\item $\mathbb{N}_{>0}$ counters $c_1, c_2, \cdots$ 
with values ranging on $\mathbb{N}_{>0}\cup\{\infty\}$.
%$1$, through all natural numbers, to infinity 
%(with infinity included). 
Each counter correspond to one dihedral point 
on the boundry of the disk of period equal to the value of the counter (with the note, that 
if counter is set to $1$ it means a trivial dihedral point - namely a non-orbi point, 
a normal point). 

We will write the state of the counters without commas, using the letter $d$. 
Note that with this convention, $c_i$ will refer to the $i$-th counter and $d_i$ will 
refet to the value of the $i$-th counter. 

So the state of the counters $d_1d_2\cdots$ correspond to the orbifold 
$*d_1d_2\cdots$ (where the trailing $1$'s are trunkated).

We will refer to the counters being "to the left" or "to the right" of each other, as 
the numbering would go from left to right.

\item a pivot pointing at some counter 
% at any time
\item a flag that can be set to: "Greater", "Searching" or "Less" corresponding to what was 
the outcome of comparing \Eoc\ of the orbifold corresponding to counters' state and 
$\frac{p}{q}$ or to the fact, that there is a need for a search of the next state of counters 
to compare with $\frac{p}{q}$.  
\end{itemize}
\subsection{What state are we starting our algorithm with}
We start with:
\begin{itemize}
\item all counters set to $1$. 
\item pivot pointing at the $c_1$
\item flag set to "Greater"
\end{itemize}
%If $\frac{p}{q}$ is of form $\frac{k}{4}$, where $k \in (-\infty,8] \cap \mathbb{Z}$ we give 
%the answear "yes" and end the whole algorithm. If $\frac{p}{q} > 2$ we give the answear "no" and 
%end the whole algorithmBecauseof this, below we assume, that \\
\subsection{Invariants claims}
Now we will state the claims of what properties the state of the counters will maintain 
during all the execution of the algorithm. The proof, that this is indeed the case will 
be performed in 
\ref{memory state proof}
\begin{claim}\label{valid state of counters}
We will do our computation such that:
\begin{itemize}
\item every state of the counters during runtime of the algorithm will have only finitely many 
counters with value non-$1$. 
\item every state in the rutime of the algorithm 
will have values on consequtive counters ordered in weakly decreasing order.
\end{itemize}
\end{claim}
From now we will 
consider only such states. 

%There is also a pivot pointing on one counter at any time.  
%The state of the counters $d_1d_2\cdots$ correspond to the orbifold 
%%of \Eoc\ equal $\chi^{orb}(*d_1d_2\cdots)$
%$*d_1d_2\cdots$ (where the trailing $1$'s are trunkated). 
\subsection{The algorithem for searching for a spectrum}
When the algorithm is in the state: 
\begin{itemize}
\item counters with values: $d_1d_2\cdots$
\item pivot: at the counter $c_p$
\item flag: set to the value $flag\_value$,
\end{itemize}
we procced as follows 
%(the term "We continue." means, that we start the following procedure from the beginning)
:
%// More on how we search for it will be told later, 
%        // for now we can think that we search one by one,
%        // starting from $d_p$ and going up till $d_p'$.
\begin{lstlisting}[firstnumber=1,consecutivenumbers=true]
In the case, the $flag\_value$ is equal to: 
{
    "Greater", then
    {
        If $\chi^{orb}(*d_1\cdots d_{p-1}\infty d_{p+1}\cdots)=\frac{p}{q}$ then
        {
            We found an orbifold and we are ending the whole
            algorithm with answer "yes, $*d_1\cdots d_{p-1}\infty d_{p+1}\cdots$".
            
            
            
        } 
        If $\chi^{orb}(*d_1\cdots d_{p-1}\infty d_{p+1}\cdots)>\frac{p}{q}$ then
        {
            We set $d_p$ to $\infty$.
            We set the flag to "Greater".
            We put the pivot at the $c_{p+1}$.
            We go to the 1st line.
        }  
        If $\chi^{orb}(*d_1\cdots d_{p-1}\infty d_{p+1}\cdots)<\frac{p}{q}$ then
        {
            We set the flag to "Searching".
            We go to the 1st line.
        }  
    }
    
    "Searching", then
    {
        We search one by one 
        for the value $d_p'$ of the $c_p$ such that
        $\chi^{orb}(*d_1\cdots d_{p-1}d_p'd_{p+1}\cdots)\leq\frac{p}{q}$ and
        $\chi^{orb}(*d_1\cdots d_{p-1}(d_p'-1)d_{p+1}\cdots)>\frac{p}{q}$.
        if $\chi^{orb}(*d_1\cdots d_{p-1}d_p'd_{p+1}\cdots)=\frac{p}{q}$ then 
        {
            We found an orbifold and we are ending the whole
            algorithm with answer "yes, $*d_1\cdots d_{p-1}d_p'd_{p+1}\cdots$".
            
            
        }
        We set $d_p$ and values of all the counters 
        to the left of $c_p$ to the value $d_p'$.
        if $\chi^{orb}(*d_1d_2d_3\cdots)=\frac{p}{q}$ then 
        {
            We found an orbifold and we are ending the whole
            algorithm with answer "yes, $*d_1d_2\cdots$".
            
            
            
        }
        If $\chi^{orb}(*d_1d_2d_3\cdots)<\frac{p}{q}$ then 
        {
            We set the flag to "Less".
            We put the pivot at the $c_{p+1}$.
            We go to the 1st line.
        }
        If $\chi^{orb}(*d_1d_2d_3\cdots)>\frac{p}{q}$ then 
        {
            We set the flag to "Greater".
            We put the pivot at the $c_1$.
            We go to the 1st line.
        }
    }
    
    "Less", then 
    {
        If $d_p = 1$ and the values of all the counters 
        on the left of $c_p$ are equal to 2 then 
        {
            We end the whole algorithm with the answer "no".
        }
        We increase $c_p$ by one ($d_p \coloneqq d_p + 1$) and
        we set the value of all counters on the left of $c_p$ to $d_p$.
        If $\chi^{orb}(*d_1d_2d_3\cdots)=\frac{p}{q}$ then
        {
            We found an orbifold and we are ending the whole
            algorithm with answer "yes, $*d_1d_2\cdots$".
            
            
            
        }
        If $\chi^{orb}(*d_1d_2d_3\cdots)>\frac{p}{q}$ then  
        {
            We set the flag to "Greater".
            We put the pivot at the $c_1$. 
            We go to the 1st line.
        } 
        If $\chi^{orb}(*d_1d_2d_3\cdots)<\frac{p}{q}$ then
        {
            We set the flag to "Less".
            We put the pivot at the $c_{p+1}$.
            We go to the 1st line.
        } 
    }
}
\end{lstlisting}
\section{The idea of the algorithm}\label{the idea of the algorithm}
We will now present in more detail what the algorithm is intented to do. 
To do this and for the later sections, we will first introduce an order on the states 
of counters satisfying \ref{valid state of counters} (as mentioned in 
\ref{valid state of counters} we will consider only such states) and prove several lemmas about it. 
\subsection{Order on the space of states of the counters}
\begin{definition}
We define a linear order $\preceq$ on the states of counters as follows:

Let $D_1$ be a state of counters equal to $d_1^1d_2^1\cdots$ and $D_2$ be a state of counters 
equal to $d_1^2d_2^2\cdots$. Let $i$ be the greatest index where $D_1$ and $D_2$ differ, then:\\
$bullet$ If $d_i^1 \leq d_i^2$ then $D_1 \preceq D_2$. 
\end{definition}

This is a suborder of the lexicographical order 
of states of counters after truncation of trailing 1's 
with the counters to the right being more significant. 

\begin{observation}
In general it is not true that if $D_1 \preceq D_2$ then 
$\cho{*D_1} \leq \cho{*D_2}$ nor that if $D_1 \preceq D_2$ then 
$\cho{*D_1} \geq \cho{*D_2}$.
\end{observation}

\begin{observation}\label{good lexicographical order}
Since $\preceq$ is a suborder of a lexicographical order it is a good order. 
\end{observation}

We can explicitly write the form of the successor of any state $d_1d_2d_3\cdots$ in $\preceq$:
%(Quotation marks around the counters' states in the following 
%lemma are added here only for readability 
%and they bare no particular meaning.)
\begin{observation}\label{form of the successor}
The successor of 
the state $d_1d_2d_3\cdots$, of the form
\begin{equation}
\underbrace{\infty\infty\cdots\infty}_{k-1 \rm\ times} d_kd_{k+1}d_{k+2}\cdots,
\end{equation}
where $k$ is 
such that $c_k$ is the first counter 
from the left that is not set to $\infty$,
% in the state $d_1d_2d_3\cdots$, 
is
\begin{equation} 
\underbrace{(d_k+1)(d_k+1)\cdots(d_k+1)}_{k-1\rm\ times}(d_k+1)d_{k+1}d_{k+2}\cdots,
\end{equation}

\end{observation}
%\textbf{Proof.}
\begin{definition}
We will call the state $d_1d_2d_3\cdots$, such that no $d_k$ is equal to $\infty$ a 
\textbf{finite} state. 

We will call the state $d_1d_2d_3\cdots$, such that at least one of $d_k$ is equal to $\infty$ 
an \textbf{infinite} state.  
\end{definition}
\begin{observation}
Using \ref{valid state of counters} we have that 
for the state $d_1d_2d_3\cdots$ to be finite (resp. infinite), it is equivalent to 
$d_1$ being different from (resp. being equal to) $\infty$. 
\end{observation}
%\begin{observation}

%\end{observation}
\begin{observation}\label{finiteness of the successor}
For any state $D$, the successor of $D$ is a finite state.
\end{observation}
\begin{definition}
We will call the ascending sequence $\{D_n\}$ 
in $\preceq$, such 
that for all $n$, we have that $D_{n+1}$ is the succesor of $D_n$, a \textbf{connected} sequence in 
$\preceq$.   
\end{definition}
\begin{observation}
Every connected sequence of the finite states is of the form $\{(d_1+n)d_2d_3\cdots\}$, 
where all $d_n$ are different from $\infty$.
\end{observation}
\begin{observation}\label{Successor lemma}
Let $D_1$ and $D_2$ be finite states and let $D_2$ be the successor of $D_1$ in 
$\preceq$. Then $\cho{*D_1} > \cho{*D_2}$.
\end{observation}
\textbf{Proof.}
From \ref{form of the successor} we know, that taking the successor 
of the finite state always chages  
only first counter and it is changing it by increasing it by 1. 
Increasing the order of the orbipoint  
decreases \Eoc. $_\square$

\begin{corollary}\label{connected sequences corollary}
The sequence $\{\cho{*D_n}\}$ is descending for every connected sequence 
of finite states $\{D_n\}$ in $\preceq$. 
\end{corollary}
\begin{lemma}\label{chi supp functoriality}
%Let $d_1$ be different form $\infty$. 
The supremum of the connected sequence 
of finite states $\{(d_1+n)d_2d_3\cdots\}$ is $\infty d_2d_3\cdots$, and 
%the following diagram commutes:
the infimum of the corresponding sequence $\{\chi(*(d_1+n)d_2d_3\cdots)\}$ is  
$\chi(*\infty d_2d_3\cdots)$.
\end{lemma}
\textbf{Proof.} 
%From \ref{good lexicographical order} we know, that every bounded sequence in $\preceq$ have 
%sup

For every $n$ we have that $(d_1+n)d_2d_3\cdots\preceq\infty d_2d_3\cdots$. Furthermore for 
every $d_1'd_2'd_3'\cdots$ such that $d_1'd_2'd_3'\cdots\preceq \infty d_2d_3\cdots$, there 
exists $n$, such that $d_1'd_2'd_3' \cdots\preceq(d_1+n)d_2d_3\cdots$. Thus, $\infty d_2d_3\cdots$ 
is the supremum of $\{(d_1+n)d_2d_3\cdots\}$.

For every $n$ we have that: 
\begin{align}
\chi(*(d_1+n)d_2d_3\cdots) &= \chi(*d_1d_2d_3\cdots) 
- \frac{(d_1+n)-1}{2(d_1+n)} + \frac{d_1-1}{2d_1} \notag\\ 
&= \chi(*d_1d_2d_3\cdots) - \frac{1}{2d_1} + \frac{1}{2(d_1+n)}
\end{align}
We also have that:
\begin{align}
\chi(*\infty d_2d_3\cdots) &= \chi(*d_1d_2d_3\cdots) 
- \frac{1}{2} + \frac{d_1-1}{2d_1} \notag\\ 
&= \chi(*d_1d_2d_3\cdots) - \frac{1}{2d_1} + 0.
\end{align}
Thus $\chi(*\infty d_2d_3\cdots)$ is the infimum of $\{\chi(*(d_1+n)d_2d_3\cdots)\}$.
$_\square$

\begin{lemma}
The state of the counters in the algorithm is weakly increasing with respect to order $\preceq$. 
\end{lemma}
\textbf{Proof.} \\
The state of the counters is changed in lines 10-11, 46, 66-67. In each of these lines 
the counter with the gretest index of all changed counters increases in value, so 
the resulting state is bigger with respect to order $\preceq$. $_\square$

\subsection{Basic idea}
The basic idea of the algorithm is to search through all the states of the counters going 
from the smallest (in the sense of $\preceq$) state of counters, which will be when all counters 
are set to $1$, up to some upper limit beyond which we are sure that no configuration of 
counters will yield the \Eoc\ that we are looking for. 

Now we will go through several obstacles of how to do so and solutions for them, answearing for 
example the 
questions how we go through all the states and what can be this upper limit. 

%More on the upper limit will be addre

%However this can't be done directly as there are infinite ascending sequences in $\preceq$. 
\subsection{Checking all the states}
This can't be done directly as there are infinite ascending sequences in $\preceq$. 
However, it can be done with some use of the properties we derived in the previous subsection.
\subsubsection{Checking infinite connected sequences in finitely many steps}
We will now present the method how to check any infinite connected sequence for solutions 
in finite number of steps.

%However, by \ref{Successor lemma} we know, that for every ascending sequence $\{a_n\}$ 
%in $\preceq$, such 
%that for all $n$, we have that $a_{n+1}$ is the succesor of $a_n$, we have that the sequence 
%$\{\cho{*a_n}\}$ is strictly descending. 

First, we will perform a reduction from 
arbitrary infinite connected sequence to the inifite connected sequence of finite states. 

Let us observe, that, by \ref{finiteness of the successor}, 
there can be at most one infinite state in 
any connected sequence, and if it is present it must be the first one. If such state 
$D_0$ is present, 
we can chceck it whether $\chi(*D_0))$ is equal to $\frac{p}{q}$ or not (one step), 
and then all states that are left to be checked are finite and form infinite 
connected sequence of finite states, thus ending our reduction. 
%separately from the rest of the sequence 

As this from this point we will present a method or checking for solutions any 
infinite connected sequence of finite states.

First, let us observe that thanks to \ref{connected sequences corollary}, 
when we are searching through the infinite connected sequence of finite states in 
$\preceq$, once we get (without fiding any solution) 
to the state $D_n$ for which $\cho{*D_n} < \frac{p}{q}$, we know 
that no state $D_m$ with $m>n$ can have $\cho{*D_m} = \frac{p}{q}$ 
and we can disregard whole sequence. 

There is, however, another problem, namely, that when we are searching through 
the infinite connected sequence of the finite state, 
%of consequtive 
%states of counters, 
initialy, we don't now, whether there will be any state 
$D_k = (d_1+k)d_2d_3\cdots$ in it, that 
will have $\cho{*(d_1+k)d_2d_3\cdots} \leq \frac{p}{q}$. 
However, thanks to \ref{chi supp functoriality} 
we can check for this, by first comparing $\frac{p}{q}$ with $\chi(*\infty d_2d_3\cdots)$. 
Since from \ref{chi supp functoriality}, we have that 
$\chi(*\infty d_2d_3\cdots)$ is the infimum of $\{*(d_1+n)d_2d_3\cdots\}$, 
we have that if $\chi(*\infty d_2d_3\cdots) < \frac{p}{q}$, then
there must be state $(d_1+n)d_2d_3\cdots$ such that $\chi(*(d_1+n)d_2d_3\cdots) < \frac{p}{q}$, 
for some $n$ and we can proceed to look for it one by one through the sequence.

One case that is left, is when $\chi(*\infty d_2d_3\cdots) > \frac{p}{q}$, but then we can 
disregard the whole sequence right away, since 
$\chi(*\infty d_2d_3\cdots)$ is the infimum of $\{*(d_1+n)d_2d_3\cdots\}$.


%From this we have the method to check any infinite connected sequence of finite states in finitely 
%many steps.
\subsection{Three "modes" of the algorithm}
The algorithm has three distinct fragments: 
\begin{itemize}
\item fragment in the lines 3-35. that will be called the "Greater" part,
\item fragment in the lines 27-62, that will be called the "Searching" part,
\item fragment in the lines 64-93, that will be called the "Less" part.
\end{itemize}
%The graph of the control flow of these parts looks like this:


\section{Proof of the correctness of the algorithm}
Firstly, let us observe, that algorithm gives the answear on lines 8, 14-15, 35-36, 55-56, 
62-63 and 
always ends immedietly after giving the answear. Thus, it will always give at most one answear.
Furthermore let us observe that these are the only places where the algorithm terminates, 
so if it terminates it will give at least one answear.
 
There are three things to be checked: \\
$\bullet$ That the algorithm never answears "yes" if there is no orbifold of the \Eoc\ 
$\frac{p}{q}$ (No false positives)\\
$\bullet$ That the algorithm never answears "no" if there is an orbifold of \Eoc\ 
$\frac{p}{q}$ (No false negatives)\\ 
$\bullet$ That the algorithm always ends in a finite number of steps (Guaranteed termination). 

%To do this, we will introduce an order on the states of counters satisfying 
%\ref{valid state of counters}:



%\begin{lemma}

%\end{lemma}
%\textbf{Proof.} \\

%\begin{lemma}

%\end{lemma}
%\textbf{Proof.} \\

\subsection{No false positives}
Algorithm gives answear "yes" at lines 7-8, 35-36, 44-45, 75-76. At each of these places, 
the answear contains the example of an orbifold with \Eoc\ equal to $\frac{p}{q}$ that was 
explicitly checked for correctness just before giving the answear (see lines 5, 33, 42, 73). 
$_\square$    
\subsection{No false negatives}
Let $d_1d_2d_3\cdots$ be such that $\cho{*d_1d_2d_3\cdots} = \frac{p}{q}$. 

%Let us assume for a contradiction, that algorithm started with $\frac{p}{q}$ answered "no". 
%
%We will show that this is impossible, by showing, that the algorithm will never 
%go beyond $d_1d_2\cdots$ in $\preceq$ order. 

First, we will show that the algorithm will never 
go beyond $d_1d_2d_3\cdots$ counter state in $\preceq$ order. 

Let us observe that the only lines where 
the counters are changed are lines 15, 40-41 and 71-72, and while changing, 
only counters 
at pivot and to the left of the pivot are changed. 

Because of that, going beyond $d_1d_2d_3\cdots$ counter state 
could happen only in lines 15, 40-41 or 71-72, while pivot would be 
%lovely
on the rightmost counter that is differrent from $d_1d_2d_3\cdots$. 
%This is because these are the only lines where the counters are changed and while changing, 
%only counters 
%at pivot and to the left of the pivot are changed. 

We will now eliminate all three options case by case. 
\subsubsection{Line 15}
\subsubsection{Lines 40-41}
\subsubsection{Lines 71-72}

\subsection{Guaranteed termination}
Let us assume that for some input $M$ and $\frac{p}{q}$ the algorithm does not anwear "yes". 
We will show, that then it will answear "no" in finite number of steps.
%\begin{enumerate}
%    \item if the pivot is on the zeroth column ($c = 0$), then 
%    \begin{enumerate}[label*=\arabic*.]
%        \item 
%        \item
%    \end{enumerate}
%    \item abcd
%\end{enumerate}
%\section{Improvements}

\section{Another questions the algorithm can answear}
\subsection{Deciding the order of accumulation}
Let $m \in \mathbb{N}$ be such that $\frac{p}{q} \in (1-\frac{m}{2},1-
\frac{m+1}{2})$
Let us denote by $r \coloneqq \frac{p}{q} - (1-\frac{m}{2})$. \\ 

We will searching in $\sigma$ as such: \\

If $\frac{p}{q} \in \sigma$, then, from the corollary \ref{predescors} we know, that there 
exist some $n \in \mathbb{N}$, such that $\frac{p}{q} + \frac{n}{2} \in \sigma$ but 
$\frac{p}{q} + \frac{n}{2} \not\in \sigma$. \\

We will be consequently checking points from $1+r$, through $1+r-\frac{l}{2}$, for 
$0 \leq l \leq m$, to the $\frac{p}{q}$. We stop at the first found point. 
If one of these point is in the spectrum, then all smaller (so also $\frac{p}{q}$) are in 
the spectrum and $\frac{p}{q}$ is the accumulation point of the spectrum of order $m-l$ 
(from this, 
we can see some heuristic, that the points that have smaller order will be generally 
harder to find in some sense). If none of this points are in in the spectrum, then $\frac{p}{q}$ 
is not. \\

\section{Implementation}
This algorithm is a part of the algorithm from \ref{Counting orbifolds -- arithmetical part} 
where the implementation of the whole will be discused in \ref{implementation}.

../LaTeX/TeX_files/counting_orbifolds.tex
% mainfile: ../praca_magisterska_orbifoldy.tex
\chapter{Counting orbifolds -- arithmetical part}\label{Counting orbifolds -- arithmetical part}


%\section{Arithmetical part}

%We want to determine this $n$. If $n = 0$, then $\frac{p}{q}$ is not in $\sigma$. 
%If $n > 0$, then $$
%\subsection{Deciding number of occurences}
%Searching for all occurences 

%The difficulty here is to carefully step other an occurence. 

%Compared to the previous version, we also use an occurance counter, starting with it set to 0 
%and with the list of orbifolds, wich is empty at the start.
\section{The idea of the algorithm}
This is an extention of the algorithm from \ref{Searching the spectrum}. It only differs by 
lines after finding the solution -- 7-11, 37-41 and 68-72. They all send the control flow 
to line 57th.
Instead terminating the algorithm the sollution is appended to the 
initialy empty list and the algorithm proceeds to search through the states as if the 
the state that the solution was changed to 
%after some repetitions of changes 
at lines 64-65, together with the pointer placement and flag value was  
the starting configuration. 

\begin{lstlisting}[firstnumber=1,consecutivenumbers=true]
In the case, the $flag\_value$ is equal to: 
{
    "Greater", then
    {
        If $\chi^{orb}(*d_1\dots d_{p-1}\infty d_{p+1}\dots)=\frac{p}{q}$ then
        {
            We found an orbifold, we add it to a list 
            and increase the occurence counter by 1. 
            We set the flag to "Less".
            We put pivot to the $c_{p+1}$ counter.
            We go to the 1st line.
        } 
        If $\chi^{orb}(*d_1\dots d_{p-1}\infty d_{p+1}\dots)>\frac{p}{q}$ then
        {
            We set $d_p$ to $\infty$.
            We set the flag to "Greater".
            We put the pivot at the $c_{p+1}$.
            We go to the 1st line.
        }  
        If $\chi^{orb}(*d_1\dots d_{p-1}\infty d_{p+1}\dots)<\frac{p}{q}$ then
        {
            We set the flag to "Searching".
            We go to the 1st line.
        }  
    }
    
    "Searching", then
    {
        We search one by one 
        for the value $d_p'$ of the $c_p$ such that 
        $\chi^{orb}(*d_1\dots d_{p-1}d_p'd_{p+1}\dots)\leq\frac{p}{q}$ and 
        $\chi^{orb}(*d_1\dots d_{p-1}(d_p'-1)d_{p+1}\dots)>\frac{p}{q}$.
        We set $c_p$ and all of the counters 
        to the left of $c_p$ to the value $d_p'$.
        if $\chi^{orb}(*d_1d_2d_3\dots)=\frac{p}{q}$ then 
        {
            We found an orbifold, we add it to a list 
            and increase the occurence counter by 1. 
            We set the flag to "Less".
            We put the pivot at the $c_{p+1}$.
            We go to the 1st line.
        }
        If $\chi^{orb}(*d_1d_2d_3\dots)<\frac{p}{q}$ then 
        {
            We set the flag to "Less".
            We put the pivot at the $c_{p+1}$.
            We go to the 1st line.
        }
        If $\chi^{orb}(*d_1d_2d_3\dots)>\frac{p}{q}$ then 
        {
            We set the flag to "Greater".
            We put the pivot at the $c_1$.
            We go to the 1st line.
        }
    }
    
    "Less", then 
    {
        If $d_p = 1$ and the values of all the counters 
        on the left of $c_p$ are equal to 2 then 
        {
            We end the whole algorithm with the answer "no".
        }
        We increase $c_p$ by one ($d_p \coloneqq d_p + 1$) and
        we set the value of all counters on the left of $c_p$ to $d_p$.
        If $\chi^{orb}(*d_1d_2d_3\dots)=\frac{p}{q}$ then
        {
            We found an orbifold, we add it to a list 
            and increase the occurence counter by 1. 
            We set the flag to "Less".
            We put pivot at the $c_{p+1}$.
            We go to the line 1..
        }
        If $\chi^{orb}(*d_1d_2d_3\dots)>\frac{p}{q}$ then  
        {
            We set the flag to "Greater".
            We put the pivot at the $c_1$. 
            We go to the 1st line.
        } 
        If $\chi^{orb}(*d_1d_2d_3\dots)<\frac{p}{q}$ then
        {
            We set the flag to "Less".
            We put pivot at the $c_{p+1}$.
            We go to the 1st line.
        } 
    }
}
\end{lstlisting}

\section{Proof of the correctness of the algorithm}
Let us observe, that whole proof from the chapter \ref{Searching the spectrum} 
was independent from the choise of 
the starting configuration -- state of counters, flag value and pivot placement, as 
long as they would hold the inveriants that were prooved in 
\ref{lemmas for the proof of the correctness} and were used in 
\ref{proof of the correctness of the algorithm} and the fact, that flag value will 
correspond to the relation between corresponding \Eoc\ and $\frac{p}{q}$. 
We know, that the found solution was satisfying all the lemmas -- as it was the state of 
the counters at some pont of the execution. The only thing left to see, is that 
the flag value will be appropriete. 

Let $D = d_1d_2d_3\cdots$ be the solution. 
Let $c_p$ be the couter at which the pointer was when the solution was found. Then, since 
\ref{same value on the counters to the left} and the fact that after each change the value of 
the counter that pivot is at is the same as value of the counters to the left of it 
and \ref{state is ordered}, we conclude 
that 
all states that have value of $c_p$ greater than $d_p$ can not be solutions. 
As such, we can proceed from the state 
\begin{equation}
D' = (d_{p+1}+1)(d_{p+1}+1)(d_{p+1}+1)\cdots (d_{p+1}+1)(d_{p+1}+1)d_{p+2}d_{p+3}\cdots.
\end{equation}
Setting flag to "Less" after finding the solution, will result in producing exectly this state. 
From this point all the invariants will be safisfied and the algorithm will proceed 
until it finds another solution or it stops. 
%In algorithm from this chapter, after finding the solution the control flow is always redirected 
%to line 57th.  
%As such, since algorithm from this chapter is the repeting iteration 
%of the algorithm from chapter \ref{Searching the spectrum} and since \ref{}, 
%above algorithm will hold all nesseserly traits. 
From \ref{second_finiteness_theorem} we know, that there will be only finitely many 
solution.
$_\square$

\section{Implementation}\label{implementation}
As an appendix in the separate document, there is a source of a program with implementation 
of this algorithm with optimisation described below.
%with full  
%enhancments described in this chapter. 
It is written in \href{https://www.rust-lang.org/}{Rust}. 
It can be also found on 
\href{https://github.com/Sooyka/praca_magisterska_orbifoldy}{Github}
%\smalltodoII{dać ref do github} 
along with the \LaTeX\ source of 
this thesis.
%It is in the separate file, as it would take too much space in this 
%document and wouldn't be readable. 

%\subsection{Optimisations}
%Binary search

%\subsection{Limitations}
%i64

%\todo{dopisać}


% mainfile: ../praca_magisterska_orbifoldy.tex
\chapter{Counting orbifolds -- combinatorical part}\label{Counting orbifolds -- combinatorical part}

%% mainfile: ../praca_magisterska_orbifoldy.tex
\chapter{Power series and generating functions}

%% mainfile: ../praca_magisterska_orbifoldy.tex
\chapter{Connection with modular forms}

../conclusions/conclusions.tex
%% mainfile: ../praca_magisterska_orbifoldy.tex
\chapter{Further directions}
\section{Asked, but unanswered questions}
\section{Unasked and unanswered questions}
\section{Power series and generating functions}
\section{Seifert manifolds}

%% mainfile: ../praca_magisterska_orbifoldy.tex
\chapter{test}
abcd

% mainfile: ../praca_magisterska_orbifoldy.tex
\nocite{*}
\bibliography{../bibliography/bibliography.bib}{}
\bibliographystyle{plain}

%~\cite{Conway2002}
\appendix
%../LaTeX/TeX_files/appendix_the_implementation_of_the_searching_algorithm.tex
% mainfile: ../praca_magisterska_orbifoldy.tex
\chapter{Appendix about good orders and accumulation points}
% and Canor-Bendixon derivatives
%\todo{Not e between equivalence of being accumulation point and being in Cantor bendixon derivative}
\section{Definition of order of accumulation points}
\label{accumulation_points_definitions} 
%\todo{zmienić to wszystko na rangę cantora bendiksona}
%We start with one technical definition of "transitive order" that will be almost what we want
%and then, there will be the the definition of "order", which is the definition that we need.
%\begin{definition}
%(Inductive). 
%%We say, that the point $a$  from the topological space $X$ is an acccumulation 
%%point of the transitive order 0, when
%We say that the point is an acccumulation point of a transitive order $0$, when it is 
%an isolated point. 
%We say that the point is an acccumulation point of a transitive order $n + 1$, when it is 
%an acccumulation point (in the usual sense) of the accumulation points of the transitive 
%order $n$. 
%\end{definition}  
%The only issue of the above definition is that the point of the transitive order $n$ 
%is also a point 
%of the transitive order $k$, for all $0< k \leq n$. We want a definition of order such that 
%for any point, there is at most one integer that is its order. So we define:
%\begin{definition}
%We say that the point is an acccumulation point of order $n$ iff it is an acccumulation point 
%of the transitive order $n$ and it is not an acccumulation point of the transitive order $n+1$. 
%If the point is an acccumulation point of the transitive order for an arbitrary large 
%$n$ we say that 
%the  point is an acccumulation point of order $\omega$.
%\end{definition}

%Corespondence between accumulation points and cantor bendixon derivative are describen 
%in appendix \smalltodoII{link to appendix}. 
This definitions will be useful for us in chapter \ref{order structure}, the exact same copy 
of it is included there \ref{accumulation_points_definitions repetition} 
as well for a readers convenience .

We start with definition of being "at least of order $n$" that will be almost what we want
and then, there will be the definition of being "order", which is the definition that we need. \\
For a given set we define as follows:
\begin{definition}
(Inductive). 
%We say, that the point $a$  from the topological space $X$ is an acccumulation 
%point of the transitive order 0, when
We say that the point $x$ is an accumulation point of a set $X$ 
of order at least $0$, when it belongs to the set $X$. 
We say that the point $x$ is an accumulation point of a set 
of order at least $n + 1$, when it is 
an accumulation point (in the usual sense) of the accumulation points each of order at least 
$n$ i.e. in every neighbourhood of $x$ there is at least one accumulation point of a set $X$ 
of order at least $n$, distinct from $x$. 
\end{definition}  
%The only issue of the above definition is that the point of the transitive order $n$ 
%is also a point 
%of the transitive order $k$, for all $0< k \leq n$. We want a definition of order such that 
%for any point, there is at most one integer that is its order. So we define:
\begin{definition}
We say that the point is an accumulation point of order $n$ iff it is an accumulation point 
of order at least $n$ and it is not an accumulation point of order at least $n+1$. 
If the point is an accumulation point of order at least $n$ for an arbitrary large 
$n$ we say that 
the point is an accumulation point of order $\omega$.
\end{definition}
When we will say that a point is an accumulation point of some set without specifying an order 
then we will mean being an accumulation point in the usual sense; from the point of view 
of above definitions, that is, an accumulation point of order at least one.
%\todo{dopisać notację do punktów skupienia różnego stopnia}


\section{Lemmas}
\begin{lemma}\label{two_sets_lemma}
If $A, B \subseteq \mathbb{R}$ have no infinite strictly ascending sequences, then set 
$A + B \coloneqq \{a+b\ |\ a \in A, b \in B\}$ also have no infinite strictly ascending sequences. 
\end{lemma}
\noindent\subsubsection{Proof.}
Let $A$, $B$ have no infinite strictly ascending sequences. 
Let $c_n \in A + B$ are elements of some sequence. With a sequence $c_n$ there are 
two associated sequences $a_n$, $b_n$, such that, for all $n$, we have $a_n \in A$, 
$b_n \in B$ and 
$a_n + b_n = c_n$. Assume (for contradiction), that $c_n$ is an infinite strictly 
ascending sequence. 
Then $\forall_n\ a_{n+1}>a_n\ \lor\ b_{n+1} > b_n$. From the assumption $a_n$ has no infinite 
ascending sequence, so $a_n$ has a weakly decreasing subsequence $a_{n_k}$. But then 
subsequence $b_{n_k}$ must be strictly increasing, as $c_{n_k}$ is strictly increasing, what gives 
us a contradiction. 
%\Lightning 
$_\square$ 

\begin{lemma}\label{sum_lemma}
If $A, B \subseteq \mathbb{R}$ have no infinite strictly ascending sequences, then set 
$A \cup B$ also have no infinite strictly ascending sequences.
\end{lemma}
\subsubsection{Proof.}
Let $A$, $B$ have no infinite strictly ascending sequences. 
For the sake of contradiction, lets assume, that $A \cup B$ has an infinite strictly 
ascending sequence $c_n$. Let $c_{n_k}$, $c_{n_l}$ be subsequences of $c_n$ consisting 
of elements from, respectively $A$ and $B$. At least one of them must be infinite and 
strictly increasing, which gives us a contradiction. $_\square$ \\


%\section{Order preserving homeomorphisms}
Concerning accumulation points, we will use the terminology, that we introduced in 
\ref{accumulation_points_definitions}

\begin{lemma}\label{order_preserving_homeomorphism_theorem}
Let $A \subseteq \mathbb{R}$ has an order type $\alpha$. 
Let $A$ be such that every accumulation point of $A$ belong to $A$. Then $A$ has not only an
order type $\alpha$ but is also homeomorphic to $\alpha$. 
\end{lemma}
%\todo{dopisać, że to równoważność}
\subsubsection{Proof.}
%Let us observe, that $\alpha$ must be countable -- as $A$ is well ordered we can assign to every 
%element $a_1 \in A$ an non-empty open interval, between $a_1$ and it's succesor $a_2$. 
%From every such interval we can pick a rational number. From this we have injection of $A$ into 
%$\mathbb{Q}$. \\
Without loss of generality, let us assume, that $A$ has no infinite descending sequence 
(case with $A$ having no infinite ascending sequence is completely analogous). \\

As $A$ has an order type $\alpha$ we have that there is an order preserving bijection 
$f : \alpha \to A$. \\
We will prove the theorem by showing that $f$ is a homeomorphism. \\

For the continuity of $f$ and $f^{-1}$ it is sufficient to show, that for every open
$U \subseteq A$ and $V \subseteq \alpha$ 
from prebases of respective topologies, $f^{-1}[U]$ and $f[V]$ are open ($\ast$). \\
Prebase open sets in $A$ are the ones inherited from the order topology on $\mathbb{R}$, 
for all $s \in \mathbb{R}$:
\begin{align*}
\{r\ &|\ r<s\} \cap A\\
\{r\ &|\ s<r\} \cap A.
\end{align*}
Prebase open sets in $\alpha$ are from order topology, for all $\nu \in \alpha$:
\begin{align*}
\{\eta\ &|\ \eta<\nu\} \\
\{\eta\ &|\ \nu<\eta\}.
\end{align*} 

%From our induction assumption we have ($\ast$) checked for following prebase sets: \\
%$\{r\ |\ r<s\} \cap A_\mu$, for all $s < f(\mu)$. \\
%$\{\eta\ |\ \eta<\nu\}$, for all $\nu < \mu$, \\
%It was left to be shown, that ($\ast$) holds for the prebase sets: \\
%$\{r\ |\ r<f(\mu)\} \cap A_\mu$, \\
%$\{r\ |\ s<r\} \cap A_\mu$, for all $s \leq f(\mu)$\\
%$\{\eta\ |\ \eta<\mu\}$, \\
%$\{\eta\ |\ \nu<\eta\}$, for all $\nu \leq \mu$, \\ 

%Prebase set -- $\{r\ |\ r<f(\mu)\} \cap A_\mu$: \\
%We have that:
%\begin{align*}
%f^{-1}[\{r\ |\ s<f(\mu)\} \cap A_\mu] = \{\eta\ |\ \eta <\mu\},
%\end{align*}
%which is open. \\
Now, we will prove ($\ast$) case by case: \\

$\bullet$ Prebase set -- $\{r\ |\ r<s\} \cap A$: \\
Let $\nu \in \alpha$ be the smallest, that $s \leq f(\nu)$, then:
\begin{align*}
f^{-1}[\{r\ |\ r<s\} \cap A] = \{\eta\ |\ \eta<\nu\},
\end{align*}
which is open. \\ 

$\bullet$ Prebase set -- $\{r\ |\ s<r\} \cap A$: \\
Let $s < f(\mu)$. We have two cases: \\
-- $s \in A$: then let $\nu$ be such that $f(\nu)=s$. Then we have that:
\begin{align*}
f^{-1}[\{r\ |\ s<r\} \cap A] = \{\eta\ |\ \nu < \eta \},
\end{align*}
which is open. \\
-- $s \not\in A$: then, by the assumption of the theorem we know that $s$ is not an accumulation 
point of $A$. From this we conclude, that 
$\exists_{t\in A} (t<s \land \neg \exists_{t' \in A} t<t'<s)$. Let $\nu$ be such that $f(\nu) = t$. 
Then we have that:
\begin{align*}
f^{-1}[\{r\ |\ s<r\} \cap A] = \{\eta\ |\ \nu < \eta \},
\end{align*}
which is open. \\

$\bullet$ Prebase set -- $\{\eta\ |\ \eta<\nu\}$: \\
\begin{align*}
f[\{\eta\ |\ \eta<\nu\}] = \{r\ |\ r<f(\nu)\} \cap A,
\end{align*}
which is open. \\

$\bullet$ Prebase set -- $\{\eta\ |\ \nu<\eta\}$: \\
\begin{align*}
f[\{\eta\ |\ \nu<\eta\}] = \{r\ |\ f(\nu)<r\} \cap A,
\end{align*}
which is open.$_\square$

\textbf{Remark.} 
The reverse is also true: If $A\subseteqq \mathbb{R}$ is homeomorphic to $\alpha$, then 
every accumulation point of $A$ belongs to $A$. 




%For an ordinal $\mu < \alpha$, let us denote $f_\mu 
%\coloneqq f\raisebox{-0.5\depth}{\big|}_{\mu + 1}$ so 
%$f_\mu$ is defined on all ordinals less or equal to $\mu$. Then we have that
%$f = \bigcup\limits_{\mu < \alpha} f_\mu$. Let us also denote $A_\mu \coloneqq f_\mu[\mu+1]$ --
%the image of $\mu + 1$ (as a set of all ordinals less or equal to $\mu$) in $A$. 
%Let us remark, that $A_\mu$ has an order type $\mu + 1$.\\ 

%We will prove the theorem by inductively showing for all $\mu < \alpha$, that $f_\mu$ is a 
%homeomorphism, and by showing that this is sufficient for $f$ to be a homeomorphism. \\


%%For simplycity, we will take reverse (decreasing) order on $A$ i.e. $1$ will be the smallest 
%% element 
%%(so for example $0 > 1$ in this order)
%%% and in general $x < y $ in the ussual order iff $x > y$ in 
%%the reverse order).  \\
%%%\todo{dopisać dowód}
%%%Since $A$ is well ordered (as we know from \ref{well_order}) it has an order preserving 
%%%bijection with some ordinal number. 

%%We will inductively show that the family of maps  $f_\mu$, 
%%indexed by ordinal numbers less 
%%or equal
%%%\smalltodoII{fix this}
%%than 
%%$\alpha$. 
%%For every $\mu < \alpha$ homeomorphism $f_\mu$ will be between 
%%prefix of $\alpha$ homeomorphic to $\nu + 1$ (so on all ordinals less 
%%or equal to $\nu$) 
%%and some prefix of $A$. 
%%We will construct them in such a way, that for any $\mu_1 < \mu_2 
%%< \alpha$ function $f_{\mu_2}$ restricted to the ordinals less or equal to 
%%$\mu_1$ coincides with $f_{\mu_1}$.
%%Then we will take $f \coloneqq \bigcup\limits_{0 \leq \mu < 
%%\alpha} f_\mu$ (so $f(\mu) \coloneqq f_\mu(\mu)$). \\
%Our inductive assumption for a given $\mu$ will be that 
%for all $\nu < \mu$ function $f_\nu$ is a
%%n order preserving 
%homeomorphism 
%between $\nu + 1$ (so defined on all ordinals less 
%or equal to $\nu$) 
%and $A_\nu$; and that for every $\nu_1 < \nu_2 < \mu$ we have 
%$f_{\nu_2}\raisebox{-0.5\depth}{\big|}_{\nu_1 + 1} = f_{\nu_1}$.
%%Then we will put $f = f_\alpha$
%\\[8pt]
%%restricted to the ordinals less or equal than $\nu_1$ coincides with $f_{\nu_1}$. \\ 
%%Our inductive assumtion will be, that for all $\mu \leq \alpha$ 
%%function $f_\mu$ is a homeomorphism 
%%between prefix of $\alpha$ homeomorphic to $\mu$ and some \\
%%$\bullet$ $\mu = 0$: Function $f_0$ is an empty function and as such it is an order preserving 
%%homeomorphism. \\
%$\bullet$ $\mu = 0$: \\
%%Let $a_0$ be the smallest element of $A$. 
%%We take $f_0$ as a function on $\{0\}$ taking value $a_0$. 
%%Both $0$ and $a_0$ are the smallest elements of, respectively, $\alpha$ and $A$ 
%%so $f_0$ is defined between prefix of $\alpha$ of all ordinals less or equal to $0$, and 
%%some prefix of $A$. 
%%Function $f_0$ also preserves order on one element set.
%Function $f_0$ is an homeomorphism between one element sets, both with discreate topology. \\[8pt]
%$\bullet$ $\mu < \alpha$ is a successor ordinal:\\ 
%%From an inductive assumption 
%%we have an order preserving homeomorphism $f_{\mu - 1}$ between all ordinals 
%%less or equal to $\mu - 1$ and some prefix of $A$. 
%%We define $f_\mu$ on all numbers less or equal to $\mu - 1$ to be equal $f_{\mu-1}$. \\
%%%Now we have two cases: \\ 
%%It remains to define $f_\mu(\mu)$. 
%%As $A$ is well ordered it is well defined to take successor of an element of $A$. 
%%We define $f_\mu(\mu)$ to be a succesor of $f_{\mu - 1}(\mu-1)$ in $A$. As such (and from 
%%inductive assumption) it is indeed defined as a function between prefix of $\alpha$ of all 
%%ordinals less or equal to $\mu$, and 
%%some prefix of $A$.
%%\\
%%Now we want to prove, that $f_\mu$ preservse the order. From the inductive assumption 
%%it preserves the order up to $\mu - 1$. As $\mu$ is the successor of $\mu - 1$ and 
%%$f_\mu(\mu)$ is a successor of $f_\mu(\mu-1)$, we have that $f_\mu$ is indeed an order preserving 
%%function. \\
%%Now we want to prove that $f_\mu$ is a homeomorphism. 
%Since $\mu$ is a successor ordinal, $\mu-1$ exists. \\
%From inductive assumtion we know,
%that $f_{\mu - 1} : \mu \to A_{\mu - 1}$ is a homeomorphism, 
%so it is sufficient to show that preimages of open 
%sets containing $f_\mu(\mu)$ and images of open sets containing $\mu$ are open. \\
%Since $f_\mu(\mu)$ is a successor (because $f_\mu$ preserves the order) 
%and since $A_\mu$ is well ordered, we have, that $f_\mu(\mu)$ 
%is an isolated point in $A_\mu$. \\
%Simmilarly $\mu$ is an isolated point in $\mu + 1$ as an successor ordinal. \\
%From this we have, that open sets containing $f_\mu(\mu)$ (resp. $\mu$) are of the form 
%\begin{equation}\label{form of open sets}
%U \cup \{f_\mu(\mu)\} \textrm{ (resp. } V \cup \{\mu\}\textrm{)} 
%\end{equation}
%for some $U$ -- open set in $A_{\mu-1}$. 
%(resp. $V$ -- open set in $\mu$). \\
%Let $U$ be an open set in $A_{\mu - 1}$ and $V = f_{\mu - 1}^{-1}[U]$ an open set in $\mu$. \\
%we have that 
%%$f_\mu[V] = $
%%\smalltodoII{może rozwinąć} 
%%From this this is clear. \\ 
%\begin{gather}
%f_\mu^{-1}\left[ U \cup \{f_\mu(\mu)\} \right] = f_\mu^{-1}\left[ U \right] \cup 
%f_\mu^{-1}\left[ \{f_\mu(\mu)\} \right] =
%%\underbrace{f_{\mu - 1}^{-1}\left[ U\right]}_{\textrm{open in } \mu} 
%f_{\mu - 1}^{-1}\left[ U\right] \cup \{\mu\} = V \cup \{\mu\}
%\end{gather}
%and
%\begin{gather}
%f_\mu \left[ V \cup \{\mu\} \right] = f_\mu \left[ V \right] \cup f_\mu\left[ \{\mu\} \right] = 
%%\underbrace{f_{\mu - 1} \left[ V \right]}_{\textrm{open in } A_{\mu - 1}} 
%f_{\mu - 1} \left[ V \right] \cup \{f_\mu (\mu)\} = U \cup \{f_\mu(\mu)\}
%\end{gather}
%so by \ref{form of open sets}, we have that preimages of open 
%sets containing $f_\mu(\mu)$ and images of open sets containing $\mu$ are indeed open. \\[8pt]
%%Let us observe, that 
%%Lets take \todo{finish}
%$\bullet$ $\mu < \alpha$ is a limit ordinal: \\ 
%%\todo{net and so on}
%%From the inductive assumption, for each $\nu < \mu$ we have an order preserving homeomorphism 
%%$f_\nu$ on the ordinals less or equal to $\nu$ and those functions pairwise coincide 
%%on the intersections 
%%of their domains. For every ordinal $\nu < \mu$ we define 
%%$f_\mu(\nu) \coloneqq f_\nu(\nu)$. It remains to define $f_\mu(\mu)$. \\
%%We consider a net $\phi_\mu \coloneqq \{f_\nu(\nu)\}_{\nu<\mu} \subset \mathbb{R}$
%%%, indexed by all $\nu < \mu$
%%. From the inductive assumption we know that the domain of the net $\phi_\mu$, as 
%%well as it's image is well ordered and that the net $\phi_\mu$ 
%%is an order preserving homeomorphism.
%%% is  this is a well ordered net. \todo{napisać jakoś lepiej tę własność}
%%Now we will show that the net $\phi_\mu$ has a limit in $A$. \\
%%First we will show, that $\phi_\mu$ has a limit in $\mathbb{R}$. For this, we will show that 
%%$\phi_\mu$ is bounded. \\
%%Order type of the image of $\phi_\nu$ is equal to $\mu$ and it is a prefix of $A$. 
%%For now, let us assume, that 
%Let $f_{\nu<\mu} \coloneqq f_{\mu}\Big|_{\{\nu\,|\,\nu<\mu\}}$. Then $f_{\nu < \mu}$ 
%is a well ordered net from $\mu$ to $\mathbb{R}$ with image $A_\mu\setminus\{f(\mu)\}$. 
%It is bounded by $f(\mu)$, so,
%as $\mathbb{R}$ is Hausdorff, from \cite{Kelley1975} (chapter 2, 
%theorem 3, page 67) we know, that $f_{\nu<\mu}$ has a unique limit as a net. Let us call this 
%limit $r$. This limit is 
%as well an accumulation point of $A$, so by the assumption of the theorem, we have that 
%$r \in A$.
%\begin{observation}
%We have that $r = f_\mu(\mu)$.
%\end{observation}
%\textit{Proof.} \\
%For the sake of contradiction, let us assume, that $r > f_\mu(\mu)$. 
%Then, as r is an accumulation 
%point of $f_{\nu<\mu}$, we have that $\exists_{\nu<\mu}\ f_\mu(\mu)<f_\mu(\nu)$. 
%Which is a contradiction as $f$ preservers the order. Hence, $r \leq f_\mu(\mu)$.
%Now, for the sake of contradiction, let us assume, that $r < f_\mu(\mu)$. \\
%As we have, that $r \in A$, we have, that there exist some $\eta$ such that $r = f(\eta)$. 
%Since $f$ preserves order and we assumed that $r < f_\mu(\mu)$, we have, that $\eta < \mu$. 
%But then, as $\mu$ is a limit ordinal, we have, that $\eta +1 < \mu$ as well. From this, 
%we conclude that there exist some ordinal $<\mu$, namely $\eta +1$, such that $f_\mu(\eta+1) > r$. 
%This however is a contradiction, as $f$ preserves the order and $r$ is an accumulation point. 
%Hence, $r \geq f_\mu(\mu)$. \\
%From this we conclude, that indeed $r = f_\mu(\mu)$. $_\square$\\
%%We have that $\forall_{\nu<\mu}\  f(\nu)<r$ \\ 
%%The limit is $f_\mu(\mu)$ as it is the 
%%biggest element of $A_\mu$ and $f_\mu$ is well ordered net. 
%%\\
%%As we have \ref{accumulation_points_of_the_set} 
%%%\newpage
%%- show that if $\mu < \omega^n$, then it is earlier than $\frac{n}{2}$ something something\\
%%\smalltodo 
%%\\
%%- done\\
%%XD \\
%%$le\ XD$ \\ 
%%For the sake of contradiction, let us assume, that $\phi_\nu$ is unbounded. \\
%%There exist $n$ such that $\phi_\nu < \omega^n$. \\
%%The only unbounded in $\mathbb{R}$ prefix of $A$ is a whole $A$. \\ 
%%From this we conclude that $A$ schould have order type of $\alpha$.
%%\\ 
%%Then $f_\mu$ is a ordinal indexed sequence.
%For the continuity of $f$ and $f^{-1}$ it is sufficient to show, that for every 
%$U \subseteq A_\mu$ and $V \subseteq \mu+1$ 
%in prebases of respective topologies, $f_\mu^{-1}[U]$ and $f_\mu[V]$ are open ($\ast$). \\
%Prebase open sets in $A_\mu$ are the ones inherited from the order topology on $\mathbb{R}$, 
%for all $s \in \mathbb{R}$:
%\begin{align*}
%\{r\ &|\ r<s\} \cap A_\mu\\
%\{r\ &|\ s<r\} \cap A_\mu.
%\end{align*}
%Prebase open sets in $\mu+1$ are from order topology, for all $\nu \in \mu+1$:
%\begin{align*}
%\{\eta\ &|\ \eta<\nu\} \\
%\{\eta\ &|\ \nu<\eta\}.
%\end{align*}
%From our induction assumption we have ($\ast$) checked for following prebase sets: \\
%$\{r\ |\ r<s\} \cap A_\mu$, for all $s < f(\mu)$. \\
%$\{\eta\ |\ \eta<\nu\}$, for all $\nu < \mu$, \\
%It was left to be shown, that ($\ast$) holds for the prebase sets: \\
%$\{r\ |\ r<f(\mu)\} \cap A_\mu$, \\
%$\{r\ |\ s<r\} \cap A_\mu$, for all $s \leq f(\mu)$\\
%$\{\eta\ |\ \eta<\mu\}$, \\
%$\{\eta\ |\ \nu<\eta\}$, for all $\nu \leq \mu$, \\ 

%Prebase set -- $\{r\ |\ r<f(\mu)\} \cap A_\mu$: \\
%We have that:
%\begin{align*}
%f^{-1}[\{r\ |\ s<f(\mu)\} \cap A_\mu] = \{\eta\ |\ \eta <\mu\},
%\end{align*}
%which is open. \\
%Prebase set -- $\{r\ |\ s<r\} \cap A_\mu$: \\
%Let $s < f(\mu)$. We have two cases: \\
%-- $s \in A$: then let $\nu$ be such that $f(\nu)=s$. Then we have that:
%\begin{align*}
%f^{-1}[\{r\ |\ s<r\} \cap A_\mu] = \{\eta\ |\ \nu < \eta \},
%\end{align*}
%which is open. \\
%-- $s \not\in A$: then, by the assumption of the theorem we know that $s$ is not an accumulation 
%point of $A$. From this we conclude, that 
%$\exists_{t\in A} (t<s \land \neg \exists_{t' \in A} t<t'<s)$. Let $\nu$ be such that $f(\nu) 
%= t$. 
%Then we have that:
%\begin{align*}
%f^{-1}[\{r\ |\ s<r\} \cap A_\mu] = \{\eta\ |\ \nu < \eta \},
%\end{align*}
%which is open. \\
%Prebase set -- :
%a\\
%This concludes the inductive part of the proof, that for every $\mu < \alpha$ we have that 
%$f_\mu$ is a homeomorphism. Now we will show, that this is sufficient for $f$ to be a 
%homeomorphism. \\
%\todo{wszystkie ograniczone otwarte działają}
%\todo{wszystkie nieograniczone też}
%%Firstly we will determine the order type of $A$. 
%%From the lemma \ref{well_order} we know, that $A$ is well ordered, so it has order type 
%%of some ordinal number. From this and 
%%from the theorem \ref{greatest \apots} we know, that for the point $1-\frac{n}{2}$ there exist 
%%a neighborhood $U=(1-\frac{n}{2}-\varepsilon,1-\frac{n}{2}+\varepsilon)$ such that $U \cap 
%%A$ is homeomorphic to $\omega^n$. From this, and again from theorem \ref{greatest \apots} 
%%we have that $A \cap [1,1-\frac{n}{2})$ is homeomorphic with $\omega^n$. 
%%From this $A$ is homeomorphic with $\alpha$.

\begin{lemma}\label{accumulation_points_and_order}
Let $A \subseteq \mathbb{R}$ be a bounded, well ordered set. 
Then $A$ has an accumulation point $a$ of order 
$n \in \mathbb{N}$ (it may be that $a \notin A$) iff order type of $A$ is at least $\omega^n$. 
\end{lemma}

\subsubsection{Proof.} 

Inductive, with respect to $n$ in $\omega^n$.

$\bullet$ $n=0$
Let us suppose, that $A$ has an accumulation point of order $0$.
Having an accumulation point of order $0$ means that $A$ is non-empty. As that it has an order 
type of at least $\omega^0 = 1$. \\

Let us suppose, that $A$ has order type at least $\omega^0=1$. Then it is non-empty, so it 
has at least one accumulation point of order $0$. 

$\bullet$ Induction step

Let us suppose that $A$ has an accumulation point $a$ of order $n+1$. This means that every 
neighbourhood of $a$ we can find infinitely many accumulation points of $A$ of order $n$. 
Let take one such neighbourhood and one such family $\{b_i\}_{i \in \mathbb{N}}$ 
of accumulation points of order $n$. 
Let us then take family of pairwise disjoint neighbourhoods $\{U_i\}_{i\in\mathbb{N}}$ of 
$\{b_i\}_{i \in \mathbb{N}}$. Let $A_i \coloneqq U_i\cap A$. 

From the induction assumption for all $i$, we have that $A_i$ is of order type at least 
$\omega^n$. As that, we managed to show an pairwise disjoint inclusions of countably many sets 
of order type at least $\omega^n$ into $A$. As that we have the order preserving inclusion of 
$\omega^{n+1}$ into $A$, so $A$ is of order type at least $\omega^{n+1}$. \\

Let us now suppose that $A$ has the order type of at least $\omega^{n+1}$. Then, we can find 
a family $\{A_i\}_{i \in \mathbb{A}}$ 
of pairwise disjoint subsets of $A$, each of order type $\omega^{n}$, with the property $(\ast)$, 
that $\forall_{i,j\in\mathbb{N}}i<j \implies \forall_{x \in A_i,y\in A_j} x< y$. 

From the inductive 
assumption, for all $i$, we have that $A_i$ has an accumulation point of order $n$. Let 
$\{b_i\}_{i\in\mathbb{N}}$ be the set of those accumulation points.
Because of the property $(\ast)$, those accumulation points are pairwise distinct, between 
$A_i, A_j$, with $i\neq j$. Since $A$ is bounded, we have that, the set $\{b_i\}_{i\in\mathbb{N}}$ 
is bounded, so it has an accumulation point $a$. As an accumulation point of the accumulation 
points of order $n$, it is an accumulation point of order $n+1$. $_\square$ 

\begin{corollary}\label{that_important_corollary}
Let $A \subseteq \mathbb{R}$ be a bounded, well ordered set of the order type $\omega^n$. Then 
it has exactly one accumulation point $a'$ of order $n$. This point has the property that 
$\forall_{a\in A}\ a<a'$.  
\end{corollary}
\subsubsection{Proof.} 
From \ref{accumulation_points_and_order} we know that $A$ has at least one accumulation point 
$a'$ of order $n$. 

For the sake of contradiction, let us assume, that there exists an accumulation 
point $\bar{a}$ of order $n$ such that $\exists_{a\in A}\ a\geq \bar{a}$. We have that $A$ has 
the order type $\omega^n$, which means that $\forall_{a_1\in A}\exists_{a_2 \in A}\ a_1 < a_2$. 
From this, we have, that $\exists_{a_0} a_0 > \bar{a}$. But then, we would have that the prefix 
$(-\infty, \bar{a}] \cup A$ of $A$ has an accumulation point $\bar{a}$ of order $n$. 
From this, from \ref{accumulation_points_and_order} we would conclude, 
that $(-\infty, \bar{a}] \cup A$ is of 
order type at least $\omega^n$, which leads to the contradiction, as $(-\infty, \bar{a}] \cup A$ 
is a proper subset of $A$. Thus, we have, that for all accumulation points $\bar{a}$ of $A$ of 
order $n$ we have that $\forall_{a\in A}\ a<\bar{a}$. 

It remains to show that there is only one 
such accumulation point - $a'$. For the sake of contradiction, let us assume, that 
there exists an accumulation point of $A$ of order $n$, named $\bar{a}$, such that 
$\bar{a} \neq a'$. Let us assume that $\bar{a} < a'$. Then, as in every neighbourhood 
of $a'$ there is a point from $A$, we have 
that $\exists_{a_0} a_0 > \bar{a}$. The absurdity of this statement is shown above. 
Case where $\bar{a} > a'$ is completely analogous. $_\square$


\begin{lemma}\label{The order does not grow lemma}
For $A, B \subseteq \mathbb{R}$, if $r \in \mathbb{R}$ is an accumulation point 
of order $m$ for $A$ and $n$ for $B$ and $m \leq n$, then $r$ 
is an accumulation point of order at most $n$ for $A \cup B$.  
\end{lemma}%\todo{Zapytać się kogoś z topologii/teorii mnogości}
\subsubsection{Proof.}
Inductive. 

$\bullet$ $n = 0$. Then $r$ is an isolated point of $B$ and either $r$ is isolated point of $A$ 
or $r \not\in A$. From this we have that there exists $U_1, U_2$ such that $B \cap U_1 = \{r\}$ and 
$A \cap U_2 \subseteq \{r\}$. From this we have that $(A\cup B) \cap (U_1\cap U_2) = \{r\}$. 
So $r$ is an isolated point of $A \cup B$.

$\bullet$ Inductive step. Let us suppose that for all $k < n$, the statement holds. Let $r$ 
be an accumulation point of order $n$ of $B$ and order $m$ of A, where $m \leq n$. From this 
we have that there exists $U_1, U_2 \ni r$ 
such that in $B\cap U_1$ there are only accumulation points 
of $B$ of order at most $n-1$ and in $A \cup U_2$ there are only accumulation points of $A$ 
of order at most $m-1$. From this, from the inductive assumption we have that in 
$(A \cup B) \cap (U_1 \cap U_2)$ there are only accumulation points of order at most $n-1$ of 
$A \cup B$.  
This means that $r$ is an accumulation point of order at most $n$ of $A\cup B$. 

We also know that, in every $U_1, U_2 \ni r$, there are accumulation points of order exactly $n-1$ 
of $B$ and exactly $m-1$ for $A$. From the inductive assumption we have then, that in 
$(A \cup B) \cap (U_1 \cap U_2)$ there are accumulation points of order $n-1$ of $A\cup B$. 
This means that $r$ is an accumulation point of order exactly $n$ of $A\cup B$.
%This means 
$_\square$   

%colorally
%coloraly
%corolary
%corollary

\begin{corollary}\label{derivative and sum is commutative}
Let $A^{(n)}$ be the set of all accumulations point of order $n$ of $A$. Then
for every $n\in\mathbb{N}$ we have that $(A\cup B)^{(n)} = A^{(n)}\cup B^{(n)}$.
\end{corollary}
\subsubsection{Proof.}
%````````````````````````````````````````````````````````````````````````````````````````````````````
Every accumulation point of either $A$ or $B$ is also an accumulation point of $A\cup B$, so 
$(A\cup B)^{(n)} \supseteq A^{(n)}\cup B^{(n)}$.

From \ref{The order does not grow lemma} we know, that for any point $r\in \mathbb{R}$, if 
$r \in (A\cup B)^{(n)}$, then $r \in  A^{(n)}\cup B^{(n)}$. $_\square$

\begin{lemma}\label{key_lemma}
For two bounded, well ordered sets $A, B \subseteq \mathbb{R}$, with order types, respectively 
$\omega^m$ 
and $\omega^n$, 
such that $m < n$, and that $\forall_{x\in A\cup B}\exists_{b\in B}x<b$, we have 
that order type of $A \cup B$ is well defined and 
equal to $\omega^n$.
\end{lemma}
\subsubsection{Proof.}
From \ref{sum_lemma}, we know, that $A\cup B$ is well ordered. As such its order type is well 
defined and equal to some ordeal number $\gamma$. 

We will show that $\gamma \leq \omega^n$ and $\gamma \geq \omega^n$, 
thus showing that $\gamma = \omega^n$.

Let $f : \omega^n \to B$ and $g : A \cup B \to \gamma$ be order preserving bijections. 
%Through this proof when we will write "corresponding" in regards of some subset of $\omega^n$

$\bullet$ $\omega^n \leq \gamma$: 

%Let us assume for the sake of contradiction, that $\gamma < \omega^m$. Then, we have an one-to-one 
%function that preserves the order from a strictly greater ordinal $\omega^m$ to a strictly 
%smaller one $\gamma$, which is a contradiction.  
We have that  
$g\circ f : \omega^n \to \gamma$ 
is an order preserving injection, thus, $\omega^n \leq \gamma$. 

$\bullet$ $\omega^n \geq \gamma$: 

From \ref{that_important_corollary} we know, that $B$ has exactly one accumulation point $b'$ of 
order $n$. This point has the property that $\forall_{b\in B}\ b<b'$.
%From the assumptions, we know that there exists $b_0 \in B$, such that $\forall_{a\in A}\ a<b_0$. 
%From this, we know that $(A\cup B) \cap (b_0,\infty) = B \cap (b_0, \infty)$. 
%Let us name the suffix $(A\cup B) \cap (b_0,\infty)$ of $A\cup B$ as $S$.
%As for the given point, the property of being an accumulation point od order $n$ of 
%some set depends on the arbitrary small neighbourhood of that point, and $B$ and $A\cup B$ 
%does not differ on the neighbourhood $S$ of $b'$, 
%we have, that $b'$ is an accumulation point od order $n$ also for $A \cup B$. 
As $b'$ is the only accumulation point of 
order $n$ for $B$ and from \ref{that_important_corollary} we know also that 
$A$ has no accumulation points of order $n$, from \ref{The order does not grow lemma} we know, 
that $A\cup B$ has exactly one accumulation point of order $n$, namely $b'$. 

%From this, we have that $g^{-1}[B \cap (b_0, \infty)]$ is a suffix of $\gamma$. Let us name this 
%suffix $S$. We know that $B \cap (b_0, \infty)$ is a suffix of $B$ and that $B$ is of 
%the order type $\omega^n$, so $B \cap (b_0, \infty)$ has no greatest element. We have also that 
%$g$ is order preserving bijection, so $S$ also has no greatest element. 
%This means that 
%$\forall_{s\in S}\exists_{s'\in S}\ s < s'$. 
%From \ref{that_important_corollary} we know, that $B$ has exctly one accumulation point $b'$ of 
%order $n$ with the propertie that $\forall_{b\in B}\ b<b'$. From this we 
%We have thatLet $s = \max\{b_0, b_1\}$. 
%This $b_2$ has the property that 
%$\exists_{\epsilon > 0 }\forall_{p \in P}\ p < b_2-\varepsilon$. 

For the sake of contradiction, let us assume that $\omega^n < \gamma$. But then, 
there is some proper prefix of $A\cup B$ with order type $\omega^n$. Let us name that 
prefix as $P$. From \ref{accumulation_points_and_order} we know, that 
$P$ has an accumulation point $p'$ of order $n$. 
%so there is some $x \in \mathbb{R}$, 
%such that $\exists_{b\in B} x<b$ and that $C \coloneqq (A\cup B)\cap (-\infty, x)$ 
%has an order type $\omega^n$.
Let $b_1 \in B$ be such that $\forall_{p \in P}\ p<b_1$. Such $b_1$ exists, because $P$ is 
a proper prefix of $A\cup B$, so $\exists_{x\in A\cup B}\forall{p\in P} p < x$, and 
%and $(A\cup B) \cap (b_0,\infty) = B \cap (b_0, \infty)$, so 
from the assumptions of the lemma we have that 
$\forall_{x \in A\cup B}\exists_{b \in B} x < b$.
We have that $p' \leq b_1$. But we have also that $b_1 < b'$, so $p' \neq b'$. This gives us the 
contradiction, as $b'$ is the only accumulation point of order $n$ in $A\cup B$. $_\square$

% This gives us a contradiction with \ref{} though, as 
%the only accumulation point of order $n$ in $B$ is strictly greater than $s$ (as it is 
%greater than ) and 
%$\exists_{\epsilon > 0 }\forall_{p \in P}\ p < b_2-\varepsilon$, so any accumulation point 
%of $P$ is smaller or 






%% mainfile: ../praca_magisterska_orbifoldy.tex
\chapter{Graphs of the control flows of the algorithms}

\end{document}



