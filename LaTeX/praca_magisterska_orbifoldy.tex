\documentclass[a4paper, 12pt]{report}
\synctex=1
\usepackage{polski}
\usepackage[T1]{fontenc}
\usepackage{amssymb}
\usepackage[polish,english]{babel}
\usepackage[utf8]{inputenc}
\usepackage{mathtools}
\usepackage{microtype}
\usepackage{amsthm}
\usepackage{amsmath}
\usepackage{amsfonts}
\usepackage{empheq}
\usepackage{xcolor}
\usepackage[pdftex,
            pdfauthor={Bartosz Sójka},
            pdftitle={Orbifoldy},
        %    pdfsubject={Orbifoldy}
            ]{hyperref}
\usepackage{siunitx}
\usepackage{geometry}
\usepackage{lipsum}
\usepackage{marvosym}
%\usepackage{enumitem}
%\usepackage{alltt}
\usepackage{listings}
\setlength\parindent{0pt}

%\topmargin = -1in

\geometry{a4paper, 
twoside,
%asymmetric,
top = 25mm,
bottom = 40mm,
inner = 35mm,
outer = 25mm
}

\lstset{numbers=left, numberstyle=\small, stepnumber=1, numbersep=5pt, mathescape=true}

%\linespread{1.24}

%top = 25mm,
%bottom = 30mm,
%inner = 30mm,
%outer = 25mm

\newtheorem{theorem}{Theorem}[subsection]
\newtheorem{observation}[theorem]{Observation}
\newtheorem{definition}[theorem]{Definition}
\newtheorem{lemma}[theorem]{Lemma}
\newtheorem{corollary}[theorem]{Corollary}

\newcommand{\todo}[1]{\hfill \break \textbf{\Huge \textcolor{violet}{TO DO: #1} \hfill \break}
\normalsize}
\newcommand{\content}[1]{\hfill \break \textbf{\large \textcolor{violet}{#1} \hfill \break}
\normalsize}
\newcommand{\smalltodo}[1]{\textbf{\ \textcolor{violet}{To do}}}
\newcommand{\smalltodoII}[1]{\hfill \break \textbf{\ \textcolor{violet}{To do: #1}}\hfill \break}
\newcommand{\missingpicture}[1]{\hfill \break\\[16pt] \Huge \textbf{\textcolor{violet}
{Missing picture 
\normalsize
#1}} \hfill
\break \\[16pt] \normalsize}

%\newcommand{\beginproof}{{\noindent\textbf{Proof.}\\}}
%\newcommand{\endproof}{{$_\square$}}

\newcommand{\sIS}{{\mathcal{\sigma}^I(S^2)}}
\newcommand{\sbD}{{\mathcal{\sigma}^b(D^2)}}

\newcommand{\Eoc}{Euler orbicharacteristic} 
\newcommand{\apots}{accumulation point of the set}

\renewcommand{\labelenumii}{\theenumii}
\renewcommand{\theenumii}{\theenumi.\arabic{enumii}.}

\title{Orbifoldy}
\author{Bartosz Sójka}
%\date{}
\begin{document}
% mainfile: ../main/praca_magisterska_orbifoldy.tex
\newpage
\thispagestyle{empty}
\begin{center}
\textbf{\large Uniwersytet Wrocławski\\
Wydział Matematyki i Informatyki\\
Instytut Matematyczny}\\
\textit{\large specjalność: teoretyczna}\\
\vspace{4cm}
\textbf{\textit{\large Bartosz Sójka}\\
\vspace{0.5cm}
{\Large Two dimentional orbifolds' volumes' spectrum}}\\
\end{center}
\vspace{3cm}
{\large \hspace*{6.5cm}Praca magisterska\\
\hspace*{6.5cm}napisana pod kierunkiem\\
\hspace*{6.5cm}prof. dr hab. Tadeusza Januszkiewicza }\\
\vfill
\begin{center}
{\large Wrocław 2021}\\
\end{center}

\tableofcontents
\begin{abstract}
\setcounter{page}{4}
%\begin{center}
%Orbifolds! Yeah! \\
%Spectrums! Yeah! 
%\end{center}
Orbifoldy
\end{abstract}
% mainfile: ../praca_magisterska_orbifoldy.tex
\chapter{Introduction}
\setcounter{page}{9}
\section{Motivations}
Orbifols are geometrical spaces that encodes some of groups action properties. 

They played a crucial role in Thurstons geometrisation prgram leading to very important 
results in geomerty
%Quotiens by a groups
They are also correlated and find use in many different subjects, 
from geometry to physics and signal analysis, and art -- se e esher art analysis -- 
differnt patterns of fresks and os on,freski.

They are also beatiful symmetrical structers on they own, provideing 
nice and neat  and uniform language to describe platonic solids, parkietarze of the euklidean 
and hyperbolid plane, asher pictures 
as we ll as general notion of symmety

%Praca o ujemnym ale z najwyzszą orbicharakterystyką

The partcular focus of this thesis is on ppossible volumes of orbifodls.
Such analyssis was performed for example in dimension three, where 
cite cite cite.

\todo{dopisać}

\section{Scope}
In this thesis we would like to give some description of the spectrum of the volumes 
of two dimensional orbifolds, in particlar thouse of negative \Eoc. 
We will examine order type and topology of the spectrum \ref{order structure} 
as well as the structure 
of the spectrum based on spectra corresponding to different manifolds \ref{spectra}, 
\ref{reduction_to_arithmetical}.
We will also provide tools for determining whether a given number is in spectrum of volumes 
\ref{Searching the spectrum} as 
well as tools to compute how many orbifolds correspond to a given volume in the spectrum 
\ref{counting occurrences}. 

\section{Technical introductions}
Now, we will proceed to give technical introductions about orbifolds, \Eoc\ and the 
technics we will use in this thesis, alongside with some definitions and notation and 
naming conventions. 
 
\section{Orbifolds}
%\todo{jak sie juz wszystko zbierze co ma tu być, to to dopisać}
\subsection{Definition}
The definition of the orbifold is taken from Thurston \cite{Thurston1979} (chapter 13), 
with slight modification described in \ref{compactness}. 
We briefly recall the concept, but for full discussion we refer to \cite{Thurston1979}. 

An orbifold is a generalisation of a manifold. As manifold, it consists of a Hausdorf space 
(which we will call a base space)
with some additional structure. 
Compared to manifolds, one allows more variety of local behaviour. 
On a manifold a map is a homeomorphism between $\mathbb{R}^n$ and some open set on a manifold. 
On an orbifold a map is a homeomorphism between a quotient of $\mathbb{R}^n$ by some 
finite group and some open set on an orbifold. 
In addition to that, the orbifold structure consist the informations about that finite group 
and a quotient map for any such open set. 

We can make an observation, that since in dimension 2, quatient of $\mathbb{R}^2$ by a finite 
group is topologically always either $\mathbb{R}^2$ or $\mathbb{R}\times\mathbb{R}_{\geq 0}$, 
we have that in dimension 2, the underlying Hausdorf space of any orbifold is a topological 
manifold (possibly with a boundary). For an orbifold $O$ we will call this underlying manifold $M$
a base manifold of $O$ and we will call $O$ an $M$-orbifold.  

\subsection{Good and bad orbifolds}
Above definition says that an orbifold is locally homeomorphic do the quotient of $\mathbb{R}^n$ 
by some finite group. 

When an orbifold as a whole is quotient of some finite group acting on a manifold we say, that 
it is 'good'. Otherwise we say, that it is 'bad'. 

%For this subsection we will also adopting notation from \cite{Thurston1979}. 

In two dimentions there are only four types of bad orbifolds, namely
(adopting notation from \cite{Thurston1979}): 
\begin{itemize}
\item $S^2(n)$ 
\item $D^2(;n)$ 
\item $S^2(n_1,n_2)$ for $n_1 < n_2$ 
\item $D^2(;n_1,n_2)$ for $n_1 < n_2$. 
\end{itemize}
All other orbifolds are good -- \cite{Thurston1979} (theorem 13.3.6).
%As manifolds are special case of orbifolds with all ...
%Write about isomorphism of all spectra

%$M$ - orbifold

Manifolds without boundary can be treated as orbifolds with trivial group for every map and will. 
Manifolds with boundary can be treated as orbifolds with trivial group on all maps from the 
interior and with group $\mathbb{Z}_2$ on the boundary, as described in \cite{Thurston1979} 
(example 13.2.2.).

\subsection{Therminology}
We differ from Thurston in the terms of naming points with maps with non-trivial groups. 
We will call them orbipoints. If the group acts as the group of rotations (so a 
cyclic group) we will call them rotational points. If the group is a dihedreal group we will 
call them 
dihedreal points. And if the point is on the boundry that stabilises reflection we will call it a 
reflection point. 

If a group associated to the orbipoint has degree $n$, we will say that tha orbipoint is 
of degree $n$. 
\subsection{Finite number of orbipoints}\label{finite number of orbipoints}
In this thesis we will consider only orbifolds with finitely many orbipoints and all orbifolds 
mentioned from this point are meant as such without further notice. Reason for this 
choice will be described in \ref{Eoc}.  

\subsection{Compactness}\label{compactness}
Orbifold as a topological space is the same as its base manifold.
We, would like to restrict our interest only to compact orbifolds. 
However, noncompact orbifolds, such as ones from \cite{Conway2016} (chapter 18) 
which are quotients 
of a group action of an infinite group acting on $\mathbb{R}^2$ (that will 
be also frequently interpreted in this setting as a hyperbolic plane $\mathbb{H}^2$), 
also interests us. We would like to accomodate some of noncompact orbifolds that 
will satisfy the condition simmilar to the \ref{finite number of orbipoints}. 
To do this we will slightly expand our definition of an orbifold. 

Let us start with a following construction.

For an noncompact orbifold $O$, let us take it's one point compactification. 
Let the compactification point be named $x_O$.  
For some set $X$, let $\#(X)$ be the number of connected components of $X$.
Now let us consider some connected open set $U \ni x_O$. The set $U\setminus\{x_O\}$ is 
not neccesserly connected.
 
Let 
\begin{equation}
C(O) \coloneqq \sup\limits_{\substack{U \ni x_O \\ \rm connected, \\ \rm open }} 
\#(U\setminus\{x_O\}).
\end{equation}
We will be interested only in the case, when $C(O)$ is finite.
If $C(O)$ is finite, we take some $U$ that realise the supremum and compactify each of 
connected components of $U\setminus\{x_O\}$ with a separate point. We will call these points 
"cusps". If for some cusp $x$, in every $U \ni x$ there are points from the boundary of $O$, 
we will call it a cusp on a boundary. In the other case, we will call it a cusp in the interior.

The result is a compact topological space. We will treat it as an orbifold, with 
cusps as orbipoits in extended definition. 
We extend the definition as such, that the map from the compactification of 
some open subset consisting point $x$ can go to the quationt of compactification of 
$\mathbb{H}^2$ by $S^1$. The group, we will take to act on $\mathbb{H}^2$ will 
be:
\begin{itemize} 
\item in the case of $x$ being in the interior -- infinite cyclic group $\mathbb{Z}$, 
where the generator acts as translation by 1 
in the half-plane model of hyperbolic plane   
\item in the case of $x$ being on the boundary -- infinite dihedral group $D_\infty$, where 
generators will be reflections with respect to vertical lines spaced by 1 in a half-plane 
model on a hyperbolic plane.
\end{itemize}

Note that in both cases, the is exactly one point in the compactification of $\mathbb{H}$, that is 
fixed for every elemnt of the acting group -- point from the compactification at 
the infinity on the top of the plane. This point is will be always mapped to the orbipoint $x$. 
We will call $x$ an, respectively, rotational, or dihedral orbipoint of infinite degree.

%Further implications of this extention of the definition will be 
%the group associated with a map of 
%a compactification  of a cusp is $\mathbb{Z}$ acting 

%However, there 
%is a class of noncompact orbifolds that interests us and we would like 
%to accomodate them. Noncompact orbifolds interests us that are quotiens 
%of a group action of an infinite group acting on $\mathbb{R}^2$ (that will 
%be also frequently interpreted in this setting as a hyperbolic plane $\mathbb{H}^2$). 
%Examples can be found in \cite{Conway}.
%welcome it in our discussion.
%quotiens of groups 
%\subsection{Maps between orbifolds}
%Map between orbifolds can be defined as follows:
%%\cite{kleiner2014geometrization}
\subsection{Equivalence}\label{sameness}
As stated in \cite{Conway16} (chapter 18) two two-dimensional orbifolds are the same iff they have:
\begin{itemize}
\item the same base manifold, 
\item the same number of orbipoints for each type and degree 
\item orbipoints on the boundaries 
in the pairwise same cyclic orders (up to orientation if the base manifold is non-orientable).
\end{itemize}
%From this we can treat that and that orbifolds as the same.

%Without boundary -- uniquely determined by orbipoints list. 

%With boundary -- in general not. 

\section{Euler (orbi)characteristic}\label{E_orb}
\label{\Eoc_as_a_sum}
%\todo{dać cytowanie do charakterystyki}
%quotiens
\subsection{Euler characteristic}
On CW-complexes we can define Euler characteristic as additive topological invariant 
normalised on symplexes.

On a CW-complex it is then an alternating sum of numbers of cells in 
a consecutive dimentions: 
\begin{equation}
\chi(C) = \sum_{d = 0}^n (-1)^d k_d
\end{equation}

Among other properties we have that if a compact manifold $N$ is a quotient of a compact 
manifold $M$, 
by the action of a finite group $G$, that acts properly discontinuously and freely on $M$, then
\begin{equation}
\chi(N) = \frac{\chi(M)}{|G|}
\end{equation}   
%we will treat 
%as we will treat manifolds as orbifolds we will always refer 
%we will 

%from of \Eoc on two dim orbifolds\label{\Eoc on 2d}

\subsection{\Eoc}\label{Eoc}\label{extended_Euler_orbicharacteristic}
We would like to extend the definition of Euler characteristic to orbifolds in a way 
that will reflect on their structre. 
We will call the resulting additive topological invariant the Euler orbicharacteristic 
and denote it by $\chi^{\rm orb}$.

%For spaces where Euler characteristic is already defined \Eoc\ will be the same as 
%Euler characteristic. 

Following \cite{Thurston1979} (definition 13.3.3.), but extending 
definition to orbifolds with cusps, we will define it as follows:
\begin{definition}
When an orbifold $O$ has a cell-division of the base space $X$, such that for each
open cell the group associated to
the interior points of a cell is constant, then the Euler number $\cho{O}$ is defined by
the formula:
\begin{equation}
\cho{O} \coloneqq \sum_{c_i} (-1)^{ \mathrm{dim}(c_i)}\frac{1}{|\Gamma(c_i)|},
\end{equation}
where $c_i$ ranges over all cells and $|\Gamma(c_i)|$ is the order of the group $\Gamma(c_i)$ 
associated to each cell, if $c_i$ is a cusp and $|\Gamma(c_i)| = \infty$ we put 
$\frac{1}{|\Gamma(c_i)|} = 0$. We will call $\frac{1}{|\Gamma(c_i)|}$ a weight of a cell $c_i$.
\end{definition} 

This definition results in the propertie, that if an orbifold $O_2$ is a quotient 
of a orbifold $O_1$, by the action of the finite group $G$, that acts properly 
discontinuosly, but not neceserly freely on $O_1$, then:
\begin{equation}
\cho{O_2} = \frac{\chi(O_1)}{|G|}.
\end{equation}
For a two dimentional orbifolds, the possible cells with weights different than $1$ are 
only in dimensions $0$ and $1$. In dimention 0 they are rotational and dihedral 
orbipoints. In dimention 1 they are fragments of the boundary that stabilises reflections. 
The weights of these cells are (with the convention that $\frac{1}{\infty} = 0$):
\begin{itemize}
\item for a rotational point of degree $n$, the weight is $\frac{1}{n}$,
\item for a dihedral point of degree $n$, the weight is $\frac{1}{2n}$,
\item for a reflection line, the weight is $\frac{1}{2}$.
\end{itemize}
We can see that weights of rotational and dihedral orbipoints are monotonously decrising 
and converges to $0$, as degree diverges to infinity. Moreover, the 
cusps -- rbipoints of an inifinte degree, 
that are stabilised by a groups of infinite degree, has weight $0$.

From this we will obtain the formula for an Euler orbicharacteristic of a two dimensional 
orbifold with rotational points of degrees $r_1, r_2, \cdots, r_n$ and dihedral points 
of degrees $d_1, d_2, \cdots, d_m$:
\begin{align}\label{chi orb expression}
\cho{O} &= \chi (M) - n + \sum_{i=1}^n \frac{1}{r_i} - \frac{m}{2} + \sum_{j=1}^m \frac{1}{2d_j} \\
&= \chi (M) - \sum_{i=1}^n \frac{r_i-1}{r_i} - \sum_{j=1}^m \frac{d_j-1}{2d_j}
\end{align}
For $O$ with only rotational orbipoints:
\begin{equation}\label{chi orb expression rotational}
\cho{O} = \chi (M) - \sum_{i=1}^n \frac{r_i-1}{r_i}.
\end{equation}
For $O$ with only dihedral orbipoints:
\begin{equation}\label{chi orb expression dihedral}
\cho{O} = \chi (M) - \sum_{j=1}^m \frac{d_j-1}{2d_j}.
\end{equation}

From these formulas we can see, that as number of orbipoints diverges to infinity, the \Eoc\ 
diverges to minus infinity. For this reason, we restric ourselves only to orbifolds 
with finitely many orbipoints.
\begin{observation}\label{orbifolds have smaller Eoc than their base manifolds}
A $M$-orbifold that is different than $M$ always have strictly smaller \Eoc\ than $M$. 
\end{observation}

%\subsection{Extended Euler orbicharacteristic}\label{extended_Euler_orbicharacteristic} 
%%(with cusps)
%%Write about cusp as a limit.
%We can extent our definition of \Eoc\ to orbifolds with cusps. 

\section{Metric structures on the orbifolds and volumes of the orbifolds}
%\subsection{Metric structure on the orbifolds}
As described and proved in \cite{Thurston1979} (13.3.6.) all good orbifolds have either:
\begin{itemize}
\item Elliptic structure, if $\cho{O} > 0$,
\item Parabollic structure, if $\cho{O} = 0$,
\item Hyperbolic structure, if $\cho{O} < 0$.
\end{itemize}
With an elliptic or a hyperbollic metric structre, we can meassure the volume of the orbifold as 
\begin{equation}
V(O) = |\int_O K dA|
\end{equation} 
Also, as stated in \cite{Thurston1979} (13.3.5.), the Gauss-Bonnet theorem works also for orbifolds 
and we have that:
\begin{equation}
\int_O K dA = 2\pi|\cho{O}|,
\end{equation}
thus having that 
\begin{equation}\label{volume to chi}
V(O) = 2\pi|\cho{O}|.
\end{equation}
Our main goal in this thesis is to describe possible volumes of two dimensional orbifolds, 
especially thouse with negative \Eoc. As from \ref{volume to chi} we have direct correspondence 
between volumes and \Eoc s, and with \Eoc, we can restrict ourselves to rational numbers, 
we will try to describe possible \Eoc\ of two dimensional orbifolds.  

\section{Classification of two dimensional manifolds}\label{2 dim manifolds}
%\todo{twierdzenie o klasyfikacji powierzchni}
In this thesis we aim to better recognise possible two dimensional orbifolds. 
The fundament that we are relaying on is the classification of two dimensional manifolds. 

It is phrased as a well known theorem:
\cite{}


%\section{Classification of orbifolds}
%The list of all orbifolds with non-negative Euler orbicharacteristic
%%Powiedzieć coś o tym, że orbicharatkeryttyka odpowiada polom (Gauss Bonett itd.)
%\cite{}
%\subsection{Non-negative \Eoc}



\section{Therminology and notation}\label{therminology and notation}
\subsection{Orbifold notation}
For the rest of this thesis we will use slightly modified notation from \cite{Conway2016}.
%"feature"
%Write about manifold Euler characterestic
%We treat manifolds and orbifolds as a sphere with some features added by the operations.
As we know from \ref{sameness}, two dimentional orbifold can be defined 
by specifying:
\begin{itemize} 
\item its base manifold, 
\item the list of degrees of rotational points (we will 
always write them in the decreasing order) 
\item the list of the lists of dihedral points for each boundary component
\end{itemize}
Furthermore from \ref{2 dim manifolds} we know, two dimensional manifold 
can be 
defined, by specifying:
\begin{itemize}
\item how many boundary components it has,
\item how many handles it has,
\item how many cross-caps it has.
\end{itemize} 
\todo{dopisać}

When we will write $\Delta(feature)$ we would mean the difference in \Eoc

%dać na sferę $\varepsilon$ słowo puste.
%We will regard parts of that notation not only as features on an orbifold but also as 
%an operations 
%on orbifolds transforming one to another by adding particular feature. \\
We will denote the difference in Euler characteristic which is made by modifying 
an orbifold by such a feature as $\Delta(modification)$.
%\todo{rozwinąć} 
%dopisać, że w Conwayowej >= 2

\subsection{Expressions involving $\infty$}
If not stated othewise, in the expressions containing $\infty$ symbol, their value is understood 
as $\varphi(\infty) \coloneqq \lim\limits_{n\to \infty}\varphi(n)$. Only expressions 
where such limits exists will occour without futther notice.
%\todo{}

\subsection{Addition of sets and numbers}
For $A, B \subseteqq \mathbb{R}$, we define 
\begin{equation}
A+B \coloneqq \{a+b\ |\ a \in A, b\in B\}.
\end{equation}

\subsection{Two dimensionality}
From this point, throught the whole thesis we will consider only two dimensional manifolds and 
orbifolds, for this reason words "two-dimensional" will be sometimes ommited, nevertheless 
from this poinnt we will always mean only two dimensional manifolds and orbifolds.  

%delta
%h c b

%$\srS, \sdD$ ???

\section{Spectra}\label{spectra}
We will call the set of all possible \Eoc\ of a $M$-orbifolds, the spectrum of $M$ and 
we will denote it by $\spebr{M}$. We will denote the set of all possible \Eoc\ of a $M$-orbifolds 
that have only rotational orbipoints by $\spe^r(M)$. 
We would denote the set of all possible \Eoc\ of a $M$-orbifolds 
that have only dihedral orbipoints by $\spe^d(M)$, but from this section, it follows that 
we always have $\spe^d(M) = \spebr{M}$. 

We will also denote the sum of spectra of all two dimentional manifolds by:
\begin{equation}
\spe \coloneqq \bigcup_{M\textrm{ : 2d manifold}} \spebr{M}.
\end{equation}
This will be the main interest of this thesis. 
%$\spe$. dadada

Now we want to derive the form of the $\spebr{M}$.
For a manifold $M$ with $h$ handles, $c$ cross-cups and $b$ boundary components, it's 
Euler characteristic is given by:
\begin{equation}
\chi(M) = 2-2h-c-b.
\end{equation}
The set of $\Delta$ for possible orbifold features are:\\
$\bullet$ for $b\neq 0$:
\begin{equation}
\{-\frac{n-1}{2n}\ \big|\ n\in\mathbb{N}_{>0}\cup \{\infty\}\}
\end{equation}
$\bullet$ for $b = 0$:
\begin{equation}
\{-\frac{n-1}{n}\ \big|\ n\in\mathbb{N}_{>0}\cup \{\infty\}\}.
\end{equation} 
%with the convention, that $\varphi(\infty) \coloneqq \lim\limits_{k\to \infty} \varphi(k)$.

Thus, we have that the form of the spectrum of two dimensional manifold $M$ is:\\
$\bullet$ for $b\neq 0$: 
\begin{equation}
\spe(M) = \chi(M) - \left\{\sum\limits_{i=1}^n\frac{d_i-1}{2d_i}\ 
\big|\ n\in\mathbb{N}_0,\ d_i\in\mathbb{N}_{>0}\cup \{\infty\}\right\}
\end{equation}
$\bullet$ for $b = 0$:
\begin{equation}
\spe(M) = \chi(M) - \left\{\sum\limits_{i=1}^n \frac{r_i-1}{r_i}\ \big|\ n\in\mathbb{N}_0,\ 
r_i\in\mathbb{N}_{>0}\cup \{\infty\}\right\}.
\end{equation} 
\label{two dim manifold spectrum}


\begin{observation}\label{2times homeomorphism}
We have that $\speS = 2\speD$.
\end{observation}
\subsubsection{Proof.}

Indeed, since: 
\begin{align}
\spe(S^2) = 2 -\left\{\sum\limits_{i=1}^n \frac{r_i-1}{r_i}\ \big|\ n\in\mathbb{N}_0,\ 
r_i\in\mathbb{N}_{>0}\cup \{\infty\}\right\}
\end{align}
and 
\begin{align}
\spe(D^2) = 1 - \left\{\sum\limits_{j=1}^m\frac{d_j-1}{2d_j}\ 
\big|\ m\in\mathbb{N}_0,\ d_j\in\mathbb{N}_{>0}\cup \{\infty\}\right\}._\square
\end{align}

\begin{observation}\label{all_spectra_are_isomorphic}\label{spe_M}
For every two dimentional manifold $M$, we have that $\spe(M)$ is homeomorphic to $\speD$. 
%For $M$ with $h$ handles, $c$ cross-cups and $b$ boundary components, 
This homeomorphism is: \\
$\bullet$ for $b \neq 0$:
\begin{equation}
\spe(M) = \speD + \chi(M) - 1, 
\end{equation}
$\bullet$ for $b = 0$:
\begin{equation}
\spe(M) = 2\speD + \chi(M) - 2.
\end{equation}  
\end{observation}

%Ok, all isomorphic with $\speD$ \\
%here write about it \\
%Tell me about it!
\subsubsection{Proof.}
For a manifold $M$ with $h$ handles, $c$ cross-cups and $b$ boundary components, it's 
$\spebr{M}$ is given by:\\
$\bullet$ for $b\neq 0$: 
\begin{equation}
\spe(M) = \chi(M) - \left\{\sum\limits_{i=1}^n\frac{d_i-1}{2d_i}\ 
\big|\ n\in\mathbb{N}_0,\ d_i\in\mathbb{N}_{>0}\cup \{\infty\}\right\}
\end{equation}
$\bullet$ for $b = 0$:
\begin{equation}
\spe(M) = \chi(M) - \left\{\sum\limits_{i=1}^n \frac{r_i-1}{r_i}\ \big|\ n\in\mathbb{N}_0,\ 
r_i\in\mathbb{N}_{>0}\cup \{\infty\}\right\}.
\end{equation} 

On the other hand, we have that:
\begin{equation}
\speD = 1-\left\{\sum\limits_{i=1}^n\frac{d_i-1}{2d_i}\ 
\big|\ n\in\mathbb{N}_0,\ d_i\in\mathbb{N}_{>0}\cup \{\infty\}\right\}
\end{equation}
\begin{equation}
\speS = 2 - \left\{\sum\limits_{i=1}^n \frac{r_i-1}{r_i}\ \big|\ n\in\mathbb{N}_0,\ 
r_i\in\mathbb{N}_{>0}\cup \{\infty\}\right\}
\end{equation}
and
\begin{equation}
\speS = 2\speD.
\end{equation}
From this, the observation follows immedietly. $_\square$

\begin{observation}
For every manifold $M$, for every $x \in \spebr{M}$, we have that $x \leq \chi(M)$.
\end{observation}

%\todo{dopisać, że spectrum jest poniżejchi rozmaitości}

\section{Egyptian franctions}
Egyptian frantion is a finite sum of fractions, all with numerators one. 
Most of the time it is also required, that the fractions in the sum have pairwise distinct 
denominators. We will however take less usual convention and will drop that requirement, 
calling an egyptian fraction any sum of unitary fractions. 
%\section{Translating questions to ones about Egyptian fractions}\label{Egyptian_fractions}
\subsection{Connection between spectra and Egyptian fractions}\label{Egyptian_fractions}
The terms $-\frac{r_i-1}{r_i}$ in the sum \ref{chi orb expression rotational} 
can be expressed as $-1+ \frac{1}{r_i}$ 
and the term $-\frac{d_j-1}{2d_j}$ in the sum \ref{chi orb expression dihedral} can be expressed as 
$-\frac{1}{2} + \frac{1}{2d_j}$. 
Then the sums become:
\begin{equation}\label{Egyptian S2 sum}
\chi(M) - n + \underbrace{\sum_{i=1}^n \frac{1}{r_i}}_{
\substack{\textrm{Egyptian} \\ \textrm{fraction}}}
\end{equation}
and
\begin{equation}\label{Egyptian D2 sum}
\chi(M) - \frac{m}{2} + \frac{1}{2}
\underbrace{\sum_{j=1}^m \frac{1}{d_j}}_{
\substack{\textrm{Egyptian} \\\textrm{fraction}}}.
\end{equation}

%In expressions there are 
In this form, the egyptian fractions are explicitly present in expresions of 
points in $\spebr{M}$.
%This is very simmilar to the notion of the egyptian franction, 
%which is a rational number expressed as a sum of a fractions with numerator $1$ and 
%different denominators. Sometimes the "distinct denominators" condition is dropped and we are 
%following that convention in this thesis. 

The $-n$ and $-\frac{m}{2}$ terms provide constraints on the number of fractions that 
can apprear in the sum.
%The particular interest is in translating questions related to spectra to the questions 
%of egyptian franctions. 
%What the data of the spectra imposes is the number of fractions to be summed. 

%We are interested in translating 
We will now translate the questions of being in the spectrum 
to the questions of being expressable as egyptian fraction with the particular number 
of summands. It will be used in \ref{saturation theorem} and \ref{unboundness}.

%We can make following corollaries:
%\begin{corollary}\label{from Egyptian fractions}
%If $x$ can be expressed as an egyptian fraction with $n$ summands, then 
%\begin{equation}
%2 - n + x \in \speS
%\end{equation}
%and
%\begin{equation}
%1 - \frac{n}{2} + \frac{x}{2} \in \speD.
%\end{equation}
%\end{corollary}
%\begin{corollary}
%If $y \in \speS$ as an \Eoc of an orbifold which all orbipoints are $n$ rotational orbipoints, 
%then 
%\begin{equation}
%y + n - 2
%\end{equation} 
%can be expressed as an egyptian fraction with $n$ 
%(not nessecerely distincs) summands. 
%
%If $y \in \speD$ as an \Eoc of an orbifold which all orbipoints are $m$ rotational orbipoints, 
%then \begin{equation}
%2y + \frac{m}{2} - 1
%\end{equation}
%can be expressed as an egyptian fraction with $m$ 
%(not nessecerely distincs) summands. 
%\end{corollary}
%We may now expand our corollaries about egyptian fractions to arbitrary $M$:

We will now state two corrolaries that follows immediately from the 
form of expressions \ref{Egyptian S2 sum} and \ref{Egyptian D2 sum}, and from 
\ref{two dim manifold spectrum}.
\begin{corollary}\label{from Egyptian fractions}
If $x$ can be expressed as an egyptian fraction with $n$ summands, then for any two dimensional 
manifold $M$ we have: 
\begin{equation}
\chi(M) - n + x \in \spebr{M}
\end{equation}
and, if $M$ has at least one boundary component also:
\begin{equation}
\chi(M) - \frac{n}{2} + \frac{1}{2}x \in \spebr{M}.
\end{equation}
\end{corollary}
\begin{corollary}\label{to egyptian fractions}
If for some two dimensional manifold $M$ we have that $y \in \spebr{M}$ as an \Eoc\ of 
an orbifold which has $n$ rotational orbipoints and not any other, 
then 
\begin{equation}
y + n - \chi(M)
\end{equation} 
can be expressed as an egyptian fraction with $n$ 
(not nessecerely distincs) summands. 

If for some two dimensional manifold $M$ with at least one boundary component 
we have that $y \in \spebr{M}$ as an \Eoc of an orbifold which has $m$ dihedral orbipoints and 
not any other, 
then 
\begin{equation}
2y + \frac{m}{2} - 2\chi(M)
\end{equation}
can be expressed as an egyptian fraction with $m$ 
(not nessecerely distinct) summands. 
\end{corollary}

\section{Operations on orbifolds}\label{Operations}
As stated in \ref{therminology and notation} we will often see two dimentional orbifold 
as a sphere $S^2$ with a collection of features.
Throughout the thesis we will frequently refer to performing the "operation" on an orbifold 
consisting of removing and adding thouse features. 
What we mean by this is giving as a result of an operation on one orbifold, defined 
by some list of features on a sphere, another one 
with a modified list of features according to the described operation. 

When we will be talking about "adding" or "removing" a feature from an orbifold, we 
will mean adding or removing this feature from a list defining this orbifold and 
taking the orbifold defined by resultin list as a result of the operation.
 
As our main interest is to figure out, for a given rational number 
$\frac{p}{q}$ which orbifolds have $\frac{p}{q}$ as their \Eoc\ and for a given 
orbifold $O$, which orbifolds 
have the same \Eoc\ as $O$, we will be particulary interested in such operations 
that do not change \Eoc, which will be used in \ref{sufficiency of D2 and S2}.



%One of a usefull approuches will be to treat 
%Write about the general 
%operations we are interested in i.e. taking any number of features (handles 
%cross caps, parts of boundry components with orbipoints on it, orbipoints in the interior) and
%replacing it by any other feauters
%(Some preserve the area)
%Write about operations nesseserie for reduction of cases
%write that every operation reduces \Eoc.

%This does not have any particular deep geometrical meaning

\label{moving from interior to boundary}


\section{Questions asked}
There will be two main parts of question: 

$\bullet$ Ones regarding $\spe$ as a set, where we will be asking 
of its order type and topology and relation to other sets such as $\speD$ and $\speS$. 
We will focus on these questions in \ref{order structure}. 

$\bullet$ Ones regarding $\spe$ as an image of a $\chi^{orb}$, sending orbifolds to their \Eoc s. 
There, we will ask for example how namy orbifolds have particular \Eoc\ and 
related questions. We will focus on these questions in the chapter \ref{counting occurrences}.  

%Reduction presented in \ref{reduction_to_arithmetical} are with . 






../LaTeX/TeX_files/definition_of_an_orbifold.tex
% mainfile: ../praca_magisterska_orbifoldy.tex
\chapter{Characteristics, classification and properties of the orbifolds}
\section{Euler orbicharacteristic}
\subsection{Classification of orbifolds with non-negative Euler orbicharacteristic}
\subsection{Extended Euler orbicharacteristic}
\section{Uniformisation theorem (formulation)}
\section{Surgeries, modifications and constructions on orbifolds}
(Some preserve an area)





% mainfile: ../praca_magisterska_orbifoldy.tex
%\synctex=1
\chapter{Order structure}

In this chapter we will discuss order type of the set of all possible Euler orbicharacteristics 
of two dimentional orbifolds. \\
For now, until Chapter \ref{Counting_occurrences} Counting occurrences, we will not pay attension 
to how many orbifolds have the same Euler orbicharacteristic. \\ 
Because of that and since Euler orbicharacteristic does not depend on the cyclic order 
of points on the conponents of the boundry we introduce an extension of a notation from 
\cite{Conway2008}. 

%We will write $*\{a,b,c,d,\dots\}$ to denote not a particular orbifold, but a type 
%of orbifold that have a cones on a component of the boundry of orders $a,b,c,d,\dots$, but 
%in any order. \\ 

We will write $*\{a,b,c,d,\dots\}$ to denote a type of a boundry (of an orbifold) that have 
kaleidoscopic points of periods $a,b,c,d,\dots$, but in any order. \\


From what we wrote above (that Euler orbicharacteristic does not depend on the cyclic order 
of points on the conponents of the boundry), we can see that Euler 
orbicharacteristic is well defined 
%on such types of orbifolds. \\ 
when we specify only such a type of the components of the boundry of an orbifold and not 
a particular cyclic order.  \\

\section{Reductions of cases}
Now we want to make some reductions to limit number of cases that we will be dealing with. \\
For this chapter we will consider orbifolds according to a definition from 
(\ref{Disk_and_sphere_with_defects}). \\
Let us observe, that:
\begin{align*}
\hspace{3cm}\Delta(\circ) =& \hspace{-1cm} &-2& \hspace{-1cm} &= \Delta(*(^*2)^4) \\
\hspace{3cm}\Delta(*) =& \hspace{-1cm} &-1& \hspace{-1cm} &= \Delta((^*2)^4) \\
\hspace{3cm}\Delta(n) =& \hspace{-1cm} &\frac{n-1}{n}& \hspace{-1cm} &= \Delta((^*n)^2)
\end{align*}

From this we can conclude, that every Euler orbicharacteristic can be obtained 
by an orbifold of signature of a type ($n$ and $m$ are arbitrary):

\begin{align*}
I_1I_2\dots I_n & \textrm{or} \\
*b_1b_2\dots b_m &.
\end{align*}

Let us denote the set of all possible Euler orbicharacteristics of orbifolds of the form 
$I_1I_2\dots I_n$ by $\sIS$ and the set 
of all possible Euler orbicharacteristics of orbifolds of the form $*b_1b_2\dots b_m$ 
as $\sbD$

Let us observe that the topological structure of $\sIS$ and $\sbD$ are the same since 

\begin{equation*}
2\sbD=\sIS
\end{equation*}
So multiplying by $2$ is the homeomorphism. 
\section{Determining the order structure}
In this chapter we will justify, that the order type of all possible Euler orbicharacteristics 
of two dimentional orbifolds is $\omega^\omega$. 
We will also describe precisely where condensation points lie and of which order 
(see below \ref{condensation_points_definitions}) they are.
\subsection{Definitions regarding order of condensation points}
\label{condensation_points_definitions} 
We start with one technical definition of "transitive order" that will be almost what we want
and then, there will be the the definition of "order", which is the definition that we need. \\ 
\begin{definition}
(Inductive). 
%We say, that the point $a$  from the topological space $X$ is a condensation 
%point of the transitive order 0, when
We say that the point is a condensation point of a transitive order $0$, when it is 
an isolated point. 
We say that the point is a condensation point of a transitive order $n + 1$, when it is 
a condensation point (in the usual sense) of the condensation points of the transitive order $n$. 
\end{definition}  
The only issue of the definition is that the point of the transitive order $n$ is also a point 
of the transitive order $k$, for all $0< k \leq n$. We want a definition of order such that 
for any point, there is at most one integer that is its order. So we define:
\begin{definition}
We say that the point is a condensation point of order $n$ iff it is a condensation point 
of the transitive order $n$ and it is not a condensation point of the transitive order $n+1$. 
If the point is a condensation point of the transitive order for an arbitraly large $n$ we say that 
the  point is a condensation point of order $\omega$.
\end{definition}

\subsection{$\sbD$}
\subsubsection{Some preliminary observations.}
Let us observe, that $\lim\limits_{n \to \infty} \Delta(^*n) = -\frac{1}{2}$. From that, we see, 
that for every point $x \in \sbD$, the point $x - \frac{1}{2}$ is a condensation point. 
Let us observe, that also, for every point $x \in \sbD$, we have that $x - \frac{1}{2} \in \sbD$, 
because $\Delta((^*2)^2) = -\frac{1}{2}$. \\

Now we will show that the order type of $\sbD$ is $\omega^\omega$ and where exactly are 
its condensation points of which orders. For this we will use  
a handfull of lemmas. 

\begin{lemma}
If $x$ is a condensation point of the set $\sbD$ of order $n$, then $x-\frac{1}{2}$ is a
 condensation point of the set $\sbD$ of order at least $n+1$. 
\end{lemma}
\textbf{Proof.} \\
Inductive. \\
$\bullet$ $n = 0$: If $x$ is an isolated point of the set $\sbD$, then $x \in \sbD$. From that, we 
have, that points $x - \frac{k-1}{2k}$ are in $\sbD$, from that, that $x-\frac{1}{2}$ is a 
condensation point of $\sbD$. \\
$\bullet$ inductive step: Let $x$ be a condensation point of the set $\sbD$ of an order $n > 0$. 
Let $a_k$ be a sequence od condensation points of order $n-1$ convergent to $x$. From the 
inductive assumption, we have, that $a_k - \frac{1}{2}$ is a sequence of condensation points 
of order at least $n$. From the basic sequence arithmetic it is convergent to $x-\frac{1}{2}$. 
From that, we have that $x-\frac{1}{2}$ is a condensation point of the set $\sbD$ of order 
at least $n+1$. $_\square$
\begin{lemma}
If $x$ is a condensation point of the set $\sbD$ of order $n+1$, then $x+\frac{1}{2}$ is 
a condensation point of the set $\sbD$ of order at least $n$.  
\end{lemma}
\textbf{Proof.} \\
Inductive \\
$\bullet$ n = 0: 







% mainfile: ../praca_magisterska_orbifoldy.tex
\chapter{Algorithm for counting occurences}\label{algorithm}


\section{Counting occurences algorithm}

%We want to determine this $n$. If $n = 0$, then $\frac{p}{q}$ is not in $\sigma$. 
%If $n > 0$, then $$
%\subsection{Deciding number of occurences}
Searching for all occurences 

The difficulty here is to carefully step other an occurence. 

Compared to the previous version, we also use an occurance counter, starting with it set to 0 
and with the list of orbifolds, wich is empty at the start.
\begin{lstlisting}[firstnumber=1,consecutivenumbers=true]
In the case, the flag is set to: 
{
    "Less", then 
    {
        We increase the pivot counter by one ($b_c \coloneqq b_c + 1$).
        If $b_c = 2$ and the values of all the counters 
        on the left are also equal 2 then 
        {
            We end the whole algorithm with the answer "no".
        }
        We set the value of all counters on the left to $b_c$
        If $\chi^{orb}(*b_1b_2b_3\dots)=\frac{p}{q}$ then
        {
            We found an orbifold, we add it to a list 
            and increase the occurence counter by 1. 
            We set the flag to "Less".
            We put pivot to the $c+1$ counter.
            We go to the line 1..
        }
        If $\chi^{orb}(*b_1b_2b_3\dots)>\frac{p}{q}$ then  
        {
            We set the flag to "Greater".
            We put the pivot on the first counter. 
            We go to the line 1..
        } 
        If $\chi^{orb}(*b_1b_2b_3\dots)<\frac{p}{q}$ then
        {
            We set the flag to "Less".
            We put pivot to the $c+1$ counter.
            We go to the line 1..
        } 
    }

    "Greater", then
    {
        If $\chi^{orb}(*b_1\dots b_{c-1}\infty b_{c+1}\dots)=\frac{p}{q}$ then
        {
            We found an orbifold, we add it to a list 
            and increase the occurence counter by 1. 
            We set the flag to "Less".
            We put pivot to the $c+1$ counter.
            We go to the line 1..
        } 
        If $\chi^{orb}(*b_1\dots b_{c-1}\infty b_{c+1}\dots)>\frac{p}{q}$ then
        {
            We set $b_c$ to $\infty$.
            We set the flag to "Greater".
            We move pivot to the $c+1$ counter.
            We go to the line 1..
        }  
        If $\chi^{orb}(*b_1\dots b_{c-1}\infty b_{c+1}\dots)<\frac{p}{q}$ then
        {
            We search for value $b_c'$ of the $c$ counter 
            such that $\chi^{orb}(*b_1\dots b_{c-1}b_c'b_{c+1}\dots)\leq\frac{p}{q}$ 
            and $\chi^{orb}(*b_1\dots b_{c-1}(b_c'-1)b_{c+1}\dots)>\frac{p}{q}$.
            // More on how we search for it will be told later, 
            // for now we can think that we search one by one,
            // starting from $b_c$ and going up till $b_c'$.
            We set $b_c$ to $b_c'$.
            if $\chi^{orb}(*b_1b_2b_3\dots)=\frac{p}{q}$ then 
            {
                We found an orbifold, we add it to a list 
                and increase the occurence counter by 1. 
                We set flag to "Less".
                We go to the line 1..
            }
            We set all the counters to the left to value $b_c$.
            if $\chi^{orb}(*b_1b_2b_3\dots)=\frac{p}{q}$ then 
            {
                We found an orbifold, we add it to a list 
                and increase the occurence counter by 1. 
                We set flag to "Less".
                We move the pivot to the column $c+1$.
                We go to the line 1..
            }
            If $\chi^{orb}(*b_1b_2b_3\dots)<\frac{p}{q}$ then 
            {
                We set flag to "Less".
                We move the pivot to the column $c+1$.
                We go to the line 1..
            }
            If $\chi^{orb}(*b_1b_2b_3\dots)>\frac{p}{q}$ then 
            {
                We set the flag to "Greater".
                We move the pivot to the first counter.
                We go to the line 1..
            }
        }  
    }
}
\end{lstlisting}

\section{Why this works}

\section{Deciding the order}
Let $m \in \mathbb{N}$ be such that $\frac{p}{q} \in (1-\frac{m}{2},1-
\frac{m+1}{2})$
Let us denote by $r \coloneqq \frac{p}{q} - (1-\frac{m}{2})$. \\ 

We will searching in $\sigma$ as such: \\

If $\frac{p}{q} \in \sigma$, then, from the corollary \ref{predescors} we know, that there 
exist some $n \in \mathbb{N}$, such that $\frac{p}{q} + \frac{n}{2} \in \sigma$ but 
$\frac{p}{q} + \frac{n}{2} \not\in \sigma$. \\

We will be consequently checking points from $1+r$, through $1+r-\frac{l}{2}$, for 
$0 \leq l \leq m$, to the $\frac{p}{q}$. We stop at the first found point. 
If one of these point is in the spectrum, then all smaller (so also $\frac{p}{q}$) are in 
the spectrum and $\frac{p}{q}$ is the accumulation point of the spectrum of order $m-l$ 
(from this, 
we can see some heuristic, that the points that have smaller order will be generally 
harder to find in some sense). If none of this points are in in the spectrum, then $\frac{p}{q}$ 
is not. \\




\section{Implementation}
As an appendix in the separate document, there is a source of a program with implementation 
of this algorithm 
with full  
enhancments described in this chapter. It is written in Rust. 
It can be also found on \smalltodoII{dać ref do github} along with the \LaTeX\ source of 
this thesis.
%It is in the separate file, as it would take too much space in this 
%document and wouldn't be readable. 




% mainfile: ../praca_magisterska_orbifoldy.tex
\chapter{Counting occurrences} \label{counting occurrences}
Our ultimate goal is to give the answer to the questions such as: \\
- For a given $x \in \sigma$, how many orbifolds have $x$ as their \Eoc?\\
- Why? Is there some underlying geometrical reason for that?\\
- Can we characterise points $x \in \sigma$ that has the most orbifolds corresponding to them? \\
- Is there any reasonable normalisation to counter the effect that there are 'more' points as we go 
to lesser values. (What we mean by 'more' was sted in) \\
The first equstion we can tackle is steaming from the chapter \ref{order structure} 
and it is -- Do $\sbD$ and $\sIS$ coincide? It is easy to answer that $\sbD \neq \sIS$ 
(and we will do that along some harder questions in the moment), but do they coincide 
starting from a sufficiently distant point? Or maybe, for every denominator, do they coincide 
from a sufficiently distatn point? (Yes.) \\
\section{Finitenes}
First we will show that for any $x \in \sigma$ there are always only finitely many orbifolds 
with an \Eoc\ equal to $x$. \\ 
Let us observe, that we only need to show this for $S^2$ orbifolds. It is like that, because, 
as discussed in \ref{surgeries} every orbifold can be obtained be modyfying the sphere and 
%the set of differences in \Eoc\ made by
%all possible modifications that are not adding orbipoints is bounded. 
there is only finitely many possible modifications that are not adding an orbipoint, each 
changing \Eoc\ by non-zero value. \\ 

\begin{theorem}
For any $x \in \sigma$ there are always only finitely many orbifolds 
with an \Eoc\ equal to $x$.
\end{theorem}
\textbf{Proof:} \\
% Suppouse, that there exists an orbifold with an \Eoc\ qual to $x$. Then, 
According to the note above, we only need to proof this for $S^2$ orbifolds. \\ 
Let $x$ be a rational number. 
For the sake of contradiction, assume, that there exists an infinite family of orbifolds 
$\{\mathcal{O}\}_{i \in I}$ with an \Eoc\ of each qual to $x$. For each $i$, tet $m_i$ be the 
order of the orbipoint with the highest order of $\mathcal{O}_i$. As for every $n \in \mathbb{N}$ 
there are only finitely many $S^2$ orbifolds with all orbipoints of order less than $n$, we have 
that the set $\{m_i\}_{i \in I}$ is unbounded. Let $\{m_n\}_{n\in \mathbb{N}}$ be some strictly 
increasing sequence 
of elements of $\{m_i\}_{i \in I}$ that diverges into infinity. \\
Let $\{a_n\}$ be the sequence of differences in \Eoc\ caused by points corresponding to $\{m_i\}$. 
Let $\{b_n\}$ be the sequence of differences in \Eoc\ caused by other points on those orbifolds. 
So for every $n$ we have $\cho{\mathcal{O}_n} = 2 + a_n + b_n$. As $\{m_n\}$ is strictly 
increasing we have that $a_n$ is strictly decreasing, so $b_n$ must be strictly 
increasing, because $\cho{\mathcal{O}_n}$ is constant for all $n$ (all $\{\mathcal{O}_n\}$ 
are from the family with \Eoc\ equal to $x$). \\ 
But $\{b_n\} \subseteq \sIS - 2$, so it is well ordered as $\sIS$ is well ordered. 
From \ref{well_order} and \ref{times_two_fact} we know that $\sIS$ has no infinite 
strongly increasing sequences, so 
$\sIS - 2$ has no infinite strongly increasing sequences. That gives us a contradiction. 
\Lightning $_\square$ 
 
\section{Some connections between \Eoc\ and geometry of corresponding orbifolds}
Here we will state some observations and corollaries derived from previous chapters 
about ... 
\begin{observation}
If an \Eoc\ is an accumulation point of order $n$ in $\sbD$ \dsa{$\sIS$}, there exist an 
orbifold of the type ... \dsa{} with $n$ cone \dsa{gyration} points of that \Eoc. 
\end{observation}
prrof. from chapter 3. (todo: dopisać)
\begin{observation}\label{adding_multiplied_differences}
If $x \in \sbD$ \dsa{$\sIS$}, then $1-x$ \dsa{$2-x$} is a difference in \Eoc\ resulting 
from some set of cone \dsa{gyration} points. From that $1-n(1-x) \in \sbD$ \dsa{$2-n(2-x)\in \sIS$} 
for all $n \in \mathbb{N}$. 
\end{observation}
%\begin{observation}

%\end{observation}
\section{$\sbD$ and $\sIS$}
In this section we would like to develop the tools and answer some questions about 
%interrelationships
relations between $\sbD$ and $\sIS$. \\
The first, stated in \ref{order structure} is that $2\sbD=\sIS$. 
This tells us all about simmilarities of their topological structures -- namely, they are the same, 
but it does not directly answers questions about how they lie in $\mathbb{R}$, relative 
to each other. 
\subsection{$-\frac{1}{84}$ and $-\frac{1}{42}$}
//Why it is how it is//
%\subsection{All the accumulation points of the $\sIS$ are in $\sbD$}
\subsection{Accumulation points of the $\sIS$}
\begin{theorem}
All accumulation points of the $\sIS$ are in $\sbD$.
\end{theorem}
There are two proofs of this theorem showing nice correnpondence -- one arithmetical and 
one geometrical. 
\\
\textbf{Proof I.}
%\subsubsection{Arithmetical reason}
\textbf{Arithmetical reason} \\
We assume that $x \in \sIS$ is an \apots\ $\sIS$.\\
%If $x \in \sIS$, then $\frac{x}{2} \in \sbD$. Then $1 - \frac{x}{2}$ 
%is a difference in \Eoc\ resulting from some set of cone points. We can add to the disc twice as 
%many cone points and resulting orbifold $\mathcal{O}$ will have an \Eoc\ equal to 
%$1 - 2(1-\frac{x}{2}) = x - 1$. From \ref{third_order_lemma} for the thesis it is sufficient 
%to $x - 1$ to be an \apots\ $\sbD$ of order at least two. \\
%%$2-x$ is a difference in \Eoc\ resulting from some set of gyration points. We can 
%We asumed that $x \in \sIS$ is an \apots\ $\sIS$, so, 
By \ref{times_two_fact} we have, that 
$\frac{x}{2} \in \sbD$ is an \apots\ $\sbD$. From \ref{third_order_lemma} we have that 
$\frac{x}{2} + \frac{1}{2} \in \sbD$. From that, from \ref{adding_multiplied_differences} we have, 
that $1-\overbrace{2}^{\substack{"n"\rm\ from\ \\ 
\rm \ref{adding_multiplied_differences}}} 
(\ \overbrace{1-(\frac{x}{2}+\frac{1}{2})}^{\substack{"1-x"\rm\ 
from\ \\ \rm \ref{adding_multiplied_differences}}}\ ) \in \sbD$. But $1 - 2(1-(\frac{x}{2}+
\frac{1}{2})) = x$, so $x \in \sbD$. $_\square$
% for some $y \in \sbD$, so $\sbD$.
\\[2pt]
\textbf{Proof II.}
%\subsubsection{Geometrical reason}
\textbf{Geometrical reason} \\
We assume that $x \in \sIS$ is an \apots\ $\sIS$.\\
%If $x \in \sIS$
From \ref{predescors} we know, that $x$ can be expressed as $y - 1$ for some $y \in \sIS$. \\
Let $\mathcal{O}$ be an orbifold with the base manifold $S^2$, such that $\cho{\mathcal{O}} 
= y$. \\
Let $\mathcal{O}_c$ be the orbifold created from $\mathcal{O}$ by adding one cusp. 
Then $\cho{\mathcal{O}_c} = y - 1 = x$. Topologically $\mathcal{O}_c$ with the cusp point 
removed (which do not change an orbicharacteristic) is $\mathbb{R}^2$. 
We can compactify it with $S^1$. This will not change an \Eoc\ since $\cho{S^1} = 0$ and 
\Eoc\ is additive.
\\ What we get is an orbifold $\mathcal{O}_D$ with the base 
manifold $D^2$ and the same 
orbipoints as $\mathcal{O}$. Since orbipoints of $\mathcal{O}$ create a difference 
in \Eoc\ equal to $2-y$, we have that $\cho{\mathcal{O}_D} = 1 - (2-y) = y - 1 = x$. 
We can then move all orbipoints from the interior of $\mathcal{O}_D$ to its boundry 
by doubling them, so $x \in \sbD$. $_\square$

\section{Translating questions to ones about Egyptian fractions}\label{Egyptian_fractions}

\section{Estimations of the number of occurences}









\section{Deformations on orbifolds?}












% mainfile: ../praca_magisterska_orbifoldy.tex
\chapter{Power series and generating functions}

%% mainfile: ../praca_magisterska_orbifoldy.tex
\chapter{Connection with modular forms}

../conclusions/conclusions.tex
%% mainfile: ../praca_magisterska_orbifoldy.tex
\chapter{test}
abcd

% mainfile: ../praca_magisterska_orbifoldy.tex
\nocite{*}
\bibliography{../bibliography/bibliography.bib}{}
\bibliographystyle{plain}

%~\cite{Conway2002}
\end{document}



