% mainfile: ../praca_magisterska_orbifoldy.tex
\chapter{Order preserving homeomorphism}
Conserning accumulation points, we will use the terminology, that we introduced in 
\ref{accumulation_points_definitions}

\begin{theorem}\label{order_preserving_homeomorphism_theorem}
Let $A \subseteq \mathbb{R}$ has order type $\alpha$. 
Let $A$ be such that every accumulation point of $A$ belong to $A$. Then $A$ has not only 
order type $\alpha$ but is also homeomorphic to $\alpha$. 
\end{theorem}

\textbf{Proof.} \\
Let us observe, that $\alpha$ must be countable -- as $A$ is well ordered we can assign to every 
element $a_1 \in A$ an non-empty open interval, between $a_1$ and it's succesor $a_2$. 
From eery such interval we can pick a rational number. Fromthis we have injection of $A$ into 
$\mathbb{Q}$. \\
Without loss of generality, let us assume, that is $A$ has no infinite descending sequence 
(case with $A$ having no infinite ascending sequence is completly analogus). \\

As $A$ has an order type $\alpha$ we have that there is an order preserving map 
$f : \alpha \to A$. 
We will prove the theorem by inductively showing, that $f$ is a homeomorphism. \\


%For simplycity, we will take reverse (decreasing) order on $A$ i.e. $1$ will be the smallest 
% element 
%(so for example $0 > 1$ in this order)
%% and in general $x < y $ in the ussual order iff $x > y$ in 
%the reverse order).  \\
%%\todo{dopisać dowód}
%%Since $A$ is well ordered (as we know from \ref{well_order}) it has an order preserving 
%%bijection with some ordinal number. 
Let us denote $f_\mu \coloneqq f\raisebox{-0.5\depth}{\big|}_{\mu_1 + 1}$ so 
$f_\mu$ is defined on all ordinals lass or equal to $\mu$.
We will inductively construct the family of order preserving homeomorphisms $f_\mu$, 
indexed by ordinal numbers less 
or equal
%\smalltodoII{fix this}
than 
$\alpha$. 
%For every $\mu < \alpha$ homeomorphism $f_\mu$ will be between 
%prefix of $\alpha$ homeomorphic to $\nu + 1$ (so on all ordinals less 
%or equal to $\nu$) 
%and some prefix of $A$. 
%We will construct them in such a way, that for any $\mu_1 < \mu_2 
%< \alpha$ function $f_{\mu_2}$ restricted to the ordinals less or equal to 
%$\mu_1$ coincides with $f_{\mu_1}$.
%Then we will take $f \coloneqq \bigcup\limits_{0 \leq \mu < 
%\alpha} f_\mu$ (so $f(\mu) \coloneqq f_\mu(\mu)$). \\
Our inductive assumption for a given $\mu$ will be that 
for all $\nu < \mu$ function $f_\nu$ will be an order preserving homeomorphism 
between prefix of $\alpha$ homeomorphic to $\nu + 1$ (so defined on all ordinals less 
or equal to $\nu$) 
and some prefix of $A$; and that for every $\nu_1 < \nu_2 < \mu$ function 
$f_{\nu_2}\raisebox{-0.5\depth}{\big|}_{\nu_1 + 1} = f_{\nu_1}$. \\ 
Then we will put $f = f_\alpha$\\[8pt]
%restricted to the ordinals less or equal than $\nu_1$ coincides with $f_{\nu_1}$. \\ 
%Our inductive assumtion will be, that for all $\mu \leq \alpha$ 
%function $f_\mu$ is a homeomorphism 
%between prefix of $\alpha$ homeomorphic to $\mu$ and some \\
%$\bullet$ $\mu = 0$: Function $f_0$ is an empty function and as such it is an order preserving 
%homeomorphism. \\
$\bullet$ $\mu = 0$: Let $a_0$ be the smallest element of $A$. 
We take $f_0$ as a function on $\{0\}$ taking value $a_0$. 
Both $0$ and $a_0$ are the smallest elements of, respectively, $\alpha$ and $A$ 
so $f_0$ is defined between prefix of $\alpha$ of all ordinals less or equal to $0$, and 
some prefix of $A$. 
Function $f_0$ also preserves order on one element set.
Function $f_0$ is an homeomorphism between one element sets, both with discreate topology. \\
$\bullet$ $\mu$ is a successor ordinal less than $\alpha$: From an inductive assumption 
we have an order preserving homeomorphism $f_{\mu - 1}$ between all ordinals 
less or equal to $\mu - 1$ and some prefix of $A$. 
We define $f_\mu$ on all numbers less or equal to $\mu - 1$ to be equal $f_{\mu-1}$. \\
%Now we have two cases: \\ 
It remains to define $f_\mu(\mu)$. 
As $A$ is well ordered it is well defined to take successor of an element of $A$. 
We define $f_\mu(\mu)$ to be a succesor of $f_{\mu - 1}(\mu-1)$ in $A$. As such (and from 
inductive assumption) it is indeed defined as a function between prefix of $\alpha$ of all 
ordinals less or equal to $\mu$, and 
some prefix of $A$.
\\
Now we want to prove, that $f_\mu$ preservse the order. From the inductive assumption 
it preserves the order up to $\mu - 1$. As $\mu$ is the successor of $\mu - 1$ and 
$f_\mu(\mu)$ is a successor of $f_\mu(\mu-1)$, we have that $f_\mu$ is indeed an order preserving 
function. \\
Now we want to prove that $f_\mu$ is a homeomorphism. As from inductive assumtion we know, 
that $f_{\mu - 1}$ was a homeomorphism it is sufficient to show that preimages of open 
sets containing $f_\mu(\mu)$ and images of open sets containing $\mu$ are open. \\
Since $f_\mu(\mu)$ is a successor and since $A$ is well ordered, we have, that $f_\mu(\mu)$ 
is an isolated point in $A$. \\
Simmilarly $\mu$ is an isolated point in $\mu + 1$ as an successor ordinal. \\
From this we have, that open sets containing $f_\mu(\mu)$ (resp. $\mu$) are of the form 
$U \cup \{f_\mu(\mu)\}$ (resp. $V \cup \{\mu\}$) for some $U$ -- open set in $A$. 
(resp. $V$ -- open set in $\mu + 1$). 
%Holding the notation of $U$ and $V$ we have that $f_\mu[V] = $
\smalltodoII{może rozwinąć} 
From this this is clear. \\ 
%Let us observe, that 
%Lets take \todo{finish}
$\bullet$ $\mu$ is a limit ordinal less than $\alpha$:  
%\todo{net and so on}
From the inductive assumption, for each $\nu < \mu$ we have an order preserving homeomorphism 
$f_\nu$ on the ordinals less or equal to $\nu$ and those functions pairwise coincide 
on the intersections 
of their domains. For every ordinal $\nu < \mu$ we define 
$f_\mu(\nu) \coloneqq f_\nu(\nu)$. It remains to define $f_\mu(\mu)$. \\
We consider a net $\phi_\mu \coloneqq \{f_\nu(\nu)\}_{\nu<\mu} \subset \mathbb{R}$
%, indexed by all $\nu < \mu$
. From the inductive assumption we know that the domain of the net $\phi_\mu$, as 
well as it's image is well ordered and that the net $\phi_\mu$ 
is an order preserving homeomorphism.
% is  this is a well ordered net. \todo{napisać jakoś lepiej tę własność}
Now we will show that the net $\phi_\mu$ has a limit in $A$. \\
First we will show, that $\phi_\mu$ has a limit in $\mathbb{R}$. For this, we will show that 
$\phi_\mu$ is bounded. \\
Order type of the image of $\phi_\nu$ is equal to $\mu$ and it is a prefix of $A$. 

As we have \ref{accumulation_points_of_the_set} 
As $\mathbb{R}$ is Hausdorff, from \cite{Kelley1975} (chapter 2, 
theorem 3, page 67) we know, that .  
\\ 
%\newpage
- show that if $\mu < \omega^n$, then it is earlier than $\frac{n}{2}$ something something\\
\smalltodo 
\\
- done\\
XD \\
$le\ XD$ \\ 
For the sake of contradiction, let us assume, that $\phi_\nu$ is unbounded. \\
There exist $n$ such that $\phi_\nu < \omega^n$. \\
The only unbounded in $\mathbb{R}$ prefix of $A$ is a whole $A$. \\ 
From this we conclude that $A$ schould have order type of $\alpha$.
\\ 
%Firstly we will determine the order type of $A$. 
%From the lemma \ref{well_order} we know, that $A$ is well ordered, so it has order type 
%of some ordinal number. From this and 
%from the theorem \ref{greatest \apots} we know, that for the point $1-\frac{n}{2}$ there exist 
%a neighborhood $U=(1-\frac{n}{2}-\varepsilon,1-\frac{n}{2}+\varepsilon)$ such that $U \cap 
%A$ is homeomorphic to $\omega^n$. From this, and again from theorem \ref{greatest \apots} 
%we have that $A \cap [1,1-\frac{n}{2})$ is homeomorphic with $\omega^n$. 
%From this $A$ is homeomorphic with $\alpha$.
