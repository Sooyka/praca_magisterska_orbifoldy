% mainfile: ../praca_magisterska_orbifoldy.tex
\chapter{Appendix about well orders}
\section{Usefull lemmas}
\begin{lemma}\label{two_sets_lemma}
If $A, B \subseteq \mathbb{R}$ have no infinite strictly ascending sequences, then set 
$A + B \coloneqq \{a+b\ |\ a \in A, b \in B\}$ also have no infinite strictly ascending sequences. 
\end{lemma}
\noindent\textbf{Proof.} \\
Let $A$, $B$ have no infinite strictly ascending sequences. 
Let $c_n \in A + B$ are elements of some sequence. With a sequence $c_n$ there are 
two associated sequences $a_n$, $b_n$, such that, for all $n$, we have $a_n \in A$, 
$b_n \in B$ and 
$a_n + b_n = c_n$. Assume (for contradiction), that $c_n$ is an infinite strictly 
ascending sequence. 
Then $\forall_n\ a_{n+1}>a_n\ \lor\ b_{n+1} > b_n$. From the assumption $a_n$ has no infinite 
ascending sequence, so $a_n$ has a weakly decreasing subsequence $a_{n_k}$. But then 
subsequence $b_{n_k}$ must be strictly increasing, as $c_{n_k}$ is strictly increasing, what gives 
us a contradiction. 
%\Lightning 
$_\square$ 

\begin{lemma}\label{sum_lemma}
If $A, B \subseteq \mathbb{R}$ have no infinite strictly ascending sequences, then set 
$A \cup B$ also have no infinite strictly ascending sequences.
\end{lemma}
\textbf{Proof.} \\
Let $A$, $B$ have no infinite strictly ascending sequences. 
For the sake of contradiction, lets assume, that $A \cup B$ has an infinite strictly 
ascending sequence $c_n$. Let $c_{n_k}$, $c_{n_l}$ be subsequences of $c_n$ consisting 
of elements from, respectively $A$ and $B$. At least one of them must be infinite and 
strictly increasing, which gives us a contradiction. $_\square$ 
\section{Order preserving homeomorphisms}
Conserning accumulation points, we will use the terminology, that we introduced in 
\ref{accumulation_points_definitions}

\begin{theorem}\label{order_preserving_homeomorphism_theorem}
Let $A \subseteq \mathbb{R}$ has order type $\alpha$. 
Let $A$ be such that every accumulation point of $A$ belong to $A$. Then $A$ has not only 
order type $\alpha$ but is also homeomorphic to $\alpha$. 
\end{theorem}

\textbf{Proof.} \\
Let us observe, that $\alpha$ must be countable -- as $A$ is well ordered we can assign to every 
element $a_1 \in A$ an non-empty open interval, between $a_1$ and it's succesor $a_2$. 
From every such interval we can pick a rational number. From this we have injection of $A$ into 
$\mathbb{Q}$. \\
Without loss of generality, let us assume, that $A$ has no infinite descending sequence 
(case with $A$ having no infinite ascending sequence is completly analogus). \\

As $A$ has an order type $\alpha$ we have that there is an order preserving bijection 
$f : \alpha \to A$. \\
For an ordinal $\mu < \alpha$, let us denote $f_\mu 
\coloneqq f\raisebox{-0.5\depth}{\big|}_{\mu + 1}$ so 
$f_\mu$ is defined on all ordinals less or equal to $\mu$. Then we have that
$f = \bigcup\limits_{\mu < \alpha} f_\mu$. Let us also denote $A_\mu \coloneqq f_\mu[\mu+1]$ --
the image of $\mu + 1$ (as a set of all ordinals less or equal to $\mu$) in $A$. 
Let us remark, that $A_\mu$ has an order type $\mu + 1$.\\ 

We will prove the theorem by inductively showing for all $\mu < \alpha$, that $f_\mu$ is a 
homeomorphism, and by showing that this is sufficient for $f$ to be a homeomorphism. \\


%For simplycity, we will take reverse (decreasing) order on $A$ i.e. $1$ will be the smallest 
% element 
%(so for example $0 > 1$ in this order)
%% and in general $x < y $ in the ussual order iff $x > y$ in 
%the reverse order).  \\
%%\todo{dopisać dowód}
%%Since $A$ is well ordered (as we know from \ref{well_order}) it has an order preserving 
%%bijection with some ordinal number. 

%We will inductively show that the family of maps  $f_\mu$, 
%indexed by ordinal numbers less 
%or equal
%%\smalltodoII{fix this}
%than 
%$\alpha$. 
%For every $\mu < \alpha$ homeomorphism $f_\mu$ will be between 
%prefix of $\alpha$ homeomorphic to $\nu + 1$ (so on all ordinals less 
%or equal to $\nu$) 
%and some prefix of $A$. 
%We will construct them in such a way, that for any $\mu_1 < \mu_2 
%< \alpha$ function $f_{\mu_2}$ restricted to the ordinals less or equal to 
%$\mu_1$ coincides with $f_{\mu_1}$.
%Then we will take $f \coloneqq \bigcup\limits_{0 \leq \mu < 
%\alpha} f_\mu$ (so $f(\mu) \coloneqq f_\mu(\mu)$). \\
Our inductive assumption for a given $\mu$ will be that 
for all $\nu < \mu$ function $f_\nu$ is a
%n order preserving 
homeomorphism 
between $\nu + 1$ (so defined on all ordinals less 
or equal to $\nu$) 
and $A_\nu$; and that for every $\nu_1 < \nu_2 < \mu$ we have 
$f_{\nu_2}\raisebox{-0.5\depth}{\big|}_{\nu_1 + 1} = f_{\nu_1}$.
%Then we will put $f = f_\alpha$
\\[8pt]
%restricted to the ordinals less or equal than $\nu_1$ coincides with $f_{\nu_1}$. \\ 
%Our inductive assumtion will be, that for all $\mu \leq \alpha$ 
%function $f_\mu$ is a homeomorphism 
%between prefix of $\alpha$ homeomorphic to $\mu$ and some \\
%$\bullet$ $\mu = 0$: Function $f_0$ is an empty function and as such it is an order preserving 
%homeomorphism. \\
$\bullet$ $\mu = 0$: \\
%Let $a_0$ be the smallest element of $A$. 
%We take $f_0$ as a function on $\{0\}$ taking value $a_0$. 
%Both $0$ and $a_0$ are the smallest elements of, respectively, $\alpha$ and $A$ 
%so $f_0$ is defined between prefix of $\alpha$ of all ordinals less or equal to $0$, and 
%some prefix of $A$. 
%Function $f_0$ also preserves order on one element set.
Function $f_0$ is an homeomorphism between one element sets, both with discreate topology. \\[8pt]
$\bullet$ $\mu < \alpha$ is a successor ordinal:\\ 
%From an inductive assumption 
%we have an order preserving homeomorphism $f_{\mu - 1}$ between all ordinals 
%less or equal to $\mu - 1$ and some prefix of $A$. 
%We define $f_\mu$ on all numbers less or equal to $\mu - 1$ to be equal $f_{\mu-1}$. \\
%%Now we have two cases: \\ 
%It remains to define $f_\mu(\mu)$. 
%As $A$ is well ordered it is well defined to take successor of an element of $A$. 
%We define $f_\mu(\mu)$ to be a succesor of $f_{\mu - 1}(\mu-1)$ in $A$. As such (and from 
%inductive assumption) it is indeed defined as a function between prefix of $\alpha$ of all 
%ordinals less or equal to $\mu$, and 
%some prefix of $A$.
%\\
%Now we want to prove, that $f_\mu$ preservse the order. From the inductive assumption 
%it preserves the order up to $\mu - 1$. As $\mu$ is the successor of $\mu - 1$ and 
%$f_\mu(\mu)$ is a successor of $f_\mu(\mu-1)$, we have that $f_\mu$ is indeed an order preserving 
%function. \\
%Now we want to prove that $f_\mu$ is a homeomorphism. 
Since $\mu$ is a successor ordinal, $\mu-1$ exists. \\
From inductive assumtion we know,
that $f_{\mu - 1} : \mu \to A_{\mu - 1}$ is a homeomorphism, 
so it is sufficient to show that preimages of open 
sets containing $f_\mu(\mu)$ and images of open sets containing $\mu$ are open. \\
Since $f_\mu(\mu)$ is a successor (because $f_\mu$ preserves the order) 
and since $A_\mu$ is well ordered, we have, that $f_\mu(\mu)$ 
is an isolated point in $A_\mu$. \\
Simmilarly $\mu$ is an isolated point in $\mu + 1$ as an successor ordinal. \\
From this we have, that open sets containing $f_\mu(\mu)$ (resp. $\mu$) are of the form 
\begin{equation}\label{form of open sets}
U \cup \{f_\mu(\mu)\} \textrm{ (resp. } V \cup \{\mu\}\textrm{)} 
\end{equation}
for some $U$ -- open set in $A_{\mu-1}$. 
(resp. $V$ -- open set in $\mu$). \\
Let $U$ be an open set in $A_{\mu - 1}$ and $V = f_{\mu - 1}^{-1}[U]$ an open set in $\mu$. \\
we have that 
%$f_\mu[V] = $
%\smalltodoII{może rozwinąć} 
%From this this is clear. \\ 
\begin{gather}
f_\mu^{-1}\left[ U \cup \{f_\mu(\mu)\} \right] = f_\mu^{-1}\left[ U \right] \cup 
f_\mu^{-1}\left[ \{f_\mu(\mu)\} \right] =
%\underbrace{f_{\mu - 1}^{-1}\left[ U\right]}_{\textrm{open in } \mu} 
f_{\mu - 1}^{-1}\left[ U\right] \cup \{\mu\} = V \cup \{\mu\}
\end{gather}
and
\begin{gather}
f_\mu \left[ V \cup \{\mu\} \right] = f_\mu \left[ V \right] \cup f_\mu\left[ \{\mu\} \right] = 
%\underbrace{f_{\mu - 1} \left[ V \right]}_{\textrm{open in } A_{\mu - 1}} 
f_{\mu - 1} \left[ V \right] \cup \{f_\mu (\mu)\} = U \cup \{f_\mu(\mu)\}
\end{gather}
so by \ref{form of open sets}, we have that preimages of open 
sets containing $f_\mu(\mu)$ and images of open sets containing $\mu$ are indeed open. \\[8pt]
%Let us observe, that 
%Lets take \todo{finish}
$\bullet$ $\mu < \alpha$ is a limit ordinal: \\ 
%\todo{net and so on}
%From the inductive assumption, for each $\nu < \mu$ we have an order preserving homeomorphism 
%$f_\nu$ on the ordinals less or equal to $\nu$ and those functions pairwise coincide 
%on the intersections 
%of their domains. For every ordinal $\nu < \mu$ we define 
%$f_\mu(\nu) \coloneqq f_\nu(\nu)$. It remains to define $f_\mu(\mu)$. \\
%We consider a net $\phi_\mu \coloneqq \{f_\nu(\nu)\}_{\nu<\mu} \subset \mathbb{R}$
%%, indexed by all $\nu < \mu$
%. From the inductive assumption we know that the domain of the net $\phi_\mu$, as 
%well as it's image is well ordered and that the net $\phi_\mu$ 
%is an order preserving homeomorphism.
%% is  this is a well ordered net. \todo{napisać jakoś lepiej tę własność}
%Now we will show that the net $\phi_\mu$ has a limit in $A$. \\
%First we will show, that $\phi_\mu$ has a limit in $\mathbb{R}$. For this, we will show that 
%$\phi_\mu$ is bounded. \\
%Order type of the image of $\phi_\nu$ is equal to $\mu$ and it is a prefix of $A$. 
%For now, let us assume, that 
Let $f_{\nu<\mu} \coloneqq f_{\mu}\Big|_{\{\nu\,|\,\nu<\mu\}}$. Then $f_{\nu < \mu}$ 
is a well ordered net from $\mu$ to $\mathbb{R}$ with image $A_\mu\setminus\{f(\mu)\}$. 
It is bounded by $f(\mu)$, so,
as $\mathbb{R}$ is Hausdorff, from \cite{Kelley1975} (chapter 2, 
theorem 3, page 67) we know, that $f_{\nu<\mu}$ has a unique limit as a net. Let us call this 
limit $r$. This limit is 
as well an accumulation point of $A$, so by the assumption of the theorem, we have that 
$r \in A$.
\begin{observation}
We have that $r = f_\mu(\mu)$.
\end{observation}
\textit{Proof.} \\
For the sake of contradiction, let us assume, that $r > f_\mu(\mu)$. 
Then, as r is an accumulation 
point of $f_{\nu<\mu}$, we have that $\exists_{\nu<\mu}\ f_\mu(\mu)<f_\mu(\nu)$. 
Which is a contradiction as $f$ preservers the order. Hence, $r \leq f_\mu(\mu)$.
Now, for the sake of contradiction, let us assume, that $r < f_\mu(\mu)$. \\
As we have, that $r \in A$, we have, that there exist some $\eta$ such that $r = f(\eta)$. 
Since $f$ preserves order and we assumed that $r < f_\mu(\mu)$, we have, that $\eta < \mu$. 
But then, as $\mu$ is a limit ordinal, we have, that $\eta +1 < \mu$ as well. From this, 
we conclude that there exist some ordinal $<\mu$, namely $\eta +1$, such that $f_\mu(\eta+1) > r$. 
This however is a contradiction, as $f$ preserves the order and $r$ is an accumulation point. 
Hence, $r \geq f_\mu(\mu)$. \\
From this we conclude, that indeed $r = f_\mu(\mu)$. $_\square$\\
%We have that $\forall_{\nu<\mu}\  f(\nu)<r$ \\ 
%The limit is $f_\mu(\mu)$ as it is the 
%biggest element of $A_\mu$ and $f_\mu$ is well ordered net. 
%\\
%As we have \ref{accumulation_points_of_the_set} 
%%\newpage
%- show that if $\mu < \omega^n$, then it is earlier than $\frac{n}{2}$ something something\\
%\smalltodo 
%\\
%- done\\
%XD \\
%$le\ XD$ \\ 
%For the sake of contradiction, let us assume, that $\phi_\nu$ is unbounded. \\
%There exist $n$ such that $\phi_\nu < \omega^n$. \\
%The only unbounded in $\mathbb{R}$ prefix of $A$ is a whole $A$. \\ 
%From this we conclude that $A$ schould have order type of $\alpha$.
%\\ 
%Then $f_\mu$ is a ordinal indexed sequence.
For the continuity of $f$ and $f^{-1}$ it is sufficient to show, that for every 
$U \subseteq A_\mu$ and $V \subseteq \mu+1$ 
in prebases of respective topologies, $f_\mu^{-1}[U]$ and $f_\mu[V]$ are open ($\ast$). \\
Prebase open sets in $A_\mu$ are the ones inherited from the order topology on $\mathbb{R}$, 
for all $s \in \mathbb{R}$:
\begin{align*}
\{r\ &|\ r<s\} \cap A_\mu\\
\{r\ &|\ s<r\} \cap A_\mu.
\end{align*}
Prebase open sets in $\mu+1$ are from order topology, for all $\nu \in \mu+1$:
\begin{align*}
\{\eta\ &|\ \eta<\nu\} \\
\{\eta\ &|\ \nu<\eta\}.
\end{align*}
From our induction assumption we have ($\ast$) checked for following prebase sets: \\
$\{r\ |\ r<s\} \cap A_\mu$, for all $s < f(\mu)$. \\
$\{\eta\ |\ \eta<\nu\}$, for all $\nu < \mu$, \\
It was left to be shown, that ($\ast$) holds for the prebase sets: \\
$\{r\ |\ r<f(\mu)\} \cap A_\mu$, \\
$\{r\ |\ s<r\} \cap A_\mu$, for all $s \leq f(\mu)$\\
$\{\eta\ |\ \eta<\mu\}$, \\
$\{\eta\ |\ \nu<\eta\}$, for all $\nu \leq \mu$, \\ 

Prebase set -- $\{r\ |\ r<f(\mu)\} \cap A_\mu$: \\
We have that:
\begin{align*}
f^{-1}[\{r\ |\ s<f(\mu)\} \cap A_\mu] = \{\eta\ |\ \eta <\mu\},
\end{align*}
which is open. \\
Prebase set -- $\{r\ |\ s<r\} \cap A_\mu$: \\
Let $s < f(\mu)$. We have two cases: \\
-- $s \in A$: then let $\nu$ be such that $f(\nu)=s$. Then we have that:
\begin{align*}
f^{-1}[\{r\ |\ s<r\} \cap A_\mu] = \{\eta\ |\ \nu < \eta \},
\end{align*}
which is open. \\
-- $s \not\in A$: then, by the assumption of the theorem we know that $s$ is not an accumulation 
point of $A$. From this we conclude, that 
$\exists_{t\in A} (t<s \land \neg \exists_{t' \in A} t<t'<s)$. Let $\nu$ be such that $f(\nu) = t$. 
Then we have that:
\begin{align*}
f^{-1}[\{r\ |\ s<r\} \cap A_\mu] = \{\eta\ |\ \nu < \eta \},
\end{align*}
which is open. \\
Prebase set -- :
a\\
This concludes the inductive part of the proof, that for every $\mu < \alpha$ we have that 
$f_\mu$ is a homeomorphism. Now we will show, that this is sufficient for $f$ to be a 
homeomorphism. \\
\todo{wszystkie ograniczone otwarte działają}
\todo{wszystkie nieograniczone też}
%Firstly we will determine the order type of $A$. 
%From the lemma \ref{well_order} we know, that $A$ is well ordered, so it has order type 
%of some ordinal number. From this and 
%from the theorem \ref{greatest \apots} we know, that for the point $1-\frac{n}{2}$ there exist 
%a neighborhood $U=(1-\frac{n}{2}-\varepsilon,1-\frac{n}{2}+\varepsilon)$ such that $U \cap 
%A$ is homeomorphic to $\omega^n$. From this, and again from theorem \ref{greatest \apots} 
%we have that $A \cap [1,1-\frac{n}{2})$ is homeomorphic with $\omega^n$. 
%From this $A$ is homeomorphic with $\alpha$.
