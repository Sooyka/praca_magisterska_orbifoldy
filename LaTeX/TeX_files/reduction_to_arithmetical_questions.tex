% mainfile: ../praca_magisterska_orbifoldy.tex
\chapter{Reduction to arithmetical questions}\label{reduction_to_arithmetical}
Reductions presented in this chapter will be more in the spirit of chapter \ref{order structure}, 
in the sense that 
%We will see (in the observation \ref{boils_down}) that the problem of determining this boils
% down to the 
% analysis of all 
%the possible 
%values of the expressions:
%\begin{equation}
%2 - \sum_{i=1}^n \frac{I_i-1}{I_i}
%\end{equation}
%and 
%\begin{equation}
%1 - \sum_{j=1}^m \frac{b_j-1}{2b_j},
%\end{equation}
%where $I_i, b_j$ varies over $\mathbb{N}_{>0} \cup \{\infty\}$. \\
%As
%\begin{equation}
%2 - \sum_{i=1}^n \frac{I_i-1}{I_i} = 2 - n + \sum_{i=1}^n \frac{1}{I_i}
%\end{equation}
%and 
%\begin{equation}
%1 - \sum_{j=1}^m \frac{b_j-1}{2b_j} = 1 - m + \sum_{j=1}^m \frac{1}{2b_j},
%\end{equation}
%some questions about the spectrum are equivalent to some regarding Egyptian fractions. 
%More on this connection is discussed in \ref{Egyptian_fractions}.
%\\[4pt]
%\textbf{Disclaimer}\\
for now, until chapter \ref{counting occurrences} 
%named "Counting occurrences" 
, we will not pay attention 
to how many orbifolds have the same Euler orbicharacteristic, only whether a particular 
number is an \Eoc\ for at least one orbifold or not. 

In chapter \ref{counting occurrences} we will explain how these reductions will be relevant 
to the discussion holded there. 
%Let us note, that 
%Euler orbicharacteristic does not depend on the cyclic order of points on 
%the components of the boundary. 
%Because of that and since Euler orbicharacteristic does not depend on the cyclic order 
%of points on the components of the boundary we introduce an extension of a notation from 
%\cite{Conway2008}. 

%%We will write $\ast \{a,b,c,d,\cdots\}$ to denote not a particular orbifold, but a type 
%%of orbifold that have a dihedral points on a component of the boundry of orders 
%$a,b,c,d,\cdots$, but 
%%in any order. \\ 

%We will write $\ast \{a,b,c,d,\cdots\}$ to denote a type of a boundary (of an orbifold) that have 
%kaleidoscopic points of periods $a,b,c,d,\cdots$, but in any order. \\


%From what we wrote above (that Euler orbicharacteristic does not depend on the cyclic order 
%of points on the components of the boundary), we can see that Euler 
%orbicharacteristic is well defined 
%%on such types of orbifolds. \\ 
%when we specify only such a type of the components of the boundary of an orbifold and not 
%a particular cyclic order.  \\

\section{Reductions of cases}
The aim of following reductions is to make it easier to answear the question of which 
points lie in $\spe$ and which not. 

The first problem with the structure of $\spe$ is that it is the sum of $\spebr{M}$, for 
every two dimentional manifold $M$. 
\begin{equation}
\spe = \bigcup_{M\textrm{ : 2d manifold}} \spebr{M}.
\end{equation}

%In this section we want to make some reductions to 
%limit number of cases that we will be dealing with. 

We aim to find a minimal set $\mathcal{M}$ of base manifolds 
%that "covers all the cases" i.e.  
such that:
\begin{equation}\label{minimal calM}
\spe = \bigcup_{M\in\mathcal{M}}\spebr{M}.
\end{equation}

% any for any $x \in \spe$ there 
%is an orbifold $O$ with a base manifold from $B$ such that $\cho{O} = x$.
%We aim to limit the number of base manifolds as much as possible while keeping entire spectrum. 
It will turn out that $\mathcal{M} = \{S^2, D^2\}$ satisfies \ref{minimal calM} and 
that both $S^2$ and $D^2$ are neccesery. 
%there are
% it can be done such that we are left with $B = \{S^2, D^2\}$ and there is 
%no futher reductions possible. 

%For this chapter we will consider orbifolds according to a definition from 
%(\ref{Disk_and_sphere_with_defects}). \\

\section{Sufficiency of $S^2$ and $D^2$}\label{sufficiency of D2 and S2}

Given an orbifold $O_1$, we want to perform some operations from \ref{Operations} on it, 
such that the resulting orbifold 
$O_2$ will have the same Euler orbicharactristic, but the base manifold of $O_2$ would 
be $S^2$ or $D^2$. We would then say, that $O_1$ got reduced to $O_2$.
In following subsection, we allow only such operations, that do not 
change \Eoc. When writing that we "can" do something we mean that there is 
possible one of the operations from \ref{Operations}.  %have as 
%big Euler characteristic as possible. 

The Euler characteristic of base manifold depends only on the number of handles, cross caps 
and boundry components. And, as stated in \ref{E_orb} it is: 
\begin{equation}
2-2h-c-b,
\end{equation}
%\smalltodoII{opisać i wybrać oznaczenia}
for $h$ - number of handles, $c$ - number of cross-caps, $b$ - number of boundary components. 

For every such a manifold feature we want to find an orbifold features with the same 
Euler orbicharacteristic delta. 

We will take to approuches, depending on whether the orbifold in question has a boundary or not.

%To do this we want to eliminate handles 
%Let us observe, that:
\subsection{Orbifold without boundary}
We can observe that:
\begin{align}
\hspace{3cm}\Delta(\circ) =& \hspace{-1cm} &-2& \hspace{-1cm} &= \Delta(2^4) \\
\hspace{3cm}\Delta(\times) =& \hspace{-1cm} &-1& \hspace{-1cm} &= \Delta(2^2)
\end{align}
From this we can see that we can remove handles and cross-caps from any orbifold without 
the boundary. 
After such reductions we are left with a $S^2$ orbifold with all orbipoints being rotational 
in the interior.

\subsection{Orbifold with boundary}7\label{orbifold_with_boundary}
We can observe that:
\begin{align}
\hspace{3cm}\Delta(\circ) =& \hspace{-1cm} &-2& \hspace{-1cm} &= \Delta((^\ast 2)^8) \\
\hspace{3cm}\Delta(\ast) =& \hspace{-1cm} &-1& \hspace{-1cm} &= \Delta((^\ast 2)^4) \\
\hspace{3cm}\Delta(\times) =& \hspace{-1cm} &-1& \hspace{-1cm} &= \Delta((^\ast 2)^4)
\end{align}
From this we can see that we can remove handles and cross-caps 
from any orbifold with a boundary. 
We can also remove all boundary components exept one.         
We can further observe that:
\begin{align}
\hspace{3cm}\Delta(n) =& \hspace{-1cm} &\frac{n-1}{n}& \hspace{-1cm} 
&= 2\frac{n-1}{2n} &= \Delta((^\ast n)^2)\label{third_reduction}
\end{align}
From this we see that we can remove all the rotational orbipoints in favor for 
dihedral orbipoints.
After such reductions we are left with a $D^2$ orbifold with all orbipoints being dihedral on 
the boundary or being reflectional on the boundary.
%\smalltodoII{reflectional?}

As a fact not necessary for our reductions, but interestung on its own, we can furthermore, 
observe that:
\begin{observation}
If $O_1$ has not $S^2$ as its base manifold it can be reduced to a $D^2$-orbifold.
\end{observation}
\textbf{Proof.} \\
If $O_1$ has not $S^2$ as its base manifold $M$, then $M$ has at least one handle or a cross-cup. We can observe that:
\begin{align}
\hspace{3cm}\Delta(\circ) =& \hspace{-1cm} &-2& \hspace{-1cm} &= \Delta(\ast (^*2)^4) \\
\hspace{3cm}\Delta(\times) =& \hspace{-1cm} &-1& \hspace{-1cm} &= \Delta(\ast).
\end{align}
From this we have that the handle or the cross-cap can be replaced by a boundary component  
and some number of boundary orbipoints. After this reduction, we can proceed with all the 
other reductions from the \ref{orbifold_with_boundary} and obtain an $D^2$-orbifold 
with the same \Eoc as the original one. $_\square$

%\begin{align}
%\Delta(n) =& &\frac{n-1}{n}&  
%&= \Delta((^\ast n)^2)\label{third_reduction}
%\end{align}

%From this we can conclude that every Euler orbicharacteristic can be obtained 
%by an orbifold with base manifold $S^2$ or $D^2$. 
%Examples of rational numbers from $\speS \setminus \speD$ and $\speD \setminus \speS$ are:
%We will provide examples 
%
%\ref{counting_occurences}

%From this we can conclude, that every Euler orbicharacteristic can be obtained 
%by an orbifold of signature of a type ($n$ and $m$ are arbitrary):

%\begin{align*}
%I_1I_2\cdots I_n & \textrm{\ or} \\
%\ast b_1b_2\cdots b_m &.
%\end{align*}

%Let us denote the set of all possible Euler orbicharacteristics of orbifolds of the form 
%$I_1I_2\cdots I_n$ by $\speS$ and the set 
%of all possible Euler orbicharacteristics of orbifolds of the form $\ast b_1b_2\cdots b_m$ 
%as $\speD$. 
%So we have that $\spe = \speD \cup \speS$. \\
%% Let us denote the set of all possible \Eoc s of two-dimentional orbifolds as $\spe$. \\
%Let us also observe that the order type and topology of $\speS$ and $\speD$ are 
%the same since 

%\begin{equation}\label{times_two_fact}
%2\speD=\speS
%\end{equation}
%and multiplying by $2$ is the order preserving homeomorphism of $\mathbb{R}$. \\[16pt]

%Now we can make aforementioned observation:
%\todo{dopisać to czytelniej}
%w szeczgólności żeby było jasne jaką redukcję robimy

The results of our reductions, can be summarised as:
%In the terms of set relations:
\begin{observation}\label{sum of spectras}
If two-dimentional manifold $M$ has no boundry, then
%\begin{equation} 
%\spe{M} = \chi(M) - (\speS - 2) 
%\end{equation}
%and
\begin{equation} 
\spebr{M} \subseteq \speS 
\end{equation} 

If, in addition, $M \neq S^2$, then 
\begin{equation} 
\spebr{M} \subseteq \speD. 
\end{equation}

\end{observation}
%\textbf{Proof} \\
%Let $M$ be a two-dimentional manifold with no boundry, from \ref{\Eoc_as_a_sum} we have, that: 
%\begin{equation}
%\spe{M} = \{\chi(M) - \sum_{i=1}^n 
%\frac{I_i-1}{I_i}\ |\ n \in \mathbb{N} \land \forall_i I_i \in \mathbb{N} \cup \infty\}.
%\end{equation}
%And from 
%$\chi(M) - (\speS - 2)$
\begin{observation}
If two-dimentional manifold  $M$ has a boundry, then 
%\begin{equation}
%\spe{M} = \chi(M) + (\speD - 1)
%\end{equation}
%and 
\begin{equation}
\spebr{M} \subseteq \speD
\end{equation}
\end{observation}
%\textbf{Proof} \\
%\begin{equation}
%\{\chi(M) - \sum_{j=1}^m 
%\frac{b_j-1}{2b_j}\ |\ m \in \mathbb{N} \land \forall_j b_j \in \mathbb{N} \cup \infty\} 
%\end{equation}
%\begin{observation}
%If two-dimentional manifold  $M$ has no boundry, then $\spe{M} \subseteq \speS$. 
%If, in addition, $M \neq S^2$, then 
%$\spe{M} \subseteq \speD$.
%\end{observation}
%\begin{observation}
%If two-dimentional manifold  $M$ has a boundry, then $\spe{M} \subseteq \speD$.
%\end{observation}

%In the terms of arithmetical expressions:
%\todo{sums are the form od sped and spes}

\begin{corollary}
We have that $\spe = \speS \cup \speD$.
\end{corollary}

We will postpone our discussion of neccessity of both $S^2$ and $D^2$ to 
\ref{neccessity of d2 and s2}, after 
the 
section \ref{Reduction to arithmetical questions section} which will provide us 
with more convenient language. 

\section{Reduction to arithmetical questions}\label{Reduction to arithmetical questions section}
As written in \ref{\Eoc on 2d}, we can express an \Eoc of a $M$-orbifold $O$ as:
\begin{equation}
\cho{O} = \chi (M) - \sum_{i=1}^n \frac{r_i-1}{r_i} - \sum_{j=1}^m \frac{d_j-1}{2d_j},
\end{equation}
where $r_i$ and $d_j$ are degrees of the, respectively, rotational and diheadral orbipoints 
of $O$.

From this we can express $\spebr{M}$ as:
\begin{align}
\spe(M) = \chi (M) &-\left\{\sum\limits_{i=1}^n \frac{r_i-1}{r_i}\ \big|\ n\in\mathbb{N}_0,\ 
r_i\in\mathbb{N}_{>0}\cup \{\infty\}\right\}+ \\
&- \left\{\sum\limits_{j=1}^m\frac{d_j-1}{2d_j}\ 
\big|\ m\in\mathbb{N}_0,\ d_j\in\mathbb{N}_{>0}\cup \{\infty\}\right\} .
\end{align}

As from \ref{sufficiency of D2 and S2} we know that $\spe = \speS \cup \speD$, and that 
$\chi (S^2) = 2$ and $\chi (D^2) = 1$, we can expresss $\spe$ as a sum ($\cup$) of two sets:
\begin{equation}
2 -\left\{\sum\limits_{i=1}^n \frac{r_i-1}{r_i}\ \big|\ n\in\mathbb{N}_0,\ 
r_i\in\mathbb{N}_{>0}\cup \{\infty\}\right\} = \spe(S^2)
\end{equation}
and
\begin{equation}
1 - \left\{\sum\limits_{j=1}^m\frac{d_j-1}{2d_j}\ 
\big|\ m\in\mathbb{N}_0,\ d_j\in\mathbb{N}_{>0}\cup \{\infty\}\right\} = \spe(D^2) .
\end{equation}

From this we see, that the core of understanding $\spe$ through arithmetical viewpoint 
is to understand possible values of expression:
%We now know, that $\spe = \speS \cup \speD$. To determin 

%\begin{observation}\label{boils_down}
%From above reductions we can conclued that our problem boiles down to the analysis of all
% the possible 
%values of the expressions:
\begin{equation}\label{S2_sum}
2 - \sum_{i=1}^n \frac{r_i-1}{r_i}
\end{equation}
and 
\begin{equation}
1 - \sum_{j=1}^m \frac{d_j-1}{2d_j},
\end{equation}
with $r_i$ and $d_j$ ranging over $\mathbb{N}_{>0}\cup \{\infty\}$.
%, with a convention 
%that $\varphi(\infty) \coloneqq \lim\limits_{k\to \infty} \varphi(k)$. 
%\end{observation}

As stated in \ref{cusp_reduction} we can perform futher reductions to have an orbifold with 
particular orbicharacteristic without cusps (if needed) and then (after these reductions) 
we can analyse only expressions with $r_i$ and $d_j$ ranging over $\mathbb{N}_{>0}$ and 
they will still give us full spectrum. 
However, as stated later, it will be more convenient to us to include orbifolds with cusps 
so we are stating this observation only as a side remark.
%for readers information. 
%The fact that this agrees with the definition of the \Eoc\ on the geometrical terms was 
%addressed in \ref{extended_Euler_orbicharacteristic}. 
\begin{observation}\label{2times homeomorphism}
We have that $\speS = 2\speD$.
\end{observation}
\textbf{Proof.}\\

Indeed, since: 
\begin{align}
\spe(S^2) = 2 -\left\{\sum\limits_{i=1}^n \frac{r_i-1}{r_i}\ \big|\ n\in\mathbb{N}_0,\ 
r_i\in\mathbb{N}_{>0}\cup \{\infty\}\right\}
\end{align}
and 
\begin{align}
\spe(D^2) = 1 - \left\{\sum\limits_{j=1}^m\frac{d_j-1}{2d_j}\ 
\big|\ m\in\mathbb{N}_0,\ d_j\in\mathbb{N}_{>0}\cup \{\infty\}\right\}._\square
\end{align}

\section{Neccessity of $S^2$ and $D^2$}\label{neccessity of d2 and s2}
As we know from \ref{Operations} adding an orbipoint to a manifold decreases it's 
orbicharacteristic. As $S^2$ has the highest Euler characteristic - $2$ of all 
two dimentional manifolds, there is no other orbifold with \Eoc\ equal to $2$. 
$S^2$ is then necesery to include $2$. 

As known from \cite{największy orbifold}, the number $-\frac{1}{84}\in\speD$ and it is 
the greatest negative 
\Eoc any two dimensional orbifold can have. We will now show, that 
$-\frac{1}{84} \not\in \speS$. For the sake of contradiction let us assume, that 
$-\frac{1}{84} \in \speS$, then, from \ref{2times homeomorphism} we know, that 
$\frac{1}{2}\left(-\frac{1}{84}\right)\in \speD$. This is a contradiction as 
$0 > \frac{1}{2}\left(-\frac{1}{84}\right) > -\frac{1}{84}$. $_\square$  

Futher examination of connections between $\speD$ and $\speS$ is performed in \ref{D_and_S}.

\section{Translating questions to ones about Egyptian fractions}\label{Egyptian_fractions}
The term in the sums can be expressed as $1- \frac{1}{I_i}$, then the sums become:
$2 - n + \sum_{i=1}^n \frac{1}{I_i}$. This is very simmilar in notion to the egyptian franction, 
which is a rational number that can be expressed as a sum of a fractions with numerator $1$ and 
different denominators. Sometimes the "distinc denominators" condition is dropped and we will 
follow that convention here. 
The particular interest is in translating questions related to spectra to the questions 
of egyptian franctions. 
What the data of the spectra imposes is the number of fractions to be summed.  





