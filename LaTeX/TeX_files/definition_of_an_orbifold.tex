% mainfile: ../praca_magisterska_orbifoldy.tex
\chapter{Definition of an orbifold}

In the definition of the orbifold we are following Thurston from \cite{Thurston1979} (chapter 13). 
We will briefly present here the concept, but we encourae the reader to consult 
\cite{Thurston1979}. \\
%We define an orbifold as a generalisation of manifold, where, istead of having homeomorphisms 
%between open sets of a manifold an $\mathbb{R}^n$ we have that for every point in the orbifold 
%there is a neighbourhood of it that is the image of the quotient map from the finito ... 
%\smalltodoII{rozwinąć powyżej}
%It is descriped very well and approunchable in \cite{Thurston1979} (chapter 13) and we would like 
%to redirect reader there.
We define an orbifold as a generalisation of a manifold. The difference is in maps. \\
On a manifolds a map is a homeomorphism between $\mathbb{R^n}$ and some open set on a manifold. 
On an orbifold a map is a homeomorphism between a quotient of $\mathbb{R^n}$ by some 
finite group and some open set on a orbifold.
In addition to that, the orbifold structure consist the informations about that finite group 
and a quotient map for any such open set. \\ 



Wersje: \\
- iloraz powierzchni przez grupe (wychodzą tylko dobre) \\
- rozmaitość z dodanymi punktami osobliwymi \\

Jak ma wyglądać:
jedna definicja (Thurstona)
notacja może byc od Conwaya

skleić drugi i trzeci rozdział

zwięźle własności


dobre i złe w trzecim rozdziale 

orbikompleksy może nie


This chapter and the next will be technical chapters. Later on we will evoke some terms and 
definitions without 
explicitly saying what they mean instead we will put a reference to this chapter with explicit 
saying to what definition it refers. \\
For example in the later chapters there will be phrases like "adding a defect of order $\dots$" or 
"gluing orbifolds by boundaries" and they are explained in this and the next chapter.
 

We will explore various definitions of an orbifold, partially proving they are equivalent, partially 
linking to the sources. \\
Some of these definitions apply only to the special cases. Some of them contain constructions 
with which not all orbifolds can be made (at least some of them can't be derived as such a priori)
. \\

\section{Where did orbifolds come from}


\section{Quotients of manifolds with the respect to group action}
\subsection{Hyperbolic plane tilling}

\section{Manifolds with defects}
\subsection{Disk and sphere with defects}\label{Disk_and_sphere_with_defects}



\subsection{Conway notation}
\cite{Conway2008}

When it is necessary to avoid a confusion, on parts such as $*abcd$, we will be writing 
$*^*a^*b^*c$ instead. \\
We will propose some extension to a notation from \cite{Conway2008}.
We will regard parts of that notation not only as features on an orbifold but also as an operations 
on orbifolds transforming one to another by adding particular feature. \\
We will denote the difference in Euler characteristic which is made by modifying 
an orbifold by such a feature as $\Delta(modification)$ 
which have less capitalistic vibes than "cost". 
For example $\Delta(*^*2) = \frac{1}{4}$. \\
We will denote by $*$ an operation of cutting out a disk and by $^{\beta*}n$ an operation of 
adding a kaleidoscopic point of period $n$ on the boundary component $\beta$. 
Last operation is defined only on orbifolds with boundaries.
%identified by Id. 
%When we will not concern ourselves on which component of the boundry kaleidoscopic point will 
%be added we


\section{Quotients of planes}

\section{Generalised manifolds}
This approach is very similar to the previous one. It differs slightly where we put the 
definition burden. 

%\section{Complexes?}

