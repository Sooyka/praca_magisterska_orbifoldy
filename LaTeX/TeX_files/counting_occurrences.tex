% mainfile: ../praca_magisterska_orbifoldy.tex
\chapter{Counting occurrences} \label{counting occurrences}
Our ultimate goal is to give the answer to the questions such as: \\
- For a given $x \in \sigma$, how many orbifolds have $x$ as their \Eoc?\\
- Why? Is there some underlying geometrical reason for that?\\
- Can we characterise points $x \in \sigma$ that has the most orbifolds corresponding to them? \\
- Is there any reasonable normalisation to counter the effect that there are 'more' points as we go 
to lesser values. (What we mean by 'more' was sted in) \\
The first equstion we can tackle is steaming from the chapter \ref{order structure} 
and it is -- Do $\sbD$ and $\sIS$ coincide? It is easy to answer that $\sbD \neq \sIS$ 
(and we will do that along some harder questions in the moment), but do they coincide 
starting from a sufficiently distant point? Or maybe, for every denominator, do they coincide 
from a sufficiently distatn point? (Yes.) \\
\section{Finitenes}
First we will show that for any $x \in \sigma$ there are always only finitely many orbifolds 
with an \Eoc\ equal to $x$. \\ 
Let us observe, that we only need to show this for sphere orbifolds. It is like that, because, 
as discussed in \ref{surgeries} every orbifold can be obtained be modyfying the sphere and 
%the set of differences in \Eoc\ made by
%all possible modifications that are not adding orbipoints is bounded. 
there is only finitely many possible modifications that are not adding an orbipoint, each 
changing \Eoc\ by non-zero value. \\ 

Suppouse, that there exists an orbifold with an \Eoc\ qual to $x$. Then 
\section{Some connections between \Eoc\ and geometry of corresponding orbifolds}
Here we will state some observations and corollaries derived from previous chapters 
about ... 
\begin{observation}
If an \Eoc\ is an accumulation point of order $n$ in $\sbD$ \dsa{$\sIS$}, there exist an 
orbifold of the type ... \dsa{} with $n$ cone \dsa{gyration} points of that \Eoc. 
\end{observation}
prrof. from chapter 3. (todo: dopisać)
\begin{observation}\label{adding_multiplied_differences}
If $x \in \sbD$ \dsa{$\sIS$}, then $1-x$ \dsa{$2-x$} is a difference in \Eoc\ resulting 
from some set of cone \dsa{gyration} points. From that $1-n(1-x) \in \sbD$ \dsa{$2-n(2-x)\in \sIS$} 
for all $n \in \mathbb{N}$. 
\end{observation}
%\begin{observation}

%\end{observation}
\section{$\sbD$ and $\sIS$}
In this section we would like to develop the tools and answer some questions about 
%interrelationships
relations between $\sbD$ and $\sIS$. \\
The first, stated in \ref{order structure} is that $2\sbD=\sIS$. 
This tells us all about simmilarities of their topological structures -- namely, they are the same, 
but it does not directly answers questions about how they lie in $\mathbb{R}$, relative 
to each other. 
\subsection{$-\frac{1}{84}$ and $-\frac{1}{42}$}
//Why it is how it is//
%\subsection{All the accumulation points of the $\sIS$ are in $\sbD$}
\subsection{Accumulation points of the $\sIS$}
\begin{theorem}
All accumulation points of the $\sIS$ are in $\sbD$.
\end{theorem}
There are two proofs of this theorem showing nice correnpondence -- one arithmetical and 
one geometrical. 
\\
\textbf{Proof I.}
%\subsubsection{Arithmetical reason}
\textbf{Arithmetical reason} \\
We assume that $x \in \sIS$ is an \apots\ $\sIS$.\\
%If $x \in \sIS$, then $\frac{x}{2} \in \sbD$. Then $1 - \frac{x}{2}$ 
%is a difference in \Eoc\ resulting from some set of cone points. We can add to the disc twice as 
%many cone points and resulting orbifold $\mathcal{O}$ will have an \Eoc\ equal to 
%$1 - 2(1-\frac{x}{2}) = x - 1$. From \ref{third_order_lemma} for the thesis it is sufficient 
%to $x - 1$ to be an \apots\ $\sbD$ of order at least two. \\
%%$2-x$ is a difference in \Eoc\ resulting from some set of gyration points. We can 
%We asumed that $x \in \sIS$ is an \apots\ $\sIS$, so, 
By \ref{times_two_fact} we have, that 
$\frac{x}{2} \in \sbD$ is an \apots\ $\sbD$. From \ref{third_order_lemma} we have that 
$\frac{x}{2} + \frac{1}{2} \in \sbD$. From that, from \ref{adding_multiplied_differences} we have, 
that $1-\overbrace{2}^{\substack{"n"\rm\ from\ \\ 
\rm \ref{adding_multiplied_differences}}} 
(\ \overbrace{1-(\frac{x}{2}+\frac{1}{2})}^{\substack{"1-x"\rm\ 
from\ \\ \rm \ref{adding_multiplied_differences}}}\ ) \in \sbD$. But $1 - 2(1-(\frac{x}{2}+
\frac{1}{2})) = x$, so $x \in \sbD$. $_\square$
% for some $y \in \sbD$, so $\sbD$.
\\[2pt]
\textbf{Proof II.}
%\subsubsection{Geometrical reason}
\textbf{Geometrical reason} \\
We assume that $x \in \sIS$ is an \apots\ $\sIS$.\\
%If $x \in \sIS$
From \ref{predescors} we know, that $x$ can be expressed as $y - 1$ for some $y \in \sIS$. \\
Let $\mathcal{O}$ be an orbifold with the base manifold $S^2$, such that $\cho{\mathcal{O}} 
= y$. \\
Let $\mathcal{O}_c$ be the orbifold created from $\mathcal{O}$ by adding one cusp. 
Then $\cho{\mathcal{O}_c} = y - 1 = x$. Topologically $\mathcal{O}_c$ is $\mathbb{R}^2$. 
We can compactify it with $S^1$. This will not change an \Eoc\ since $\cho{S^1} = 0$ and 
\Eoc\ is additive.
\\ What we get is an orbifold $\mathcal{O}_D$ with the base 
manifold $D^2$ and the same 
orbipoints as $\mathcal{O}$. Since orbipoints of $\mathcal{O}$ create a difference 
in \Eoc\ equal to $2-y$, we have that $\cho{\mathcal{O}_D} = 1 - (2-y) = y - 1 = x$. 
We can then move all orbipoints from the interior of $\mathcal{O}_D$ to its boundry 
by doubling them. $_\square$



\section{Estimations of the number of occurences}









\section{Deformations on orbifolds?}











