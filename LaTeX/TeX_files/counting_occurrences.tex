% mainfile: ../praca_magisterska_orbifoldy.tex
\chapter{Counting occurrences} \label{counting occurrences}
Our ultimate goal is to give the answer to the questions such as: \\
- For a given $x \in \sigma$, how many orbifolds have $x$ as their \Eoc?\\
- Why? Is there some underlying geometrical reason for that?\\
- Can we characterise points $x \in \sigma$ that has the most orbifolds corresponding to them? \\
The first equstion we can tackle is steaming from the chapter \ref{order structure} 
and it is -- Do $\sbD$ and $\sIS$ coincide? It is easy to answer that $\sbD \neq \sIS$ 
(and we will do that along some harder questions in the moment), but do they coincide 
starting from a sufficiently distant point? Or maybe, for every denominator, do they coincide 
from a sufficiently distatn point? (Yes.) \\
\section{Some connections between \Eoc\ and geometry of corresponding orbifolds}
Here we will state some observations and corollaries derived from previous chapters 
about ... \\
\begin{observation}
If an \Eoc\ is an accumulation point of order $n$ in $\sbD$ ($\sIS$), there exist an 
orbifold of the type ... () with $n$ cone points (... points) with that \Eoc. 
\end{observation}
prrof. from chapter 3. (todo: dopisać)
\section{$\sbD$ and $\sIS$}
In this section we would like to develop the tools and answer some questions about 
interrelationships between $\sbD$ and $\sIS$. \\
The first, stated in \ref{order structure} is that $2\sbD=\sIS$. 
This tells us all about simmilarities of their topological structures -- namely, they are the same, 
but it does not directly answers questions about how they lie in $\mathbb{R}$, relative 
to each other. \\
\subsection{$-\frac{1}{84}$ and $-\frac{1}{42}$}
//Why it is how it is//
\subsection{All the accumulation points of the $\sIS$ are in $\sbD$}
\subsubsection{Arithmetical reason}
If $x \in \sIS$, then $2-x$ is a difference in \Eoc\ resulting from some set of gyration points. 
We can 
\subsubsection{Geometrical reason}
If $x \in \sIS$ 











\section{Deformations on orbifolds?}
