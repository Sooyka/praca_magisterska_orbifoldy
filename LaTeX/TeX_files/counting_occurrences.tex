% mainfile: ../praca_magisterska_orbifoldy.tex
\chapter{Counting occurrences} \label{counting occurrences}
%In this chapter we would care not only whether the pointis or is not in the spectrum -- we will 
%give a closer look of how many orbifolds have particular \Eoc. \\
%Our ultimate goal is to give the answer to the questions such as: \\
%- For a given $x \in \sigma$, how many orbifolds have $x$ as their \Eoc?\\
%- Why? Is there some underlying geometrical reason for that?\\
%- Can we characterise points $x \in \sigma$ that has the most orbifolds corresponding to them? \\
%- Is there any reasonable normalisation to counter the effect that there are 'more' 
%points as we go 
%to lesser values. (What we mean by 'more' was sted in) \\
%The first equstion we can tackle is steaming from the chapter \ref{order structure} 
%and it is -- Do $\speD$ and $\speS$ coincide? It is easy to answer that $\speD \neq \speS$ 
%(and we will do that along some harder questions in the moment), but do they coincide 
%starting from a sufficiently distant point? Or maybe, for every denominator, do they coincide 
%from a sufficiently distatn point? (Yes.) \\
%\todo{przenieść (przekopiować?) część może do futher direction}
%write about cyclic order

The central question of this section is: "given a rational number, how many orbifolds 
have that \Eoc ?". The warning for this chapter is, that this question will remain unansweared, 
however, we can provide some glimpse into the partial answear. 

%To answear this, we should 
%first check, whether for any number, there are only fihis numer is always finite.
%Secondly:
In the first section, we will show that for any number, there are only finitely many 
orbifolds with that \Eoc. 
In sebsequent section, we will separate the problem of finding the exact number into 
more arithmetical 
part and one more combinatorial.

%There are two aspects of counting occurensec
%Even three.
%One, we can state some finitness conditions -- this is useful, to sort out what questions 
%are meaningfull
%Then there are study of expressions of he form:
%arithmetical expresions
%and the study of cmbinatorics.

\section{Finitenes}

%Here we will show that for any $x \in \sigma$:
%\begin{itemize}
%\item for every $n \in \mathbb{N}$ there are only finitely many orbifolds 
%with the \Eoc\ greater or equal to $x$ and all orbipoints of order at most $n$
%\item there are finitely many orbifolds with an \Eoc\ equal to $x$. 
%\end{itemize}

\begin{observation}\label{first_finiteness_theorem}
For any $x \in \sigma$ and $n \in \mathbb{N}$ there are only finitely many orbifolds 
with the \Eoc\ greater or equal to $x$ and all orbipoints of order at most $n$.
\end{observation}
\noindent\textbf{Proof:} 

For a given $x$, there are only finitely many manifolds 
with an Euler characteristic $y \geq x$. Only them can be 
base manifolds for an orbifold with an \Eoc\ $y'\geq x$, as adding orbipoints always 
decreases an \Eoc. 

It remains to prove then, that for any base manifold $M$, there are only finetly many orbifolds, 
with $M$ as a base manifold, that have 
 an \Eoc\ $y \geq x$, and all orbipoints of order at most $n$.

We proceed now simmilarly to the proof of \ref{finiteness_lemma} -- 
on the orbifold with an \Eoc\ $y \in [x,2]$, there can be at most 
$\max \{\lfloor 4(1-x) \rfloor, \lfloor 2(2-x) \rfloor\}$ orbipoints. 
Thus, for a given manifold $M$ and a given $x$ and $n$, there can be at most 
$(n-1)^{\max \{\lfloor 4(1-x) \rfloor, \lfloor 2(2-x) \rfloor\}}$ orbifolds with an \Eoc\ 
$y \geq x$, 
all orbipoints of order at most $n$ and $M$ as a base manifold.$_\square$ 

\begin{theorem}\label{second_finiteness_theorem}
For any $x \in \sigma$ there are only finitely many orbifolds 
with the \Eoc\ equal to $x$.
\end{theorem}
\noindent\textbf{Proof:} 

% Suppouse, that there exists an orbifold with an \Eoc\ qual to $x$. Then, 
Let $x$ be a rational number. 
Let $\mathcal{O}$ be the set of all orbifolds with 
an \Eoc\ qual to $x$. 
Those orbifolds can have different base manifolds. However, the set of base manifolds of 
orbifolds from $\mathcal{O}$ is finite, as there are only finitely many 
two dimentional manifolds with an Euler characteristic greater or equal to $x$ and an 
orbifold always has 
an \Eoc\ less or equal to the Euler characteristic of its underlying manifold. 

%We will now provide an argument that shows that for any base manifold $M$, the number 
%of $M$ orbifolds with \Eoc\ equal to $x$ is finite. It will complete the proof of our theorem.\\
It remains to proof, that for any base manifold $M$, the number of $M$ orbifolds 
with \Eoc\ equal to $x$ is finite: 

Let $M$ be a two dimentional manifold. 

For the sake of contradiction, assume, that there exists an infinite set $\mathcal{O}_M$
of $M$-orbifolds such that $\mathcal{O}_M \subseteq \mathcal{O}$ ($\ast$).

If $M$ has some boundary, orbifolds in $\mathcal{O}_M$ 
can have both rotational and dihedral orbipoints. 
For the simplicity of the following part of the proof we want now to reduce this case 
to a case 
where only one type of the orbipoints is present. 

Let us observe, that every rotational orbipoint can be replaced by two dihedral orbipoints 
of the same order without changing the \Eoc. Thus, if there would be infinitely many 
obifolds in $\mathcal{O}_M$ having both rotational and dihedral orbipoints, there would be 
also infinitely many orbifolds in $\mathcal{O}_M$ having only dihedral orbipoints. 
Thus it is sufficient to prove that there are finetly many orbifolds in $\mathcal{O}_M$ 
that have only one type of the orbipoints. 

We will now perform the proof of above statement in the case of dihedral orbipoints.
The proof for rotational orbipoints is completely analogous. 

Let us call the subset of $\mathcal{O}_M$ 
that consists only of orbifolds with only dihedral orbipoints by $\mathcal{O}_M^d$.

Let $\mathcal{O}_M = \{O_i\}_{i \in I}$. 
For each $i$, let $s_i = (b^0_i, \cdots, b^{l_i}_i)$ be the signature 
of $O_i$ written with decreasing orders of 
dihedral points. So for each $i$ we have, that $b^0_i$ is 
the order of the orbipoint with the highest order of all the dihedral orbipoints of 
$O_i$. 
By \ref{first_finiteness_theorem} we know that if the set $\{b^0_i\}_{i \in I}$ 
would be bounded 
by some $n \in \mathbb{N}$ it would mean, that $\mathcal{O}_M^d$ would be finite. As 
from ($\ast$) this is not the case, we know
that the set $\{b^0_i\}_{i \in I}$ is unbounded. 
Let $\{i_n\}_{n\in \mathbb{N}} \subseteq I$ be a sequence of indeces such that
$\{b^0_{i_n}\}_{n\in \mathbb{N}}$ is strictly increasing. 

%Let $\{a_n\}$ be the sequence such that $a_n = \frac{m_n-1}{m_n} \end{cases}$ if point 
%corresponding 
%to $m_n$ is an rotational point and $a_n = \frac{m_n-1}{2m_n}$ if point corre 
Let $\{x_n\}$ be the sequence such that $x_n = \Delta(b^0_{i_n})$.
Let $\{y_n\}$ be the sequence such that $y_n = \Delta(b^1_{i_n},
\cdots, b^{l_{i_n}}_{i_n})$. 
So for every $n$ we know that $\cho{O_{i_n}} = \chi(M) + a_n + b_n$. As $\{b^0_{i_n}\}$ 
is strictly 
increasing, we know 
that $a_n$ is strictly decreasing, so $b_n$ must be strictly 
increasing  
%(*), 
(we have that $\cho{O_{i_n}}$ is constant for all $n$, since all $O_{i_n}$ 
are from the family with \Eoc\ equal to $x$). 

But $\{b_n\} \subseteq \spebr{M} - \chi(M)$. 
From \ref{well_order} and \ref{times_two_fact} we know that $\spebr{M}$ has no infinite 
strongly increasing sequences, so 
$\spebr{M} - \chi(M)$ has no infinite strongly increasing sequences. That gives us a 
contradiction. 
%*. 
%\Lightning 
$_\square$ 


%First we will show that for any $x \in \sigma$ there are always only finitely many orbifolds 
%with an \Eoc\ equal to $x$. \\ 
%Let us observe, that we only need to show this for $S^2$ orbifolds. It is like that, because, 
%as discussed in \ref{surgeries} every orbifold can be obtained be modyfying the sphere and 
%%the set of differences in \Eoc\ made by
%%all possible modifications that are not adding orbipoints is bounded. 
%there is only finitely many possible modifications that are not adding an orbipoint, each 
%changing \Eoc\ by non-zero value. \\ 

%\begin{theorem}
%For any $x \in \sigma$ there are always only finitely many orbifolds 
%with an \Eoc\ equal to $x$.
%\end{theorem}
%\textbf{Proof:} \\
%% Suppouse, that there exists an orbifold with an \Eoc\ qual to $x$. Then, 
%According to the note above, we only need to proof this for $S^2$ orbifolds. \\ 
%Let $x$ be a rational number. 
%For the sake of contradiction, assume, that there exists an infinite family of orbifolds 
%$\{\mathcal{O}\}_{i \in I}$ with an \Eoc\ of each qual to $x$. For each $i$, tet $m_i$ be the 
%order of the orbipoint with the highest order of $\mathcal{O}_i$. As for every $n \in \mathbb{N}$ 
%there are only finitely many $S^2$ orbifolds with all orbipoints of order less than $n$, we have 
%that the set $\{m_i\}_{i \in I}$ is unbounded. Let $\{m_n\}_{n\in \mathbb{N}}$ be some strictly 
%increasing sequence 
%of elements of $\{m_i\}_{i \in I}$ that diverges into infinity. \\
%Let $\{a_n\}$ be the sequence of differences in \Eoc\ caused by points corresponding 
%to $\{m_i\}$. 
%Let $\{b_n\}$ be the sequence of differences in \Eoc\ caused by other points on those orbifolds. 
%So for every $n$ we have $\cho{\mathcal{O}_n} = 2 + a_n + b_n$. As $\{m_n\}$ is strictly 
%increasing we have that $a_n$ is strictly decreasing, so $b_n$ must be strictly 
%increasing, because $\cho{\mathcal{O}_n}$ is constant for all $n$ (all $\{\mathcal{O}_n\}$ 
%are from the family with \Eoc\ equal to $x$). \\ 
%But $\{b_n\} \subseteq \speS - 2$, so it is well ordered as $\speS$ is well ordered. 
%From \ref{well_order} and \ref{times_two_fact} we know that $\speS$ has no infinite 
%strongly increasing sequences, so 
%$\speS - 2$ has no infinite strongly increasing sequences. That gives us a contradiction. 
%\Lightning $_\square$ 
 
%\section{Some connections between \Eoc\ and geometry of corresponding orbifolds}
%\todo{zobaczyć, czy ten rozdział ma sens}
%Here we will state some observations and corollaries derived from previous chapters 
%about ... 

%%\begin{observation}

%%\end{observation}


\section{Infinitness}
\subsection{Unboundeness of some number of occurences}
We know, that for any $x$, there are only finetely many orbifolds with $x$ as an \Eoc . 
However, we can ask about some boundness of number of these orbipoints. 
In particular, we could ask, whether near any accumulation point, there will be $x$ with an 
arbitrary large number of orbifolds corresponding to it. 
The answer will be yes, and it can be formulated as such:
\begin{theorem}\label{unboundness}
For any neighbourhood $U$ of any accumulation point of $\speD$ of order at least $2$, for any 
$n\in \mathbb{N}$, 
there exists an $x\in U$ such that there are at least $n$ orbifolds with $x$ as their 
\Eoc.
\end{theorem}
\textbf{Proof.}\\
This will follow from the theorem about the sums of egiptian fractions from \cite{Browning2011}.
It states that for ...
%Napisać, że dowolnie blisko każdego punktu skupienia da się znaleźć liczbę o dowolnie wielu 
%odpowiadających jej orbifoldach.

%\todo{dac jakieś źródła i ok}



\section{Dividing the problem into an arithmetical and combinatorial parts}

Given a number $x$, the question about how many orbifolds have $x$ as an \Eoc\ can be 
partially answeared by asking the question how many of sums of the form:
\begin{equation}\label{counting D2}
1-\sum_{i=1}^m \frac{d_j-1}{2d_j} 
%-\sum_{i=1}^n \frac{r_i-1}{r_i}
\end{equation} 
and
\begin{equation}\label{counting S2}
2-\sum_{i=1}^n \frac{r_i-1}{r_i}
\end{equation}
with $n\in \mathbb{N}$ and $\forall_i\ d_i,r_i\in\mathbb{N}\cup\{\infty\}$, are equal to $x$. 

It is a matter of convention (and then coherently translating this convention to the final result) 
what sums are we treating as "the same". The convention we will take, is that a sum is determined 
uniquly by the tuple $(d_1,\dots,d_n)$ \rba{or $(r_1,\dots,r_n)$} of orders 
of orbipoints, ordered in deacreasing order, appearing in the sum. 

%This covers the "arithmetical" 
%and combinatorics 
%of the "orbifold" 
%part of the problem. 
In this sums there is a structure of what \Eoc\ orbipoints can produce. 
The question "How many different sums (understood by above convention) are equal to a given $x$?" 
This is the first part of the problem -- the arithmetical part, 
%adressed in 
%\ref{conting_arithmetical}. 
%This is the hard and only partially answered part.
%This however, does not give us the full information. 
%It gives an aswear to the question "How many $S^2$ orbifolds have an \Eoc\ equal to $x$" and 
%"How many $D^2$ orbifolds have an \Eoc\ equal to $x$"
For once, without changing \Eoc\ some 
orbipoints can be replaced 
by orbipoints of a different type giving orbifolds with both rotational and dihedral orbipoints; or 
by a features on a manifold such as handles, cross-cups and 
boundry components, giving orbifold of the same \Eoc, but different base manifold. 
Secondly, when the orbipoints lie on the boundry components, their 
order of placement around the boundary component matters as orbifolds with orbipoints on boundry components with different order 
are not neccesery the same.  

This is a combinatorical part. Here we would make an assumption that we know 
the answear to the arithmetical question -- given $x$ how many sums of the form 
\ref{counting D2} and \ref{counting S2} are equal to $x$. Then we will derive from 
this the proper number of orbifolds of a given \Eoc\ $x$. 
%This gives us a question -- "How many different orbifolds produce the same sum?".
%This will be answeared in the section \ref{counting_combinatorics}.
%The other question is -- "How many manifold with some sums correspont to $x$?" and this will 
%be treated in \ref{conting_diff_man} together with more precise notions of these questions. 

\todo{Write reductions first to separately computing for different manifolds}

\section{Different manifolds}\label{conting_diff_man}
When answearing the question "Given number $x$, how namy orbifolds have $x$ as their \Eoc?" we will first 
divide it into the series of questions with specified base manifold each. 

For each manifold $M$, we will answear the question:
"Given number $x$, how namy $M$-orbifolds have $x$ as their \Eoc?" 
Note, that for $M$ such that $\chi(M) < x$, the answear is always $0$, since 
orbifolds have smaller \Eoc\ than their base manifolds 
(\ref{orbifolds have smaller Eoc than their base manifolds}).

This question about different manifolds can be changed to the question about order of 
accumulation. 

\begin{lemma}

\end{lemma}

For a point $x$, if it is of order $n$, then all $x+1,...,x+n$ are also in the spcterum
as such we can take differences

all that n can go into manifold features.
 
To answear the question how much 

%\section{Deformations on orbifolds?}

\section{Arithmetical part}\label{conting_arithmetical}
%This is the part of unansweared question --



In \ref{algorithm} we provide and algorithm to compute.

%Algorithm that is proved to stop is a very elaborated equation. 
%We treat the problem as partialy solved however as it does not gives any particular glimpse 
%into the structure
%of why it is such a number. 
%It is however computable.
%How much sums correspond also is reducible to $D^2$.

%The result will hav emore algorithmical nature

\section{Combinatorial part}\label{counting_combinatorics}
%This case is simple in 













