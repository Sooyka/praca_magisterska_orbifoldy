% mainfile: ../praca_magisterska_orbifoldy.tex
%\synctex=1
\chapter{Order structure}

In this chapter we will discuss order type of the set of all possible Euler orbicharacteristics 
of two dimentional orbifolds. \\
For now, until Chapter \ref{Counting_occurrences} Counting occurrences, we will not pay attension 
to how many orbifolds have the same Euler orbicharacteristic. \\ 
Because of that and since Euler orbicharacteristic does not depend on the cyclic order 
of points on the conponents of the boundry we introduce an extension of a notation from 
\cite{Conway2008}. 

%We will write $*\{a,b,c,d,\dots\}$ to denote not a particular orbifold, but a type 
%of orbifold that have a cones on a component of the boundry of orders $a,b,c,d,\dots$, but 
%in any order. \\ 

We will write $*\{a,b,c,d,\dots\}$ to denote a type of a boundry (of an orbifold) that have 
kaleidoscopic points of periods $a,b,c,d,\dots$, but in any order. \\


From what we wrote above (that Euler orbicharacteristic does not depend on the cyclic order 
of points on the conponents of the boundry), we can see that Euler 
orbicharacteristic is well defined 
%on such types of orbifolds. \\ 
when we specify only such a type of the components of the boundry of an orbifold and not 
a particular cyclic order.  \\

\section{Reductions of cases}
Now we want to make some reductions to limit number of cases that we will be dealing with. \\
For this chapter we will consider orbifolds according to a definition from 
(\ref{Disk_and_sphere_with_defects}). \\
Let us observe, that:
\begin{align*}
\hspace{3cm}\Delta(\circ) =& \hspace{-1cm} &-2& \hspace{-1cm} &= \Delta(*(^*2)^4) \\
\hspace{3cm}\Delta(*) =& \hspace{-1cm} &-1& \hspace{-1cm} &= \Delta((^*2)^4) \\
\hspace{3cm}\Delta(n) =& \hspace{-1cm} &\frac{n-1}{n}& \hspace{-1cm} &= \Delta((^*n)^2)
\end{align*}

From this we can conclude, that every Euler orbicharacteristic can be obtained 
by an orbifold of signature of a type ($n$ and $m$ are arbitrary):

\begin{align*}
I_1I_2\dots I_n & \textrm{or} \\
*b_1b_2\dots b_m &.
\end{align*}

Let us denote the set of all possible Euler orbicharacteristics of orbifolds of the form 
$I_1I_2\dots I_n$ by $\sIS$ and the set 
of all possible Euler orbicharacteristics of orbifolds of the form $*b_1b_2\dots b_m$ 
as $\sbD$

Let us observe that the topological structure of $\sIS$ and $\sbD$ are the same since 

\begin{equation*}
2\sbD=\sIS
\end{equation*}
So multiplying by $2$ is the homeomorphism. 
\section{Determining the order structure}
In this chapter we will justify, that the order type of all possible Euler orbicharacteristics 
of two dimentional orbifolds is $\omega^\omega$. 
We will also describe precisely where condensation points lie and of which order 
(see below \ref{condensation_points_definitions}) they are.
\subsection{Definitions regarding order of condensation points}
\label{condensation_points_definitions} 
We start with one technical definition of "transitive order" that will be almost what we want
and then, there will be the the definition of "order", which is the definition that we need. \\ 
\begin{definition}
(Inductive). 
%We say, that the point $a$  from the topological space $X$ is a condensation 
%point of the transitive order 0, when
We say that the point is a condensation point of a transitive order $0$, when it is 
an isolated point. 
We say that the point is a condensation point of a transitive order $n + 1$, when it is 
a condensation point (in the usual sense) of the condensation points of the transitive order $n$. 
\end{definition}  
The only issue of the definition is that the point of the transitive order $n$ is also a point 
of the transitive order $k$, for all $0< k \leq n$. We want a definition of order such that 
for any point, there is at most one integer that is its order. So we define:
\begin{definition}
We say that the point is a condensation point of order $n$ iff it is a condensation point 
of the transitive order $n$ and it is not a condensation point of the transitive order $n+1$. 
If the point is a condensation point of the transitive order for an arbitraly large $n$ we say that 
the  point is a condensation point of order $\omega$.
\end{definition}

\subsection{$\sbD$}
\subsubsection{Some preliminary observations.}
Let us observe, that $\lim\limits_{n \to \infty} \Delta(^*n) = -\frac{1}{2}$. From that, we see, 
that for every point $x \in \sbD$, the point $x - \frac{1}{2}$ is a condensation point. 
Let us observe, that also, for every point $x \in \sbD$, we have that $x - \frac{1}{2} \in \sbD$, 
because $\Delta((^*2)^2) = -\frac{1}{2}$. \\

Now we will show that the order type of $\sbD$ is $\omega^\omega$ and where exactly are 
its condensation points of which orders. For this we will use  
a handfull of lemmas. 

\begin{lemma}
If $x$ is a condensation point of the set $\sbD$ of order $n$, then $x-\frac{1}{2}$ is a
 condensation point of the set $\sbD$ of order at least $n+1$. 
\end{lemma}
\textbf{Proof.} \\
Inductive. \\
$\bullet$ $n = 0$: If $x$ is an isolated point of the set $\sbD$, then $x \in \sbD$. From that, we 
have, that points $x - \frac{k-1}{2k}$ are in $\sbD$, from that, that $x-\frac{1}{2}$ is a 
condensation point of $\sbD$. \\
$\bullet$ inductive step: Let $x$ be a condensation point of the set $\sbD$ of an order $n > 0$. 
Let $a_k$ be a sequence od condensation points of order $n-1$ convergent to $x$. From the 
inductive assumption, we have, that $a_k - \frac{1}{2}$ is a sequence of condensation points 
of order at least $n$. From the basic sequence arithmetic it is convergent to $x-\frac{1}{2}$. 
From that, we have that $x-\frac{1}{2}$ is a condensation point of the set $\sbD$ of order 
at least $n+1$. $_\square$
\begin{lemma}
If $x$ is a condensation point of the set $\sbD$ of order $n+1$, then $x+\frac{1}{2}$ is 
a condensation point of the set $\sbD$ of order at least $n$.  
\end{lemma}
\textbf{Proof.} \\
Inductive \\
$\bullet$ n = 0: 






