% mainfile: ../praca_magisterska_orbifoldy.tex
%\synctex=1
\chapter{Order structure} \label{order structure}
%
%Order type with zanurzenie w R
%
%1537/137
%
In this chapter we will discuss order type of the set of all possible Euler orbicharacteristics 
of two dimensional orbifolds. \\

For now, until Chapter \ref{counting occurrences} Counting occurrences, we will not pay attention 
to how many orbifolds have the same Euler orbicharacteristic. Let us note, that 
Euler orbicharacteristic does not depend on the cyclic order of points on 
the components of the boundary. 
%Because of that and since Euler orbicharacteristic does not depend on the cyclic order 
%of points on the components of the boundary we introduce an extension of a notation from 
%\cite{Conway2008}. 

%%We will write $*\{a,b,c,d,\dots\}$ to denote not a particular orbifold, but a type 
%%of orbifold that have a cones on a component of the boundry of orders $a,b,c,d,\dots$, but 
%%in any order. \\ 

%We will write $*\{a,b,c,d,\dots\}$ to denote a type of a boundary (of an orbifold) that have 
%kaleidoscopic points of periods $a,b,c,d,\dots$, but in any order. \\


%From what we wrote above (that Euler orbicharacteristic does not depend on the cyclic order 
%of points on the components of the boundary), we can see that Euler 
%orbicharacteristic is well defined 
%%on such types of orbifolds. \\ 
%when we specify only such a type of the components of the boundary of an orbifold and not 
%a particular cyclic order.  \\

\section{Reductions of cases}
Now we want to make some reductions to limit number of cases that we will be dealing with. \\
%For this chapter we will consider orbifolds according to a definition from 
%(\ref{Disk_and_sphere_with_defects}). \\
Let us observe, that:
\begin{align*}
\hspace{3cm}\Delta(\circ) =& \hspace{-1cm} &-2& \hspace{-1cm} &= \Delta(*(^*2)^4) \\
\hspace{3cm}\Delta(*) =& \hspace{-1cm} &-1& \hspace{-1cm} &= \Delta((^*2)^4) \\
\hspace{3cm}\Delta(n) =& \hspace{-1cm} &\frac{n-1}{n}& \hspace{-1cm} &= \Delta((^*n)^2)
\end{align*}

From this we can conclude, that every Euler orbicharacteristic can be obtained 
by an orbifold of signature of a type ($n$ and $m$ are arbitrary):

\begin{align*}
I_1I_2\dots I_n & \textrm{\ or} \\
*b_1b_2\dots b_m &.
\end{align*}

Let us denote the set of all possible Euler orbicharacteristics of orbifolds of the form 
$I_1I_2\dots I_n$ by $\sIS$ and the set 
of all possible Euler orbicharacteristics of orbifolds of the form $*b_1b_2\dots b_m$ 
as $\sbD$. \\
Let us denote the set of all possible \Eoc s of two dimentional orbifolds as $\sigma$. \\
Let us observe that the topological structure of $\sIS$ and $\sbD$ are the same since 

\begin{equation}\label{times_two_fact}
2\sbD=\sIS
\end{equation}
So multiplying by $2$ is the homeomorphism. \\
From above reductions we conclued that our problem boiled down to analysis possible 
values of the expression:
\begin{equation}
2 - \sum_{i=1}^n \frac{I_i-1}{I_i}. 
\end{equation}
and 
\begin{equation}
1 - \sum_{j=1}^m \frac{b_j-1}{2b_j}
\end{equation}
We also have shown that all possible \Eoc s are achieved without using cusps. As such, we will use 
cusps, remembering, that we can always get rid of them, if needed. So above $I_i$ and $b_j$ 
are ranging over $\mathbb{N}_{>0}\cup \{\infty\}$, where expressions for infinity are defined as 
a limits. The fact that it agrees with the definition of the \Eoc\ on the geometrical terms was 
addressed here \ref{extended_Euler_orbicharacteristic}.  
\section{Determining the order structure}
In this chapter we will justify, that the order type of all possible Euler orbicharacteristics 
of two dimensional orbifolds is $\omega^\omega$. 
We will also describe precisely where accumulation points lie and of which order 
(see below \ref{accumulation_points_definitions}) they are.
\subsection{Definitions regarding order of accumulation points}
\label{accumulation_points_definitions} 
We start with one technical definition of "transitive order" that will be almost what we want
and then, there will be the the definition of "order", which is the definition that we need. \\ 
\begin{definition}
(Inductive). 
%We say, that the point $a$  from the topological space $X$ is an acccumulation 
%point of the transitive order 0, when
We say that the point is an acccumulation point of a transitive order $0$, when it is 
an isolated point. 
We say that the point is an acccumulation point of a transitive order $n + 1$, when it is 
an acccumulation point (in the usual sense) of the accumulation points of the transitive order $n$. 
\end{definition}  
The only issue of the definition is that the point of the transitive order $n$ is also a point 
of the transitive order $k$, for all $0< k \leq n$. We want a definition of order such that 
for any point, there is at most one integer that is its order. So we define:
\begin{definition}
We say that the point is an acccumulation point of order $n$ iff it is an acccumulation point 
of the transitive order $n$ and it is not an acccumulation point of the transitive order $n+1$. 
If the point is an acccumulation point of the transitive order for an arbitrary large $n$ we say that 
the  point is an acccumulation point of order $\omega$.
\end{definition}

When we will say that a point is an accumulation point of some set without specifying an order 
then we will mean being an accumulation point in the ussual sense; from the point of view 
of above definitions, that is, an accumulation point of order as least one.

\subsection{Order structure of $\sbD$}
\subsubsection{Some preliminary observations.}
Let us observe, that $\lim\limits_{n \to \infty} \Delta(^*n) = -\frac{1}{2}$. From that, we see, 
that for every point $x \in \sbD$, the point $x - \frac{1}{2}$ is an acccumulation point. 
Let us observe, that also, for every point $x \in \sbD$, we have that $x - \frac{1}{2} \in \sbD$, 
because $\Delta(^*\infty) = -\frac{1}{2}$. \\

Now we will show that the order type of $\sbD$ is $\omega^\omega$ and where exactly are 
its accumulation points of which orders. For this we will use  
a handful of lemmas. 

\begin{lemma}\label{first_order_lemma}
If $x$ is an acccumulation point of the set $\sbD$ of order $n$, then $x-\frac{1}{2}$ is a
 accumulation point of the set $\sbD$ of order at least $n+1$. 
\end{lemma}
\textbf{Proof.} \\
Inductive. \\
$\bullet$ $n = 0$: If $x$ is an isolated point of the set $\sbD$, then $x \in \sbD$. From that, we 
have, that points $x - \frac{k-1}{2k}$ are in $\sbD$, from that, that $x-\frac{1}{2}$ is a 
accumulation point of $\sbD$. \\
$\bullet$ inductive step: Let $x$ be an acccumulation point of the set $\sbD$ of an order $n > 0$. 
Let $a_k$ be a sequence of accumulation points of order $n-1$ convergent to $x$. From the 
inductive assumption, we have, that $a_k - \frac{1}{2}$ is a sequence of accumulation points 
of order at least $n$. From the basic sequence arithmetic it is convergent to $x-\frac{1}{2}$. 
From that, we have that $x-\frac{1}{2}$ is an acccumulation point of the set $\sbD$ of order 
at least $n+1$. $_\square$
\begin{lemma}\label{second_order_lemma}
If $x$ is an acccumulation point of the set $\sbD$ of order $n$, then $x+\frac{1}{2}$ is 
an acccumulation point of the set $\sbD$ of order at least $n-1$.  
\end{lemma}
\noindent\textbf{Proof.} \\
Inductive \\
$\bullet$ $n = 1$: We assume, that $x$ is an acccumulation point of isolated points of the set 
$\sbD$. Let us observe, that for all $m$ there are only finitely many Euler orbicharacteristics 
in the interval $[1,x]$ of orbifolds that have cone points of period equal at most $m$. \\ 
From that, for arbitrary small neighborhood $U \ni x$ and arbitrary large $m$ there exist an orbifold 
that has a cone point of period grater than $m$, whose Euler orbicharacteristic lies in $U$. 
Let us take a sequence of such \Eoc s $a_k$ convergent to $x$, such that we can choose 
a sequence divergent to infinity of periods of cone points $b_k$ of orbifolds of \Eoc s equal $a_k$. 
\smalltodoII{picture} 
Let us observe, that for all $k$, the number $a_k+\frac{b_k-1}{2b_k}$ is in $\sbD$. 
It is so, because $a_k$ is an \Eoc of an orbifold that have a cone point of period $b_k$, so 
identical orbifold, only without this cone point has an \Eoc equal to $a_k + \frac{b_k-1}{2b_k}$. 
The sequence $a_k + \frac{b_k-1}{2b_k}$ converge to $x+\frac{1}{2}$. From that we have, that 
$x + \frac{1}{2}$ is an acccumulation point of the set $\sbD$ of order at least $0$. \\
$\bullet$ inductive step: Let $x$ be an acccumulation point of the set $\sbD$ of order $n > 1$. 
Let $a_k$ be a sequence of accumulation points of the set $\sbD$ of order $n-1$ convergent to $x$. 
From the inductive assumption the sequence $a_k + \frac{1}{2}$ is a sequence of an acccumulation
 points of the set $\sbD$ of order $n-2$ convergent to $x + \frac{1}{2}$. From that 
 $x + \frac{1}{2}$ is an acccumulation point of the set $\sbD$ of order at least $n-1$. $_\square$ 
\begin{lemma}\label{third_order_lemma}
If $x$ is an acccumulation point of the set $\sbD$ of order $n+1$, then \\
$x - \frac{1}{2}$ is an acccumulation point of the set $\sbD$ of order $n+2$ and \\
$x + \frac{1}{2}$ is an acccumulation point of the set $\sbD$ of order $n$. 
\end{lemma}
\noindent\textbf{Proof.} \\
Let $x$ be an acccumulation point of the set $\sbD$ of order $n+1$. From the lemma 
 \ref{first_order_lemma} we know, that $x - \frac{1}{2}$ is an acccumulation point of the set 
 $\sbD$ of order at least $n+2$. Now let us assume (for a contradiction), that $x - \frac{1}{2}$ 
 is an \apots $\sbD$ of order $k>n+2$. But then from the lemma \ref{second_order_lemma} 
 we have that $x$ is an acccumulation point of the set $\sbD$ of order at least $n+2$ and that 
 is a contradiction. \\
Analogously, from the lemma \ref{second_order_lemma} we know, that $x + \frac{1}{2}$ is a 
accumulation point of the set $\sbD$ of order at least $n$. Let us assume (for a contradiction), 
that $x+ \frac{1}{2}$ is an acccumulation point of the set $\sbD$ of order $k>n$. But then 
from the lemma \ref{first_order_lemma} we have that $x$ is an acccumulation point of the set $\sbD$ 
of order at least $n+2$ and that is a contradiction. $_\square$ 
\begin{lemma}\label{accumulations_points_of_the_set}
For all $n \in \mathbb{N}$ all accumulation points of the set $\sbD$ of order $n$ are in $\sbD$.
\end{lemma}
\noindent\textbf{Proof.} \\
Inductive \\
$\bullet$ $n=0$: Clear, as they are isolated points of $\sbD$. \\
$\bullet$ inductive step: Let $x$ be a \apots  $\sbD$ of order $n>0$. From the lemma 
\ref{third_order_lemma} point $x+\frac{1}{2}$ is an acccumulation point of the set $\sbD$  
of order $n-1$. From the inductive assumption $x+\frac{1}{2} \in \sbD$. Then $x \in \sbD$. 
$_\square$ 
\begin{lemma}\label{two_sets_lemma}
If $A, B \subseteqq \mathbb{R}$ have no infinite ascending sequences, then set 
$A + B \coloneqq \{a+b\ |\ a \in A, b \in B\}$ also have no infinite ascending sequences. 
\end{lemma}
\noindent\textbf{Proof.} \\
Let $A$, $B$ have no infinite ascending sequences. 
Let $c_n \in A + B$ are elements of some sequence. With a sequence $c_n$ there are 
two associated sequences $a_n$, $b_n$, such that, for all $n$, we have $a_n \in A$, $b_n \in B$ and 
$a_n + b_n = c_n$. Assume (for contradiction), that $c_n$ is an infinite ascending sequence. 
Then $\forall_n\ a_{n+1}>a_n\ \lor\ b_{n+1} > b_n$. From the assumption $a_n$ has no infinite 
ascending sequence, so $a_n$ has a weakly decreasing subsequence $a_{n_k}$. But then 
subsequence $b_{n_k}$ must be strictly increasing, what gives 
us a contradiction. \Lightning $_\square$ 
\begin{lemma}\label{well_order}
In $\sbD$ there are no infinite ascending sequences.
\end{lemma}
\noindent\textbf{Proof.} \\
Let us denote by $A_n$ the set of all possible \Eoc s realised by orbifolds of type 
$*b_1,\dots,b_n$. Then $A_0 = \{1\}$ and $A_{n+1}=A_n+\{-\frac{n-1}{2n}\ |\ n\geq 2\}$. 
From that, from the lemma \ref{two_sets_lemma}, for all $n$, we have that $A_n$ do not have 
infinite ascending sequence. $\sbD = \bigcup\limits^\infty_{n=0}A_n$. Let us also observe, that 
for all $n$, we have $A_n \subseteqq [1-\frac{n}{4},1-\frac{n}{2}]$. From that we have $\sbD$ 
do not have infinite ascending sequences. $_\square$
\begin{theorem}\label{biggest \apots}
The biggest \apots\ $\sbD$ of order $n$ is $1-\frac{n}{2}$.
\end{theorem}
\noindent\textbf{Proof.}\\
Inductive \\
$\bullet$ $n=0$: $1\in \sbD$ and $1$ is the biggest element of $\sbD$. \\
$\bullet$ an inductive step: From the inductive assumption we know that $1-\frac{n}{2}$ is 
the biggest \apots  $\sbD$ of order $n$. From the lemma \ref{third_order_lemma} we have then 
that $1-\frac{n+1}{2}$ is a \apots  $\sbD$ of order $n+1$. Let us assume (for a contradiction), 
that there exist a bigger accumulation point of order $n+1$ equal to $y > 1-\frac{n+1}{2}$. 
But then, from lemma \ref{third_order_lemma}, point $y+\frac{1}{2}$ would be an acccumulation point 
of order $n$, what gives a contradiction, because $y+\frac{1}{2}>1-\frac{n}{2}$. $_\square$ 
\subsection{Order structure of the set of all possible Euler orbicharacteristics $\sigma$}
\begin{theorem}
The order type of the set of possible Euler orbicharacteristics of two dimensional orbifolds 
$\sigma$ is $\omega^\omega$. 
\end{theorem}
\noindent\textbf{Proof.} \\
From the lemma \ref{well_order} we know, that $\sbD$ is well ordered. From this and 
from the theorem \ref{biggest \apots} we know, that for the point $1-\frac{n}{2}$ there exist 
a neighborhood $U=(1-\frac{n}{2}-\varepsilon,1-\frac{n}{2}+\varepsilon)$ such that $U \cap 
\sbD$ is homeomorphic to $\omega^n$. From this, and again from theorem \ref{biggest \apots} 
we have that $\sbD \cap [1,1-\frac{n}{2})$ is homeomorphic with $\omega^n$. 
From this $\sbD$ is homeomorphic with $\omega^\omega$. From this $\sIS$ is homeomorphic 
with $\omega^\omega$. \\
$\sIS = 2\sbD$, so for all $n\in -\mathbb{N}$ set $\sIS \cap [2,n)$ has a lower order type then 
$\sbD\cap [2,n)$. From this, we have that $\sIS \cup \sbD \cong \omega^\omega$. $_\square$ \\[4pt]
From the above discussion we can conclude following:
% corollary (stated in for deifferent ways 
%as one are sometimes more useful that another): 
%\begin{corollary}\label{predescors}
%Let $x \in \sigma$. Then there exists $n \in \mathbb{N}$ such that $x + \frac{n}{2} \in \sigma$ 
%but $x+\frac{n+1}{2} \not\in \sigma$. For such $n$ we have that $x$ is an \apots\ $\sigma$ of 
%order $n$.  
%\end{corollary}
%\begin{corollary}
%\end{corollary}
%%Above corollary can be refolmulated in a way that sometimes is more useful:
%\begin{corollary}\label{predescors_variant_II}
%Let $x \in \sigma$ be an \apots\ $\sigma$ of order $n$. Then there is $y \in \sigma$ that is 
%an isolated point of $\sigma$ such that $x = y - \frac{n}{2}$.   
%\end{corollary}

\begin{corollary}\label{predescors}
Let $x \in \sigma$. Then:
\begin{itemize}
\item there exists $n_1 \in \mathbb{N}$ such that $x + \frac{n_1}{2} \in \sigma$ 
but $x+\frac{n_1+1}{2} \not\in \sigma$. \\ In other words, there exist $y \in \sigma$ and 
$n_1 \in \mathbb{N}$ such that 
$y + \frac{1}{2} \not\in \sigma$ and such that $x = y - \frac{n_1}{2}$;
\item there exists $n_2 \in \mathbb{N}$ such that $x$ is an \apots\ $\sigma$ of 
order $n_2$
%\item $x$ is an \apots\ $\sigma$ of order $n_2$, for some $n_2 \in \mathbb{N}$;  
\end{itemize}
and $n_1 = n_2$.
\end{corollary}







\section{Which points are in the $\sigma$?}
Here we will try to understand better the conditions that let us determine wether the point 
lie in $\sigma$ or not.
% We will also state some observations about reasoning which points 
%belong to $\sigma$ based on the knowlegde of other points belonging there. \\



\section{More about how this $\omega^\omega$ lies in $\mathbb{R}$}

\begin{theorem}
The first (biggest) negative \apots\ of all possible \Eoc of two dimensional orbifolds is 
$-\frac{1}{12}$. It is the accumulation point of order $1$. 
\end{theorem}
\noindent\textbf{Proof.} \\
We will show, that $-\frac{1}{12}$ is the biggest negative accumulation point of the set $\sbD$. 
From this we will obtain the thesis, as the set of all possible Euler orbicharacteristics 
of two dimensional orbifolds is equal to $\sIS \cup \sbD$ and $\sIS = 2\sbD$, so 
the biggest negative point of the set $\sIS$ is smaller than the biggest negative accumulation 
point of the set $\sbD$. \\
$\bullet$ $-\frac{1}{12}=\chi^{orb}((2,3))-\frac{1}{2}$, from this we have that $-\frac{1}{12}$ 
an acccumulation point of the set $\sbD$ of order at least $1$. \\
$\bullet$ Let us assume (for the contradiction), that there exist bigger, negative 
accumulation point of the set $\sbD$ of order at least $1$. Let us denote it by $x$. \\
However, then, from the lemma \ref{third_order_lemma} point $x+\frac{1}{2}$ is the accumulation 
point of the set $\sbD$. What is more, since $x\in (0, -\frac{1}{12})$, then $x+\frac{1}{2} 
\in (\frac{1}{2}, \frac{5}{12}$. From the lemma \ref{accumulations_points_of_the_set} we 
have that $x$ is in $\sbD$. But orbifolds of the type $*b_1$ can have \Eoc only greater or 
equal $\frac{1}{2}$. Orbifolds of the type $*b_1b_2$ can only have \Eoc $\frac{1}{2}$, 
$\frac{5}{12}$ and some smaller. Orbifolds of the type $*b_1b_2b_3\dots$ can have \Eoc only 
lower than $\frac{1}{4}$. This analysis of the cases leads us to the conclusion, that 
$(\frac{1}{2},\frac{5}{12})\cap \sbD=\emptyset$ and to the contradiction. \\
$\bullet$ Above analysis of the cases leads us also to the conclusion, that $\frac{5}{12}$ 
is 
an isolated point of the set $\sbD$, from this $-\frac{1}{12}$ is an acccumulation point 
of order $1$ of the set $\sbD$. $_\square$ \\ 
 








