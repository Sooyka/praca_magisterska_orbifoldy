% mainfile: ../praca_magisterska_orbifoldy.tex
%\synctex=1
\chapter{Order structure}

In this chapter we will discuss order type of the set of all possible Euler orbicharacteristics 
of two dimentional orbifolds. \\
For now, until Chapter \ref{Counting_occurrences} Counting occurrences, we will not pay attension 
to how many orbifolds have the same Euler orbicharacteristic. \\ 
Because of that and since Euler orbicharacteristic does not depend on the cyclic order 
of points on the conponents of the boundry we introduce an extension of a notation from 
\cite{Conway2008}. 

%We will write $*\{a,b,c,d,\dots\}$ to denote not a particular orbifold, but a type 
%of orbifold that have a cones on a component of the boundry of orders $a,b,c,d,\dots$, but 
%in any order. \\ 

We will write $*\{a,b,c,d,\dots\}$ to denote a type of a boundry (of an orbifold) that have 
kaleidoscopic points of periods $a,b,c,d,\dots$, but in any order. \\


From what we wrote above (that Euler orbicharacteristic does not depend on the cyclic order 
of points on the conponents of the boundry), we can see that Euler 
orbicharacteristic is well defined 
%on such types of orbifolds. \\ 
when we specify only such a type of the components of the boundry of an orbifold and not 
a particular cyclic order.  \\

\section{}


