% mainfile: ../praca_magisterska_orbifoldy.tex
%\synctex=1
\chapter{Order type and topology}\label{order structure}
%
%Order type with zanurzenie w R
%
%1537/137
%
In this chapter we will discuss that both the order type and the topology 
of the set of all possible Euler orbicharacteristics 
of two-dimensional orbifolds are that of $\omega^\omega$. We will call this set $\spe$.
%We will see (in the observation \ref{boils_down}) that the problem of determining this boils
% down to the 
% analysis of all 
%the possible 
%values of the expressions:
%\begin{equation}
%2 - \sum_{i=1}^n \frac{I_i-1}{I_i}
%\end{equation}
%and 
%\begin{equation}
%1 - \sum_{j=1}^m \frac{b_j-1}{2b_j},
%\end{equation}
%where $I_i, b_j$ varies over $\mathbb{N}_{>0} \cup \{\infty\}$. \\
%As
%\begin{equation}
%2 - \sum_{i=1}^n \frac{I_i-1}{I_i} = 2 - n + \sum_{i=1}^n \frac{1}{I_i}
%\end{equation}
%and 
%\begin{equation}
%1 - \sum_{j=1}^m \frac{b_j-1}{2b_j} = 1 - m + \sum_{j=1}^m \frac{1}{2b_j},
%\end{equation}
%some questions about the spectrum are equivalent to some regarding Egyptian fractions. 
%More on this connection is discussed in \ref{Egyptian_fractions}.
\\[4pt]
\textbf{Disclaimer}\\
For now, until Chapter \ref{counting occurrences} named "Counting occurrences", 
we will not pay attention 
to how many orbifolds have the same Euler orbicharacteristic. 
%Let us note, that 
%Euler orbicharacteristic does not depend on the cyclic order of points on 
%the components of the boundary. 
%Because of that and since Euler orbicharacteristic does not depend on the cyclic order 
%of points on the components of the boundary we introduce an extension of a notation from 
%\cite{Conway2008}. 

%%We will write $\ast \{a,b,c,d,\cdots\}$ to denote not a particular orbifold, but a type 
%%of orbifold that have a dihedral points on a component of the boundry of orders 
%$a,b,c,d,\cdots$, but 
%%in any order. \\ 

%We will write $\ast \{a,b,c,d,\cdots\}$ to denote a type of a boundary (of an orbifold) that have 
%kaleidoscopic points of periods $a,b,c,d,\cdots$, but in any order. \\


%From what we wrote above (that Euler orbicharacteristic does not depend on the cyclic order 
%of points on the components of the boundary), we can see that Euler 
%orbicharacteristic is well defined 
%%on such types of orbifolds. \\ 
%when we specify only such a type of the components of the boundary of an orbifold and not 
%a particular cyclic order.  \\

\section{Reductions of cases}
Now we want to make some reductions to limit number of cases that we will be dealing with. \\ 
We aim to find a minimal set $B$ of base manifolds that such that any for any $x \in \spe$ there 
is an orbifold $O$ with a base manifold from $B$ such that $\cho{O} = x$.
%We aim to limit the number of base manifolds as much as possible while keeping entire spectrum. 
It will turn out it can be done such that we are left with $B = \{S^2, D^2\}$ and there is 
no futher reduction possible. \\ 
%For this chapter we will consider orbifolds according to a definition from 
%(\ref{Disk_and_sphere_with_defects}). \\
Given an orbifold $O_1$, we want to perform some surgeries on it such that the resulting orbifold 
$O_2$ will have the same Euler orbicharactristic, but the base manifold of $O_2$ would have as 
big Euler characteristic as possible. \\ 
The Euler orbicharacteristic of base manifold depends only on the number of handles, cross caps 
and boundry components. And, as stated in \ref{} it is: \\ 
For ever such a manifold feature we want to find an orbifold features with the same 
Euler orbicharacteristic delta.  \\
%To do this we want to eliminate handles 
%Let us observe, that:
One of the ways to do that is by observing that:
\begin{align}
\hspace{3cm}\Delta(\circ) =& \hspace{-1cm} &-2& \hspace{-1cm} &= \Delta(\ast (^\ast 2)^4) \\
\hspace{3cm}\Delta(\ast) =& \hspace{-1cm} &-1& \hspace{-1cm} &= \Delta((^\ast 2)^4) \\
\hspace{3cm}\Delta(\times) =& \hspace{-1cm} &-1& \hspace{-1cm} &= \Delta((^\ast 2)^4)
\end{align}
So we see that from any orbifold we can eradicate handles ....       

\begin{align}
\hspace{3cm}\Delta(n) =& \hspace{-1cm} &\frac{n-1}{n}& \hspace{-1cm} 
&= \Delta((^\ast n)^2)\label{third_reduction}
\end{align}

%\begin{align}
%\Delta(n) =& &\frac{n-1}{n}&  
%&= \Delta((^\ast n)^2)\label{third_reduction}
%\end{align}

From this we can conclude that every Euler orbicharacteristic can be obtained 
by an orbifold with base manifold $S^2$ or $D^2$. 
Examples of rational numbers from $\speS \setminus \speD$ and $\speD \setminus \speS$ are:
We will provide examples 
Futher examination of connections between $\speD$ and $\speS$ is performed in 
\ref{counting_occurences}

%From this we can conclude, that every Euler orbicharacteristic can be obtained 
%by an orbifold of signature of a type ($n$ and $m$ are arbitrary):

%\begin{align*}
%I_1I_2\cdots I_n & \textrm{\ or} \\
%\ast b_1b_2\cdots b_m &.
%\end{align*}

%Let us denote the set of all possible Euler orbicharacteristics of orbifolds of the form 
%$I_1I_2\cdots I_n$ by $\speS$ and the set 
%of all possible Euler orbicharacteristics of orbifolds of the form $\ast b_1b_2\cdots b_m$ 
%as $\speD$. 
%So we have that $\spe = \speD \cup \speS$. \\
%% Let us denote the set of all possible \Eoc s of two-dimentional orbifolds as $\spe$. \\
%Let us also observe that the order type and topology of $\speS$ and $\speD$ are 
%the same since 

%\begin{equation}\label{times_two_fact}
%2\speD=\speS
%\end{equation}
%and multiplying by $2$ is the order preserving homeomorphism of $\mathbb{R}$. \\[16pt]

%Now we can make aforementioned observation:
In the terms of set relations:
\begin{observation}
If two-dimentional manifold $M$ has no boundry, then
%\begin{equation} 
%\spe{M} = \chi(M) - (\speS - 2) 
%\end{equation}
%and
\begin{equation} 
\spebr{M} \subseteq \speS 
\end{equation} 

If, in addition, $M \neq S^2$, then 
\begin{equation} 
\spebr{M} \subseteq \speD. 
\end{equation}

\end{observation}
%\textbf{Proof} \\
%Let $M$ be a two-dimentional manifold with no boundry, from \ref{\Eoc_as_a_sum} we have, that: 
%\begin{equation}
%\spe{M} = \{\chi(M) - \sum_{i=1}^n 
%\frac{I_i-1}{I_i}\ |\ n \in \mathbb{N} \land \forall_i I_i \in \mathbb{N} \cup \infty\}.
%\end{equation}
%And from 
%$\chi(M) - (\speS - 2)$
\begin{observation}
If two-dimentional manifold  $M$ has a boundry, then 
%\begin{equation}
%\spe{M} = \chi(M) + (\speD - 1)
%\end{equation}
%and 
\begin{equation}
\spebr{M} \subseteq \speD
\end{equation}
\end{observation}
%\textbf{Proof} \\
%\begin{equation}
%\{\chi(M) - \sum_{j=1}^m 
%\frac{b_j-1}{2b_j}\ |\ m \in \mathbb{N} \land \forall_j b_j \in \mathbb{N} \cup \infty\} 
%\end{equation}
%\begin{observation}
%If two-dimentional manifold  $M$ has no boundry, then $\spe{M} \subseteq \speS$. 
%If, in addition, $M \neq S^2$, then 
%$\spe{M} \subseteq \speD$.
%\end{observation}
%\begin{observation}
%If two-dimentional manifold  $M$ has a boundry, then $\spe{M} \subseteq \speD$.
%\end{observation}

In the terms of arithmetical expressions:
\begin{observation}\label{boils_down}
From above reductions we can conclued that our problem boiles down to the analysis of all
 the possible 
values of the expressions:
\begin{equation}\label{S2_sum}
2 - \sum_{i=1}^n \frac{I_i-1}{I_i}
\end{equation}
and 
\begin{equation}
1 - \sum_{j=1}^m \frac{b_j-1}{2b_j},
\end{equation}
\end{observation}
with $I_i$ and $b_j$ ranging over $\mathbb{N}_{>0}\cup \{\infty\}$. \\
As stated in \ref{cusp_reduction} we can perform futher reductions to have an orbifold with 
particular orbicharacteristic without cusps (if needed) and then (after these reductions) 
we can analyse only expressions with $I_i$ and $b_j$ ranging over $\mathbb{N}_{>0}$ and 
they will still give us full spectrum. \\ 
However, as stated later, it will be more convenient to us to include orbifolds with cusps 
so we are stating above remark only for readers information. \\ 
 The fact that it agrees with the definition of the \Eoc\ on the geometrical terms was 
addressed in \ref{extended_Euler_orbicharacteristic}. 

%We also have shown that all possible \Eoc s are achieved without using cusps. As such, 
%we will use 
%cusps, remembering, that we can always get rid of them, if needed. So above $I_i$ and $b_j$ 
%are ranging over $\mathbb{N}_{>0}\cup \{\infty\}$, where expressions for infinity are defined as 
%a limits. The fact that it agrees with the definition of the \Eoc\ on the geometrical terms was 
%addressed in \ref{extended_Euler_orbicharacteristic}. \\
%Here is the notation that will be used: \\
%$\spe$ - spectrum of all possible \Eoc\ of two-dimentional orbifolds \\
%$\spe{M}$ - spectrum of all possible \Eoc\ of $M$ orbifolds \\

%% We would like to make some more observations concerning subsets of $\spe$. 
%% For any two-dimentional manifold $M$, observe, that:

%Now we state what we can already conclude. 


%\begin{theorem}\label{all_spectra_are_isomorphic}
%For any two two-dimentional manifolds $M$, $N$ spectras $\spe{M}$ and $\speM(N)$ have the same 
%order type and topology. 
%\end{theorem}
%\textbf{Proof} \\



\section{Order type and topology of $\speD$}
To determine order type and topology of $\spe$ we will first study how $\speD$ looks like. 
Then, remembering that 
$\spe = \speS \cup \speD$ and $\speS = 2\speD$ we will make an argument for $\spe$. \\
In this section we will also describe precisely where accumulation points of $\speD$ lie and of 
 which order 
(see below \ref{accumulation_points_definitions}) they are. Analysis of locations of those 
accumulation points, as interesting as it is alone will also be necessery for providing 
our argument about order type and topology of $\speD$. \\
\subsection{Definition and properties of order of accumulation points}
\label{accumulation_points_definitions} 
%We start with one technical definition of "transitive order" that will be almost what we want
%and then, there will be the the definition of "order", which is the definition that we need.
%\begin{definition}
%(Inductive). 
%%We say, that the point $a$  from the topological space $X$ is an acccumulation 
%%point of the transitive order 0, when
%We say that the point is an acccumulation point of a transitive order $0$, when it is 
%an isolated point. 
%We say that the point is an acccumulation point of a transitive order $n + 1$, when it is 
%an acccumulation point (in the usual sense) of the accumulation points of the transitive 
%order $n$. 
%\end{definition}  
%The only issue of the above definition is that the point of the transitive order $n$ 
%is also a point 
%of the transitive order $k$, for all $0< k \leq n$. We want a definition of order such that 
%for any point, there is at most one integer that is its order. So we define:
%\begin{definition}
%We say that the point is an acccumulation point of order $n$ iff it is an acccumulation point 
%of the transitive order $n$ and it is not an acccumulation point of the transitive order $n+1$. 
%If the point is an acccumulation point of the transitive order for an arbitrary large 
%$n$ we say that 
%the  point is an acccumulation point of order $\omega$.
%\end{definition}
We start with definition of being "at least of order $n$" that will be almost what we want
and then, there will be the definition of being "order", which is the definition that we need. \\
For a given set we define as follows:
\begin{definition}
(Inductive). 
%We say, that the point $a$  from the topological space $X$ is an acccumulation 
%point of the transitive order 0, when
We say that the point $x$ is an acccumulation point of a set $X$ 
of order at least $0$, when it belongs to the set $X$. 
We say that the point $x$ is an acccumulation point of a set 
of order at least $n + 1$, when it is 
an acccumulation point (in the usual sense) of the accumulation points each of order at least 
$n$ i.e. in every neighbourhood of $x$ there is at least one accumulation point of a set $X$ 
of order at least $n$, distincs from $x$. 
\end{definition}  
%The only issue of the above definition is that the point of the transitive order $n$ 
%is also a point 
%of the transitive order $k$, for all $0< k \leq n$. We want a definition of order such that 
%for any point, there is at most one integer that is its order. So we define:
\begin{definition}
We say that the point is an acccumulation point of order $n$ iff it is an acccumulation point 
of order at least $n$ and it is not an acccumulation point of order at least $n+1$. 
If the point is an acccumulation point of order at least $n$ for an arbitrary large 
$n$ we say that 
the point is an acccumulation point of order $\omega$.
\end{definition}
When we will say that a point is an accumulation point of some set without specifying an order 
then we will mean being an accumulation point in the ussual sense; from the point of view 
of above definitions, that is, an accumulation point of order at least one.
%\todo{dopisać notację do punktów skupienia różnego stopnia}
\begin{lemma}

\end{lemma}
\subsection{Analysis of locations of accumulation points of $\speD$ with respect to their order}
%\subsubsection{Some preliminary observations}
We want to determine where exactly are accumulation points of the set $\speD$ with 
respect to their orders. \\
For this we will use  
a handful of observations and lemmas. 
\begin{observation}\label{accumulation_points_are_in_the_spectrum}
Let us observe, that $\lim\limits_{n \to \infty} \Delta(^\ast n) = -\frac{1}{2}$. From that, 
we see, 
that for every point $x \in \speD$, the point $x - \frac{1}{2}$ is an acccumulation point. 
Let us observe, that also, for every point $x \in \speD$, we have that $x - \frac{1}{2} 
\in \speD$, 
because $\Delta(^\ast \infty) = -\frac{1}{2}$. 
\end{observation}
%Now we will show that the order type of $\speD$ is $\omega^\omega$ and where exactly are 
%its accumulation points of which orders. For this we will use  
%a handful of lemmas. 

%\subsubsection{Finiteness lemma}
\begin{lemma}\label{finiteness_lemma}
For all $n \in \mathbb{N}_{\geq 2}$ and $x \in (-\infty, 1]$ there are only finitely 
many Euler orbicharacteristics
in the interval $[x,1] \cap \speD$ of orbifolds that have points of order equal 
at most $n$. 
\end{lemma}
\textbf{Proof.} \\
Let $x \in (-\infty, 1]$. There can be at most $\lfloor 4(1-x) \rfloor$ orbipoints on the 
$D^2$ orbifold 
with an \Eoc\ $y \in [x,1]$ since each orbipoint decreases an \Eoc\ by at least $\frac{1}{4}$ 
and the Euler characteristic of $D^2$ is $1$. \\
There are only $(n-1)^{\lfloor 4(1-x) \rfloor}$ possible sets of $\lfloor 4(1-x) \rfloor$ 
orbipoints' orders that are less or equal than $n$. Hence, there are only at most 
$(n-1)^{\lfloor 4(1-x) \rfloor}$ possible \Eoc s.


\begin{lemma}\label{first_order_lemma}
If $x$ is an acccumulation point of the set $\speD$ of order $n$, then $x-\frac{1}{2}$ is a
 accumulation point of the set $\speD$ of order at least $n+1$. 
\end{lemma}
\textbf{Proof.} \\
Inductive. \\
$\bullet$ $n = 0$: If $x$ is an isolated point of the set $\speD$, then $x \in \speD$. 
From that, we 
have, that points $x - \frac{k-1}{2k}$ are in $\speD$ for all $k \geq 1$, from that, that 
$x-\frac{1}{2}$ is a 
accumulation point of $\speD$. \\
$\bullet$ inductive step: Let $x$ be an acccumulation point of the set $\speD$ of an order 
$n > 0$. 
Let $a_k$ be a sequence of accumulation points of order $n-1$ convergent to $x$. From the 
inductive assumption, we have, that $a_k - \frac{1}{2}$ is a sequence of accumulation points 
of order at least $n$. From the basic sequence arithmetic it is convergent to $x-\frac{1}{2}$. 
From that, we have that $x-\frac{1}{2}$ is an acccumulation point of the set $\speD$ of order 
at least $n+1$. $_\square$
\begin{lemma}\label{second_order_lemma}
If $x$ is an acccumulation point of the set $\speD$ of order $n$, then $x+\frac{1}{2}$ is 
an acccumulation point of the set $\speD$ of order at least $n-1$.  
\end{lemma}
\noindent\textbf{Proof.} \\  
Inductive \\
$\bullet$ $n = 1$: We assume, that $x$ is an acccumulation point of isolated points of the set 
$\speD$. 
%Let us observe
From \ref{finiteness_lemma} we know, that for all $m$ there are only finitely many 
Euler orbicharacteristics 
in the interval $[x,1]$ of orbifolds that have dihedral points of order equal at most $m$. \\ 
%\todo{może to dać jako osobny lemat}
From that, for arbitrary small neighborhood $U \ni x$ and arbitrary large $m$ there exist 
an orbifold 
that has a dihedral point of period grater than $m$, whose Euler orbicharacteristic lies in $U$. 
Let us take a sequence of such \Eoc s $a_k$ convergent to $x$, such that we can choose 
a sequence divergent to infinity of periods of dihedral points $b_k$ of orbifolds of \Eoc s 
equal $a_k$. 
\smalltodoII{picture} 
Let us observe, that for all $k$, the number $a_k+\frac{b_k-1}{2b_k}$ is in $\speD$. 
It is so, because $a_k$ is an \Eoc\ of an orbifold that have a dihedral point of period $b_k$, so 
identical orbifold, only without this dihedral point, has an \Eoc\ equal to $a_k + 
\frac{b_k-1}{2b_k}$. 
The sequence $a_k + \frac{b_k-1}{2b_k}$ converge to $x+\frac{1}{2}$. From that we have, that 
$x + \frac{1}{2}$ is an acccumulation point of the set $\speD$ of order at least $0$. \\
$\bullet$ inductive step: Let $x$ be an acccumulation point of the set $\speD$ of order $n > 1$. 
Let $a_k$ be a sequence of accumulation points of the set $\speD$ of order $n-1$ 
convergent to $x$. 
From the inductive assumption the sequence $a_k + \frac{1}{2}$ is a sequence of an acccumulation
 points of the set $\speD$ of order $n-2$ convergent to $x + \frac{1}{2}$. From that 
 $x + \frac{1}{2}$ is an acccumulation point of the set $\speD$ of order at least 
 $n-1$. $_\square$ 
\begin{lemma}\label{third_order_lemma}
If $x$ is an acccumulation point of the set $\speD$ of order $n+1$, then \\
$x - \frac{1}{2}$ is an acccumulation point of the set $\speD$ of order $n+2$ and \\
$x + \frac{1}{2}$ is an acccumulation point of the set $\speD$ of order $n$. 
\end{lemma}
\noindent\textbf{Proof.} \\
Let $x$ be an acccumulation point of the set $\speD$ of order $n+1$. From the lemma 
 \ref{first_order_lemma} we know, that $x - \frac{1}{2}$ is an acccumulation point of the set 
 $\speD$ of order at least $n+2$. Now let us assume (for a contradiction), that $x - \frac{1}{2}$ 
 is an \apots $\speD$ of order $k>n+2$. But then from the lemma \ref{second_order_lemma} 
 we have that $x$ is an acccumulation point of the set $\speD$ of order at least $n+2$ and that 
 is a contradiction. \\
Analogously, from the lemma \ref{second_order_lemma} we know, that $x + \frac{1}{2}$ is a 
accumulation point of the set $\speD$ of order at least $n$. Let us assume (for a contradiction), 
that $x+ \frac{1}{2}$ is an acccumulation point of the set $\speD$ of order $k>n$. But then 
from the lemma \ref{first_order_lemma} we have that $x$ is an acccumulation point of 
the set $\speD$ 
of order at least $n+2$ and that is a contradiction. $_\square$ 
\begin{lemma}\label{accumulation_points_of_the_set}
For all $n \in \mathbb{N}$ all accumulation points of the set $\speD$ of order $n$ are in $\speD$.
\end{lemma}
\noindent\textbf{Proof.} \\
Inductive \\
$\bullet$ $n=0$: Clear, as they are isolated points of $\speD$. \\
$\bullet$ inductive step: Let $x$ be a \apots  $\speD$ of order $n>0$. From the lemma 
\ref{third_order_lemma} point $x+\frac{1}{2}$ is an acccumulation point of the set $\speD$  
of order $n-1$. From the inductive assumption $x+\frac{1}{2} \in \speD$. Then, 
from \ref{accumulation_points_are_in_the_spectrum}, we have that $x \in \speD$. 
$_\square$ 

\begin{theorem}\label{greatest \apots}
The greatest \apots\ $\speD$ of order $n$ is $1-\frac{n}{2}$.
\end{theorem}
\noindent\textbf{Proof.}\\
Inductive \\
$\bullet$ $n=0$: We know, that $1\in \speD$ and $1$ is the greatest element of $\speD$. \\
$\bullet$ an inductive step: From the inductive assumption we know that $1-\frac{n}{2}$ is 
the greatest \apots\ $\speD$ of order $n$. From the lemma \ref{third_order_lemma} we have then 
that $1-\frac{n+1}{2}$ is a \apots\ $\speD$ of order $n+1$. Let us assume (for a contradiction), 
that there exist a bigger accumulation point of order $n+1$ equal to $y > 1-\frac{n+1}{2}$. 
But then, from lemma \ref{third_order_lemma}, point $y+\frac{1}{2}$ would be an acccumulation 
point 
of order $n$, what gives a contradiction, because $y+\frac{1}{2}>1-\frac{n}{2}$. $_\square$ 

\subsection{Proof that $\speD$ is well ordered}

\begin{definition} 
Let $B_0 = \{1\}$.
For an $n \in \mathbb{N}_{>0}$, let $B_n$ be the set of all possible \Eoc\ realised 
by orbifolds of type 
$*b_1, \cdots, b_n$. For a given $n$ these are 
$D^2$ orbifolds with precisely $n$ non trivial orbipoits on their boundry.
\end{definition}


\begin{observation}\label{recursive_relation}
There is a recursive relation, that $B_{n+1}=B_n+\{-\frac{n-1}{2n}\ |\ n\geq 2\}$
\end{observation}
\textbf{Proof.} \\
It is so, because every orbifold with $n+1$ orbipoints can be obtained by adding one point 
to an orbifold with $n$ orbipoints and the set 
$\{-\frac{n-1}{2n}\ |\ n\geq 2\} = \{\Delta(^\ast b)\ |\ b \geq 2\}$. $_\square$
%Let take $x \in B_{n+1}$. Then there exist some orbifold $O_1$ with an \Eoc\ equal to $x$ with 
%signature $\ast b_1, \cdots, b_{n+1}$. 
%Taking an orbifold $O_2$ with signature $\ast b_1, \cdots, b_n$, its \Eoc\ $\cho{O_2}$ 
%is in $B_n$ as $O_2$ has $n$ orbipoints. The difference between $\cho{O_1}$ and $\cho{O_2}$ 
%is $-\frac{b_{n+1} - 1}{2b_{n+1}} \in \{-\frac{n-1}{2n}\ |\ n\geq 2\}$. \\
%Let take $x \in B_n+\{-\frac{n-1}{2n}\ |\ n\geq 2\}$. Then $ x = x_n + x'$ for some 
%$x_n \in B_n$ and $x' \in \{-\frac{n-1}{2n}\ |\ n\geq 2\}$. Let 


\begin{observation}\label{form_of_a_spectrum}
Observe that, as any orbifold has only finitely many orbipoints, we have that $\speD \subseteq 
\bigcup\limits^\infty_{n=0}B_n $. We defined $\speD$ as a set of all possible \Eoc\ of disk 
orbifolds, so $\speD \supseteq 
\bigcup\limits^\infty_{n=0}B_n $. From this we have that $\speD = \bigcup\limits^\infty_{n=0}B_n$.
\end{observation}

\begin{lemma}\label{fixed_number_of_orbipoints}
For any given $n \in \mathbb{N}$ the set $B_n$ is a subset of the interval 
$[1-\frac{n}{2}, 1-\frac{n}{4}]$.
\end{lemma}
\textbf{Proof.} \\


\begin{lemma}\label{two_sets_lemma}
If $A, B \subseteqq \mathbb{R}$ have no infinite strictly ascending sequences, then set 
$A + B \coloneqq \{a+b\ |\ a \in A, b \in B\}$ also have no infinite strictly ascending sequences. 
\end{lemma}
\noindent\textbf{Proof.} \\
Let $A$, $B$ have no infinite strictly ascending sequences. 
Let $c_n \in A + B$ are elements of some sequence. With a sequence $c_n$ there are 
two associated sequences $a_n$, $b_n$, such that, for all $n$, we have $a_n \in A$, 
$b_n \in B$ and 
$a_n + b_n = c_n$. Assume (for contradiction), that $c_n$ is an infinite strictly 
ascending sequence. 
Then $\forall_n\ a_{n+1}>a_n\ \lor\ b_{n+1} > b_n$. From the assumption $a_n$ has no infinite 
ascending sequence, so $a_n$ has a weakly decreasing subsequence $a_{n_k}$. But then 
subsequence $b_{n_k}$ must be strictly increasing, as $c_{n_k}$ is strictly increasing, what gives 
us a contradiction. 
%\Lightning 
$_\square$ 

\begin{lemma}\label{sum_lemma}
If $A, B \subseteqq \mathbb{R}$ have no infinite strictly ascending sequences, then set 
$A \cup B$ also have no infinite strictly ascending sequences.
\end{lemma}
\textbf{Proof.} \\
Let $A$, $B$ have no infinite strictly ascending sequences. 
For the sake of contradiction, lets assume, that $A \cup B$ has an infinite strictly 
ascending sequence $c_n$. Let $c_{n_k}$, $c_{n_l}$ be subsequences of $c_n$ consisting 
of elements from, respectively $A$ and $B$. At least one of them must be infinite and 
strictly increasing, which gives us a contradiction. $_\square$ 

\begin{observation}\label{ascending_in_B_n}
From \ref{recursive_relation} and \ref{two_sets_lemma}, we have that $B_n$ do not have 
infinite ascending sequence for all $n$. \\
Further, from \ref{sum_lemma} we conclude, 
that $\bigcup\limits_{n=0}^N B_n$ do not have infinite ascending sequence for all $N$.
\end{observation}

\begin{theorem}\label{well_order}
In $\speD$ there are no infinite strictly ascending sequences, hence, it is well ordered.
\end{theorem}
\noindent\textbf{Proof.} \\
%Being well ordered is a consequence of not having infinite strictly ascending sequences. 
For the sake of contradiction lets assume that $c_n$ is an infinite strictly ascending sequence in 
$\speD$. As $c_n$ is bounded from below by $c_0$ and whole $\speD$ is bounded from above 
by $1$, all elements of $c_n$ are in the interval $[c_0, 1]$. 
From \ref{form_of_a_spectrum} we have, that $\speD = \bigcup\limits^\infty_{n=0}B_n$. \\
Lemma 
\ref{fixed_number_of_orbipoints} says that for all $n$ we
have $B_n \subset [1-\frac{n}{2}, 1 - \frac{n}{4}]$. From this, we know, that for any 
$n$ such that $1 - \frac{n}{4} < c_0$ 
we have, that $B_n \cap [c_0,1] = \varnothing $. Let $n_0$ be such that 
$1 - \frac{n_0}{4} = c_0$ (so $n_0 = 4(1-c_0)$). 
Then for all $n > n_0$ we have $1 - \frac{n}{4} > c_0$, meaning, that 
for all $n > n_0$ we have
$B_n \cap [c_0,1] = \varnothing $, so all elements of $c_n$ are in 
$\bigcup\limits_{n=0}^{n_0} B_n$.
But this contradicts \ref{ascending_in_B_n}.  $_\square$
%\todo{rozwinąć ostatni argument}









\subsection{Proof that order structure and topology of $\speD$ are those of $\omega^\omega$}

\begin{theorem}\label{speD_theorem}
Order type and topology of $\speD$ are those of $\omega^\omega$. 
\end{theorem}

\textbf{Proof} \\
We will prove it by inductively constructing an order preserving homeomorphism $f$ between 
$\omega^\omega$ and $\speD$. \\
For simplycity, we will take reverse (decreasing) order on $\speD$ i.e. $1$ will be the smallest 
 element 
(so for example $0 > 1$ in this order)
% and in general $x < y $ in the ussual order iff $x > y$ in 
the reverse order).  \\
%\todo{dopisać dowód}
%Since $\speD$ is well ordered (as we know from \ref{well_order}) it has an order preserving 
%bijection with some ordinal number. 
We will inductively construct the family of order preserving homeomorphisms $f_\mu$, 
indexed by ordinal numbers less 
%or equal
\smalltodoII{fix this}
than 
$\omega^\omega$ each 
prefix of $\omega^\omega$ homeomorphic to $\nu + 1$ (so on all ordinals less 
or equal to $\nu$) 
and some prefix of $\speD$. We will construct them in such a way, that for any $\mu_1 < \mu_2 
< \omega^\omega$ function $f_{\mu_2}$ restricted to the ordinals less or equal to 
$\mu_1$ coincides with $f_{\mu_1}$.
Then we will take $f \coloneqq \bigcup\limits_{0 \leq \mu < 
\omega^\omega} f_\mu$ (so $f(\mu) \coloneqq f_\mu(\mu)$). \\
Our inductive assumption for a given $\mu$ will be that 
for all $\nu < \mu$ function $f_\nu$ will be an order preserving homeomorphism 
between prefix of $\omega^\omega$ homeomorphic to $\nu + 1$ (so on all ordinals less 
or equal to $\nu$) 
and some prefix of $\speD$ and that for every $\nu_1 < \nu_2 < \mu$ function $f_{\nu_2}$ 
restricted to the ordinals less or equal than $\nu_1$ coincides with $f_{\nu_1}$. \\ 
%Our inductive assumtion will be, that for all $\mu \leq \omega^\omega$ 
%function $f_\mu$ is a homeomorphism 
%between prefix of $\omega^\omega$ homeomorphic to $\mu$ and some \\
%$\bullet$ $\mu = 0$: Function $f_0$ is an empty function and as such it is an order preserving 
%homeomorphism. \\
$\bullet$ $\mu = 0$: We take $f_0$ as a function on $\{0\}$ taking value $1$. 
Both $0$ and $1$ are the smallest elements of, respectively, $\omega^\omega$ and $\speD$ 
so $f_0$ is defined between prefix of $\omega^\omega$ of all ordinals less or equal to $0$, and 
some prefix of $\speD$. 
Function $f_0$ also preserves order on one element set.
Function $f_0$ is an homeomorphism between one element sets, both with discreate topology. \\
$\bullet$ $\mu$ is a successor ordinal less than $\omega^\omega$: From an inductive assumption 
we have an order preserving homeomorphism $f_{\mu - 1}$ between all ordinals 
less or equal to $\mu - 1$ and some prefix of $\speD$. 
We define $f_\mu$ on all numbers less or equal to $\mu - 1$ to be equal $f_{\mu-1}$. \\
%Now we have two cases: \\ 
It remains to define $f_\mu(\mu)$. 
As $\speD$ is well ordered it is well defined to take successor of an element of $\speD$. 
We define $f_\mu(\mu)$ to be a succesor of $f_{\mu - 1}(\mu-1)$ in $\speD$. As such (and from 
inductive assumption) it is indeed defined as a function between prefix of $\omega^\omega$ of all 
ordinals less or equal to $\mu$, and 
some prefix of $\speD$.
\\
Now we want to prove, that $f_\mu$ preservse the order. From the inductive assumption 
it preserves the order up to $\mu - 1$. As $\mu$ is the successor of $\mu - 1$ and 
$f_\mu(\mu)$ is a successor of $f_\mu(\mu-1)$, we have that $f_\mu$ is indeed an order preserving 
function. \\
Now we want to prove that $f_\mu$ is a homeomorphism. As from inductive assumtion we know, 
that $f_{\mu - 1}$ was a homeomorphism it is sufficient to show that preimages of open 
sets containing $f_\mu(\mu)$ and images of open sets containing $\mu$ are open. \\
Since $f_\mu(\mu)$ is a successor and since $\speD$ is well ordered, we have, that $f_\mu(\mu)$ 
is an isolated point in $\speD$. \\
Simmilarly $\mu$ is an isolated point in $\mu + 1$ as an successor ordinal. \\
From this we have, that open sets containing $f_\mu(\mu)$ (resp. $\mu$) are of the form 
$U \cup \{f_\mu(\mu)\}$ (resp. $V \cup \{\mu\}$) for some $U$ -- open set in $\speD$. 
(resp. $V$ -- open set in $\mu + 1$).
%Holding the notation of $U$ and $V$ we have that $f_\mu[V] = $
\smalltodoII{może rozwinąć}
From this this is clear. \\
%Let us observe, that 
%Lets take \todo{finish}
$\bullet$ $\mu$ is a limit ordinal less than $\omega^\omega$:  
%\todo{net and so on}
From the inductive assumption, for each $\nu < \mu$ we have an order preserving homeomorphism 
$f_\nu$ on the ordinals less or equal to $\nu$ and those functions pairwise coincide 
on the intersections 
of their domains. For every ordinal $\nu < \mu$ we define 
$f_\mu(\nu) \coloneqq f_\nu(\nu)$. It remains to define $f_\mu(\mu)$. \\
We consider a net $\phi_\nu \coloneqq \{f_\nu(\nu)\}_{\nu<\mu} \subset \mathbb{R}$
%, indexed by all $\nu < \mu$
. From the inductive assumption we know that the domain of the net $\phi_\nu$, as 
well as it's image is well ordered and that the net $\phi_\nu$.  
is an order preserving homeomorphism.
% is  this is a well ordered net. \todo{napisać jakoś lepiej tę własność}
Now we will show that the net $\phi_\nu$ has a limit in $\speD$. \\
First we will show, that $\phi_\nu$ has a limit in $\mathbb{R}$. For this, we will show that 
$\phi_\nu$ is bounded. \\
Order type of the image of $\phi_\nu$ is equal to $\mu$ and it is a prefix of $\speD$. 

As we have \ref{accumulation_points_of_the_set} 
As $\mathbb{R}$ is Hausdorff, from \cite{Kelley1975} (chapter 2, 
theorem 3, page 67) we know, that .  
\\ 
Firstly we will determine the order type of $\speD$. 
From the lemma \ref{well_order} we know, that $\speD$ is well ordered, so it has order type 
of some ordinal number. From this and 
from the theorem \ref{greatest \apots} we know, that for the point $1-\frac{n}{2}$ there exist 
a neighborhood $U=(1-\frac{n}{2}-\varepsilon,1-\frac{n}{2}+\varepsilon)$ such that $U \cap 
\speD$ is homeomorphic to $\omega^n$. From this, and again from theorem \ref{greatest \apots} 
we have that $\speD \cap [1,1-\frac{n}{2})$ is homeomorphic with $\omega^n$. 
From this $\speD$ is homeomorphic with $\omega^\omega$.

From the above discussion we can also formulate following corollaries that will be useful later:
% corollary (stated in for deifferent ways 
%as one are sometimes more useful that another): 
%\begin{corollary}\label{predescors}
%Let $x \in \spe$. Then there exists $n \in \mathbb{N}$ such that $x + \frac{n}{2} \in \spe$ 
%but $x+\frac{n+1}{2} \not\in \spe$. For such $n$ we have that $x$ is an \apots\ $\spe$ of 
%order $n$.  
%\end{corollary}
%\begin{corollary}
%\end{corollary}
%%Above corollary can be refolmulated in a way that sometimes is more useful:
%\begin{corollary}\label{predescors_variant_II}
%Let $x \in \spe$ be an \apots\ $\spe$ of order $n$. Then there is $y \in \spe$ that is 
%an isolated point of $\spe$ such that $x = y - \frac{n}{2}$.   
%\end{corollary}

\begin{corollary}\label{predescors}
Let $x \in \speD$. Then:
\begin{itemize}
\item there exists $n_1 \in \mathbb{N}$ such that $x + \frac{n_1}{2} \in \speD$ 
but $x+\frac{n_1+1}{2} \not\in \speD$. \\ In other words, there exist $y \in \speD$ and 
$n_1 \in \mathbb{N}$ such that 
$y + \frac{1}{2} \not\in \speD$ and such that $x = y - \frac{n_1}{2}$;
\item there exists $n_2 \in \mathbb{N}$ such that $x$ is an \apots\ $\speD$ of 
order $n_2$
%\item $x$ is an \apots\ $\speD$ of order $n_2$, for some $n_2 \in \mathbb{N}$;  
\end{itemize}
and $n_1 = n_2$.
\end{corollary}

\section{Order type and topology of $\spebr{M}$}\label{all_spectra_are_isomorphic}
Ok, all isomorphic with $\speD$ \\
here write about it \\
Tell me about it!


\section{Order type and topology of $\spe$}
\begin{theorem}
The order type of the set of possible Euler orbicharacteristics of two-dimensional orbifolds 
$\spe$ is $\omega^\omega$. 
\end{theorem}
\todo{tutaj też dopisać dowód}
Provide some argument about being homeomorphic 
\noindent\textbf{Proof.} \\
From \ref{speD_theorem} we knowm, that $\speD$ is homeomorphic with $\omega^\omega$. From 
\ref{all_spectra_are_isomorphic}, we know, that $\speS$ is homeomorphic 
with $\omega^\omega$. \\
$\speS = 2\speD$, so for all $n\in -\mathbb{N}$ set $\speS \cap [2,n)$ has a lower order type then 
$\speD\cap [2,n)$. From this, we have that $\speS \cup \speD \cong \omega^\omega$. $_\square$ \\[4pt]

\section{Order type and topology of some subsets of $\spe$ and $\spe(M)$}
$\spe_n$ \\
taking limit points is the order type of $\omega^\omega$ but not homeomorphic anymore. 



%\section{Which points are in the $\spe$?}

%Here we will try to understand better the conditions that let us determine wether the point 
%lie in $\spe$ or not.
% We will also state some observations about reasoning which points 
%belong to $\spe$ based on the knowlegde of other points belonging there. \\


\section{More about how this $\omega^\omega$ lies in $\mathbb{R}$}
%\todo{movve to some other section maybe}
\begin{observation}
The first (greatest) negative \apots\ of $\spe$ is 
$-\frac{1}{12}$. It is the accumulation point of order $1$. 
\end{observation}
\noindent\textbf{Proof.} \\
We will show, that $-\frac{1}{12}$ is the greatest negative accumulation point of the set $\speD$. 
From this we will obtain the thesis, as the set of all possible Euler orbicharacteristics 
of two-dimensional orbifolds is equal to $\speS \cup \speD$ and $\speS = 2\speD$, so 
the greatest negative point of the set $\speS$ is smaller than the greatest negative accumulation 
point of the set $\speD$. \\
$\bullet$ $-\frac{1}{12}=\chi^{orb}((2,3))-\frac{1}{2}$, from this we have that $-\frac{1}{12}$ 
an acccumulation point of the set $\speD$ of order at least $1$. \\
$\bullet$ Let us assume (for a contradiction), that there exist bigger, negative 
accumulation point of the set $\speD$ of order at least $1$. Let us denote it by $x$. \\
However, then, from the lemma \ref{third_order_lemma} point $x+\frac{1}{2}$ is the accumulation 
point of the set $\speD$. What is more, since $x\in (0, -\frac{1}{12})$, then $x+\frac{1}{2} 
\in (\frac{1}{2}, \frac{5}{12}$. From the lemma \ref{accumulation_points_of_the_set} we 
have that $x$ is in $\speD$. But orbifolds of the type $\ast b_1$ can have \Eoc only greater or 
equal $\frac{1}{2}$. Orbifolds of the type $\ast b_1b_2$ can only have \Eoc $\frac{1}{2}$, 
$\frac{5}{12}$ and some smaller. Orbifolds of the type $\ast b_1b_2b_3\cdots$ can have \Eoc only 
lower than $\frac{1}{4}$. This analysis of the cases leads us to the conclusion, that 
$(\frac{1}{2},\frac{5}{12})\cap \speD=\emptyset$ and to the contradiction. \\
$\bullet$ Above analysis of the cases leads us also to the conclusion, that $\frac{5}{12}$ 
is 
an isolated point of the set $\speD$, from this $-\frac{1}{12}$ is an acccumulation point 
of order $1$ of the set $\speD$. $_\square$ \\ 
 








