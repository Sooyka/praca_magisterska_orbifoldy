% mainfile: ../praca_magisterska_orbifoldy.tex
\chapter{Different definitions of an orbifold}

This chapter the next will be a technical chapters. Later on we will evoke some terms and 
definitions without 
explicitly saying what they mean instead we will put a reference to this chapter with explicit 
saying to what definition it refers. \\
For example in the later chapters there will be phrases like "adding a defect of order $\dots$" or 
"gluying orbifolds by boundries" and they are explained in this and the next chapter.
 

We will explore various definitions of an orbifold, partially proving they are equivalent, partially 
linking to the sources. \\
Some of these definitions apply only to the special cases. Some of them contain constructions 
with which not all orbifolds can be made (at least some of them can't be derived as such a priori)
. \\

\section{Hiperbolic plane tilling}

\section{Manifolds with defects}
\subsection{Disk and sphere with defects}\label{Disk_and_sphere_with_defects}



\subsection{Conway notation}
\cite{Conway2008}

When it is nesseccery to avoid a confusion, on parts such as $*abcd$, we will be writting 
$*^*a^*b^*c$ instead. \\
We will propose some extension to a notation from \cite{Conway2008}.
We will regard parts of that notation not only as features on an orbifold but also as an operations 
on orbifolds transforming one to another by adding particular feature. \\
We will denote the difference in Euler characteristic which is made by modifying 
an orbifold by such a feature as $\Delta(modification)$ 
which have less capitalistic vibes than "cost". 
For example $\Delta(*^*2) = \frac{1}{4}$. \\
We will denote by $*$ an operation of cutting out a disk and by $^{\beta*}n$ an operation of 
adding a kaleidoscopic point of period $n$ on the boundry component $\beta$. 
Last operation is defined only on orbifolds with boundries.
%identified by Id. 
%When we will not concern ourselves on which component of the boundry kaleidoscopic point will 
%be added we


\section{Quationts of planes}

\section{Generalised manifolds}
This approuch is very simmilar to the previous one. It differs slightly where we put the 
difinition burden. 

\section{Complexes?}

