% mainfile: ../praca_magisterska_orbifoldy.tex
\chapter{Introduction}
\setcounter{page}{8}
\section{Motivations}
Quotiens by a groups

Praca o ujemnym ale z najwyzszą orbicharakterystyką

\section{Definitions}
\todo{jak sie juz wszystko zbierze co ma tu być, to to dopisać}
The definition of the orbifold is taken from Thurston \cite{Thurston1979} (chapter 13). 
We briefly recall the concept, but for full discussion we refer to \cite{Thurston1979}. \\
An orbifold is a generalisation of a manifold. One allows more variety of local behaviour. 
On a manifold a map is a homeomorphism between $\mathbb{R}^n$ and some open set on a manifold. 
On an orbifold a map is a homeomorphism between a quotient of $\mathbb{R}^n$ by some 
finite group and some open set on an orbifold. 
In addition to that, the orbifold structure consist the informations about that finite group 
and a quotient map for any such open set. \\

Above definition says that an orbifold is locally homeomorphic do the quotient of $\mathbb{R}^n$ 
by some finite group. \\
When an orbifold as a whole is quotient of some finite group acting on a manifold we say, that 
it is 'good'. Otherwise we say, that it is 'bad'. \\

We are also adopting notation from \cite{Thurston1979}. \\

In two dimentions there are only four types of bad orbifolds, namely: \\
- $S^2(n)$ \\
- $D^2(;n)$ \\
- $S^2(n_1,n_2)$ for $n_1 < n_2$ \\
- $D^2(;n_1,n_2)$ for $n_1 < n_2$. \\
All other orbifolds are good.
As manifolds are special case of orbifolds with all ...
We differ from Thurston in the terms of naming points with maps with non-trivial groups. 
We call them orbipoints. If the group acts as the group of rotations (so a 
cyclic group) we call them rotational points. If the group is a dihedreal group we call them 
dihedreal points. And if it is point on the boundry that stabilises relfecition it is a 
reflection point.

quotiens of groups 

Map between orbifolds can be defined as follows:
\cite{kleiner2014geometrization}
\subsection{Sameness}
From this we can treat that and that orbifolds as the same.

\section{Euler (orbi)characteristic}\label{E_orb}
\label{\Eoc_as_a_sum}
\todo{dać cytowanie do charakterystyki}
quotiens
\subsection{Euler characteristic}
We can define Euler characteristic as additive topological invariant defined 
normalised on symplexes.

we will treat 
as we will treat manifolds as orbifolds we will always refer 
we will 

from of \Eoc on two dim orbifolds\label{\Eoc on 2d}
\subsection{Classification of orbifolds with non-negative Euler orbicharacteristic}
The list of all orbifolds with non-negative Euler orbicharacteristic
Powiedzieć coś o tym, że orbicharatkeryttyka odpowiada polom (Gauss Bonett itd.)
\subsection{Extended Euler orbicharacteristic}\label{extended_Euler_orbicharacteristic} (with cusps)
Write about cusp as a limit.

Write about isomorphism of all spectra

$M$ - orbifold
\subsection{Spectra}
This will be the main interest of this thesis. $\spe$. dadada

\subsection{Definition and properties of order of accumulation points}
\label{accumulation_points_definitions} 
%\todo{zmienić to wszystko na rangę cantora bendiksona}
%We start with one technical definition of "transitive order" that will be almost what we want
%and then, there will be the the definition of "order", which is the definition that we need.
%\begin{definition}
%(Inductive). 
%%We say, that the point $a$  from the topological space $X$ is an acccumulation 
%%point of the transitive order 0, when
%We say that the point is an acccumulation point of a transitive order $0$, when it is 
%an isolated point. 
%We say that the point is an acccumulation point of a transitive order $n + 1$, when it is 
%an acccumulation point (in the usual sense) of the accumulation points of the transitive 
%order $n$. 
%\end{definition}  
%The only issue of the above definition is that the point of the transitive order $n$ 
%is also a point 
%of the transitive order $k$, for all $0< k \leq n$. We want a definition of order such that 
%for any point, there is at most one integer that is its order. So we define:
%\begin{definition}
%We say that the point is an acccumulation point of order $n$ iff it is an acccumulation point 
%of the transitive order $n$ and it is not an acccumulation point of the transitive order $n+1$. 
%If the point is an acccumulation point of the transitive order for an arbitrary large 
%$n$ we say that 
%the  point is an acccumulation point of order $\omega$.
%\end{definition}
Corespondence between accumulation points and cantor bendixon derivative are describen 
in appendix \smalltodoII{link to appendix}. 
We start with definition of being "at least of order $n$" that will be almost what we want
and then, there will be the definition of being "order", which is the definition that we need. \\
For a given set we define as follows:
\begin{definition}
(Inductive). 
%We say, that the point $a$  from the topological space $X$ is an acccumulation 
%point of the transitive order 0, when
We say that the point $x$ is an acccumulation point of a set $X$ 
of order at least $0$, when it belongs to the set $X$. 
We say that the point $x$ is an acccumulation point of a set 
of order at least $n + 1$, when it is 
an acccumulation point (in the usual sense) of the accumulation points each of order at least 
$n$ i.e. in every neighbourhood of $x$ there is at least one accumulation point of a set $X$ 
of order at least $n$, distincs from $x$. 
\end{definition}  
%The only issue of the above definition is that the point of the transitive order $n$ 
%is also a point 
%of the transitive order $k$, for all $0< k \leq n$. We want a definition of order such that 
%for any point, there is at most one integer that is its order. So we define:
\begin{definition}
We say that the point is an acccumulation point of order $n$ iff it is an acccumulation point 
of order at least $n$ and it is not an acccumulation point of order at least $n+1$. 
If the point is an acccumulation point of order at least $n$ for an arbitrary large 
$n$ we say that 
the point is an acccumulation point of order $\omega$.
\end{definition}
When we will say that a point is an accumulation point of some set without specifying an order 
then we will mean being an accumulation point in the ussual sense; from the point of view 
of above definitions, that is, an accumulation point of order at least one.
%\todo{dopisać notację do punktów skupienia różnego stopnia}

\section{Uniformisation theorem (formulation)}
\todo{twierdzenie o klasyfikacji powierzchni}
\section{Operations and constructions on orbifolds}\label{Operations}
Write about the general operations we are interested in i.e. taking any number of features (handles 
cross caps, parts of boundry components with orbipoints on it, orbipoints in the interior) and
replacing it by any other feauters
(Some preserve the area)
Write about operations nesseserie for reduction of cases
write that every operation reduces \Eoc.


\section{Notation}

"feature"

We treat manifolds and orbifolds as a sphere with some features added by the operations.

dać na sferę $\varepsilon$ słowo puste.
We will regard parts of that notation not only as features on an orbifold but also as an operations 
on orbifolds transforming one to another by adding particular feature. \\
We will denote the difference in Euler characteristic which is made by modifying 
an orbifold by such a feature as $\Delta(modification)$.
\todo{rozwinąć} 
dopisać, że w Conwayowej >= 2

If not stated othewise, in the expressions containing $\infty$ symbol, their value is understood 
as $\varphi(\infty) \coloneqq \lim\limits_{n\to \infty}\varphi(n)$.
\todo{}
Addition of sets and numbers.

Warning throught the whole thesis we will consider only two dimensional manifolds and 
orbifolds, because of that words "two-dimensional" will be usually omited 

delta
h c b

\section{Questions asked}
There will be two main parts of question: 

$\bullet$ Ones regarding $\spe$ as a set, where we will be asking 
of its order type and topology and relation to other sets such as $\speD$ and $\speS$. 
We will focus on these questions in \ref{order structure}. 

$\bullet$ Ones regarding $\spe$ as an image of a $\chi^{orb}$, sending orbifolds to their \Eoc s. 
There, we will ask for example how namy orbifolds have particular \Eoc\ and 
related questions. We will focus on these questions in the chapter \ref{counting occurrences}.  

Reduction presented in \ref{reduction_to_arithmetical} are with . 





