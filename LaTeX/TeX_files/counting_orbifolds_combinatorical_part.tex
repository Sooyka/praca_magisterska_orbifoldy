% mainfile: ../praca_magisterska_orbifoldy.tex
\chapter{Counting orbifolds -- combinatorical part}\label{Counting orbifolds -- combinatorical part}
We will go through the steps described in \ref{combinatorical part}.

We will form our answer in a form of answering the 
question for a given number $\frac{P}{q}$ and a given manifold $M$. 
As stated in \ref{combinatorical part} at the end all answers for all $M$ needs to be summed, 
and we know that for $M$ such that $\chi(M) < \frac{p}{q}$, the answer is $0$, so there 
are always only finitely many answers to be summed. 

\section{Manifolds without boundary}
In this case we ask our algorithm from \ref{Counting orbifolds -- arithmetical part}: 
"How many sums of the form
\begin{equation}
1-\sum_{j=1}^m \frac{d_j-1}{2d_j} 
\end{equation}
are equal to 
\begin{equation}
\frac{1}{2}\frac{p}{q} - \frac{1}{2}\chi(M) -1\ ?",
\end{equation}
and the result is our final answer, as (\ref{sameness}) the list of degrees of rotational orbipoints ordered 
in descending order uniquely defines a two dimensional $M$-orbifold without boundary. 
\section{Manifolds with boundary}
\subsection{Using chapter \ref{Counting orbifolds -- arithmetical part}}\label{use algorithm step}
We ask our algorithm from chapter \ref{Counting orbifolds -- arithmetical part} for the list 
of all possible sums of the form
\begin{equation}
1-\sum_{j=1}^m \frac{d_j-1}{2d_j} 
\end{equation}
that are equal to 
\begin{equation}
\frac{p}{q} - \chi(M) - 1. 
\end{equation}
As we know from: 
\begin{itemize}
\item theorem \ref{second_finiteness_theorem} 
\item the fact, that to each 
sum corresponds at least one $M$-orbifold
\item the fact, that to different sums corresponds different $M$-orbifolds,
\end{itemize}
this list of sums will be finite. 
%\subsection{Fixed}
\subsection{Reduction to only dihedral orbipoints}\label{reduction only to dihedral}
As stated in \ref{combinatorical part} in point 3.3.1
%\ref{additional things in boundary case -- rotational}
, we need to first 
take into account that an orbifold can have both dihedral and rotational orbipoints. 

We have complete list of sums corresponding to degrees of dihedral orbipoints that result in 
the $\frac{p}{q}$ orbicharacteristic on $M$-orbifold. 

We are interested in having complete list of sums corresponding to degrees of both dihedral 
and rotational 
orbipoints that result in 
the $\frac{p}{q}$ orbicharacteristic on $M$-orbifold. 
We will now propose a unique  
reduction $R$ 
%($\ast$) 
of every list corresponding to both dihedral and 
rotational orbipoints to the list of only dihedral orbipoints. 

The reduction 
($\ast$) 
$R$ goes as follows: for a list of degrees of orbipoints consisting of 
$r_1r_2\cdots r_n$ for rotational orbipoints and $d_1d_2\cdots d_m$ for dihedral orbipoints, 
we replace each $r_i$ from the list of rotational orbipoints by to entries of the same 
value on the list of dihedral orbipoints. This does not change the 
corresponding \Eoc, since $\Delta(n) = \Delta(^*n^*n)$ 
%(as stated in \ref{})
. 

This procedure is unambiguous and gives only one possible list of dihedral orbipoints degrees 
for every list of both rotational and dihedral orbipoints degrees. 

Based on this, for a given sum $d_1d_2\cdots d_n$,
we can perform the transformation of replacing $n$ by $^*n^*n$ in another direction to 
produce all possible sums consisting of both rotational and dihedral degrees, that 
would be reduced to sum $d_1d_2\cdots d_n$. The uniqueness of reduction $R$ 
%($\ast$) 
guarantees, that 
we won't arrive to the same unreducted sum from different starting lists of only dihedral degrees.

Based on this, after getting the list of sums from \ref{use algorithm step}, we need to 
add to this list all possible sums that could be reducted by 
%($\ast$) 
$R$ to some 
of the sums that we got from \ref{use algorithm step}. 

Sums on the new extended list have also rotational orbipoints taken into account. 
At this point we have full list of sums resulting in \Eoc\ equal to $\frac{p}{q}$ on a 
$M$-orbifolds. 
In further considerations we will not explicitly rotational orbipoints degrees as they play 
no role in combinatorics.

\subsection{Fixed sum}

Let us now consider a case with a manifold $M$, and a sum $d_1d_2\cdots d_n$ (as we wrote 
at the end of the previous section we are not writing rotational degrees as 
they will play no role from this point, however, they are possibly present in some sums). 
Then the last step of the procedure will be to sum over all sums produced in 
\ref{reduction only to dihedral}. 



Let $b$ be the number of boundary components of $M$. We know from the assumption, that $b > 0$. 

%Let the sum be of the form $a_1^{k_1}a_2^{k_2}\cdots a_m^{k_m}$, meaning that the 
%list $d_1d_2\cdots d_n$ looks like (let $l_i \coloneqq \sum_{j = 1}^i k_j$):
%\begin{equation}
%\underbrace{d_1d_2\cdots d_{l_1}}_{k_1\ \mathrm{times}\ a_1} 
%\underbrace{d_{l_1+1}d_{l_1+2}\cdots d_{l_2}}_{k_2\ \mathrm{times}\ a_2} \cdots 
%\underbrace{d_{l_{m-1}}d_{l_{m-1}+1}\cdots d_n}_{k_m\ \mathrm{times}\ a_m}. 
%\end{equation}

%We need to partition $a_1^{k_1}a_2^{k_2}\cdots a_m^{k_m}$ among boundary 
%components of $M$. 

We need to partition $d_1d_2\cdots d_n$ among boundary 
components of $M$. 

%In considerations of partitions, we will treat 
%At first we will give each of the boundary component a label and treat them as distinguishable. 
%At the end, we will divide the result by $b!$ reflecting the fact, that topologically, they are 
%not 
%distinguishable. 

At this moment we will treat boundary components as distinguishable. 

%Effective way to iterate over all the partitions 
Let us consider some partition of $d_1d_2\cdots d_n$ among the 
distinguishable  boundary 
components. After this, results from all partitions need to be summed together. 
The fact that boundary components are not distinguishable 
will be taken into account in the next subsection by assigning the proper weights in the summation.
%, but in the 
%case that $M$ is orientable,
%results from partitions where at least . 



\subsection{Fixed partition}
Let us now consider some fixed partition.
% and one boundary component. 
Suppose that in this partition for every $1\leq j \leq b$
boundary component $B_j$ have orbipoints of degrees: 
%$^ja_1^{^js_1}\ ^ja_2^{^js_2}\cdots\ ^ja_{n_j}^{^js_{n_j}}$
$^jd_1\ ^jd_2\cdots ^jd_{n_j}$.  
We want to know how many possible sets of cyclic orders there are on a boundary components with 
$^jd_1\ ^jd_2\cdots ^jd_{n_j}$ on $B_i$. 
%Having this answer we will multiply these answers from each boundary component in 
%a given partition
%%, this way resulting in an answer for a partition. 
%However  

%First, we will start with answearing the question when all $s_i$ are equal to $1$. 
%Then:
%\begin{itemize}
%\item if $M$ is orientable, possible number of orders is the number 
%of linear orders -- $n!$, divided by 
%the number of cyclic permutations -- $n$, giving the answear $(n-1)!$. 
%\item  if $M$ is non orientable and $n \geq 3$, possible number of orders is the number 
%of linear orders -- $n!$, divided by the number of elements in a dihedral group $D_n$, 
%that is -- $2n$, resultng in an answer $\frac{(n-1)!}{2}$.
%\item if $M$ is non orientable and $n < 3$, then the answear is the same as in orientable case 
%and there exists only one cyclic order that is inveriant to reflections.  
%%as cyclic orders of one ore two elements are invariant to reflections. 
%\end{itemize} 

%Then 

%\subsubsection{Fixed cyclic order for the type of the components}
%We can enumerate through all the cyclic orders by iterating though linear orders
We can count sets of cyclic orders on indistinguishable boundary components 
by iterating through tuples of 
linear orders on distinguishable boundary components and summing them with proper weights. 

Given the tuple of linear orders $\mathcal{L} = (L_1, \cdots, L_b)$ on
distinguishable components, to calculate 
the weight $W(\mathcal{L})$, we will first set some weights $W(L_j)$, for every 
$1\leq j \leq b$. 
%Then we will put $W(L) \coloneqq \prod_{i=1}^bL_i$.  

%We 
%, but counting each 
For a linear 
order $L_j$ we set the 
weight $W(L_j)$ to be 
%(let $t \coloneqq \sum_{i = 0}^n s_i$)
:  
%where $\Gamma$ is (let $t \coloneqq \sum_{i = 0}^n s_i$): 
\begin{itemize}
\item in case $M$ is orientable -- $\frac{|\mathbb{Z}_k|}{|\mathbb{Z}_{n_j}|} = \frac{k}{n_j}$, 
where $\mathbb{Z}_k$ 
is the biggest cyclic subgroup of $\mathbb{Z}_{n_j}$, under which $L_j$ 
is invariant as a linear order,
\item in case $M$ is not orientable -- $\frac{|\mathbb{D}_k|}{|\mathbb{D}_{n_j}|} = 
\frac{2k}{2n_j} 
=\frac{k}{n_j}$ , 
where $\mathbb{D}_k$ is 
the biggest dihedral 
subgroup of $\mathbb{D}_{n_j}$, under 
which $L_j$ is invariant as a linear order. 
\end{itemize}
%After this, we need to divide 
%First, let us answear this question for $b = 1$. 

Then, we put $W(L) \coloneqq SR\prod_{j=1}^bL_j$, where:
\begin{itemize}
\item $S \coloneqq \frac{|G|}{|S_b|}$, where $G$ is the biggest subgroup of permutations 
$S_b$ under which 
tuple of orders $\mathcal{L}$ is invariant as a tuple of cyclic orders, 
\item $R = \frac{1}{2}$ if
$M$ is orientable and  
at least one of linear orders $L_j$ is different as a cyclic order than the 
reverse of $L_j$ as a cyclic order; 
otherwise $R = 1$
\end{itemize}


%Let us start with only cosidering dihedral orbipoints.

%Let us first calculate number of possible divisions of $a_1^{k_1}a_2^{k_2}\cdots a_m^{k_m}$ 
%among boundarie components. If we have $k_i$ copies of the orbipoint of degree $a_i$, 
%the problem of counting the possible distributions of them among $b$ boundary is 
%equivalent to the problem of dividing $k_i$ elemnent set into $b$ distinguishable compartemets. 
%This is in turn equivalent to a problem of placing $b-1$ 
%undistinguishable objects into $k_i+1$ distinguishable compartements 
%(each of this compartemett is a places of divisions of 
%$k_i$ element 
%set into the next of consequtive $b$ groups). This in turn is equal to     

%\section{Counting orbifolds -- combinatorical part summary}
\subsection{Comment about possibility of a single equation}
Although, given enough effort, results from this section could be summarise in one 
equation consisting only 
of $b$ and $k_1, k_2, \cdots, k_n$ for a given sum, 
we feel that it would be long enough not to give any new 
insight into the structure of the problem. We are stopping thus at giving the above procedure.
%what is an algorithm if not very elaborate equation after all?


