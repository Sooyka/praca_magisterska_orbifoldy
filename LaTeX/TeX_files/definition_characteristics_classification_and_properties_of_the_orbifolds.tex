% mainfile: ../praca_magisterska_orbifoldy.tex
\chapter{Definition, characteristics, classification and properties of the orbifolds}
\section{Definition}
In the definition of the orbifold we are following Thurston from \cite{Thurston1979} (chapter 13). 
We will briefly present the concept here, but we encourage the reader to consult 
\cite{Thurston1979}. \\
We define an orbifold as a generalisation of a manifold. The difference is in maps. \\
On a manifold a map is a homeomorphism between $\mathbb{R}^n$ and some open set on a manifold. \\
On an orbifold a map is a homeomorphism between a quotient of $\mathbb{R}^n$ by some 
finite group and some open set on an orbifold. \\
In addition to that, the orbifold structure consist the informations about that finite group 
and a quotient map for any such open set. \\

Above definition says that an orbifold is locally homeomorphic do the quotient of $\mathbb{R}^n$ 
by some finite group. \\
When an orbifold as a whole is quotient of some finite group acting on a manifold we say, that 
it is 'good'. Otherwise we say, that it is 'bad'. \\

We are also adopting notation from \cite{Thurston1979}. \\

In two dimentions there are only four types of bad orbifolds, namely: \\
- $S^2(n)$ \\
- $D^2(;n)$ \\
- $S^2(n_1,n_2)$ for $n_1 < n_2$ \\
- $D^2(;n_1,n_2)$ for $n_1 < n_2$. \\
All other orbifolds are good.
\section{Euler orbicharacteristic}
\subsection{Classification of orbifolds with non-negative Euler orbicharacteristic}
The list of all orbifolds with non-negative Euler orbicharacteristic
Powiedzieć coś o tym, że orbicharatkeryttyka odpowiada polom (Gauss Bonett itd.)
\subsection{Extended Euler orbicharacteristic}\label{extended_Euler_orbicharacteristic} (with cusps)
Write about cusp as a limit.
\section{Uniformisation theorem (formulation)}
\section{Surgeries, modifications and constructions on orbifolds}\label{surgeries}
(Some preserve the area)
\section{Notation}
We will regard parts of that notation not only as features on an orbifold but also as an operations 
on orbifolds transforming one to another by adding particular feature. \\
We will denote the difference in Euler characteristic which is made by modifying 
an orbifold by such a feature as $\Delta(modification)$.
\todo{rozwinąć} 



