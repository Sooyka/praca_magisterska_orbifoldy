% mainfile: ../praca_magisterska_orbifoldy.tex
\chapter{Definition, characteristics, classification and properties of the orbifolds}
\section{Definition}
\todo{jak sie juz wszystko zbierze co ma tu być, to to dopisać}
The definition of the orbifold is taken from Thurston \cite{Thurston1979} (chapter 13). 
We briefly recall the concept, but for full discussion we refer to \cite{Thurston1979}. \\
An orbifold is a generalisation of a manifold. One allows more variety of local behaviour. 
On a manifold a map is a homeomorphism between $\mathbb{R}^n$ and some open set on a manifold. 
On an orbifold a map is a homeomorphism between a quotient of $\mathbb{R}^n$ by some 
finite group and some open set on an orbifold. 
In addition to that, the orbifold structure consist the informations about that finite group 
and a quotient map for any such open set. \\

Above definition says that an orbifold is locally homeomorphic do the quotient of $\mathbb{R}^n$ 
by some finite group. \\
When an orbifold as a whole is quotient of some finite group acting on a manifold we say, that 
it is 'good'. Otherwise we say, that it is 'bad'. \\

We are also adopting notation from \cite{Thurston1979}. \\

In two dimentions there are only four types of bad orbifolds, namely: \\
- $S^2(n)$ \\
- $D^2(;n)$ \\
- $S^2(n_1,n_2)$ for $n_1 < n_2$ \\
- $D^2(;n_1,n_2)$ for $n_1 < n_2$. \\
All other orbifolds are good.
As manifolds are special case of orbifolds with all ...
We differ from Thurston in the terms of naming points with maps with non-trivial groups. 
We call them orbipoints. If the group acts as the group of rotations (so a 
cyclic group) we call them rotational points. If the group is a dihedreal group we call them 
dihedreal points. And if it is point on the boundry that stabilises relfecition it is a 
reflection point.
\section{Euler orbicharacteristic}
\label{\Eoc_as_a_sum}
we will treat 
as we will treat manifolds as orbifolds we will always refer 
we will 
\subsection{Classification of orbifolds with non-negative Euler orbicharacteristic}
The list of all orbifolds with non-negative Euler orbicharacteristic
Powiedzieć coś o tym, że orbicharatkeryttyka odpowiada polom (Gauss Bonett itd.)
\subsection{Extended Euler orbicharacteristic}\label{extended_Euler_orbicharacteristic} (with cusps)
Write about cusp as a limit.

Write about isomorphism of all spectra

\section{Uniformisation theorem (formulation)}
\todo{twierdzenie o klasyfikacji powierzchni}
\section{Surgeries, modifications and constructions on orbifolds}\label{surgeries}
Write about the general sugeries we are interested in i.e. taking any number of features (handles 
cross caps, parts of boundry components with orbipoints on it, orbipoints in the interior) and
replacing it by any other feauters
(Some preserve the area)
Write about surgeries nesseserie for reduction of cases
\section{Notation}
We will regard parts of that notation not only as features on an orbifold but also as an operations 
on orbifolds transforming one to another by adding particular feature. \\
We will denote the difference in Euler characteristic which is made by modifying 
an orbifold by such a feature as $\Delta(modification)$.
\todo{rozwinąć} 



