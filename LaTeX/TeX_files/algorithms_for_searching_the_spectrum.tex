% mainfile: ../praca_magisterska_orbifoldy.tex
\chapter{Algorithms for searching the spectrum}

\section{Decidability}
\todo{oj dokończyć}
Here we will show the proof that the problem of "deciding whether a given rational number is in an 
Euler orbicharacteristic's spectrum or not" is decidable by showing algorithm for doing this. 
Later, our algorithm will have a bonus property of determining of which order of condensation 
is given point if it is in fact in $\sigma$. \\
\smalltodoII{Może od razu postawić pełny problem}
%It fill get also a performance enhancement by this added property. \\
First stated algorithm is also very inefficient and is presented, because the idea is the most 
clear in it. Right after it there is stated an algorithm with two enhancements: 
\begin{itemize}
\item determining an accumulation point of which order is a given point, if it is in fact in the 
spectrum (this enhancement gives also a performance boost)
\item faster searching, because some cases do not need to be checked. 
\end{itemize}
We start with $\frac{p}{q}$, where $p \in \mathbb{Z}$, $q \in \mathbb{N}_{>0}$. \\ 
%and $\textrm{gcd}(p,q)$. \\ 

We want to determine whether there exists $b_1,b_2,\dots,b_k$, such that $\chi^{orb}(*b_1\dots b_k) = 
\frac{p}{q}$. \\ 

In the case that $\frac{p}{q}$ is of the form $l\frac{1}{4}$, for some whole $l$ 
% $q = 4$ 
we can give the answer right away. For $l > 4$ we have that $l\frac{1}{4}$ is not in the set 
and for $l \leq 4$ it is. Moreover for an even $l$ it is a condensation point of order 
$\frac{4-l}{2}$ (see \ref{biggest \apots})
and for an odd $l$ it is a condensation point of order $\frac{3-l}{2}$ (see \ref{predescors}). \\

In the case, where $\frac{p}{q} >2$, we also can give answer rigth away and this answr is "no". \\

Now we will consider only cases when $\frac{p}{q}$ is not of the form $l\frac{1}{4}$ and is 
$\leq 2$.
\subsection{The first approach to the searching algorithm}
%The first approach of the searching algorithm is of this form: \\

%%We start with the 
%We use: 
%\begin{itemize}
%$\mathbb{N}$ counters $b_1b_2\dots$ 
%(with values ranging from $1$, through all natural numbers, to infinity 
%(with infinity included)) set to $1$. Each counter correspond to one cone point 
%on the boundry of the disk of period equal to the value of the counter (with the note, that 
%if counter is set to $1$ it means a trivial cone point - namely a none cone point, a normal point). 
%Every state of the counters during runtime of the algorith will have only finitely many 
%counters with value non-$1$. Moreover every state in the rutime of the algorithm 
%will have values on consequtive counters ordered in weakly decreasing order. From now we will 
%consider only such states. \\
%The state $b_1b_2\dots$ correspond to the orbifold of 
%\Eoc equal $\chi^{orb}(*b_1b_2\dots)$ (where the trailing $1$ are trunkated). \\ 
%%There is also a pivot pointing on one counter at any time.  

We use: 
\begin{itemize}
\item $\mathbb{N}_{>0}$ counters $b_1b_2\dots$ 
with values ranging from $\mathbb{N}_{>0}\cup\{\infty\}$.
%$1$, through all natural numbers, to infinity 
%(with infinity included). 
Each counter correspond to one cone point 
on the boundry of the disk of period equal to the value of the counter (with the note, that 
if counter is set to $1$ it means a trivial cone point - namely a none cone point, a normal point). 
\item a pivot pointing to some counter at any time
\item a flag that can be set to "Greater" or "Less" corresponding to what was 
the outcome of comparing \Eoc\ of the orbifold corresponding to counters' state and 
$\frac{p}{q}$.  
\end{itemize}

We start with:
\begin{itemize}
\item all counters set to $1$. 
\item pivot pointing at the first counter
\item flag set to "Greater"
\end{itemize}
%If $\frac{p}{q}$ is of form $\frac{k}{4}$, where $k \in (-\infty,8] \cap \mathbb{Z}$ we give 
%the answear "yes" and end the whole algorithm. If $\frac{p}{q} > 2$ we give the answear "no" and end the whole algorithmBecauseof this, below we assume, that \\
We will do our computation such that:
\begin{itemize}
\item every state of the counters during runtime of the algorithm will have only finitely many 
counters with value non-$1$. 
\item every state in the rutime of the algorithm 
will have values on consequtive counters ordered in weakly decreasing order.
\end{itemize}
From now we will 
consider only such states. \\
%There is also a pivot pointing on one counter at any time.  
The state of the counters $b_1b_2\dots$ correspond to the orbifold of 
\Eoc\ equal $\chi^{orb}(*b_1b_2\dots)$ (where the trailing $1$ are trunkated). \\ 
When the algorithm is in the state: 
\begin{itemize}
\item counters: $b_1b_2\dots$
\item pivot: on the counter $c$
\item flag: set to the value $flag\_value$,
\end{itemize}
we procced as follows 
%(the term "We continue." means, that we start the following procedure from the beginning)
:
\begin{lstlisting}[firstnumber=1,consecutivenumbers=true]
In the case, the flag is set to: 
{
    "Less", then 
    {
        We increase the counter $c$ by one ($b_c \coloneqq b_c + 1$).
        If $b_c = 2$ and the values of all the counters 
        on the left are also equal 2 then 
        {
            We end the whole algorithm with the answer "no".
        }
        We set the value of all counters on the left to $b_c$
        If $\chi^{orb}(*b_1b_2b_3\dots)=\frac{p}{q}$ then
        {
            We found an orbifold and we are ending the whole
            algorithm with answer "yes, $*b_1b_2\dots$".
        }
        If $\chi^{orb}(*b_1b_2b_3\dots)>\frac{p}{q}$ then  
        {
            We set the flag to "Greater".
            We put the pivot on the first counter. 
            We go to the line 1..
        } 
        If $\chi^{orb}(*b_1b_2b_3\dots)<\frac{p}{q}$ then
        {
            We set the flag to "Less".
            We put pivot to the $c+1$ counter.
            We go to the line 1..
        } 
    }

    "Greater", then
    {
        If $\chi^{orb}(*b_1\dots b_{c-1}\infty b_{c+1}\dots)=\frac{p}{q}$ then
        {
            We found an orbifold and we are ending the whole
            algorithm with answer "yes, $*b_1\dots b_{c-1}\infty b_{c+1}\dots$".
        } 
        If $\chi^{orb}(*b_1\dots b_{c-1}\infty b_{c+1}\dots)>\frac{p}{q}$ then
        {
            We set $b_c$ to $\infty$.
            We set the flag to "Greater".
            We move pivot to the $c+1$ counter.
            We go to the line 1..
        }  
        If $\chi^{orb}(*b_1\dots b_{c-1}\infty b_{c+1}\dots)<\frac{p}{q}$ then
        {
            We search for value $b_c'$ of the $c$ counter 
            such that $\chi^{orb}(*b_1\dots b_{c-1}b_c'b_{c+1}\dots)\leq\frac{p}{q}$ 
            and $\chi^{orb}(*b_1\dots b_{c-1}(b_c'-1)b_{c+1}\dots)>\frac{p}{q}$.
            More on how we search for it will be told later, for now  
            we can think that we search one by one starting 
            from $b_c$ and going up till $b_c'$.
            We set $b_c$ to $b_c'$.
            if $\chi^{orb}(*b_1b_2b_3\dots)=\frac{p}{q}$ then 
            {
                We found an orbifold and we are ending the whole
                algorithm with answer "yes, $*b_1b_2\dots$".
            }
            We set all the counters to the left to value $b_c$.
            if $\chi^{orb}(*b_1b_2b_3\dots)=\frac{p}{q}$ then 
            {
                We found an orbifold and we are ending the whole
                algorithm with answer "yes, $*b_1b_2\dots$".
            }
            If $\chi^{orb}(*b_1b_2b_3\dots)<\frac{p}{q}$ then 
            {
                We set flag to "Less".
                We move the pivot to the column $c+1$.
                We go to the line 1..
            }
            If $\chi^{orb}(*b_1b_2b_3\dots)>\frac{p}{q}$ then 
            {
                We set the flag to "Greater".
                We move the pivot to the first counter.
                We go to the line 1..
            }
        }  
    }
}
\end{lstlisting}
\subsection{Why this works}

%\begin{enumerate}
%    \item if the pivot is on the zeroth column ($c = 0$), then 
%    \begin{enumerate}[label*=\arabic*.]
%        \item 
%        \item
%    \end{enumerate}
%    \item abcd
%\end{enumerate}
\subsection{Improvements}
Let $m \in \mathbb{N}$ be such that $\frac{p}{q} \in (1-\frac{m}{2},1-
\frac{m+1}{2})$
Let us denote by $r \coloneqq \frac{p}{q} - (1-\frac{m}{2})$. \\ 

We will searching in $\sigma$ as such: \\

If $\frac{p}{q} \in \sigma$, then, from the corollary \ref{predescors} we know, that there 
exist some $n \in \mathbb{N}$, such that $\frac{p}{q} + \frac{n}{2} \in \sigma$ but 
$\frac{p}{q} + \frac{n}{2} \not\in \sigma$. \\

We will be consequently checking points from $1+r$, through $1+r-\frac{l}{2}$, for 
$0 \leq l \leq m$, to the $\frac{p}{q}$. We stop at the first found point. 
If one of these point is in the spectrum, then all smaller (so also $\frac{p}{q}$) are in 
the spectrum and $\frac{p}{q}$ is the accumulation point of the spectrum of order $m-l$ 
(from this, 
 we can see some heuristic, that the points that have smaller order will be generally 
harder to find in some sense). If none of this points are in in the spectrum, then $\frac{p}{q}$ 
is not. \\

%We want to determine this $n$. If $n = 0$, then $\frac{p}{q}$ is not in $\sigma$. 
%If $n > 0$, then $$

Searching for all occurences \\

\subsection{Implementation}
As an appendix, there is a source of a program with implementation of this algorithm 
with full  
enhancments described in this chapter. It is written in Rust. 
%It is in the separate file, as it would take too much space in this 
%document and wouldn't be readable. 



