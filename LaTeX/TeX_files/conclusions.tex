% mainfile: ../praca_magisterska_orbifoldy.tex
\chapter{Conclusions}

We described the spectrum of possible \Eoc s of two dimentional orbifolds. 
It has topology of $\omega^\omega$ and the problem, whether the given point is in the 
spectrum is desidable. \\

We also provided some finitess results, such as that there are always only finately many 
orbifolds for a given \Eoc. So the problem how much are for a given number is also decidable.\\

From \smalltodo{referencja} we know, that there are howe ver, blab la dowolnie dużo na \Eoc. \\

It remains unclear how Disk spectrum and Sphere spcectrum lies relative to each other, 
but some result was, shown, namely, that every accumulation point of Sphere spectum
is also in the disk spectrum.

for every denominator, do they coincide 
from a sufficiently distatn point? (Yes.) 

\section{Further directions}
\subsection{Asked, but unanswered questions}
Our ultimate goal is to give the answer to the questions such as: \\
- For a given $x \in \sigma$, how many orbifolds have $x$ as their \Eoc?\\
- Why? Is there some underlying geometrical reason for that?\\
- Can we characterise points $x \in \sigma$ that has the most orbifolds corresponding to them? \\
- Is there any reasonable normalisation to counter the effect that there are 'more' 
points as we go 
to lesser values. (What we mean by 'more' was sted in) \\
The first equstion we can tackle is steaming from the chapter \ref{order structure} 
and it is -- Do $\speD$ and $\speS$ coincide? It is easy to answer that $\speD \neq \speS$ 
(and we will do that along some harder questions in the moment), but do they coincide 
starting from a sufficiently distant point? 
%Or maybe, for every denominator, do they coincide 
%from a sufficiently distatn point? (Yes.) \\
%\todo{przenieść (przekopiować?) część może do futher direction}
write about cyclic order
%\subsection{Unasked and unanswered questions}
%\subsection{Power series and generating functions}
%\subsection{Seifert manifolds}

