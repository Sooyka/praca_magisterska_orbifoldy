% mainfile: ../praca_magisterska_orbifoldy.tex
\chapter{Conclusions}

\section{What was done}
In \ref{reduction_to_arithmetical} we proved that $\speD \not\subseteq \speS$ and 
$\speS \not\subseteq \speD$. 

In chapter \ref{order structure}, among other things, we described the spectrum of possible \Eoc s 
of two dimensional orbifolds
and, as the result, the spectrum of all possible areas 
of two dimensional hyperbolic orbifolds in a ordinal and topological manner. 
It has order type and topology (induced from $\mathbb{R}$) of $\omega^\omega$. 
We also proved, that every accumulation point of $\speS$ is in $\speD$.
%and the problem, whether the given point is in the 
%spectrum is desidable. \\

In chapter \ref{Searching the spectrum} 
we provided algorithm for deciding for a given number $x$, whether 
there exists an orbifold $O$, such that $\cho{O} = x$ and proved its correctness.

In chapter \ref{counting occurrences} we provided some finiteness results, such as that 
there are always only finitely many 
orbifolds for a given \Eoc. 
%So the problem how much are for a given number is also decidable.\\
We also proved that for every $n$, in every neighbourhood of every accumulation point 
of $\spe$ of order at least $2$, there is at least one number $x$, such that there are at least 
$n$ orbifolds such that $\cho{O} = x$.

In chapter \ref{Counting orbifolds -- arithmetical part} 
and chapter \ref{Counting orbifolds -- combinatorical part} 
we provided an algorithm 
 for counting for a given number $x$ number of orbifold such that $\cho{O} = x$,
and proved its correctness. 
 
%of this algorithm and discussed, 
We also discussed that its complexity is low enough for actual implementation 
and practical usage on a reasonably small denominators and reasonably close to zero.
%From \smalltodo{referencja} we know, that there are howe ver, blab la dowolnie dużo na \Eoc. \\



%for every denominator, do they coincide 
%from a sufficiently distatn point? (Yes.) 

\section{Further directions}
It remains unclear how Disk spectrum and Sphere spectrum lies relative to each other. 
In particular we still don't know, whether they coincide from a sufficiently distant point.
%but some result was, shown, namely, that every accumulation point of Sphere spectum
%is also in the disk spectrum.

%\subsection{Asked, but unanswered questions}
%Our ultimate goal would to give the answer to the questions such as: \\
%- For a given $x \in \sigma$, how many orbifolds have $x$ as their \Eoc?\\
We don't really know why there is exactly "this" many orbifolds for a given \Eoc? 
%Giving only the algorithm gave little 
We would like to know, whether there is some underlying geometrical reason for that?

We would like to somehow characterise points $x \in \sigma$ that has "the most" 
orbifolds corresponding to them. With reasonable normalisation of what it means for a number
to have "more" orbifolds as we go to lesser values of \Eoc. 

%- Is there any reasonable normalisation to counter the effect that there are 'more' 
%points as we go 
%to lesser values. (What we mean by 'more' was sted in) \\
%The first equstion we can tackle is steaming from the chapter \ref{order structure} 
%and it is -- Do $\speD$ and $\speS$ coincide? It is easy to answer that $\speD \neq \speS$ 
%(and we will do that along some harder questions in the moment), but do they coincide 
%starting from a sufficiently distant point? 
%Or maybe, for every denominator, do they coincide 
%from a sufficiently distatn point? (Yes.) \\
%\todo{przenieść (przekopiować?) część może do futher direction}
%write about cyclic order
%\subsection{Unasked and unanswered questions}
%\subsection{Power series and generating functions}
%\subsection{Seifert manifolds}

