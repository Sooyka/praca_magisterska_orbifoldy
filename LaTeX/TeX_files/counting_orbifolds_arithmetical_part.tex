% mainfile: ../praca_magisterska_orbifoldy.tex
\chapter{Counting orbifolds -- arithmetical part}\label{Counting orbifolds -- arithmetical part}


%\section{Arithmetical part}

%We want to determine this $n$. If $n = 0$, then $\frac{p}{q}$ is not in $\sigma$. 
%If $n > 0$, then $$
%\subsection{Deciding number of occurences}
%Searching for all occurences 

%The difficulty here is to carefully step other an occurence. 

%Compared to the previous version, we also use an occurance counter, starting with it set to 0 
%and with the list of orbifolds, wich is empty at the start.
\section{The idea of the algorithm}
This is an extention of the algorithm from \ref{Searching the spectrum}. It only differs by 
lines after finding the solution -- 7-11, 37-41 and 68-72. They all send the control flow 
to line 57th.
Instead terminating the algorithm the sollution is appended to the 
initialy empty list and the algorithm proceeds to search through the states as if the 
the state that the solution was changed to 
%after some repetitions of changes 
at lines 64-65, together with the pointer placement and flag value was  
the starting configuration. 

\begin{lstlisting}[firstnumber=1,consecutivenumbers=true]
In the case, the $flag\_value$ is equal to: 
{
    "Greater", then
    {
        If $\chi^{orb}(*d_1\dots d_{p-1}\infty d_{p+1}\dots)=\frac{p}{q}$ then
        {
            We found an orbifold, we add it to a list 
            and increase the occurence counter by 1. 
            We set the flag to "Less".
            We put pivot to the $c_{p+1}$ counter.
            We go to the 1st line.
        } 
        If $\chi^{orb}(*d_1\dots d_{p-1}\infty d_{p+1}\dots)>\frac{p}{q}$ then
        {
            We set $d_p$ to $\infty$.
            We set the flag to "Greater".
            We put the pivot at the $c_{p+1}$.
            We go to the 1st line.
        }  
        If $\chi^{orb}(*d_1\dots d_{p-1}\infty d_{p+1}\dots)<\frac{p}{q}$ then
        {
            We set the flag to "Searching".
            We go to the 1st line.
        }  
    }
    
    "Searching", then
    {
        We search one by one 
        for the value $d_p'$ of the $c_p$ such that 
        $\chi^{orb}(*d_1\dots d_{p-1}d_p'd_{p+1}\dots)\leq\frac{p}{q}$ and 
        $\chi^{orb}(*d_1\dots d_{p-1}(d_p'-1)d_{p+1}\dots)>\frac{p}{q}$.
        We set $c_p$ and all of the counters 
        to the left of $c_p$ to the value $d_p'$.
        if $\chi^{orb}(*d_1d_2d_3\dots)=\frac{p}{q}$ then 
        {
            We found an orbifold, we add it to a list 
            and increase the occurence counter by 1. 
            We set the flag to "Less".
            We put the pivot at the $c_{p+1}$.
            We go to the 1st line.
        }
        If $\chi^{orb}(*d_1d_2d_3\dots)<\frac{p}{q}$ then 
        {
            We set the flag to "Less".
            We put the pivot at the $c_{p+1}$.
            We go to the 1st line.
        }
        If $\chi^{orb}(*d_1d_2d_3\dots)>\frac{p}{q}$ then 
        {
            We set the flag to "Greater".
            We put the pivot at the $c_1$.
            We go to the 1st line.
        }
    }
    
    "Less", then 
    {
        If $d_p = 1$ and the values of all the counters 
        on the left of $c_p$ are equal to 2 then 
        {
            We end the whole algorithm with the answer "no".
        }
        We increase $c_p$ by one ($d_p \coloneqq d_p + 1$) and
        we set the value of all counters on the left of $c_p$ to $d_p$.
        If $\chi^{orb}(*d_1d_2d_3\dots)=\frac{p}{q}$ then
        {
            We found an orbifold, we add it to a list 
            and increase the occurence counter by 1. 
            We set the flag to "Less".
            We put pivot at the $c_{p+1}$.
            We go to the line 1..
        }
        If $\chi^{orb}(*d_1d_2d_3\dots)>\frac{p}{q}$ then  
        {
            We set the flag to "Greater".
            We put the pivot at the $c_1$. 
            We go to the 1st line.
        } 
        If $\chi^{orb}(*d_1d_2d_3\dots)<\frac{p}{q}$ then
        {
            We set the flag to "Less".
            We put pivot at the $c_{p+1}$.
            We go to the 1st line.
        } 
    }
}
\end{lstlisting}

\section{Proof of the correctness of the algorithm}
Let us observe, that whole proof from the chapter \ref{Searching the spectrum} 
was independent from the choise of 
the starting configuration -- state of counters, flag value and pivot placement, as 
long as they would hold the inveriants that were prooved in 
\ref{lemmas for the proof of the correctness} and were used in 
\ref{proof of the correctness of the algorithm} and the fact, that flag value will 
correspond to the relation between corresponding \Eoc\ and $\frac{p}{q}$. 
We know, that the found solution was satisfying all the lemmas -- as it was the state of 
the counters at some pont of the execution. The only thing left to see, is that 
the flag value will be appropriete. 

Let $D = d_1d_2d_3\cdots$ be the solution. 
Let $c_p$ be the couter at which the pointer was when the solution was found. Then, since 
\ref{same value on the counters to the left} and the fact that after each change the value of 
the counter that pivot is at is the same as value of the counters to the left of it 
and \ref{state is ordered}, we conclude 
that 
all states that have value of $c_p$ greater than $d_p$ can not be solutions. 
As such, we can proceed from the state 
\begin{equation}
D' = (d_{p+1}+1)(d_{p+1}+1)(d_{p+1}+1)\cdots (d_{p+1}+1)(d_{p+1}+1)d_{p+2}d_{p+3}\cdots.
\end{equation}
Setting flag to "Less" after finding the solution, will result in producing exectly this state. 
From this point all the invariants will be safisfied and the algorithm will proceed 
until it finds another solution or it stops. 
%In algorithm from this chapter, after finding the solution the control flow is always redirected 
%to line 57th.  
%As such, since algorithm from this chapter is the repeting iteration 
%of the algorithm from chapter \ref{Searching the spectrum} and since \ref{}, 
%above algorithm will hold all nesseserly traits. 
From \ref{second_finiteness_theorem} we know, that there will be only finitely many 
solution.
$_\square$

\section{Implementation}\label{implementation}
As an appendix in the separate document, there is a source of a program with implementation 
of this algorithm with optimisation described below.
%with full  
%enhancments described in this chapter. 
It is written in \href{https://www.rust-lang.org/}{Rust}. 
It can be also found on 
\href{https://github.com/Sooyka/praca_magisterska_orbifoldy}{Github}
%\smalltodoII{dać ref do github} 
along with the \LaTeX\ source of 
this thesis.
%It is in the separate file, as it would take too much space in this 
%document and wouldn't be readable. 

%\subsection{Optimisations}
%Binary search

%\subsection{Limitations}
%i64

%\todo{dopisać}

