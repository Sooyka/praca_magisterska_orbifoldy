% mainfile: ../praca_magisterska_orbifoldy.tex
\chapter{Counting orbifolds -- arithmetical part}\label{algorithm}


%\section{Arithmetical part}

%We want to determine this $n$. If $n = 0$, then $\frac{p}{q}$ is not in $\sigma$. 
%If $n > 0$, then $$
%\subsection{Deciding number of occurences}
Searching for all occurences 

The difficulty here is to carefully step other an occurence. 

Compared to the previous version, we also use an occurance counter, starting with it set to 0 
and with the list of orbifolds, wich is empty at the start.
\begin{lstlisting}[firstnumber=1,consecutivenumbers=true]
In the case, the flag is set to: 
{
    "Less", then 
    {
        We increase the pivot counter by one ($b_c \coloneqq b_c + 1$).
        If $b_c = 2$ and the values of all the counters 
        on the left are also equal 2 then 
        {
            We end the whole algorithm with the answer "no".
        }
        We set the value of all counters on the left to $b_c$
        If $\chi^{orb}(*b_1b_2b_3\dots)=\frac{p}{q}$ then
        {
            We found an orbifold, we add it to a list 
            and increase the occurence counter by 1. 
            We set the flag to "Less".
            We put pivot to the $c+1$ counter.
            We go to the line 1..
        }
        If $\chi^{orb}(*b_1b_2b_3\dots)>\frac{p}{q}$ then  
        {
            We set the flag to "Greater".
            We put the pivot on the first counter. 
            We go to the line 1..
        } 
        If $\chi^{orb}(*b_1b_2b_3\dots)<\frac{p}{q}$ then
        {
            We set the flag to "Less".
            We put pivot to the $c+1$ counter.
            We go to the line 1..
        } 
    }

    "Greater", then
    {
        If $\chi^{orb}(*b_1\dots b_{c-1}\infty b_{c+1}\dots)=\frac{p}{q}$ then
        {
            We found an orbifold, we add it to a list 
            and increase the occurence counter by 1. 
            We set the flag to "Less".
            We put pivot to the $c+1$ counter.
            We go to the line 1..
        } 
        If $\chi^{orb}(*b_1\dots b_{c-1}\infty b_{c+1}\dots)>\frac{p}{q}$ then
        {
            We set $b_c$ to $\infty$.
            We set the flag to "Greater".
            We move pivot to the $c+1$ counter.
            We go to the line 1..
        }  
        If $\chi^{orb}(*b_1\dots b_{c-1}\infty b_{c+1}\dots)<\frac{p}{q}$ then
        {
            We search for value $b_c'$ of the $c$ counter 
            such that $\chi^{orb}(*b_1\dots b_{c-1}b_c'b_{c+1}\dots)\leq\frac{p}{q}$ 
            and $\chi^{orb}(*b_1\dots b_{c-1}(b_c'-1)b_{c+1}\dots)>\frac{p}{q}$.
            // More on how we search for it will be told later, 
            // for now we can think that we search one by one,
            // starting from $b_c$ and going up till $b_c'$.
            We set $b_c$ to $b_c'$.
            if $\chi^{orb}(*b_1b_2b_3\dots)=\frac{p}{q}$ then 
            {
                We found an orbifold, we add it to a list 
                and increase the occurence counter by 1. 
                We set flag to "Less".
                We go to the line 1..
            }
            We set all the counters to the left to value $b_c$.
            if $\chi^{orb}(*b_1b_2b_3\dots)=\frac{p}{q}$ then 
            {
                We found an orbifold, we add it to a list 
                and increase the occurence counter by 1. 
                We set flag to "Less".
                We move the pivot to the column $c+1$.
                We go to the line 1..
            }
            If $\chi^{orb}(*b_1b_2b_3\dots)<\frac{p}{q}$ then 
            {
                We set flag to "Less".
                We move the pivot to the column $c+1$.
                We go to the line 1..
            }
            If $\chi^{orb}(*b_1b_2b_3\dots)>\frac{p}{q}$ then 
            {
                We set the flag to "Greater".
                We move the pivot to the first counter.
                We go to the line 1..
            }
        }  
    }
}
\end{lstlisting}

\section{Why this works}

\section{Deciding the order}
Let $m \in \mathbb{N}$ be such that $\frac{p}{q} \in (1-\frac{m}{2},1-
\frac{m+1}{2})$
Let us denote by $r \coloneqq \frac{p}{q} - (1-\frac{m}{2})$. \\ 

We will searching in $\sigma$ as such: \\

If $\frac{p}{q} \in \sigma$, then, from the corollary \ref{predescors} we know, that there 
exist some $n \in \mathbb{N}$, such that $\frac{p}{q} + \frac{n}{2} \in \sigma$ but 
$\frac{p}{q} + \frac{n}{2} \not\in \sigma$. \\

We will be consequently checking points from $1+r$, through $1+r-\frac{l}{2}$, for 
$0 \leq l \leq m$, to the $\frac{p}{q}$. We stop at the first found point. 
If one of these point is in the spectrum, then all smaller (so also $\frac{p}{q}$) are in 
the spectrum and $\frac{p}{q}$ is the accumulation point of the spectrum of order $m-l$ 
(from this, 
we can see some heuristic, that the points that have smaller order will be generally 
harder to find in some sense). If none of this points are in in the spectrum, then $\frac{p}{q}$ 
is not. \\




\section{Implementation}
As an appendix in the separate document, there is a source of a program with implementation 
of this algorithm 
with full  
enhancments described in this chapter. It is written in Rust. 
It can be also found on \smalltodoII{dać ref do github} along with the \LaTeX\ source of 
this thesis.
%It is in the separate file, as it would take too much space in this 
%document and wouldn't be readable. 



